\paragraph{System B.~Moderate potential for serendipity -- Generating conjectures and proofs} 
Suppose that rather than checking a known theorem, the user programs
the computer to come up with conjectures, and generate proof-attempts on
its own.  Furthermore, suppose that three out of 100 generated
conjectures turn out to be provable.  In this case, the user may be
interested and pleasantly surprised -- especially if one or more of
the theorems is one that he or she wouldn't have thought of.  In this
case, each of the generated conjectures is a potential
\textbf{trigger} for discovery.  Note that although the system itself
generates these conjectures, in general it cannot determine in advance which ones,
if any, will turn out to be true.  Its ability to assess the generated
conjectures and to construct proof-attempts constitutes an elementary
\textbf{prepared mind}.  The system is (fallibly) able to apply
pre-programmed methods to form a \textbf{bridge} to an interesting
\textbf{result}, namely, a new theorem.  Indeed, fallibility applies
twice over: not only may the system fail to find the interesting
conjectures, it may also fail to find proofs for all of the (true)
conjectures that it does discover.
Furthermore, note that due to its limited domain knowledge, the system
has only a weak model of the way \textbf{value} will be assigned to
any theorem it finds.
%% Conjectures arise through some
%% combinatorial or other similar process, which, although deterministic,
%% selects a sample from the population of available conjectures.
% Although the conjectures are presumably deterministically constructed,
% they may appear random; more specifically, i
In lieu of further information about the conjecture generation process
(and about those conjectures which are not generated)
it seems that every potential trigger for discovery is
encountered by definition, so that \textbf{chance} does
not play a significant role at that stage.
% \footnote{For example, in lieu of further information
%  about the conjecture-generating algorithm, we might regard a
%  conjecture's ``order of arrival'' as a proxy for its probability of
%  being encountered, after a suitable normalisation.  So that if $n$
%  conjectures are selected, the $j$th conjecture could be assigned the
%  ``subjective probability'' $2(n+1-j)/n(n+1)$.}
However, the system may make use of simple heuristics -- based, for example, on
a computed \emph{plausibility measure}
\cite[p.~193]{colton2007computational} -- to keep it from focusing on
a conjecture that it isn't likely to prove, so it is capable of a
somewhat discriminating form of \textbf{curiosity}.  It will effect a
\textbf{focus shift} to each plausible trigger independently, in turn.
This conservative behaviour also contributes to the system's
\textbf{sagacity}, which is otherwise grounded in proof-generation
techniques.  This system matches all of our criteria for serendipity,
although we should stress that its ability to generate new mathematics
will depend, in part, on the initial selection of problem domain --
and, to a considerable extent, on the programmer's ingenuity.  Among
the environmental factors from Section
\ref{sec:environmental-factors}, this system matches the description
of \textbf{multiple tasks}, but not the others.