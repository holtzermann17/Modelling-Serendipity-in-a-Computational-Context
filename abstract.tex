\begin{abstract}
%% INNOVATIVE
%% Serendipity has played a key role in fundamental scientific and
%% technical advances, but its potential to contribute to discovery and
%% innovation by computational systems is often downplayed.
%% %% CONTRIBUTION TO UNDERSTANDING AND KNOWLEDGE
%% The concept of serendipity has, however, been taken up by researchers
%% in information science, for whom it denotes the development of systems
%% that support happy chance discoveries for their users.
%% %% NEW ARGUMENTS
%% rticle, we advance a novel perspective, shifting the focus
%% from ``serendipity as a service'' to artificial systems that
%% themselves catalyse, evaluate and leverage serendipitous occurences.
%% INTERPRETATIONS OR INSIGHTS
While a system cannot be made to perform serendipitously on demand,
nevertheless, we argue that its ``serendipity potential'' can be
increased by means of a suitable system architecture.
%%%
%%%
%% %% IMAGINATIVE SCOPE
%% Rather than relying on the system's user to contribute mental or
%% physical effort, such systems are required to be suitably self-reliant
%% in relevant cognitive, conative, and affective respects.
%% %% ASSEMBLING OF INFORMATION IN AN INNOVATIVE WAY
%% Based on a survey of theoretical literature, we advance a %sequential
%%  model of serendipitous occurrences in a computational context that
%% stresses the role of creativity, invention and discovery.
%% %%
%% We also survey existing implementations and describe how the ideas
%% they embody could be reused to realise our framework's individual
%% components.
%% %%
%% This allows us to assess the current progress towards systems that
%% exhibit serendipitous behaviour, and the gaps that lie between current
%% systems and a full implementation of the model.
%%
%% NEW THEORETICAL FRAMEWORKS AND CONCEPTUAL MODELS
We elaborate a cognitively-inspired model of serendipity potential,
described in six phases:
%%
\emph{perception}, \emph{attention}, \emph{interest},
\emph{explanation}, \emph{bridge}, and \emph{valuation}.
%%
Earlier phases can be returned to from later phases as a creative
process develops, in context, with the active participation of a
creative agent, but not entirely within that agent's control.
%%
We evaluate the model by applying it to existing systems, and point to
further applications in automated programming, recommender systems,
and computational creativity.
%%
We conclude that equipping computational systems with the potential for serendipity is widely applicable across
different artificial intelligence applications.
%% %% ENHANCEMENT OF KNOWLEDGE, THINKING, UNDERSTANDING
%% We discuss how designers and could benefit from a full implementation
%% of ``serendipity in the system'', and some of the challenges that
%% designing for serendipity raises.
%% %% PRACTICE
%% Since such systems are not strictly controlled by the programmer,
%% their long-term behaviour raises a number of practical and ethical
%% concerns.  In particular, one key aspect of the framework is that it
%% envisions systems that invent their own evaluation functions.
%% %% COHERENCE
%% We argue that the concept of serendipity can be used as a safeguard in
%% meta-evaluations that ensure the system's long-term behaviour
%% benefits its user and other stakeholders over the long run.
%%% I've run out of steam here, but we should deal with the
%%% following issues in the discussion and conclusion, and then
%%% come back to the abstract and add further details. -JC
%% ANALYTICAL POWER
%% DEPTH OF SCHOLARSHIP
%% APPROPRIATE ENGAGEMENT WITH OTHER RELEVANT WORK
%% PROFOUND INFLUENCE
%% INSTRUMENTAL IN DEVELOPING NEW THINKING
%% PRACTICES
%% PARADIGMS
%% POLICIES
%% A MAJOR EXPANSION OF THE RANGE AND THE DEPTH OF RESEARCH
%% OUTSTANDINGLY NOVEL
% 4-6 keywords
%% \keywords{Serendipity,
%% Discovery Systems,
%% Computational Creativity,
%% Recommender Systems,
%% Automated Programming}
\end{abstract}


\iffalse
%OLD ABSTRACT
Building on a survey of previous theories of serendipity and
creativity, we advance a phased model of serendipitous occurrences. We distinguish between serendipity as a service and serendipity in the system itself, clarify the role of invention and discovery, and provide a measure for the serendipity potential of a system.
%%%
While a system 
can arguably not be guaranteed to be serendipitous, it can have a 
high potential for serendipity. Practitioners can use these theoretical tools to evaluate a
computational system's potential for
unexpected behaviour that may have a beneficial outcome.
%%%
In addition to a qualitative features of serendipity potential, the
model also includes quantitative ratings that can guide development
work. We show how the model is used in three case studies of existing
and hypothetical systems, in the context of evolutionary computing,
automated programming, and (next-generation) recommender systems.
From this analysis, we extract recommendations for practitioners
working with computational serendipity, and outline future directions
for research.
\fi

\newpage
