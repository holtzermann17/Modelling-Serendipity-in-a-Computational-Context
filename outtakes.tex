%% From introduction:
\begin{comment}
\citeA{van1994anatomy} -- echoing the
(negative) reflections on the potential
for a purely computational approach to mathematics from \citeA{poincare1910creation} -- claimed that:
\begin{quote}
``\emph{Like all intuitive operating, pure serendipity is not amenable
    to generation by a computer.  The very moment I can plan or
    programme `serendipity' it cannot be called serendipity
    anymore}.'' \cite[p.~646]{van1994anatomy}
\end{quote}
We believe that serendipity is not so mystical as such statements
might seem to imply.  In Section \ref{sec:discussion} we will show
that ``patterns of serendipity'' like those collected by van Andel can
be applied in the design of computational systems.  Purposive acts can
have unintended consequences \cite{merton1936unanticipated}.
Similarly, even if we cannot plan or program serendipity, we can
prepare for it.

Indeed, if serendipity was ruled out as a matter of principle, computing would
be restricted to happy or unhappy \emph{unsurprises} -- preprogrammed,
preunderstood behaviour -- that would be interspersed periodically,
perhaps, with an \emph{unhappy} surprise.  Venkatesh Rao
\cite{rao2015breaking} uses the term \emph{zemblanity} -- after
William Boyd \cite{boyd2010armadillo}: ``the
opposite of serendipity, the faculty of making unhappy, unlucky and
expected discoveries by design'' -- to describe systems that are
doomed to produce \emph{only} unhappy unsurprises.  According to Rao, this is
the implied fate of systems that are tied inextricably to a fixed
vision, from which any (inevitable) deviation constitutes a mistake.  This condition
stands at a sharp contrast with the ``second-order cybernetics''
introduced by \citeA{von2003cybernetics}, which envisions systems that
are able to specify their own purpose, and adapt it with respect to a
wider environment.  It also contrasts with Taleb's
\cite{taleb2012antifragile} notion of ``antifragility'' in
which disturbances within a certain range strengthen the system.
\citeA{minsky1967programming} argues that any sufficiently complex
computational system is bound to make decisions that its creators
could not foresee, and may not fully understand.  Demonstrably
gracefully behaviour in response to unexpected circumstances, and a
preference for ``happy'' as opposed to ``unhappy'' outcomes may be
prerequisites for the development of autonomous systems that are
worthy of our trust.  While not the same as ``Serendipity as a
Service'', such systems should at least be able to recognise
serendipity when it is happening.  Over time a system might also come
to recognise previous missed opportunities and false realisations --
what van Andel \cite[p.~639]{van1994anatomy} terms \emph{negative
  serendipity} -- and learn from them.\ednote{\textbf{R1}: sweeping unsubstantiated statements: "Demonstrably gracefully behaviour in response to unexpected circumstances, and a preference for ``happy'' as opposed to ``unhappy'' outcomes may be prerequisites for the development of autonomous systems that are worthy of our trust." "If we can build systems that can communicate with us about their unexpected discoveries and reason about the value of these discoveries in a socially responsible way, then we may be reasonably confident that they will behave in an ethical manner." This gratuitous philosophising is unnecessary. The central ideas in the paper should be assessed in their own right without such speculative claims.}

These issues are particularly relevant in light of current
high-profile discussions around the role of AI in society.  These
discussions centre on a dilemma in which the core issues described
above come sharply into focus.  Although we ``we cannot predict what
we might achieve'' with AI, nevertheless many researchers,
technologists, and policy makers subscribe to a (presumably healthy)
conservatism that says that AI systems should ``do what we want them
to do'' \cite{research-priorities}.  A computationally robust notion
of serendipity may go a considerable ways towards resolving the
underlying dilemma.  If we can build systems that can communicate with
us about their unexpected discoveries and reason about the value of
these discoveries in a socially responsible way, then we may be
reasonably confident that they will behave in an ethical manner.

Less controversial than ``programmed serendipity'', but no less worthy
of study, is serendipity that arises in the course of user
interaction.  Indeed, it could be argued that everyday social media
already offers something approaching ``Serendipity as a Service''.
The user logs on hoping, but without any guarantee, that they will
find something interesting, charming, or entertaining, and potentially
relevant to whatever is going on in their life at the moment.
Of course, it should not be assumed that any system that can accommodate
user interaction can generate serendipity; take for example the use of
a calculator, where the potential for serendipity through user
interaction is minimal.  The frameworks introduced in this paper are
broad enough to be used in the design and evaluation of sociotechnical
systems, and we will touch on some examples, however we focus on
modeling serendipity in a computational context.\ednote{\textbf{R1}: We gradually find out that the review of systems considers not individual occurrences of serendipity, but "potential for serendipity", apparently a property of computer systems.}
\end{comment}


%% From Discussion/Related work

\edcomment{As we touched on in Section \ref{sec:nextgenrec}, serendipity in the
field of recommendation systems is understood to imply that the system
suggests \emph{unexpected} items, which the user considers to be
\emph{useful}, \emph{interesting}, \emph{attractive} or
\emph{relevant} \cite{foster2003serendipity,Toms2000}.
\citeA{Herlocker2004} and \citeA{McNee2006} view serendipity to be
important component of recommendation quality, alongside accuracy and
diversity.
% \cite{Herlocker2004} \cite{Lu2012},\cite{Ge2010}.  
Definitions differ as to the requirement of \emph{novelty};
\citeA{Adamopoulos2011}, for example, describe systems that suggest
items that may already be known, but which are still unexpected in the
current context.  While standardised measures such as the $F_1$-score
or the (R)MSE are used to determine the \emph{accuracy} of a
recommendation (i.e.~whether the recommended item is very close to
what the user is already known to prefer), as yet there is no common
agreement on a measure for serendipity, although there are several
proposals
\cite{Murakami2008,Adamopoulos2011,McCay-Peet2011,iaquinta2010can}.
In terms of the framework from Section \ref{sec:by-example}, these
systems focus mainly on generating a \emph{trigger} to be processed
by the user, and prepare the ground for serendipitous \emph{discovery}.
Intelligent modelling approaches could potentially bring additional aspects
of serendipity into play in future systems, as  discussed in Section
\ref{sec:nextgenrec}.}{Christian: Remove whole paragraph? We already have some related work in the rec sys section, and there's no need to talk more specifically about domain-specific requirements.}


%% From introduction

A serendipitous discovery represents an ``unsought finding'' \cite[p. 631]{van1994anatomy}, and is thus strongly related to re-evaluation: For example, a failed attempt to develop an ultra-strong superglue resulted in a re-stickable adhesive that no one was quite sure how to use.\ednote{\label{note:logic}\textbf{R1}: The logical structure of the ideas in the paper (not the list of section topics) could be made clearer.}  After considerable trial and error, this turned out to be just the right ingredient for making the now ubiquitous Post-it\texttrademark\ notes.
%
% In this way, serendipity is related to deviations from expected or
% familiar patterns, and to new insight.
%

%Importantly, serendipity is not the same as luck. 
% * <chris.abyi@gmail.com> 2016-05-04T14:44:53.713Z:
%
% > When we consider the practical uses for weak glue, the possibility
% > that a life-saving antibiotic might be found growing on contaminated
% > petri dishes, or the idea that burdock burrs could be anything but
% > annoying, we encounter radical changes in the evaluation of what's
% > interesting.  Importantly, serendipity is not the same as luck. 
%
% Foo
%
% ^.

  

These groups would clearly benefit from serendipitous discoveries supported or even triggered by a computational system. Such ``serendipity as a service'' is of particular interest in information retrieval and recommender systems. In the context of this paper, we want to promote an alternative, system-internal view on serendipity in a computational context, in which a system can leverage serendipity as part of its inner workings, and thus improve its second-order services for the user. \citeA{minsky1967programming} argues that any sufficiently complex computational system is bound to make decisions that its creators could not foresee, and may not fully understand. We consequently adopt van Andel's view that serendipity cannot be programmed \cite[p.~646]{van1994anatomy}, in terms of not ``pre-determined''. Nevertheless, we suggest that a computational system can be designed to have a \emph{potential} for serendipitous discoveries. Re-evaluation again contributes to this goal: A system could detect previously missed opportunities and false realisations -- what van Andel \cite[p.~639]{van1994anatomy} terms \emph{negative serendipity} -- and learn from them.

In this paper, we develop a process-based model to design and evaluate computational systems with a potential for serendipity. Our model is partly based on an earlier discussion of serendipity in a computational context \cite{pease2013discussion}, and we adopt the \emph{Standardised Procedure for Evaluating Creative Systems} (SPECS) \cite{jordanous:12} to develop a stategy for assessing the potential for serendipity in a given system. Importantly, we suggest that serendipity is not only about discovery, but in many circumstances also about invention, which relates it closely to creativity.  

%%% Removing whole future work section -- perhaps it can be alluded to with a sentence in the conclusions

\subsection{Future Work} \label{sec:futurework} \label{sec:hatching}

\ednote{Christian: Remove whole section, or shorten it to one paragraph like "we are interested to apply this model to design patterns because.." as part of the conclusion/future work section. This sounds like the beginning of another paper and is too detailed. Alternatively, transform it into a proper case-study.}In looking for ways to manage and encourage serendipity, we are drawn
to the approach taken by the \emph{design pattern} community
\cite{alexander1999origins}. 
\citeA{meszaros1998pattern} describe the typical scenario for authors of design patterns:

\begin{quote}
\noindent ``\emph{You are an experienced practitioner in your field.  You
have noticed that you keep using a certain solution to a commonly
occurring problem.  You would like to share your experience with
others.}''
\end{quote}

\noindent There are many ways to describe a solution. Meszaros and Doble remark,
\begin{quote}
\noindent ``\emph{What sets patterns apart is their ability to explain the
rationale for using the solution (the `why') in addition to describing
the solution (the `how').}''
\end{quote}
Regarding the criteria that pattern writers seek to address: 
\begin{quote}
\noindent ``\emph{The most appropriate solution to a problem in a context is
the one that best resolves the highest priority forces as determined
by the particular context.}''
\end{quote}

%
%% Their article describes a number of criteria relevant to writing
%% good design patterns, e.g. \emph{Clear target audience},
%% \emph{Visible forces}, and \emph{Relationship to other patterns}.
%
Applying the solution achieves this resolution of forces in the
application domain.
The design pattern itself achieves something further: it encapsulates
knowledge in a brief, shareable form.  Tracing the steps involved, we
see that the creation of a new design pattern is always somewhat
serendipitous (Figure \ref{fig:pattern-schematic}; compare Figure
\ref{fig:1b}).
%%
To van Andel's assertion that ``The very moment I can plan or
programme `serendipity' it cannot be called serendipity anymore,'' we
reply that patterns -- and programs -- can include built-in
indeterminacy.  Moreover, we can foster circumstances that may make
unexpected happy outcomes more likely, by designing and developing
systems that increasingly address the challenges outlined in Section
\ref{sec:recommendations}.  Such systems will encounter unexpected
stimuli, become curious about them, sagaciously pursue enquiry
within a social context, and assess the value of any outcomes.
%
Figure \ref{fig:va-pattern-figure} shows one example of a design
pattern that can be used to \emph{plan for serendipity}, based on the
``\emph{Successful Error}'' pattern identified by van Andel.  In
future work, we intend to build a more complete serendipity pattern
language -- and put it to work within autonomous programming systems.

% Is ``having a stretch goal'' an example of a serendipity pattern?  I think so!

\begin{figure}[!h]
\vspace{.3cm}
\begin{center}
\begingroup
\tikzset{
block/.style = {draw, fill=white, rectangle, minimum height=3em, minimum width=3em},
tmp/.style  = {coordinate}, 
sum/.style= {draw, fill=white, circle, node distance=1cm},
input/.style = {coordinate},
output/.style= {coordinate},
pinstyle/.style = {pin edge={to-,thin,black}}
}

\begin{tikzpicture}[auto, node distance=2cm,>=latex']
    \node [sum] (A-sum1) {};
    \node [input, name=pinput, above left=.7cm and .7cm of A-sum1] (A-pinput) {};
    \node [input, name=tinput, left=2.2cm of A-sum1] (A-tinput) {};
    \node [input, name=minput, below left of=A-sum1] (A-minput) {};
    \node [input, name=minput, right of=A-sum1] (A-moutput) {};

    \draw [->] (A-pinput) -- node{{\footnotesize% $p$:
                                   problem}} (A-sum1);
    \draw [->] (A-tinput) -- node{\vphantom{{\footnotesize g}}{\footnotesize% $T$:
                                                                 context~}} (A-sum1);

    \node [sum, right=2.3cm of A-sum1] (B-sum1) {};
    \node [input, name=pinput, above left=.7cm and .7cm of B-sum1] (B-pinput) {};
    \node [input, name=tinput, left of=B-sum1] (B-tinput) {};
    \node [input, name=minput, below left of=B-sum1] (B-minput) {};
    \node [sum, right=2cm of B-sum1] (B-sum2) {};

    \node [input, name=minput, right of=B-sum2] (B-moutput) {};
    \draw [->] (B-pinput) -- node{{\footnotesize% $p^{\prime}$:
        \vphantom{{\footnotesize p}}rationale}} (B-sum1);

    \draw [->] (A-sum1) -- node{\vphantom{{\footnotesize g}}{\footnotesize %$T^{\star}$:
                                                               solution~~}} (B-sum1);

    \draw [->] (B-sum1) -- node{\vphantom{{\footnotesize g}}{\footnotesize % $R$:
                                                               pattern}} (B-sum2);
    \draw [->] (B-sum2) -- node[text width=1.5cm,execute at begin node=\setlength{\baselineskip}{1ex}]{\footnotesize % $|R|>0$:
                                                                                                       \emph{new~shared\\knowledge}} (B-moutput);
\end{tikzpicture}

\endgroup
\end{center}

\vspace{-.3cm}
\caption{The components of design patterns mapped to our process schematic\label{fig:pattern-schematic}}
\end{figure}

\begin{figure}[!h]
\setlist[description]{font=\normalfont\itshape}
{\normalsize
\begin{mdframed}
\vspace{2mm}
\textbf{\emph{Successful error}}~
\begin{description}[leftmargin=0\parindent,labelindent=0em,itemsep=10pt]
\item[{Context.}] You run an organisation with different divisions and
  contributors with varied expertise.  People routinely discover
  interesting things that no one knows how to turn into a product.
\item[{Problem.}]  How can you get the most value from this sort of discovery?
\item[{Solution.}] Allow people to work on pet projects, and encourage
  interaction between people in different divisions.  Set aside time
  for in-house seminars.
\item[{Rationale.}] Prototypes can be discussed, even if they are not
  directly marketable.  Following the interests of contributors
  preserves their autonomy.  Contact with different points of view
  brings additional knowledge to bear.
\item[{Resolution.}] 
Open discussion plays to everyone's strengths.  It can
  expose flaws at any stage, and can help to guide work in the direction of
  real innovation.  Participants in these conversations
  will learn something, and will help each other maximise the value of
  the discovery.  
\item[{Example.}] Low-tack restickable glue, discovered by a 3M
  engineer in 1968, ultimately proved useful for making
  Post-it\texttrademark\ Notes, which were launched in 1980 after
  several rounds of in-house prototyping.
\end{description}
\vspace{-1mm}
\end{mdframed}
}
\caption{A standard design pattern template applied to van Andel's serendipity pattern, \em{Successful error}\label{fig:va-pattern-figure}}
\end{figure}

%%% Removed this from related work since it's discussed in the introduction now

Paul Andr{\'e} et al.~\cite{andre2009discovery} previously
proposed a two-part model of serendipity encompassing ``the chance
encountering of information, and the sagacity to derive insight from
the encounter.''  The first phase has been automated more frequently
-- but these authors suggest that computational systems should be
developed that support both aspects.  They specifically suggest to
pursue this work by developing systems with better representational
features: \emph{domain expertise} and a \emph{common language model}.

These features seem to exemplify aspects of the \emph{prepared mind}.
However, the \emph{bridge} is a distinct step in the process that
preparation can support, but that it does not always fully determine.
Domain understanding is not always a precondition; it can be emergent.
For instance, persons involved in a dialogue may understand each other
quite poorly, while nevertheless finding the conversation interesting
and ultimately rewarding.  Misunderstandings can present learning
opportunities, and can develop \emph{new} shared language. %%  Various social strategies, ranging from Writers
%% Workshops, to open source software, to community-based approaches in
%% psychological counselling have been developed to exploit similar
%% emergent effects and to develop \emph{new} shared language
%% \cite{gabriel2002writer,seikkula2014open}.

%%% Removed from related work

\ednote{Christian: Remove paragraph. I agree with the reviewer: this
  does not relate closely enough to the main topic, and disrupts the
  flow.} Thus, serendipity is particularly relevant for thinking about
\emph{autonomous systems}.\ednote{\textbf{R1}: a tendency to ramble
  off the topic slightly, pulling other interesting issues into the
  paper on p.25, the final paragraph of Sect 6.1} There is a certain
amount of apprehension and concern in circulation around the idea of
autonomous systems.  \citeA{machine-ethics-status} suggest that these
concerns ultimately come back to the question: will these systems
behave in an ethical manner?  The more we constrain the system's
operation, the less chance there is of it ``running off the rails.''
However, constraints come with a serious downside.  Highly constained
systems will not be able to \emph{learn} anything very new while they
operate.  If this means that the system's ethical judgement is fixed
once and for all, then we cannot trust it to behave ethically when
circumstances change.  Highly constrained systems are unlikely to be
convincingly \emph{social}, inasmuch as the constraints rule out
emergent behaviour in advance.  

%% \begin{quote}
%% ``\emph{Tinkering is a process of serendipity-seeking that does not
%%     just tolerate uncertainty and ambiguity, it requires it.  When
%%     conditions for it are right, the result is a snowballing effect
%%     where pleasant surprises lead to more pleasant surprises.}''
%%   \cite[``Tinkering versus Goals'']{rao2015breaking}
%% %% What makes this a problem-solving mechanism is diversity of individual perspectives coupled with the law of large numbers (the statistical idea that rare events can become highly probable if there are enough trials going on). If an increasing number of highly diverse individuals operate this way, the chances of any given problem getting solved via a serendipitous new idea slowly rises. This is the luck of networks.

%% %% Serendipitous solutions are not just cheaper than goal-directed ones. They are typically more creative and elegant, and require much less conflict. Sometimes they are so creative, the fact that they even solve a particular problem becomes hard to recognize. For example, telecommuting and video-conferencing do more to “solve” the problem of fossil-fuel dependence than many alternative energy technologies, but are usually understood as technologies for flex-work rather than energy savings.

%% %% Ideas born of tinkering are not targeted solutions aimed at specific problems, such as “climate change” or “save the middle class,” so they can be applied more broadly. As a result, not only do current problems get solved in unexpected ways, but new value is created through surplus and spillover. The clearest early sign of such serendipity at work is unexpectedly rapid growth in the adoption of a new capability. This indicates that it is being used in many unanticipated ways, solving both seen and unseen problems, by both design and “luck”.
%% \end{quote}
%% If we control the system, at bottom the best we can hope for is
%% ``pleasant unsurprises.''  At the same time, understanding serendipity
%% may help build autonomous systems that produce fewer ``unpleasant surprises,'' a
%% serious contemporary concern
%% \cite{philosophy-machine-morality,machine-ethics-status}.

%
% Ritchie initially bases his metrics on human judgment, but points out different ways to compute them automatically, arising from practical study.  For instance, quality could be computed using a fitness score of the assessed artifacts, which should highly correlate with human-perceived quality.  The typicality of produced artifacts was calculated as their similarity to the artifacts inspiring the generative process.  Nevertheless, this requires a good distant metric.  Both fitness functions and distance metrics are subject to an ongoing debate in computational aesthetics.

%% Although the notion of serendipity that we have developed is
%% process-focused, value is a crucial dimension of serendipity, and
%% evaluations of an outcome (often an artefact) continue to be relevant.
%% Furthermore, 


\ednote{Christian: Remove rest of paragraph? Goes a bit too far..}
Establishing complex analogies between evolving problems and solutions
is one of the key strategies used by teams of human designers
\cite{Analogical-problem-evolution-DCC}.  In computational research to
date, the creation of new patterns and higher-order analogies is
typically restricted to a simple and fairly abstract ``microdomain''
\cite{hofstadter1994copycat,DBLP:journals/jetai/Marshall06}.
%
Turning over increased responsibility to the machine will be important
if we want to foster the possibility of genuine surprises.


The {\sf SerenA} system developed by Deborah Maxwell et
al.~\cite{maxwell2012designing} offers a case study in some
of these concepts.  This system is designed to support
serendipitous discovery for its (human) users
\cite{forth2013serena}.  The authors rely on a process-based
model of serendipity \cite{Makri2012,Makri2012a} that is derived
from user studies which draw on interviews with 28 researchers.
Study participants  were asked to look for instances of
serendipity from both
their personal and professional lives.  The research aims to
support the formation of bridging connections from an unexpected
encounter to a previously unanticipated but valuable outcome.
The theory focuses on the acts of reflection that support both
the creation of a bridge, and the estimation of the potential
value of the result.
%
While this description touches on all of the features of our model, {\sf
  SerenA} largely matches the description offered by Andr{\'e} et
al.~\cite{andre2009discovery} of discovery-focused systems, in which
the user experiences an ``aha'' moment and takes the
creative steps to realise the result.  {\sf SerenA}'s primary computational method is to
search outside of the normal search parameters in order to engineer
potentially serendipitous (or at least pseudo-serendipitous)
encounters.

%%% By example

  The dimension of
  sagacity can also go some way towards describing, if not explaining,
  how it is this person rather than that person makes a particular
  societal contribution.  For example, Edward Jenner was not the first
  person to observe that cowpox innoculation prevents smallpox
  contagion, but his experimental verification of this principle, and
  his development and promotion of the first ``vaccines'' were
  important scientific and social advances \cite{riedel2005edward}.

 \ednote{Christian: Remove rest of paragraph, if term not picked up later?} Juxtaposition of interrelated contexts may result in what
   novelist John Barth \cite[p.~311]{barth1992last} refers to as
   ``logistically assisted serendipity.''


%%% Notes from discussion with Simon about ATP.

%% \begin{itemize}
%% \item e.g. if you use an automated theorem prover, why is that very unlikely to pass our definition of serendipity?  If someone has 100 conjectures, are YOU more likely to be surprised?  Start with an ATP guy proving a theorem, and expand it until it’s a computational example.  Start w/ something straightforward that fails all of our criteria.
%% \item E.g. machine learning is dynamic (e.g. logs are changing minute by minute).  Software writing an algorithm (e.g. via deep learning) is trying to solve an open problem.  1+1 isn’t going to lead to serendipity.  However, if a system is working in an online way — then we move towards a.
%% \item The data set, the labelling can change.  The algorithm can change.  Classifier (output) changes.  We can offer advice to your average AI researcher who says, get a classifier, throw data in — but using e.g. crowdsourcing for positives and negatives can change day by day.  
%% \item Extract the text and send to simon —
%% \end{itemize}

%% \ednote{\textbf{R1}: Why is "focus shift"
%%   (a.k.a. "reevaluation") a "condition", while "prepared mind" is a
%%   "component"? Overall, the logical thread of the argumentation, and
%%   the status of all the concepts, should be spelled out more
%%   explicitly.}\ednote{\textbf{R2}: There were numerous examples of
%%   serendipity provided that preceded these case studies but they all
%%   were grounded in the sciences. It would have been helpful to use a
%%   more diverse set of examples beyond the usual suspects (penicillin,
%%   Velcro). I think because of the physicality of these examples it was
%%   difficult to understand how they could be translated
%%   computationally.  The authors mention the relevance of serendipity
%%   in several sectors in the second paragraph of the intro, perhaps
%%   exploring some of these would help. Examples from the growing body
%%   of research on serendipity in the information sciences may prove
%%   helpful in this respect.}



%% % \input{2c-related-work.tex}


%% : after all,
%% in our experiment the 101st conjecture and any further additions to
%% the sequence are not considered.
%% Joe: Here, I'm thinking again of Bertrand's paradox, 
%% Bernoulli's ``Principle of Insufficient Reason''
%% (aka Keynes's ``Principle of Indifference'')
%% ... and things like the notion of ``chaos''
%% and ``plausibility'' \cite[p.~193]{colton2007computational}



%\vspace{-.3cm}
\subsubsection*{Step 1: Identify a definition of serendipity that your system should satisfy to be considered serendipitous.}
%~\\
%\vspace{-.1cm}

\noindent We adopt the definition of serendipity from the previous subsection
\ref{sec:modelDefinition}.

%\vspace{-.3cm}
\subsubsection*{Step 2: Using Step 1, clearly state what standards you use to evaluate the serendipity of your system.}
%~\\
%\vspace{-.1cm}

\noindent With our definition and other features of the model in mind, we propose the following standards for evaluating serendipity in computational systems. These criteria allow the evaluator to assess the degree of seredipity that is present in a given system's operation.

%% Serendipity relies on a reassessment or reevaluation -- a \emph{focus shift} in which something that was previously uninteresting, of neutral, or even negative value, becomes interesting.

\begin{description}[itemsep=16pt]
\item[{(\textbf{A - Definitional characteristics})}] {The system can
  be said to have a {\textbf{prepared mind}}, consisting of previous
  experiences, background knowledge, a store of unsolved problems,
  skills, expectations, readiness to learn, and (optionally) a current
  focus or goal.  It then processes a {\textbf{trigger}} that is at
  least partially the result of factors outside of its control,
  including randomness or unexpected events.  It classifies this
  trigger as interesting, constituting a {\textbf{focus shift}}.  The
  system then uses reasoning techniques and/or social or otherwise
  externally enacted alternatives to create a {\textbf{bridge}} from
  the trigger to a result.  The {\textbf{result}} is evaluated as
  useful, by the system and/or by an external source.}  The evaluator
  should specify all of these aspects relative to the system under
  consideration at a sufficient degree of precision to show their
  processual interconnection.
%%%%%%%%%%%%%%%%%%%%%%%%%%%%%%%%%%%%%%%%%%%%%%%%%%%%%%%%%%%%%%%%%%%%%
\item[{(\textbf{B - Dimensions})}] {Serendipity, and its various
  dimensions, can be present to a greater or lesser degree.  If the
  criteria above have been met, we consider the system (and
  optionally, generate ratings as estimated probabilities) along
  several dimensions:
%
{($\mathbf{a}$: \textbf{chance})} how likely was this trigger to appear to
  the system?
%
{($\mathbf{b}$: \textbf{curiosity})} On a population basis, comparing
similar circumstances, how likely was the trigger to be identified as
interesting?
%
{($\mathbf{c}$: \textbf{sagacity})} On a population basis, comparing
similar circumstances, how likely was it that the trigger would be
turned into a result?
%
Finally, we ask, again, comparing similar results where possible:
{($\mathbf{d}$: \textbf{value})} How valuable is the result that
is ultimately produced?}
%
%Then combining $\mathbf{a}\times\mathbf{b}\times\mathbf{c}$ gives a
 % likelihood score: 
\begin{mdframed}
\vspace{.1cm} {\textbf{\emph{Likelihood score and final ruling.}} A low (nonzero) likelihood score $\mathbf{a}\times\mathbf{b}\times\mathbf{c}$ and
  high value $\mathbf{d}$ imply that the event was ``serendipitous.''
  In all other cases, the event was ``unserendipitous.''}
\end{mdframed}
%%%%%%%%%%%%%%%%%%%%%%%%%%%%%%%%%%%%%%%%%%%%%%%%%%%%%%%%%%%%%%%%%%%%%
\item[{(\textbf{C - Factors})}] {Finally, if the criteria from Part A
  are met, and if the event is deemed sufficiently serendipitous to
  warrant further investigation according to the criteria in Part B,
  then in order to deepen our qualitative understanding of the
  serendipitous behaviour, we ask: To what extent does the system
  exist in a {\textbf{dynamic world}}, spanning {\textbf{multiple
      contexts}}, featuring {\textbf{multiple tasks}}, and
  incorporating {\textbf{multiple influences}}?}\ednote{\textbf{R1}: the status of the additional "factors" (dynamic world, etc.) is not totally clear. Are these necessary conditions for something to count as serendipity, or is there a hypothesis that these factors tend to promote serendipity?}
\end{description}

%\vspace{-.3cm}
\subsubsection*{Step 3: Test your serendipitous system against the standards stated in Step 2 and report the results.}
%~\\
%\vspace{-.1cm}

\noindent In Section \ref{sec:computational-serendipity}, we will pilot our framework by examining the degree of serendipity of existing and hypothetical computational systems. 

\subsection{Using SPECS to evaluate computational serendipity}\label{specs-overview}

In this section, we use the elements of the conceptual framework
described in Section \ref{sec:by-example} help to flesh out this
definition, to develop quite detailed evaluation criteria.
We adapt the \emph{Standardised Procedure for Evaluating Creative Systems} (SPECS),
a high-level, customisable evaluation strategy that was devised to judge the creativity
of computational systems \cite{jordanous:12}.\ednote{\textbf{R1}: The allusions to Jordanous' SPECS methodology are specious. In effect, the authors argue that they have defined their terms, given at least rough operational descriptions for these terms, and then stuck to their own specifications. This is just general good practice in any empirical investigation, and to promote this to the status of a special methodology is pretentious. The Jordanous articles cited may have more to say about this, particular with respect to creativity, but what has been demonstrated here -- i.e. merely being relatively careful about one's concepts when using them empirically -- is hardly a special methodology invented in recent years and attributable to Jordanous. (In fact, paragraph A on page 13 passes the buck on operationalisation, and simply repeats the abstract definition, along with an edict that the evaluator should do the work for this step.)}  

In the three step SPECS process, the evaluator defines the concepts
and behaviours that signal creativity, converts this definition into
clear standards, and then applies them to evaluate the target systems.
%
We follow a slightly modified version of Jordanous's earlier evaluation
guidelines, in that rather than attempt a definition and evaluation of
{\em creativity}, we follow the three steps for \emph{serendipity}.


%%% Moving related work content here


%% Let's focus on applications.

Recent work has examined the related topics of \emph{curiosity}
\cite{wu2013curiosity} and \emph{surprise} \cite{grace2014using} in
computing.  The latter example seeks to ``adopt methods from the field
of computational creativity [$\ldots$] to the generation of scientific
hypotheses.''  This provides a useful example of an effort focused on
computational \emph{invention}.  Another related area of contemporary
computing in which serendipitous events may be found is
bioinformatics: ``Instead of waiting for the happy accidents in the
lab, you might be able to find them in the data''
\cite[p.~70]{kennedy2016inventology}.\ednote{Christian: remove? Not
  immediately related, and we didn't talk enough about the invention
  aspect. Grace and Maher use intrinsic motivation which is more
  domain-independent, but also not more about invention than many
  other CC examples.}

As we indicated earlier, creativity and serendipity are often
discussed in related ways.  A further terminological clarification is
warranted.  The word \emph{creative} can be used to describe a
``creative product'', a ``creative person'', a ``creative process''
and even the broader ``creative milieu.''  Computational creativity
must take acount all of these aspects \cite{jordanous2016four}.  In
contrast, the model we have presented focuses only on serendipity as
an attribute of a particular kind of process.  Most often, we speak of
a system's \emph{potential} for serendipity.  In the current work, we
do not use the term to describe an artefactual property (like novelty
or usefulness), or a system trait (like skill).\ednote{Christian: Either erase or move shorter version of this paragraph into section 3 (Serendipity and Creativity)}

Inspired by social systems that capitalise on this effect, we have investigated the feasibility
of building multi-agent systems that learn by sharing and discussing
partial understandings \cite{corneli2015computational,corneli2015feedback}.

Figueiredo and Campos \cite{Figueiredo2001} describe serendipitous
``moves'' from one problem to another, which transform a problem that
cannot be solved into one that can.  However, it is important to
notice that progress with problems does not always mean transforming a
problem that cannot be solved into one that can.  Progress may also
apply to growth in the ability to \emph{posit} problems.  In keeping
track of progress, it would be useful for system designers to record
(or get their systems to record) what problem a given system solves,
and the degree to which the computer was responsible for coming up
with this problem.
%
As Pease et al. \cite[p. 69]{pease2013discussion} remark,
anomaly detection and outlier analysis are part of the standard
machine learning toolkit -- but recognising \emph{new} patterns and
defining \emph{new} problems is more ambitious (compare von Foerster's
\cite{von2003cybernetics} second-order cybernetics). 


%%% From Heuristics section

\ednote{\label{note:profound}R1: The definition offered for
  serendipity includes some profound and potentially difficult
  concepts such as "prepared mind" and "sagacity" but no detailed
  operationalisation of these terms is offered, leaving the authors
  free to make their own subjective judgements about when and where
  these concepts appear. When the authors examine specific computer
  systems, this allows some rather mundane aspects of these systems to
  be labelled as constituting "prepared mind", "sagacity", etc. What
  computer system could be said *not* to have a "prepared mind", given
  the looseness of the way this term can be used? Do any
  knowledge-based systems *not* manifest
  "sagacity"?}\ednote{\textbf{R1}: On p.13, there are two clear
  mentions of a *system* (not a sequence of events) being "considered
  serendipitous". It would be good if "potential for serendipity" was
  clearly and explicitly defined, and it was clearer what sort of
  entity "serendipitous" applied to. }\ednote{Christian, referring to
  R1: Possibly refer to distinction of serendipity as service
  vs. serendipity in system as made in introduction.}

This means that the system can be ascribed a degree of
{\textbf{curiosity}}, indicating, on a suitable population basis, how
likely it is that the trigger (assuming that it appeared) would be
identified as interesting.

%%%%%%%%%%%%%%%%%%%%%%%%%%%%%%%%%%%%%%%%%%%%%%%%%%%%%%%%%%%%%%%%%%%%%%%%%%%%%%%%%%%%%%%%%%%%%%%%%%%%

\begin{quote}
``\emph{Intuition, insight, and learning are no longer exclusive possessions of humans: any large high-speed computer can be programmed to exhibit them also.}'' \cite[p.~6]{simon1958heuristic}
\end{quote}

Philosopher and media theorist Vil\'em Flusser
\cite[p.~10]{flusser2011into} envisions a hypothetical computer
program that could rediscover all of the improbable things that have
happened in the history of the universe.\ednote{\textbf{Joe}: This may
  be ``uncalled for philosophizing'' but anyway, check page number.}
However, we presumably don't need a special computer for this -- the
universe itself has ``discovered'' many improbable things so far.

It is, however, clear
that computers have both accelerated the rate of discovery, and that
they can be used to study the conditions of discovery
\cite{arbesman2011eurekometrics}.

Curious machines, Rob Saunders... -- since curiosity is an important
keyword for us, it's worth looking at this prior art.
\cite{manaris2003evolutionary} is also relevant regarding music.

As several of the theorists examined in Section
\ref{sec:literature-review} remarked, serendipity is not the same as
chance: neither does it reduce to value.

...a conventional multi-tasking operating system which can respond to interrupts, or an email server with a spam filter "evaluating" unexpected items...

\ednote{\textbf{R1}: a tendency to ramble off the topic
  slightly, pulling other interesting issues into the paper, such as
  the last paragraph on p.24, the paragraph starting "In sum..."}

%%%%%%%%%%%%%%%%%%%%%%%%%%%%%%%%%%%%%%%%%%%%%%%%%%%%%%%%%%%%%%%%%%%%%%%%%%%%%%%%%%%%%%%%%%%%%%%%%%%%

We return to this constellation of issues related to system autonomy
below.

%%%

definitions of
``serendipity'', drawing on previous theories and collecting
historical examples from various domains. We used this survey to
develop a collection of criteria which we propose to be
computationally salient.
%
We used this evaluation model to analyse the potential for serendipity in case
studies of evolutionary computing, automated
programming, and recommender systems.  We saw that the proposed framework can be used both
retrospectively, as an evaluation tool, and prospectively, as a design
tool. In every case, the model surfaced themes that can help to guide
implementation.
%
We then reviewed related work: like \citet{andre2009discovery}, we
propose a two-part definition of serendipity: \emph{discovery}
followed by \emph{invention}. Our process-focused model of
serendipity elaborates both stages, and our case studies show how to
use this model to evaluate existing and hypothetical computer systems.
In contrast to most prior work in \emph{computational creativity}, for
\emph{computational serendipity}, evaluation of various forms needs to
be embedded inside of computational systems.

%In our discussion, we reflected back over the proposed model and case
%studies, and outlined a programme of research into computational
%serendipity.  We have also drawn attention to broader considerations
%in system design. 

%In Section \ref{sec:computational-serendipity}, we applied our model
%to evaluate the serendipity of an evolutionary music improvisation
%system, a system for automatically assembling flowcharts, and a
%hypothetical class of next-generation recommender systems.

%The model has helped to highlight directions for development that
%would increase a system's potential for serendipity, either
%incrementally or more transformatively.  Our model outlines a path
%towards the development of systems that can observe events that would
%otherwise not be observed, take an interest in them, and transform the
%observations into artefacts with value.\ednote{\textbf{Joe}: Note that
%  I've combined the discussion and conclusion sections.  Need suitable
%  section intro text here.}

%%%

\ednote{Christian: There's a lot of potential to cut in this section. I'd remove it entirely, if it wasn't also about the relationship of serendipity and CC. I'd suggest to trim this down to 1-2 paragraphs, and connect it with sec. 2 on serendipity, creativity and invention.}

Reviewing the components of serendipity introduced in Section
\ref{sec:by-example} and crystalised in the definition of serendipity
presented in Section \ref{sec:our-model} in light of the practice
scenarios discussed in Section \ref{sec:computational-serendipity}, we
can describe the following challenges for research in computational
serendipity.\ednote{\textbf{R1}: a
  tendency to ramble off the topic slightly … much of the verbiage in
  Sect 6.2}

Our case studies in Section
  \ref{sec:computational-serendipity} highlight the potential value of
  increased autonomy on the system side.  The search for connections
  that make raw data into ``strategic data'' is an appropriate theme
  for research in computational intelligence and machine learning to
  grapple with.  In the standard cybernetic model, we control
  computers, and we also control the computer's operating context.
  There is little room for serendipity if there is nothing outside of
  our direct control.  This mainstream model stands in contrast to von
  Foerster's \cite{von2003cybernetics} second-order
  cybernetics. 


%The elements in row 5, {\em preparation}, {\em incubation}, {\em insight}, etc., are person-centred, as they refer to actions on the part of the discoverer (albeit potentially  passive or unconscious actions).
%
% The elements in row 4 refer to elements of the social environment.
%
%The \emph{new connection} (row 7) or \emph{trigger} and \emph{result} (row 8) describe an event.
%
%In a survey of 30 definitions of serendipity in English language dictionaries from 1909 to 2000, Merton and Barber \cite[pp. 246-249]{Merton} found that 23 define it as a personal attribute, 6 define it as both personal attribute and as a kind of phenomenon (an event), and 1 defines it solely as an event.
%
% These divisions into event, people and environment centred elements of serendipity will be useful when we consider how when we might model and measure serendipity in a computational context. 

%%%%%%%%%%%%%%%%%%%%%%%%%%%%%%%%%%%%%%%%%%%%%%%%%%%%%%%%%%%%%%%%%%%%%%%%%%%%%%%%%%%%%%%%%%%%%%%%%%%%

The ellipses in the final box at the end of the workflow in Figure \ref{fig:1b} suggest
open-ended applications.  One important sub-class of these will result
in changes to one or more of the system's modules, either by expanding
the knowledge base that it draws on, or adjusting its methods.  
Accordingly, even though the process in Figure \ref{fig:1a} is linear,
the model as a whole can accommodate Merton's \cite{merton1948bearing}
notion of ``extending an existing theory,'' and to the loops noted by
\citet{lawley2008maximising} and \citet{Makri2012a}.


Although Figures \ref{fig:model} treats the case of
successful serendipity, as the earlier remarks suggest, each step is
fallible, as is the system as a whole. 
% Thus, for example, a trigger that has been initially tagged as interesting may prove to be fruitless in the verification stage.  Similarly, a system that implements all the procedural steps in Figure \ref{fig:1b} but that for whatever reason is never able to achieve a result of significant value cannot be said to have potential for serendipity.  However, a system that only produces results of high value would also be suspect, since this would suggest a tight coupling between trigger and outcome.
Indeed, fallibility is a ``meta-criterion'' that transcends the criteria from
Section \ref{sec:by-example}. 
%%
Summarising, we propose the following definition, which is revolves
around the the dimensions of chance, curiosity, and sagacity described
earlier.  These are mapped to the system's main points of possible
failure (Figure \ref{fig:1a}).

%% If the system learns an $N$th fact or
%% If applied to a system which could be described as minimally
%% serendipitous at best, and perhaps not at all serendipitous, does our
%% model identify the lack or presence of serendipity?  
%% %% As example, a spellchecker
%% %% program identifies spelling errors in text input and optionally can
%% %% correct spelling automatically. The only situation we can conceive of
%% %% where serendipity could possibly occur is tenuous; perhaps a suggested
%% %% correction may be incorrect, but may lead the user to interpret the
%% %% correction in an unexpected way. In all other aspects that we have
%% %% considered, spellchecker software would be a decidedly unlikely
%% %% candidate for harbouring serendipitous opportunities.  
%% Traditional spellchecker programs could be said to have a
%% \textbf{prepared mind}, in that they are constructed with internal
%% dictionaries with which to check spelling and ways of deciding what a
%% misspelled word might be.  Given our above discussion of how the
%% system might be serendipitous, the \textbf{serendipity trigger} could
%% be seen as the user misspelling a word and the system suggesting
%% alternative possibilities that the user had not previously conceived.
%% However, the \textbf{bridge} from trigger to serendipitous result (if
%% any) would have been built by the user, not by the system.  With
%% adaptive context-aware text completion tools, we can imagine a
%% ``Cyrano de Bergero'' effect in which the machine finds a
%% serendipitous bridge and offers the \textbf{result} to the user.
%% However, the current generation of text completion tools are known
%% more for infelicities than for exceptional wit.


%% \begin{table}[t]
%% {\centering
%% \begin{tabular}{l@{\hspace{.5cm}}l}
%% \textbf{\emph{Autonomy}} (\emph{chance}) & \textbf{\emph{Learning}} (\emph{curiosity}) \\
%% ``seizing opportunities''                & ``making mental space''\\
%% ``being observant''                      &  ``varying routines''  \\[.5cm]
                              
%% \textbf{\emph{Sociality}} (\emph{sagacity}) & \textbf{\emph{Judgement}} (\emph{value})\\
%% ``relaxing boundaries''                     & ``drawing on previous experiences''\\
%% ``looking for patterns'' &
%% \end{tabular}

%% \par}
%% \caption{Strategies from \citet{makri2014making} drawn into our framework\label{tab:Strategies}}
%% \end{table}

%% \begin{figure}[h]
%% {\small
%% \begin{verbatim}
%%            AUTONOMY (CHANCE)                  LEARNING (CURIOSITY)
%%            seizing opportunities              making mental space
%%            being observant                    varying routines 
                              
%%            SOCIALITY (SAGACITY)               JUDGEMENT (VALUE)
%%            relaxing boundaries                drawing on previous experiences
%%            looking for patterns
%% \end{verbatim}

%% \par}
%% \caption{Heuristics for increasing serendipity potential, theorised}
%% \end{figure}

%% Maybe bring in \cite{hill2012remixing} about generativity.
%% Maybe mention Flusser's thought experiment about computing improbable things

%%% I propose that we restructure the introduction considerably.

%% Jenifer Widom (http://cs.stanford.edu/people/widom/paper-writing.html)
%% suggests the following outline:

%% 1. What is the problem?
%% 2. Why is it interesting and important?
%% 3. Why is it hard? (E.g., why do naive approaches fail?)
%% 4. Why hasn't it been solved before? (Or, what's wrong with previous proposed solutions? How does mine differ?)
%% 5. What are the key components of my approach and results? Also include any specific limitations.

%% 6. Then have a final paragraph or subsection: "Summary of
%% Contributions". It should list the major contributions in bullet
%% form, mentioning in which sections they can be found. This material
%% doubles as an outline of the rest of the paper, saving space and
%% eliminating redundancy.


%% The perspective developed in the current paper sharpens these
%% understandings in two ways: firstly, we point out that work is
%% involved in both phases of the process (even when chance plays a
%% role), and secondly, following Bergson we defer true ``novelty'' to
%% the invention phase.
%% In other words, serendipity involves creative making.  Furthermore, we
%% emphasise the importance of active, agential discernment over more
%% passive stumbling.

%% According to Arthur Cropley, creative thinking involves:
%% \begin{quote}
%% ``{[}N{]}\emph{ovelty generation followed by (or accompanied by) exploration of the novelty from the point of view of workability, acceptability, or similar criteria, in order to determine if it is effective.}'' \cite{cropley2006praise}
%% \end{quote}


%%% Wednesday, Oct 26, 2016

\begin{figure}[h]
\vspace{2mm}
\captionsetup[subfigure]{justification=centering}

\begin{subfigure}{\textwidth}
%%%%%%%%%%%%%%%%%%%%%%%%%%%%%%%%%%%%%%%%%%%%%%%%%%%%%%%%%%%%%%%%%%%%%%%%%%%%%%%%%%%%%%%%%%%%%%%%%%%%
\begin{minipage}[b]{\textwidth}
{\centering
\begingroup
\tikzset{
block/.style = {draw, fill=white, rectangle, minimum height=3em, minimum width=3em},
tmp/.style  = {coordinate}, 
sum/.style= {draw, fill=white, circle, node distance=1cm},
input/.style = {coordinate},
output/.style= {coordinate},
pinstyle/.style = {pin edge={to-,thin,black}}
}

\begin{tikzpicture}[auto, node distance=2cm,>=latex']
    \node [sum] (A-sum1) {};
    \node [input, name=pinput, above left=.7cm and .7cm of A-sum1] (A-pinput) {};
    \node [sum, name=tinput, left=1.3cm of A-sum1] (A-tinput) {};
    \node [input, name=minput, below left of=A-sum1] (A-minput) {};
    \node [input, name=minput, right of=A-sum1] (A-moutput) {};
    \draw [->] (A-pinput) -- node{$p$} (A-sum1);
    \draw [->] (A-tinput) -- coordinate[midway](T) (A-sum1);
%    \draw [->] (A-tinput) -- (A-sum1);
%    \draw [->] (A-sum1) -- node{\vphantom{{\tiny g}}$T^{\star}$}  (A-moutput);

    \node [sum, right=2cm of A-sum1] (B-sum1) {};
    \node [input, name=pinput, above left=.7cm and .7cm of B-sum1] (B-pinput) {};
    \node [input, name=tinput, left of=B-sum1] (B-tinput) {};
    \node [input, name=minput, below left of=B-sum1] (B-minput) {};
    \node [sum, right=1.3cm of B-sum1] (B-sum2) {};
    \node [input, name=minput, right=2.3cm of B-sum2] (B-moutput) {};
    \draw [->] (B-pinput) -- node{$p^{\prime}$} (B-sum1);

    \draw [->] (A-sum1) -- coordinate[midway](Tstar) node{\vphantom{{\tiny g}}$T^{\star}$} (B-sum1);

    \draw [->] (B-sum1) -- coordinate[midway](R) node{\vphantom{{\tiny g}}$B$} (B-sum2);
%    \draw [->] (B-sum2) -- coordinate[midway](evaluation) node{$R$~$(|R|>0)$}  (B-moutput);

%    \node [below=.15cm of T] (chance) {\vphantom{$c^{\star}$}\textcolor{gray}{$c$}};
    \node [below=-.02cm of A-tinput] (chance) {\vphantom{$c^{\star}$}\textcolor{gray}{$a$}};
    \node [below=-.02cm of A-sum1] (curiosity) {\vphantom{$c^{\star}$}\textcolor{gray}{$b$}};
    \node [below=-.02cm of B-sum1] (sagacity) {\vphantom{$c^{\star}$}\textcolor{gray}{$c$}};
    \node [below=-.02cm of B-sum2] (value) {\vphantom{$c^{\star}$}\textcolor{gray}{$d$}};
    \node [above=.02cm of B-sum2] (result) {\vphantom{$c^{\star}$}$R$};
    \node [above=.02cm of A-tinput] (trigger) {\vphantom{$c^{\star}$}$T$};
\end{tikzpicture}
\endgroup



\par}

\subcaption{~\parbox{\widthof{A boxes-and-arrows diagram, showing one possible implementation architecture}}{
A simplified process schematic, showing the key components of the model: the trigger ($T$), prepared mind ($p$, $p^\prime$), focus shift ($T^\star$), bridge ($B$), and result ($R$); and the corresponding dimensions: chance ($a$), curiosity ($b$), sagacity ($c$), and value ($d$).}
\label{fig:1a}}
\end{minipage}
\end{subfigure}
\medskip

%%%%%%%%%%%%%%%%%%%%%%%%%%%%%%%%%%%%%%%%%%%%%%%%%%%%%%%%%%%%%%%%%%%%%%%%%%%%%%%%%%%%%%%%%%%%%%%%%%%%
\begin{minipage}[b]{\textwidth}
{\centering
\begin{tikzpicture}[
single/.style={draw, anchor=text, rectangle},
]
\node (discovery) {\textbf{\emph{Discovery:}}};
% ``poet generates poem''
\node[single, right=8mm of discovery.east,text width=1.6cm] (poet) {\emph{generation\\ module}};
\node[single, right=6mm of poet.east] (poem) {\tikz{\node[draw,circle]{};}};
\draw [->] (poet.east) -- (poem.west);

% ``critic listens to poem and offers feedback''
\node[ellipse, draw, right=9mm of poem.east,text width=1.3cm] (critic) {\emph{feedback}};
\draw [->] (poem.east) -- (critic.west);
\node[single, above=8mm of critic.north,text width=1.4cm] (experience) {\emph{reflection\\ module}};
\node[draw,diamond,inner sep =.3mm,above right=4mm and 3mm of critic] (comment) {\phantom{pp}};
\node[draw,circle,double] (dcircle) at (comment) {};
\node[draw,diamond, inner sep =.3mm, above left=4mm and 3mm of critic] (reflection) {\phantom{pp}};
\node[draw,circle,double] (Tcircle) at (reflection){};
\node[draw,circle,scale=.7] (tcircle) at (reflection){};

\draw[->,thick] ([yshift=1mm]critic.east) to [out=0,in=270] (comment.south) ;
\draw[->,thick] (comment.north) to [out=90,in=0] (experience.east) ;
\draw[->,thick] (experience.west) to [out=180,in=90] (reflection.north) ;
\draw[->,thick] (reflection.south) to [out=270,in=140] ([yshift=1.5mm]critic.west) ;

% nonprinting point to use to bend curve
\coordinate[below right=3mm and 7mm of critic] (mid1);

\node[single, below left=6mm and 4mm of critic] (feedback) {\raisebox{.1em}{\shifttext{-.1em}{\tikz{\node[draw,circle,circular drop shadow,fill=white]{};}}}};
\node[below=.65cm of discovery] (focusshift) {{\small \textbf{\emph{[Focus shift]}}}};

% draw the first curve into focus shift
\draw [->] ([yshift=-1mm]critic.east) to[out=0,in=90] (mid1) to[out=270,in=0] (feedback.east);

%%% Next phase
\node[below=2cm of discovery] (invention) {\textbf{\emph{Invention:}}};

% ``poet integrates feedback''
\node[ellipse, draw, right=12mm of invention.east,text width=1.51cm] (integrator) {\emph{validation}};

% nonprinting point to use to bend curve
\coordinate[above left=2mm and 9mm of integrator] (mid2);

% draw the second curve out from focus shift
\draw [->] (feedback.west) to[out=180,in=90] (mid2) to[out=270,in=160] (integrator.west);

% ``poet asks questions about the feedback''

\node[single, below=9mm of integrator.south,text width=2.5cm] (explainer) {\emph{experimentation\\ module}};

\node[draw,diamond,inner sep =.3mm, below right=4mm and 5mm of integrator] (question) {\phantom{pp}};
\node[draw,regular polygon,regular polygon sides=4,scale=.9] (dsquare) at (question){};

\node[draw,diamond, inner sep =.3mm, below left=4mm and 5mm of integrator] (answer) {\phantom{pp}};
\node[draw,regular polygon,regular polygon sides=3,scale=.6] (dtriangle) at (answer){};

\draw[->,thick] ([yshift=-1mm]integrator.east) to [out=0,in=90] (question.north) ;
\draw[->,thick] (question.south) to [out=270,in=0] (explainer.east) ;
\draw[->,thick] (explainer.west) to [out=180,in=270] (answer.south) ;
\draw[->,thick] (answer.north) to [out=90,in=200] ([xshift=1mm,yshift=-1.8mm]integrator.west) ;

\node[yshift=1mm,single, right=10mm of integrator.east] (problem) {\evaluationcompositeinternal};

\draw [->] ([yshift=1mm]integrator.east) -- (problem.west);

% ``poet reflects on feedback and updates codebase''

\node[single, right=6mm of problem.east,text width=1.6cm] (pgrammer) {\emph{evaluation}\\ \emph{module}};
\draw [->] (problem.east) -- (pgrammer.west);
\node[single, right=4mm of pgrammer.east,text width=.3cm] (etc) {\hphantom{\rule{.07pt}{.15pt}}...};
\draw [->] (pgrammer.east) -- (etc.west);
\end{tikzpicture}


\par}
\smallskip

\subcaption{A boxes-and-arrows diagram, showing one possible implementation architecture}\label{fig:1b}
\end{minipage}

\caption{Schematic representations of a potentially serendipitous process}\label{fig:model}
\end{figure}

%\afterpage{\clearpage}

%% Figure \ref{fig:1a} is a heuristic map of the features of serendipity
%% introduced in Section \ref{sec:by-example}.
%% %
%% Dashed paths ending in `\ymark' show some of the things that can go
%% wrong.

%% \ednote{Christian: remove this paragraph and adjust beginning of next one? It's described similarly (and in a more appropriate place?) later in this section, starting with 'Although Figures..'}
%% It is worth remarking that many things might go wrong.
%% %
%% A serendipity trigger might not arise, or might not attract interest.
%% If interest is aroused, a path to a useful result may not be sought,
%% or may not be found.  If a result is developed, it may turn out to be
%% of little value.  Prior experience with a related problem could be
%% informative, but could also hamper innovation.  Similarly, multiple
%% tasks, influences, and contexts can help to foster an inventive frame
%% of mind, but they may also be distractions.

%\newpage

Figure \ref{fig:1a} focuses on the key features of ``successful'' serendipity.
%
A potential \textbf{trigger} for serendipity, denoted here by $T$, has been perceived.  
%
The \textbf{prepared mind} corresponds to those preparations, labelled
$p$ and $p^{\prime}$, that are relevant to the discovery and invention
phases, respectively, which may include training,
current attitude, and access to relevant knowledge sources.
%
A \textbf{focus shift} indicates that the trigger has now been discovered to be
interesting.  The newly-interesting trigger is denoted $T^\star$, and is
common to both the discovery and the invention phases.
%
%
The \textbf{bridge} $B$ consists of the actions based on $p^{\prime}$
that are taken on $T^\star$ leading to the \textbf{result} $R$, which is
then given a positive evaluation.

Fallibility is a ``meta-criterion'' for serendipity, and our
definition of the serendipity potential of a system will revolve
around examining the system's main points of possible failure.  As
indicated in Figure \ref{fig:1a},
($a$) due to \textbf{chance}, the trigger may not arise or be observed;
($b$) due to insufficient \textbf{curiosity}, a trigger may not arouse the system's interest;
($c$) due to insufficient \textbf{sagacity}, the system may not be able to transform a trigger that has captured its interest into a tangible result;
and ($d$), a result may not be of any significant \textbf{value}.

%%% Wednesday, Oct 26, 2016

While feedback loops are present in several previous definitions of
serendipity (and in Figure \ref{fig:model}), we will not explicitly
include them in our definition of the serendipity potential of a
system.
%%
Nevertheless, assuming that the system is updated (or updates itself)
over time, there is nothing to prevent its serendipity potential
changing over the course of repeated runs.
% As a system is
%updated or as outside conditions change, we might find that
%\emph{System $S$} and a new variant, \emph{System $S^{\prime}$} differ
%in their serendipity potential.  Importantly, this has to do not just
%with changing values along the several evaluation dimensions, but in
% changes to the definition of a ``trigger.''
%% Related issues will be discussed below before we turn to
%% the case studies in Section \ref{sec:computational-serendipity}.
Our model remains open-ended about the uses to which the
valuable result may be put.

Along with these structural assertions, we propose several qualitative
concepts that describe environments and architectures that can make
serendipitous occurrences more likely.  We suspect that serendipity
will be more likely for agents who experience and participate in a
\textbf{dynamic world}, who are active in \textbf{multiple contexts},
occupied with \textbf{multiple tasks}, and who avail themselves of
\textbf{multiple influences}.  These features are not included in the
following definition of the serendipity potential of a system, but
they may feature in more detailed descriptions of modules or operating
environments.

%\vspace{.5cm}

%%% THIS NEEEDS CHECKING.  DOES IT MAKE SENSE?  DOES IT DO WHAT WE NEED?
%%% - JOE 21 May 2016

\begin{mdframed}
\begin{defn}[The serendipity potential of a system]\label{def:serendipity}
~\emph{The evaluator should choose a lower threshold of value,
    $\epsilon$, and an upper threshold of likelihood, $\delta$, both
    in the interval $(0,1)$.
A family of comparison systems, $\mathcal{F}$, is selected.
Then observations are made:}
%%
\emph{
\begin{enumerate}[label=(\arabic*)]
%% \item The system can be said to have a {\textbf{prepared mind}},
%%   consisting of previous experiences, background knowledge, a store of
%%   unsolved problems, skills, expectations, readiness to learn, and
%%   (optionally) a current focus or goal.  This corresponds to a
%%   {\textbf{context}} of evaluation.
\item\label{crit:trigger} The systems process data that arises at least partially as the
  result of factors outside of their control.  Amongst this data a \textbf{trigger}
  is observed by some proportion of the systems, $0\leq a \leq 1$.
\item\label{crit:focus} Assuming a trigger has been observed, some proportion of
  the systems, $0\leq b \leq 1$, classify it as being especially interesting and
  subject it to further processing, constituting a {\textbf{focus
      shift}}.
\item\label{crit:result} Assuming that a trigger has been marked as interesting, some proportion of the systems, $0\leq c \leq 1$, transform the trigger into a
  \textbf{result}.
\item\label{crit:value} This result is then rated by the evaluator as having a
  {\textbf{value}}, $d$, such that $\epsilon<d\leq 1$.
\end{enumerate}
\noindent 
$\mathcal{F}$ should be expanded as long as this increases the size of the 
sub-population that ever becomes ``successful'' for each of the criteria \ref{crit:trigger}--\ref{crit:value}.
If the system we are interested in assessing belongs to this sub-population, and if $a\times b\times c<\delta$, then $1-a\times b\times c$ is the system's
\emph{serendipity potential}.  (Otherwise, the system does not have serendipity potential.)}
\end{defn}
\end{mdframed}

%% I don't think it's necessary to say that the result is rated ``by
%% the system,'' it could be rated by a third party.
%% (E.g. NumbersWithNames doesn't know which of its conjectures is
%% most interesting, just which is most plausible.)  But I think we
%% have to be a bit careful, e.g. with the spell-checker.

\bigskip

%%% MAYBE SAY A BIT MORE HERE BEFORE LAUNCHING INTO "HEURISTICS" SECTION

%Intuitively, the likelihood score $a\times b\times c$ captures the
%probability of ``success.''  When the likelihood score is sufficiently
%low, success is highly improbable.

%It will be noted that the definition could be ``abused'' by choosing a
%very \emph{low} threshold for value ($\epsilon$) and a \emph{high}
%threshold for likelihood ($\delta$).  In this case, arbitrarily many
%systems can be seen to have arbitrarily low serendipity potential.
%The threshold parameters are useful for drawing a distinction between
%this long but not especially informative tail and those systems which
%exhibit interesting effects due to their strong serendipity potential.

The primary challenge presented by a frequentist definition like the
one above is that serendipity tends to be discussed with reference to
one-off occurrences: thus, we consider the historical discovery of
penicillin by Alexander Fleming or the historical invention of
Velcro\texttrademark\ by George de Mestral, and so on.  After the
fact, we might be inclined to judge the probability of success at
100\%, and beforehand, we might not even think to ask the question.
This is where examples like the case history of Eugen Semmer are
particularly useful.

Semmer (1843-1906) was a veterinary pathologist known for research
into fowl cholera and anthrax \cite{saunders1980veterinary}, but
perhaps better known for what failed to happened after he planned to
conduct a post mortem analysis for two unwell horses.  As it turned
out, there was one crucial problem with his plan: ``when he arrived in
the morning he discovered that the animals had unexpectedly and
inexplicably recovered'' \cite[p.~75]{cropley2013creativity}.  Semmer
determined that this was caused by the unintended presence of
\emph{penicillum notatum} spores in his laboratory.  He proceeded to
test this theory with further \emph{in vivo} experiments on other
animals.

\begin{quote}
 ``\emph{However, apparently blinded by the
   narrow nature of his special knowledge {\upshape\ldots}\ he did not
   recognise that he had stumbled on an important life-saver (what we
   now call `antibiotics'), and instead went to considerable lengths to
   eradicate the spores from his laboratory.}'' \cite[p.~76]{cropley2013creativity}
\end{quote}
  
In terms of our definition, we can understand the horses' recovery to
be a trigger, and Semmer's further experiments to be a focus
shift.  In these respects his experience is comparable to that
of Fleming's famous encounter with bare patches in his dirty petri dishes.
However, Semmer's failure to translate his experience into a valuable
result allows us to bound $c$ from above, at 50\% or less,
for the ``lab biology'' systems shared by both Semmer and Fleming.

In a setting where a given computation system runs repeatedly, a
population of separate runs can allow us to estimate scores on the
relevant dimensions.  Similarly, if the system involves multiple
threads or agents, then frequentist modelling approach is a natural one
to use.

This conception of serendipity potential allows us to specify a single
number in place of the Likert-type data suggested by Makri and
Blandford:
\vspace{-\baselineskip}
\begin{aquote}{\cite[p.~7]{Makri2012b} [emphasis modified]}
\emph{
\begin{enumerate}
\item How {\upshape \textbf{unexpected}} were the circumstances that led to the connection being made? (Not at all/Somewhat/Very)
\item How {\upshape \textbf{insightful}} was the making of the connection itself? (Not at all/Somewhat/Very)
\item How {\upshape \textbf{valuable}} was or do you expect the outcome to be? (Not at all/Somewhat/Very)
\end{enumerate}
}
\end{aquote}

Note, however, that the result's value, $d$, does not factor into our
definition of serendipity potential, except as part of a thresholding
step that is used to determine whether serendipity potential is
defined for a given system.  This is because, in reflecting on prior
descriptions of serendipity, it seems that what matters is the fact that
the result is of value, rather than its absolute or overall value.  In applying our
definition, system designers and evaluators are asked to specify a
threshold of value that they deem to be sufficiently significant.

\subsubsection{Serendipity potential}

% If the population is ``runs,'' then we can pick out one of them and look at how/whether it effects a focus shift.

The associated likelihood score is $\mathit{high}\times \mathit{high} \times\mathit{high}$.
This does not make a convincing case for
the system having serendipity potential.
Furthermore, until the system can produce
results that a third party can judge
to be valuable, it will not satisfy Criterion \ref{crit:value} from our
definition.  For now, we must conclude that
with the set-up described above, the system
does not have serendipity potential.
This motivates a new set of experiments that meaningfully
judge the value of generated flowcharts, generated texts, and
explanatory heuristics, and that involve agents that
are able to give specific, differentiated,
attention to specific kinds of triggers.
These changes would both result in and require
increased curiosity and sagacity on the part of the system.  Depending on the
implementation strategy, the role of chance may shift as well.  
%%% JAC - add citations to ICCC papers if they have been accepted.


\subsubsection{Serendipity potential}

In this case, we compute a likelihood measure of
$\mathit{low}\times\mathit{low}\text{-}\mathit{variable}\times\mathit{low}$, with outcomes
of potentially high value, so that the envisioned system would have a
relatively high serendipity potential.  Realising such a system should
be understood as a computational grand challenge.  Thinking long
term (Heuristic \hyperref[heur:4]{4}) were such a system
was ever realised, in order to maintain high value,
continued adaptations would presumably be required.


We compare this hypothetical system with current recommender systems.
All such systems have imperfect knowledge of user preferences and
interests.  The \textbf{chance} of a recommender system noticing
some particular salient pattern in user behaviour
seems quite ``$\mathit{low}$,'' at least at the moment.
When a given pattern is noticed,
adapting the recommendation strategy accordingly could be described as
\textbf{curiosity}.  It is important to note that these adaptations
may work to the detriment of
user satisfaction -- and business metrics -- over the short term.  In
principle, the system's curiosity could be set as a parameter,
depending on how much coherence is permitted to suffer for the sake of
gaining new knowledge.  However, most current systems are unlikely to
be able to adapt significantly, even if new patterns are noticed.  We rate this dimension
as ``$\mathit{low}\text{-}\mathit{variable}$''  Measures of \textbf{sagacity} would relate to
the system's ability to develop useful experiments and draw sensible
inferences from user behaviour.  For example, the system would have to
select the best time to initiate an A/B test.  A significant amount of
programming would have to be invested in order to make this sort of
judgement autonomously, and currently such systems are beyond rare,
so we rate this dimension as ``$\mathit{low}$.''
The \textbf{value} of recommendation strategies can be measured in
terms of traditional business metrics or other organisational
objectives.


% JC: We already used the term ``focus'' above; we might want to pick
% two different words or possibly compress the two stages, although it
% may also be smart to use the term twice in connection with the idea
% of a SHIFT.

%% JC: Is this what's intended?
% We do not use the notion \textbf{focus shift}, as it implies some
% absolute shift from another event to this, whereas it's perfectly
% reasonable to focus on a few things at the same time.

%% Pek van Andel gives the example of an agricultural worker
%% who noticed a tree that was taller and healthier than
%% others, and subsequently

%% As an example from the history of science Fleming
%% \cite{fleming} noted: ``There are thousands of
%% different moulds'' -- and ``that chance put the mould in
%% the right spot at the right time was like winning the
%% Irish sweep.''  However, Fleming's discovery was not a
%% complete coincidence: his training and work

% JC: I question whether we should use the word ``trigger'' or
% ``serendipity'' at this stage.  Can we really call an event a
% ``trigger'' at this stage?  Also, since we're defining serendipity,
% we can't use it yet.

%An event that is perceptually ``louder,'' in the sense
%that it covers more of the sensory field, will typically
%be more likely to gain attention.

%% Comparisons
%% between systems can help identify the features that can be changed,
%% and to which extent.  For instance, in the class of Jazz improvisation
%% systems, perception may be more or less high fidelity.

%% In seeking to develop systems with high serendipity potential, developers may encounter an exploration vs.~exploitation trade-off here: you can develop one system, you could compare with lots of different approaches.

%% \begin{itemize}
%% \item That it makes sense to think of \textbf{chance} playing a role
%%   both in the perception stage, and also in the way interest is
%%   allocated, since chance events in previous time steps will have
%%   contributed to shaping the system's prepared mind.
%% \item The \textbf{prepared mind} could come from multiple systems, if
%%   we consider embodied and situated systems, each with a distinct
%%   perspective on the world. The process depends on a \textbf{prepared
%%     mind} at several stages: when determining the interestingness of
%%   an event, when explaining it, when building the bridge and when
%%   assessing its value.
%% \item In the most straightforward examples, attentional
%%   focus as described in Definition \ref{def:attention} is
%%   initially extrospective, i.e. focused on data that is
%%   newly encountered by the system.  After a focus shift
%%   takes place as outlined in Definition
%%   \ref{def:interest}, attention is refocused to become
%%   introspective.  At this stage it has retrieved an
%%   initial context for the event, within which it is now
%%   positioned.  Note that it is also possible to carry out
%%   an internal focus shift, and retrieve context from one
%%   mental model in order to flesh out another.
%% \end{itemize}


%% JC: I'm commenting this text from Alison, because I think we deal with
%% it sufficiently above.  But someone else could double check, especially
%% comparing Alison's idea of a focus shift.

%% \subsection{Definitions}\label{sec:definitions}
%% \begin{newpart}{Outline of definitions from Alison, align text with Figure \ref{fig:gap-diagram} and Christian's accompanying text}
%% 1. Computer Science Definitions

%% (2)

%% 2. Serendipity Definitions

%% (1) Relating to events:

%% (a) A \textbf{trigger} is a piece of data without which a result would not have
%% been found...

%% (b) A \textbf{bridge} is the path from trigger to result /set of mechanisms used to
%% form the path from trigger to result...

%% (c) A \textbf{result} is ...

%% (d) A \textbf{potentially serendipitous incident} is one which can be decomposed
%% into Trigger, Bridge and Result. [do we want FS in here?]

%% (2) Relating to \textbf{serendipity space} (in which incidents may
%% lie):

%% (a) \textbf{chance} is a dimension (used in relation to the trigger - how likely was
%% the trigger to occur in that context?) Relates to the environment.

%% (b) \textbf{sagacity} is a dimension (used in relation to the
%% discoverer - about the insight used to see the relevance of the
%% trigger (focus shift), and the skill and knowledge did it take to get
%% from the trigger to the result.)

%% Relates to the discoverer.

%% The outdated word “Sagacity” can also be seen as wisdom, or perhaps
%% as insight. These related to abilities on the part of the discoverer,
%% without which they would not have made the discovery.

%% \textbf{Sagacity}:

%% (i) ability to Focus Shift (ability to evaluate and to re-evaluate in
%% a new way)

%% (ii) ability to take consequential note of anomalies to arrive at
%% unanticipated discoveries [is this just performing the focus shift?]

%% (iii) set of mechanisms which will form the bridge between trigger and
%% result (do Pek van Andel's patterns fit in here? ) ability to link
%% together chance events to arrive at a valuable discovery.

%% (iv) enhanced evaluation capacities: initial evaluation and
%% re-evaluation

%% (FS), ability to generate and evaluate between possible directions
%% that developing the trigger might go; ability to evaluate result
%% and know that it is good.

%% (v) ability of form an “unexpected connection”

%% (vi) prepared mind [?] - relates to knowledge rather than ability

%% (c) value is a dimension (used in relation to the result - how good is it?)
%% Relates to the discovery.

%% (d) serendipity space is the space defined by the three dimensions chance,
%% sagacity and value.

%% (e) We measure position in serendipity space by where in serendipity space
%% a potentially serendipitous event lies.

%% how is context different to prepared mind etc?

%% (3) \textbf{Relating to a system}: We measure the serendiptous
%% potential of a system by ...

%% (a) \textbf{Prepared Mind}. The collection of knowledge, mechanisms,
%% .... that constitute a system. This is not something that systems have
%% or not: we can describe the contents of any system in terms of
%% prepared mind.

%% However, we can identify aspects that should be helpful: open problems
%% - a set of open problems or half-formed gernations. Egs, conjectures
%% in HRL (a new example might suddenly be interesting since it’s a
%% counter-example), an unfinished poem, or proof, ... eg new proof
%% technique - “this might be useful for the other proof I was trying to
%% do”. [What else?] what isn’t a set of open problems? Chess player: The
%% prepared mind will normally develop over a session of a system run, so
%% a trigger might be missed at the start of a run where the same system
%% might have done a focus shift on the same trigger later on in its run.

%% The Prepared Mind is the “environmental factors” which relate to
%% internal knowledge of the system (discoverer).

%% value generator - reuse or invent

%% (b) \textbf{Bridging techniques}. These are a subset of the prepared
%% mind and constitute those techniques actually used to

%% (c) \textbf{Ability to Focus Shift}: Let $E(o, c)$ be the evaluation
%% performed by the system according to a set of evaluation criteria of a
%% given object $o$ in a given context $c$. A focus shift occurs when, for
%% object $o_1$ and context $c_1$ , $E(o_1 , c_1 ) \leq \theta$ for a given
%% threshold $\theta$; and the system either:

%% (i) retrieves an existing context $c_2$ such that
%% $E(o_1 , c_2 ) > \theta$

%% (ii) generates a new context $c_2$ such that
%% $E(o_1 , c_2 ) > \theta$, or

%% (iii) changes its evaluation criteria to $E^{\prime}$ such that
%% $E^{\prime} (o_1 , c_1) > \theta$

%% (or some combination of (i) - (iii))

%% (d) \textbf{Curiosity}

%% (e) \textbf{Sagacity}

%% (4) We say that a system is serendipitous if the proportion of its
%% discoveries is over a threshold in serendiptity space. [cf Ritchie’s
%%   criteria but we take the process into account by our prior
%%   definitions.]

%% (5) \textbf{Relating to an environment}: We measure the serendiptous
%% potential of an environment by ....

%% (a) \textbf{Dynamic world}: world changes over time, eg web services, ...

%% (b) \textbf{Multiple contexts}

%% (c) \textbf{Multiple tasks}

%% (d) \textbf{Multiple influences}. Multiple sources of data, eg system
%% interface with twitter, ....

%% \textbf{inventing a problem}
%% \end{newpart}

\iffalse
\afterpage{\clearpage}
\begin{table}[p]
{\centering \renewcommand{\arraystretch}{1.5}
\scriptsize
\begin{tabular}{p{1.4in}@{\hspace{.1in}}p{1.4in}@{\hspace{.1in}}p{1.4in}}
\multicolumn{1}{c}{\textbf{{\footnotesize Evolutionary music}}}
& \multicolumn{1}{c}{\textbf{{\footnotesize Flowchart assembly}}}
& \multicolumn{1}{c}{\textbf{{\footnotesize Next-gen.~rec.~sys.}}}
\\[.05in]
\multicolumn{3}{l}{\em {\textbf{Condition}}} \\
\cline{1-3}
\multicolumn{3}{l}{\em Focus shift} \\[-.1cm]
Driven by (currently, human) evaluation of samples
& Find a pattern to explain a successful combination of nodes
& Unexpected behaviour in the aggregate
\\
\cline{1-3}
~\\[-.1cm]
\multicolumn{3}{l}{\em {\textbf{Components}}} \\
\cline{1-3}
\multicolumn{3}{l}{\em Trigger} \\[-.1cm]
% \textbf{Trigger}
Previous evolutionary steps, in combination with user input
& Trial and error in combinatorial search
& Input from user behaviour
\\
% \cline{1-3}
\multicolumn{3}{l}{\em Prepared mind} \\[-.1cm]
% \textbf{Prepared mind}
Musical knowledge, evolution mechanisms
& Constraints on node inputs and outputs; history of successes and failures
& Through user/domain model\\
% \cline{1-3}
%\textbf{Bridge}
\multicolumn{3}{l}{\em Bridge} \\[-.1cm]
Newly-evolved Improvisors
& Try novel combinations
& Elements identified outside clusters\\
% \cline{1-3}
%\textbf{Result}
\multicolumn{3}{l}{\em Result} \\[-.1cm]
Music generated by the fittest Improvisors
& Non-empty or more highly qualified output
& Dependent on organisation goals \\ \cline{1-3}
~\\[-.1cm]
%%%%%%%%%%%%%%%%%%%%%%%%%%%%%%%%%%%%%%%%%%%%%%%%%%%%%%%%%%%%%%%%%%%%%%%%%%%%%%%%%%%%%%%%%%%%%%%%%%%%
\multicolumn{3}{l}{\em \textbf{Dimensions}}  \\
\cline{1-3}
%\textbf{Chance}
\multicolumn{3}{l}{\em Chance} \\[-.1cm]
Looking for rare gems in a huge search space
& Changing state of the outside world; random selection of nodes to try
& Imperfect knowledge of user preferences and behaviour\\
% \cline{1-3}
%\textbf{Curiosity}
\multicolumn{3}{l}{\em Curiosity} \\[-.1cm]
Aiming to have a particular user take note of an Improvisor
& Search for novel combinations
& Making unusual recommendations\\
% \cline{1-3}
%\textbf{Sagacity}
\multicolumn{3}{l}{\em Sagacity} \\[-.1cm]
Enhance user appreciation of Improvisor over time, using a fitness function
& Don't try things known not to work; consider variations on successful patterns
& Update recommendation model after user behaviour \\
% \cline{1-3}
%\textbf{Value} &
\multicolumn{3}{l}{\em Value} \\[-.1cm]
Via fitness function (as a proxy measure of creativity)
& Currently ``non-empty results''; more interesting evaluation functions possible
& Per business metrics/objectives\\
\cline{1-3}
%%%%%%%%%%%%%%%%%%%%%%%%%%%%%%%%%%%%%%%%%%%%%%%%%%%%%%%%%%%%%%%%%%%%%%%%%%%%%%%%%%%%%%%%%%%%%%%%%%%%
~\\[-.1cm]
\multicolumn{3}{l}{\em \textbf{Factors}} \\
\cline{1-3}
%\textbf{Dynamic world}
\multicolumn{3}{l}{\em Dynamic world} \\[-.1cm]
Changes in the user tastes
& Changing data sources and growing domain knowledge
& As precondition for testing system's influences on user behaviour\\
%\cline{1-3}
%\textbf{Multiple contexts}
\multicolumn{3}{l}{\em Multiple contexts} \\[-.1cm]
Multiple users' opinions would change what the system is curious about and require greater sagacity
& Interaction between different heuristic search processes would increase unexpectedness
& User model, domain model, model of its own behaviour\\
% \cline{1-3}
%\textbf{Multiple tasks}
\multicolumn{3}{l}{\em Multiple tasks} \\[-.1cm]
Evolve Improvisors, generate music, collect user input, carry out fitness calculations
& Generate new heuristics and new domain artefacts
& Make recommendations, learn from users, update models\\
% \cline{1-3}
%\textbf{Multiple influences}
\multicolumn{3}{l}{\em Multiple influences} \\[-.1cm]
Through programming of fitness function and musical parameter combinations
& Learning to combine new kinds of ProcessNodes
& Experimental design, psychology, domain understanding\\
\cline{1-3}
\end{tabular}
\par}
\normalsize
%\bigskip
\caption{Summary: applying our computational serendipity model to three case studies\label{caseStudies}}
\end{table}
\fi

%% Table \ref{caseStudies} summarises how the condition, components,
%% dimensions and factors in our model of serendipity appear in in our
%% current work on a flowchart-assembly system, in an evolutionary music
%% system, and in hypothetical ``next-generation'' recommender
%% systems.

% \subsubsection{Application of heuristic criteria}

%% \begin{figure}
%% {\centering
%% \begingroup
\tikzset{
pblock/.style = {font={\ttfamily},rectangle split, rectangle split horizontal,
                 rectangle split parts=3, rectangle split part fill={gray,white,gray}},
eblock/.style = {font={\ttfamily},rectangle split, rectangle split horizontal,
                 rectangle split parts=1, rectangle split part fill={gray,white,gray}}
}

\begin{tikzpicture}
    % PERCEPTION
    \node [pblock] (perception)
                              {\nodepart{one}    \phantom{3456780234567}
                               \nodepart{two}    \phantom{123789013}
                               \nodepart{three}  \phantom{9999012356780}};
    \node[left=of perception.west] (perceptionL) {Perception};

    % EVENT
    \node[above=.5cm of perception.90,draw,circle] (event) {};
    \node[right=of event.east] (eventL) {Event};

    % ATTENTION
    \node [pblock,below=of perception.west,anchor=west]
                   (attention)
                              {\nodepart{one}    \phantom{3456780234567}
                               \nodepart{two}    \phantom{123789013}
                               \nodepart{three}  \phantom{9999012356780}};
    \node[anchor=west] (attentionL) at (attention -| perceptionL.west)  {Attention};

    % INTEREST
    \node [pblock,below=of attention.west,anchor=west]
                   (interest)
                              {\nodepart{one}    \phantom{3456780234567}
                               \nodepart{two}    \phantom{123789013}
                               \nodepart{three}  \phantom{9999012356780}};
    \node[anchor=west] (interestL) at (interest -| perceptionL.west) {Interest};

    %%%%%%%%%%%%%%%%%%%%%%%%%%%%%%%%%%%%%%%%%%%%%%%%%%%%%%%%%%%%%%%%%%%%%%%%%%%%%%%%%%%%%%%%%%%%%%%%%%%%

    % EXPLANATION
    \node [eblock,below=of interest.west,anchor=west,rectangle split part fill={gray}]
                   (explanation)
                              {\nodepart{one}    \phantom{6789013561234567905600090124567999890\hspace{.4em}}};
    \node[anchor=west] (explanationL) at (explanation -| perceptionL.west) {Explanation};

    % BRIDGING
    \node [eblock,below=of explanation.west,anchor=west,rectangle split part fill={gray}]
                   (bridging)
                              {\nodepart{one}    \phantom{9012346781234567913456345678023495690\hspace{.4em}}};
    \node[anchor=west] (bridgingL) at (bridging -| perceptionL.west) {Bridging};

    % VALUATION
    \node [eblock,below=of bridging.west,anchor=west,rectangle split part fill={gray}]
                   (valuation)
                              {\nodepart{one}    \phantom{2290134567888127890145612345680129930\hspace{.4em}}};
    \node[anchor=west] (valuationL) at (valuation -| perceptionL.west) {Valuation};

    % EVENT+PERCEIVED
    \node[below=1cm of event,draw,circle,double] (eventY) {};

    % EVENT+CONSCIOUS
    \node[below=2cm of event,draw,circle,double] (eventX) {};
    \node[draw,circle,scale=.7] (eventXdec) at (eventX){};

    % EVENT+FOCUS
    \node[below=3cm of event,draw,circle,circular drop shadow,fill=white] (eventZ) {};

    % EVENT+EXPLANATION
    %% \begin{scope}[transform canvas={xshift = -.2cm}]
    %% \node[below=4cm of event,draw,circle] (eventB) {};
    %% \node[right=.1cm of eventB,draw,regular polygon,regular polygon sides=4] (explanationB) {};
    %% \end{scope}

    % EVENT+EXPLANATION+PROBLEM
    %% \begin{scope}[transform canvas={xshift = -.45cm}]
    %% \node[below=5cm of event,draw,circle] (eventC) {};
    %% \node[right=.1cm of eventC,draw,regular polygon,regular polygon sides=4] (explanationC) {};
    %% \node[right=.13cm of explanationC,draw,regular polygon,regular polygon sides=3,scale=.65,yshift=-.08cm] (problemC) {};
    %% \end{scope}

    % EVENT+EXPLANATION+PROBLEM+VALUATION
    %% \begin{scope}[transform canvas={xshift = -.45cm}]
    %% \node[below=6.4cm of event,draw,circle] (eventD) {};
    %% \node[right=.1cm of eventD,draw,regular polygon,regular polygon sides=4] (explanationD) {};
    %% \node[right=.13cm of explanationD,draw,regular polygon,regular polygon sides=3,scale=.65,yshift=-.08cm] (problemD) {};

    %% \node[left=.01cm of eventD,xshift=.1cm] (evaluation1) {$v($};
    %% \node[right=1.07cm of evaluation1] (evaluation2) {$)$};

    %% \coordinate[right=1.15cm of evaluation1] (evaluationO);
    %% \end{scope}

    %% Flow of events
    \path[draw,->,shorten <= 0.05cm, shorten >= 0.05cm] (event) -> (eventY);
    \path[draw,->,shorten <= 0.05cm, shorten >= 0.05cm] (eventY) -> (eventX);
    \path[draw,->,shorten <= 0.05cm, shorten >= 0.05cm] (eventX) -> (eventZ);
%    \path[draw,->,shorten <= 0.07cm, shorten >= 0.05cm] (eventZ) -> (eventB);
%    \path[draw,->,shorten <= 0.05cm, shorten >= 0.05cm] (eventB) -> (eventC);
%    \path[draw,->,shorten <= 0.05cm, shorten >= 0.05cm] (eventC) -> (eventD);

\end{tikzpicture}
\endgroup


%% \par}
%% \vspace{.2cm}
%% \caption{Schematic analysis of the next-generation recommender system
%% \label{fig:gap-diagram-recommender}}
%% \end{figure}

%% With the above challenge in mind, we ask how serendipity could be
%% achieved within a next-generation recommender system.  In terms of our
%% model, current systems have at least the makings of a \textbf{prepared
%%   mind}, comprising both a user- and a domain model, both of which can
%% be updated dynamically.  In systems that are designed to induce
%% serendipity for the user, the system can potentially bring about a
%% focus shift for the user by presenting recommendations that are
%% neither too close, nor too far away from what the user already knows.
%% Here we consider what happens when the flow of information is the
%% other way around.  A new recommendation strategy that addresses the
%% organisation's goals would be a valuable \textbf{result}. %CHRISTIAN
%% FIXED \ednote{\textbf{R1}: On p.23, "Current systems seek to induce
%%   serendipity by...". Do current systems have a goal of serendipity?
%%   Really??}

%% \subsubsection{Qualitative assessment}

%% Recommender systems have to cope with a \textbf{dynamic world} of
%% changing user preferences and a changing collection of items to
%% recommend.  A dynamic environment which nevertheless exhibits some
%% degree of regularity represents a precondition for useful A/B
%% testing.  The system's \textbf{multiple contexts} include the user
%% model, the domain model, as well as an evolving model of its own
%% programmatic organisation.  A system matching the description here
%% would have \textbf{multiple tasks}: making useful recommendations,
%% generating new experiments to learn about users, and improving its
%% models.  In order to make effective decisions, a system would have
%% to avail itself of \textbf{multiple influences} related to
%% experimental design, psychology, and domain understanding.
%% Pathways for user feedback that go beyond answers to the question
%% ``Was this recommendation helpful?'' could be one way make the
%% relevant expertise available.


%% \subsubsection{Qualitative assessment}

%% The {\sf GAmprovising} system operates in a \textbf{dynamic world},
%% insofar as the user's tastes and judgements may change over time.
%% From the point of view of our definition of serendipity potential, the
%% most obvious weak point of the system is that it isn't able to
%% ``listen'' to the music that it generates, which would allow it to
%% much more fully exploit the dynamics of its own evolutionary process,
%% and effect its own focus shift without consulting an outside oracle.
%% Greater dynamism and, especially, more differentiation amongst the
%% population of Improvisors could add the potential for serendipity to
%% ``{\sf GAmprovising 2}''.  More differentiation among Improvisers
%% would mean that selective attention could (sometimes) be exceptional;
%% e.g., a given hereditary line might become interested in a new style
%% of play, much as in the real-world history of jazz music.  A
%% subsequent version of the system that could simultaneously cater to
%% the tastes of multiple listeners would be occupied with a considerably
%% more complex problem, spanning and integrating \textbf{multiple
%%   contexts}.  The current version of the system uses one global
%% fitness function; the system would be more convincing if the fitness
%% function evolved to match the listener's taste.  Juggling two or more
%% fitness functions would give the future version of the system
%% \textbf{multiple tasks}.  Currently, \textbf{multiple influences} are
%% present only in the design of the fitness function and settings for
%% generative parameters, but, as above, these could be augmented with an
%% evolving sense of musical taste.

%% JC: I'm commenting this text from Alison, because I think we deal with
%% it sufficiently above.  But someone else could double check, especially
%% comparing Alison's idea of a focus shift.

%% \subsection{Definitions}\label{sec:definitions}
%% \begin{newpart}{Outline of definitions from Alison, align text with Figure \ref{fig:gap-diagram} and Christian's accompanying text}
%% 1. Computer Science Definitions

%% (2)

%% 2. Serendipity Definitions

%% (1) Relating to events:

%% (a) A \textbf{trigger} is a piece of data without which a result would not have
%% been found...

%% (b) A \textbf{bridge} is the path from trigger to result /set of mechanisms used to
%% form the path from trigger to result...

%% (c) A \textbf{result} is ...

%% (d) A \textbf{potentially serendipitous incident} is one which can be decomposed
%% into Trigger, Bridge and Result. [do we want FS in here?]

%% (2) Relating to \textbf{serendipity space} (in which incidents may
%% lie):

%% (a) \textbf{chance} is a dimension (used in relation to the trigger - how likely was
%% the trigger to occur in that context?) Relates to the environment.

%% (b) \textbf{sagacity} is a dimension (used in relation to the
%% discoverer - about the insight used to see the relevance of the
%% trigger (focus shift), and the skill and knowledge did it take to get
%% from the trigger to the result.)

%% Relates to the discoverer.

%% The outdated word “Sagacity” can also be seen as wisdom, or perhaps
%% as insight. These related to abilities on the part of the discoverer,
%% without which they would not have made the discovery.

%% \textbf{Sagacity}:

%% (i) ability to Focus Shift (ability to evaluate and to re-evaluate in
%% a new way)

%% (ii) ability to take consequential note of anomalies to arrive at
%% unanticipated discoveries [is this just performing the focus shift?]

%% (iii) set of mechanisms which will form the bridge between trigger and
%% result (do Pek van Andel's patterns fit in here? ) ability to link
%% together chance events to arrive at a valuable discovery.

%% (iv) enhanced evaluation capacities: initial evaluation and
%% re-evaluation

%% (FS), ability to generate and evaluate between possible directions
%% that developing the trigger might go; ability to evaluate result
%% and know that it is good.

%% (v) ability of form an “unexpected connection”

%% (vi) prepared mind [?] - relates to knowledge rather than ability

%% (c) value is a dimension (used in relation to the result - how good is it?)
%% Relates to the discovery.

%% (d) serendipity space is the space defined by the three dimensions chance,
%% sagacity and value.

%% (e) We measure position in serendipity space by where in serendipity space
%% a potentially serendipitous event lies.

%% how is context different to prepared mind etc?

%% (3) \textbf{Relating to a system}: We measure the serendiptous
%% potential of a system by ...

%% (a) \textbf{Prepared Mind}. The collection of knowledge, mechanisms,
%% .... that constitute a system. This is not something that systems have
%% or not: we can describe the contents of any system in terms of
%% prepared mind.

%% However, we can identify aspects that should be helpful: open problems
%% - a set of open problems or half-formed gernations. Egs, conjectures
%% in HRL (a new example might suddenly be interesting since it’s a
%% counter-example), an unfinished poem, or proof, ... eg new proof
%% technique - “this might be useful for the other proof I was trying to
%% do”. [What else?] what isn’t a set of open problems? Chess player: The
%% prepared mind will normally develop over a session of a system run, so
%% a trigger might be missed at the start of a run where the same system
%% might have done a focus shift on the same trigger later on in its run.

%% The Prepared Mind is the “environmental factors” which relate to
%% internal knowledge of the system (discoverer).

%% value generator - reuse or invent

%% (b) \textbf{Bridging techniques}. These are a subset of the prepared
%% mind and constitute those techniques actually used to

%% (c) \textbf{Ability to Focus Shift}: Let $E(o, c)$ be the evaluation
%% performed by the system according to a set of evaluation criteria of a
%% given object $o$ in a given context $c$. A focus shift occurs when, for
%% object $o_1$ and context $c_1$ , $E(o_1 , c_1 ) \leq \theta$ for a given
%% threshold $\theta$; and the system either:

%% (i) retrieves an existing context $c_2$ such that
%% $E(o_1 , c_2 ) > \theta$

%% (ii) generates a new context $c_2$ such that
%% $E(o_1 , c_2 ) > \theta$, or

%% (iii) changes its evaluation criteria to $E^{\prime}$ such that
%% $E^{\prime} (o_1 , c_1) > \theta$

%% (or some combination of (i) - (iii))

%% (d) \textbf{Curiosity}

%% (e) \textbf{Sagacity}

%% (4) We say that a system is serendipitous if the proportion of its
%% discoveries is over a threshold in serendiptity space. [cf Ritchie’s
%%   criteria but we take the process into account by our prior
%%   definitions.]

%% (5) \textbf{Relating to an environment}: We measure the serendiptous
%% potential of an environment by ....

%% (a) \textbf{Dynamic world}: world changes over time, eg web services, ...

%% (b) \textbf{Multiple contexts}

%% (c) \textbf{Multiple tasks}

%% (d) \textbf{Multiple influences}. Multiple sources of data, eg system
%% interface with twitter, ....

%% \textbf{inventing a problem}
%% \end{newpart}



%%% OLD INTRODUCTION, REVISED BY CHRISTIAN
\iffalse
\section{Introduction} \label{sec:introduction}

%%%%%%%%%%%%%%%%%%%%%%%%%%%%%%%%%%%%%%%%%%%%%%%%%%%%%%%%%%%%%%%%%%%%%%%%%%%%%%%%%%%%%%%%%%%%%%%%%%%%
% 1. What is the problem?
%\subsection{The problem this paper addresses}
%%%%%%%%%%%%%%%%%%%%%%%%%%%%%%%%%%%%%%%%%%%%%%%%%%%%%%%%%%%%%%%%%%%%%%%%%%%%%%%%%%%%%%%%%%%%%%%%%%%%


%%%%%%%%%%%%%%%%%%%%%%%%%%%%%%%%%%%%%%%%%%%%%%%%%%%%%%%%%%%%%%%%%%%%%%%%%%%%%%%%%%%%%%%%%%%%%%%%%%%%
% 3. Why is it hard? (E.g., why do naive approaches fail?)
%\subsection{Why this is hard}
%%%%%%%%%%%%%%%%%%%%%%%%%%%%%%%%%%%%%%%%%%%%%%%%%%%%%%%%%%%%%%%%%%%%%%%%%%%%%%%%%%%%%%%%%%%%%%%%%%%%
Most previous research on serendipity in a computing context focuses
on stimulating serendipitous discovery on the user side.  Paul
Andr{\'e} et al~\cite{andre2009discovery} previously proposed a
two-part model of serendipity encompassing ``the chance encountering
of information, and the sagacity to derive insight from the
encounter.''
%% JC: Added ref to Kotkov survey. 20 March 2017
The first phase is the one most frequently automated (or
semi-auto\-mated).  For example, the {\sf SerenA} project, developed
by Deborah Maxwell et al~\cite{maxwell2012designing}, aimed to
support users in forming bridging connections from an unexpected
encounter to a previously unanticipated but valuable outcome by
drawing on linked data from around the web. Together with many other
projects in the domain of recommender systems~\cite{Zhang2011}, the
{\sf SerenA} system realises what we call \emph{serendipity as a
  service}.  This constitutes the most immediate backdrop for our
work, so we summarise it briefly before proceeding.

\paragraph{State of the art: Serendipity as a service}
This idea is widespread in the recommendation literature, where serendipity is understood to involve ``a positive emotional response of the user about a previously unknown (novel) item''  \cite{Adamopoulos:2014:URS:2699158.2559952}.  Moreover,
\begin{quote}
``\emph{a serendipitous recommendation involves an item that the user would not be likely to discover otherwise, whereas the user might autonomously discover novel items.}'' \cite{Adamopoulos:2014:URS:2699158.2559952}
\end{quote}
Similar remarks can be made about web search or casual browsing of blogs and social media \cite{Andre:2009:XSP:1518701.1519009}.
 A recent survey of serendipity in recommender systems is offered by \citet{kotkov2016survey}. 

\paragraph{Our focus: Serendipity in the system itself}
By contrast, rather than revisiting the topic of serendipitous discoveries by a system's user which are triggered by the system's behaviour, we focus on understanding efforts to model \emph{serendipity on the system side}.  Indeed, serendipity as a service is in some sense a special case.  We will use existing literature to theorise how a system -- whether human, computer, human+computer, or otherwise distributed -- may potentially exhibit serendipitous behaviour.

\citet{Andre:2009:XSP:1518701.1519009} mention the following
motivations for building systems that may support the user's experience of serendipity, which carry over desiderata for autonomous systems:
\begin{quote}
``\emph{reinforcing an existing problem or solution or taking it
in a new direction, rejection or confirmation of ideas,
identifying information relevant to a latent goal, or just finding
information of interest}''
\end{quote}

\paragraph{Relevant prior efforts}
Efforts to build systems that can be said to behave
serendipitously in some way have typically focused on engineering aspects \cite{wilkins1985recovering,washington1999autonomous,muscettola1997board} -- and in
some cases aesthetics \cite{cybernetic-serendipity} -- rather than on
theory.  Attempts to develop a more systematic treatment of
serendipity include the work of \citet{Figueiredo2001}, who describe
several types of serendipitous ``moves'' that effect the
transformation of a problem that cannot be solved into one that can.

\citet{andre2009discovery} consider the issue from a broad theoretical vantage point.  In their recommendations, they suggest that required features of sagacity and insight in
computational systems may be improved with added domain expertise and
a common language model.  However, it is not clear that the
presence or absence of useful talents like domain expertise and linguistic
ability would allow us to draw a distinction between serendipitous and
non-serendipitous discovery.

Nevertheless, it is certainly true that preparation matters.
Louis Pasteur, who is known for
several lucky experiments \cite{roberts,gaughan2010accidental},
famously remarked: ``Dans les champs de l'observation le hasard ne
favorise que les esprits pr\'epar\'es'' (``In the fields of
observation chance favors only the prepared mind'')
\cite[p.~131]{pasteur-chance}.  The question of how the mind is prepared matters; and here it is useful to consider von Foerster's \cite{von2003cybernetics} notion of a
\emph{second-order cybernetics}, which theorises system participants with the ability to specify their own purpose.
Many historical examples of
serendipity are centred on learning something new, rather than
simply reasoning from facts that are already known -- or simply solving pre-given problems.  This suggests the relevance of mental training to develop a range of attributes including perceptiveness, assertiveness, and adaptability.

\citet{lawley2008maximising} develop a
well articulated process model of serendipity as a sequence of events,
and \citet{Makri2012a} adapt this to focus on connections amongst
data.  Our model takes inspiration from these
approaches and makes the application
to computational systems more systematic.  To do so, we draw
on a range of literature from creativity research
and cognitive science as well as computational examples.

%%%%%%%%%%%%%%%%%%%%%%%%%%%%%%%%%%%%%%%%%%%%%%%%%%%%%%%%%%%%%%%%%%%%%%%%%%%%%%%%%%%%%%%%%%%%%%%%%%%%
% 5. What are the key components of my approach and results? Also include any specific limitations.
%\subsection{Key aspects of our approach, including limitations}

\paragraph{Overview of our findings}
When we consider classic examples of serendipity, such as the
practical uses for weak glue, the possibility that a life-saving
antibiotic could be found growing on contaminated petri dishes, or the
idea that burdock burrs could be anything but annoying, we notice
radical changes in evaluation.   Asking what weak glue could possibly be useful for might kick
off an exploratory process of invention.  A system that can detect
previously missed opportunities and false realisations -- what
\citet[p.~639]{van1994anatomy} terms \emph{negative serendipity} --
may be able to learn from them.

In this paper, we propose a qualitative process model of serendipity, and a definition of the serendipity potential of a system in terms of the components the model is made of, their parameters and their interplay. 
Although technical realisations already exist for most of the proposed components, more work is needed to realise systems that fully integrate them and thus meet the model's
criteria.  We include pointers in this direction in the final
sections.
% 6. "Summary of Contributions". 

\subsection*{Summary of contributions} \label{sec:contributions}

\begin{itemize}
\item In Section \ref{sec:literature-review}, drawing on a brief review of prior literature on the concept of ``serendipity,'' we juxtapose existing theories and models of serendipity, and summarise the logical structure of serendipitous occurrences. We suggest to understand serendipity in terms of discovery, invention and creativity, and thus augment the existing literature with references from creativity research.
\item In Section \ref{sec:our-model}, we synthesise the understanding gained in the previous section in a process-oriented model that points to a definition of the serendipity potential of a system.  We ground this model in relevant computing and cognitive science literature.
\item We provide a demonstration of our model and evaluation procedure in Section \ref{sec:computational-serendipity}, by applying it to three case studies in evolutionary jazz improvisation, automated
programming, and next-generation recommender systems.
\item Section \ref{sec:discussion} concludes the paper with recommendations for researchers working on computational serendipity.
\end{itemize}
\fi


%%%%%%%%%%%%%%%%%%%%%%%%%%%%%%%%%%%%%%%%%%%%%%%%%%%%%%%%%%%%%%%%%%%%%%

%% \begin{itemize}
%% \item The explanation phase will play a particularly crucial role in
%%   the development of automatic programming systems (with whatever
%%   architecture), not just to safeguard stakeholders, but also to make
%%   systems that are understandable to themselves.  Given the
%%   high-profile successes of non-symbolic AI, an emphasis on symbolic
%%   AI may seem counter-intuitive, but it is in fact just as important
%%   to the developmnet of the field. Autonomous evaluation and
%%   meta-evaluation are areas of particular interest, insofar as
%%   self-programming systems may come up with new frameworks for
%%   evaluating the programs it creates, will need ways to evaluate
%%   these.  Again, this can be understood both in terms of safeguards
%%   and user modelling.
%% \item In recommender systems, the short-term value of recommendations
%%   to users should potentially be allowed to suffer as long as the
%%   expected value of future, more intelligently informed,
%%   recommendations remains higher.  In order for these matters to be
%%   computable, recommender systems need contextual
%%   temporally-meaningful models.  In this context, serendipity
%%   potential might be modelled as a payoff function.
%% \item Creative systems should take on more responsibilities related to
%%   valuation, also which will drive the systems' interest and
%%   corresponding focus shifts. Transferring skills or problems from one
%%   domain to another may seen as another ``fundamental'' facet of
%%   creativity.  To be convincing, creative systems need to be able to
%%   address in order to explain their behaviour convincingly.
%% \end{itemize}

%% \textbf{[Do we still have these formal notations in here? I suggest to abandon any attempt to quantify serendipity potential here and make it pure qualitative. Adjust / trim text? Also next paragraph. - CG]}
%% Serendipity potential could also be
%% defined straightforwardly for a family of systems, again using the
%% concept of a layered architecture and indicator functions, now
%% defining $\iota_{\mathit{perception}}$ and the other $\iota_{\cdot}$'s
%% in Definition \ref{def:serendipity-potential} as sums of the
%% corresponding functions over the population. In this way, a wider
%% range of perceptions and other abilities can be brought online.  In
%% such a setting, ``alignment'' between the several components variously
%% takes on a message passing, format-shifting, or broader social
%% interpretation.  Historical examples like the development of
%% Post-it\texttrademark\ notes show how multiple perspectives can
%% contribute to serendipitous outcomes.  Indeed, our Definition
%% \ref{def:interest} suggests that this phenomenon lies at the heart of
%% serendipity -- although like each of the steps in our model, it is
%% necessary but not sufficient.

%% Populations are important in another respect: even if there is only a small
%% chance that some particular event will happen, with sufficient trials,
%% it will likely happen.  This corresponds to Bergson's
%% \cite{bergson1946creative} comment that every discovery will
%% happen ``sooner or later.''  By contrast, as we discussed in Section
%% \ref{sec:literature-review}, the Bergsonian theory of invention states
%% some in-principle possibilities may fail to come into existence.

%% \textbf{[Move into conclusions subsection? Also next paragraph? - CG]}
%% So, is ``pure serendipity'', amenable to generation by a computer?
%% %%
%% The answer is, as in van Andel's introductory quote, ``no''.  Though, we have shown in our analysis that we can
%% describe computational systems that are, at least in principle, as
%% capable of realising serendipitous outcomes as human beings, if not
%% more so.  Although, no one -- human or machine -- can guarantee that any
%% specific effort will lead to an unexpectedly beneficial outcome (by
%% definition, this outcome may fail to materialise), we suggest that we can build systems with greater potential for serendipity. Our contribution can therefore be seen as an necessary extension to van Andel's denial.

%% One broad class of examples, computer-supported serendipity, which we
%% have herein referred to as ``serendipity as a service,'' is already
%% well-studied.  That work has been accompanied by collections of
%% heuristics for system users who may want to increase their own
%% potential for serendipity \cite{makri2014making}.  However, previous
%% scholarly and empirical attention to the potential for serendipity in
%% computational systems themselves has been much more constrained.  Our case
%% studies suggest that serendipity on the system side is within the
%% reach, if not yet the grasp, of contemporary computing practice.

%% \textbf{[Todo: introduce the next few paragraphs as parallels between our and other frameworks, e.g. from CogSci. Potentially move into dedicated Related Work section? - CG]}
%% \citet{Makri2012a} propose that there is a continuum between
%% pseudoserendipity and serendipity, which may correspond to the degree
%% of novelty of the problem and solution\ldots

%% For philosopher and media theorist Vil\'em Flusser,  \emph{automation} in a
%% universe otherwise ruled by entropy is defined as follows:
%% \begin{quote}
%% ``\emph{a self-governing computation of accidental events, excluding
%%     human intervention and stopping at a situation that human beings
%%     have determined to be informative.}'' \cite[p.~19]{flusser2011into}
%% \end{quote}
%% The reason these events are informative is that they are improbable
%% ``from the standpoint of the universe.''  However, Flusser observes
%% that ``from the receiver's standpoint, they are still probable.''  The
%% default operation of most modern technologies is to be thoroughly
%% predictable.

%% Our model considers not only discoveries but also the unexpected
%% invention of new and unanticipated patterns of discovery.
%% This framework is consistent with the theory of the predictive mind,
%% which suggests that `enminded' beings
%% (\citealp[pp.~170--171]{ingold2000perception};
%% \citealp{brandt2011enminded}) are conscious and act precisely on the
%% basis of surprising, unexpected data.

%% Our layered model is also reminiscent of earlier models in AI: we
%% point to \citet{singh2005architecture} and
%% \citet{sloman1998architectures} as relevant prior art.  Sloman and
%% Logan's division into \emph{reactive}, \emph{deliberative}, and
%% \emph{reflective} layers map coherently onto the six-layer division
%% we've presented.  Moreover, Singh and Minsky's six layers --
%% \emph{instinctive reactions}, \emph{learned reactions},
%% \emph{deliberative thinking}, \emph{reflective thinking},
%% \emph{self-reflective thinking} and \emph{self-conscious thinking} --
%% could be seen as nearly analogous to the framework we presented above.
%% %%  This, and related work surveyed in
%% %% \citet{mccarthy2002architecture}, concerns commonsense reasoning.
%% The schematic similarities notwithstanding, we believe that an
%% analysis of serendipity gives a new way to think about these
%% established problems in AI.

%% According to \citet[p.~6]{simon1958heuristic}, ``[i]ntuition, insight,
%% and learning'' have been close at hand in the computing field for some
%% time.
%% %%
%% And yet, computer scientists working in the field of
%% computational creativity --
%% like media artists working in the field of creative computing -- often focus on product evaluations rather than on
%% developing systems with these capabilities.  As alluded to in our
%% case study on {\sf GAmprovising}, \citeA{ritchie07} proposed metrics
%% that depend on properties that a reasonably sophisticated judge can
%% ascribe to generated artefacts: ``typicality'', i.e., the extent to
%% which an artefact belongs to a certain genre, and ``quality.''  These
%% are used as atomic measures from which more complex metrics, including
%% ``novelty,'' can be derived.  In recent years, artefact-centred
%% evaluations are increasingly complemented by methods that consider
%% process \cite{colton2008creativity} or a combination of product and
%% process \cite{jordanous:12,colton-assessingprogress}.  Systems
%% increasingly evaluate their own work in light of an audience model or
%% through other means, and may adapt their goals and behaviour
%% accordingly (e.g., \citealp{gervas2016integrating}). However, ``accidents'' arising outside of the control of
%% the system (and ultimately, outside of the control of the researcher)
%% might be deemed out of scope for computational creativity.  Unexpected
%% external effects could even be seen to invalidate research in this
%% area.

%% We claim that the concept of serendipity brings autonomous creative
%% systems into clearer focus: not via an abstract notion of creativity
%% \emph{ex nilio} or \emph{ex se}, but as creativity in interaction with
%% the world.  This requires a different mindset, and a different
%% approach to system building and evaluation.  Our model provides basic
%% outlines that system designers and developers can use to guide their
%% efforts to develop the potential for serendipity in their systems.

%% %%% \paragraph{\textbf{Chance requires Autonomy}.}
%% Although this was not the main focus of our research, this model may
%% also prove relevant to understanding and averting ``unexpected
%% failures.''
%% %%
%% \citeA{research-priorities} advise AI researchers to build features of
%% \emph{verification}, \emph{validity}, \emph{security} and
%% \emph{control} into their systems.  An uncertain world requires
%% \emph{autonomy} to be added to the list, which means giving up some
%% control, and accepting that there will be missed opportunities on the
%% one hand, and that computers will devote resources to problems we
%% would not have thought of on the other.  Serendipity can
%% be seen as particularly relevant to the future development of
%% ``socially adaptive electronic partners'' \cite{van2015creating}.

%% %%% \paragraph{\textbf{Curiosity implies Learning}.}
%% As we considered ways to enhance the measure of serendipity potential
%% in our case studies, we were led to consider computational agents that
%% increasingly participate in ``our world'' rather than in a
%% circumscribed and highly controlled microdomain.  AI researchers need
%% to think about how to design systems that can not only face ``more
%% complex challenges'' \cite{dontdespair} -- but that can also
%% gracefully learn how to tackle new challenges as they arise.

%% \textbf{[What's still missing from the discussion? - CG]}
%% \begin{itemize}
%% \item How complete is our model?
%% \item How strongly can it guide implementation?
%% \item Serendipity potential across / within different domains: we have observed that comparing different systems based on individual components will be hard and potentially domain-dependent: what does it mean that one system is better at ``perception'' than another? 
%% \item How does our review of technical papers confirm these difficulties?
%% \item What are the challenges of turning the qualitative concept of ``serendipity potential'' into a quantitative notion in future work. 
%% \end{itemize}

%% \subsection{Conclusion}
%% \begin{itemize}
%% \item Distinguished ``serendipity as a service'' vs. ``serendipity in the system'' or ``serendipity \emph{for} a service''.
%% \item Pointed out ambiguity in ``serendipity support'', need to point out whether system supports serendipity in a person or vice versa.
%% \item Resolved ambiguity in the concept of serendipity in a computational context by surveying definitions and etymology.
%% \item Proposed the qualitative concept of ``serendipity potential'' and a sequential framework emphasising the roles of creativity, discovery and invention to evaluate it.
%% \item The framework has been informed by the theoretical literature on serendipity, and by research in computational creativity.
%% \item Compared existing systems in computational creativity, automated programming and recommender systems against our framework to assess their serendipity potential and identify challenges towards a fully independent serendipitous system.
%% \item Discussed how users in these three example domains could benefit from ``serendipity in the system''.
%% \item ``Serendipity in the system'' is an underexplored field of research, hoping that our framework will help researchers to find their way around and conceptualise their approaches.
%% \end{itemize}

%% \subsection{Future Work}
%% \begin{itemize}
%% \item Full implementation of the framework
%% \item Find subfactors of each component in our sequential model to increase the granularity of the serendipity potential concept. 
%% \end{itemize}



%% %% FOR CHRISTIAN! 



%% %%%%%%%%%%%%%%%%%%%%%%

%% %%% \paragraph{\textbf{Sagacity invokes Sociality}.}
%% %% The four supportive factors for
%% %% serendipity described in this paper -- a \emph{dynamic world},
%% %% \emph{multiple contexts}, \emph{multiple tasks}, and \emph{multiple
%% %%   influences} -- give some clues.
%% %% Collectively, these factors strongly resemble our social reality, in which
%% %% structure and meanings are emergent \cite{mead1932philosophy}.
%% %% Whereas \citeA{d2012creativity} and
%% %% \citeA{saunders2002curious}, among others, consider social agents in
%% %% creative contexts, modelling the conditions of emergence presents
%% %% a range of theoretical challenges \cite{Loreto2016}.
%% %% %%% \paragraph{\textbf{Value relies on Judgement}.}
%% %% \citeA{stakeholder-groups-bookchapter}
%% %% outline a general programme for computational creativity, and examine
%% %% perceptions of computational creativity among members of the
%% %% public, computational creativity researchers, and existing creative
%% %%   communities.  We should now add a fourth important ``stakeholder''
%% %%   group in computational creativity research: computer systems
%% %%   themselves.

%% %% %We believe that these categories can also help to theorise the subjective experience of serendipity.
%% %% Our model offers systematic guidance on ways to encourage serendipity
%% %% in computational systems, and a new way to approach heuristics like
%% %% ``seizing opportunities'' and ``looking for patterns'' from
%% %% \citet{makri2014making}.

% \textbf{[Insert a brief paragraph here to outline this section, to ensure consistency with rest of the document. -CG]}
%%% JC: These notes from Christian could be blended into the Discussion.

%% Since 1994, ``network'' models and metaphors have increasingly
%% informed the way we think about brains, bodies, and ecosystems, as
%% well as commerce, creativity, social life, and, indeed, computing.  It
%% is natural to see creativity as happening ``online'' in one form or
%% another.  To be sure, an always-on culture also presents challenges
%% for old-fashioned attention and reflection, the cornerstones of
%% sagacity and insight.  It is not by spurious reconsideration or by
%% recontextualisation willy-nilly that serendipitous findings are
%% formed, but careful reconsideration following, and followed by, due
%% reflection.
%% %% A private concern meets a social context
%% %% in which it can be resolved, an overheard conversation elicits a
%% %% helpful remark and so on.  Thus, for example,
%% %% \citet[p.~339]{Edmonds1994} quotes \cite{gombrich1966norm}: ``in
%% %% searching for a new solution Leonardo projected new meanings into the
%% %% forms he saw in his old discarded sketches.''


%% \citet{corneli2015feedback} took preliminary steps towards the system
%% orientation that we will develop, and also considered how social
%% infrastructures might implement several of the serendipity patterns
%% noted by \citet{van1994anatomy}.  We are aware of recent frameworks
%% designed to help build systems that support the experience of
%% serendipity in their users \cite{niu2017framework,melo2018}: that work
%% testifies to the broader interest that modelling serendipity holds
%% within current computing research, but is different from our present aim.

%% We evaluate the concept of serendipity potential qualitatively by assessing representative existing systems against our framework.  This allows us to identify specific challenges for those who would seek the full and robust implementation of automated systems with serendipity potential.   We argue that users and other stakeholders of autonomous systems could benefit from the effort that will be needed to produce such implementations, and describe potential applications in the domains of recommender systems, computational creativity and automated programming.

%% %% These indications suggest that enhancing systems' serendipity potential
%% %% could enhance their potential to make useful discoveries in various application domains.
%% Artificial intelligence is already making great strides in discovery, ranging from autonomous scientific discoveries in biology \cite{lobo2015inferring} to mixed-initiative applications in drug discovery in which the AI functions ``like a research assistant''  \cite{better-drugs}.
%% % Some scientists have downplayed the role of serendipity because it is seen as non-scientific \cite{ramakrishnan1999data}.
%% Could computational systems make more interesting discoveries if they were able to frame their investigations with a backstory, as is the case for some of the most interesting discoveries by human scientists?

%% %% ******
%% %% Paragraph on discovery
%% %% In 1996, \citet{fayyad1996data} wrote that ``The state of the art in
%% %% automated methods in data mining is still in a fairly early stage of
%% %% development.''  The 20 intervening years have seen considerable
%% %% progress.  Although computational serendipity may seem like a reach
%% %% beyond the current state of the art, a robust discussion of the
%% %% considerations involved can contribute to development of this future
%% %% research area.
%% %% *****

%% NEW SECTION TO EXPLAIN WHAT WE'RE DOING
\subsection{Distilling the literature into a framework} \label{sec:distill}

The different treatments of serendipity in many cases appear to build
on one another, and in all cases appear to be roughly aligned.
Accordingly we can distil from the foregoing survey into a framework
that describes serendipitous phenomena in terms of six phases:
\emph{perception}, \emph{attention}, \emph{interest},
\emph{explanation}, \emph{bridge}, and \emph{valuation}.  Table
\ref{tab:theory-summary} shows graphically how we have drawn out these concepts.
%% In Section
%% \ref{sec:connections} we will explain the analysis in more
%% detail. Section \ref{sec:theory-considerations} summarises the key
%% requirements that a computational perspective imposes on each phase.
%% Section \ref{sec:our-model} will then describes a model derived from
%% this framework.
%\subsubsection{Connections across prior frameworks} \label{sec:connections}
In the following paragraphs, we trace through the rows of Table
\ref{tab:theory-summary} line by line, resummarising earlier
perspectives on serendipity and drawing connections between
these earlier theories and our framework.
Here we use boldface to distinguish elements of earlier theories, and
italics to distinguish elements of our framework.
\begin{enumerate}[label=(\arabic*)]
\item We take Bergson's \cite{bergson1946creative} notion of
  \textbf{discovery} to entail \emph{perception} and \emph{attention},
  triggering \emph{interest}.  In cases of serendipity, we understand
  \textbf{invention} to build on a discovery, through the generation
  of a novel \emph{explanation} and a \emph{bridge} to a newly
  identify problem that the explanation solves.  The solution is then
  \emph{evaluated} positively.
\item Andre et al's \cite{andre2009discovery} \textbf{chance
  encountering of information} explicitly indicates \emph{perception}
  of a chance event.  We take \emph{attention} and
  \emph{interest} to be implicit.  We understand the terse phrase
  \textbf{sagacity to derive insight} to encapsulate what we mean by
  \emph{explanation}, \emph{bridge}, and \emph{valuation}.
\item Cs\'ikszentmih\'alyi's
  \cite{csikszentmihalyi1997flow} three-part model of creativity concerns
  interactions between a Domain, a Field, and an Individual (often
  collectively abbreviated as``DFI'').  In cases of serendipitous
  creativity, we understand the following to occur.
  A \emph{chance event is perceived} that cannot be fully explained
  when \emph{attended to} through the rubric of known \textbf{symbolic
    rules} which comprise a specific cultural Domain.
  A creative Individual is then inspired by the event's
  \textbf{novelty} to achieve a \emph{focus shift}, namely, to examine
  the unexplained details and generate an---\emph{a fortiori} also-novel---\emph{explanation
    of the event}.  Finally, their finding is \textbf{validated} 
  by a Field of experts when the
  explanation can be \emph{bridged} to some (new or existing) problem
  that it solves, in which case the process is deemed creative, and
  given a positive \emph{evaluation}.
\item Allen et al's \cite{Allen:2013:LOD:2655780.2655790}
  category of \textbf{mentioned findings} suggests \emph{perception of a
    chance event} and \emph{attention to salient detail}; their
  category \textbf{inspiration} suggests \emph{interest and focus
    shift} leading to an effort to \emph{explain the event} with a
  research design that explores the serendipitous inspiration; their
  category \textbf{research focus} focuses in on better understanding a
  ``fortuitous discovery'' or ``unanticipated finding'' in order to establish a
  \emph{bridge to a problem} that the discovery solves and thereby contribute
  to \emph{evaluating the result}.
\item Merton's \cite{merton1948bearing} description of
  the observation of an \textbf{unanticipated} datum aligns with
  \emph{perception} of a chance event that captures our
  \emph{attention}: it is a ``fortuitous'' discovery (p. 506).
  Subsequent \emph{interest} in the \textbf{anomalous} nature of the
  datum causes our focus to shift towards a \textbf{strategic}
  \emph{explanation} of the anomaly, leading to the \emph{bridge} from
  the anomalous detail to new theoretical insights.  The new (or
  extended) \textbf{theory initiated} by these investigations receives
  an at least preliminarily positive \emph{valuation}.
\item As described by \citet{wallas1926art}, \textbf{preparations}
  (among with we include observations) afford the
  \emph{perception of a chance event}.  Note that such preparations
  are relevant both to observing the event, and to recognising it as
  unexpected.  During a period of \textbf{incubation}, the perceiver's
  \emph{attention} may be turned towards \emph{salient details} that
  can lead to an \textbf{insight} which then leads to an
  \emph{explanation of the event}.  Here we run into some
  terminological collisions.  What we call the \emph{bridge to a
    problem} could be linked to the insight stage, but we may also
  think of it as rather close to what Wallace calls \textbf{evaluation},
  insofar as the problem that is identified at this stage is what makes
  the insight useful.  In the phase of \textbf{elaboration} (introduced
  by Cs\'ikszentmih\'alyi) the finding undergoes
  further \emph{evaluation} in new contexts.
\item In Lawley and Tompkins's model \cite{lawley2008maximising}, the
  \textbf{prepared mind} is relied upon at several stages in the
  process; indeed, as we described above, we see the prepared mind as
  vitally active throughout.  In the first instance, we can connect it
  with these authors' usage of the term ``\emph{perception}.''  As we
  noted earlier with reference to Clark's theory of predictive
  processing, the mind's previous preparations are what make the
  \textbf{unexpected event} unexpected.  Previous preparations can
  either prevent or allow \textbf{recognising potential} in a given
  observation, in part because these preparations constrain how and
  whether the individual \emph{pays attention} to the event, and
  whether or not they become \emph{interested}. Only when the
  aforementioned steps have occurred might the person \textbf{seize the
    moment} to form a contextual \emph{explanation of the event}; and
  \textbf{amplify effects} by finding a \emph{bridge to a problem}
  that the explanation can solve.  Once all of this is done, then the
  agent may \textbf{evaluate effects}.  Note the role for a prepared
  mind in our sense---as active throughout the process---in
  supporting the ``iterative circularity'' that Lawley and Tompkins
  say may motivate several passes of recursion over the steps between
  evaluating effects and recognising potential, as well as the role of
  chance in producing opportunities to learn.
%% I didn't revise this one yet -JC
\item \citet{Makri2012a} follow Lawley and Tompkins in including
  feedback loops explicitly in their model.  We understand their
  \textbf{new connection} to be formed by the \emph{perception of a
    chance event} and \emph{attention to salient detail} which then
  leads the potential experiencer of serendipity to \textbf{project
    value}.  This subsequently leads to active \emph{interest} when
  the individual in question \textbf{exploits the new connection} in
  some \emph{explicable} way.  Makri and Blandford assert that this
  result is already a \textbf{valuable outcome}, i.e., it solves some
  problem directly; by \textbf{reflecting on its value} the agent may
  \emph{bridge to a (further) problem}.  An interesting aspect of the
  Makri and Blandford model is that \emph{valuation} is somewhat
  ongoing, and that reflecting on value may feed back into the earlier
  part of the process that projected value, leading to renewed
  interest.  As the process iterates, it would seem that additional
  bridges to new problems are created, or that some particular problem
  is understood in more detail.
\end{enumerate}


% For example, both ``valuable outcome'' and ``reflect on value'' from \cite{Makri2012a} have to do with the concept of \emph{valuation}, but we have highlighted that the ``outcome'' has to do with finding a problem that the new connection solves.  Other rows have had similar readings to make sense of earlier frameworks in terms of several common concepts.

\subsection{Design requirements for each phase} \label{sec:theory-considerations}

Here we work from a computational perspective to justify the need for
each of the six dimensions, and to outline what would be required at
each phase in order to support its functioning in a computational
system.  The considerations below are expanded in Section
\ref{sec:our-model}.  However, the description here already shows the
significant demands that the concept of serendipity will impose on any
computational implementation.

\paragraph{Perception \emph{of a chance event}}

People can perceive a wide range of inputs using the six senses;
however machines have a much more varied set of possible inputs. To
enable computers to ``perceive,'' we need to equip them with sensing
apparatus and, more broadly, interfaces with the world and other
systems.  In order to have any potential for serendipity, the machine
will need to have a \emph{prepared mind} that can cope with input
streams that exhibit unexpected \emph{chance} behaviour.
%%
%% In the historical literature on serendipity, chance is often, though
%% not always, thought of in connection with accidental occurrences over
%% which the agent has limited control.
A computational perspective
allows us to explicitly bring to light system-internal events as one
possible source of chance.

\paragraph{Attention \emph{to salient detail}}

The system must not only process events, but also find patterns, as
these arise by chance.  The system should not simply wait to encounter
special predefined keywords or trigger-phrases.  A system should be
able to use knowledge from its foundational {\em prepared mind} to
identify what is salient (and what is not), and accordingly engage
selectively with its input.

\paragraph{Focus shift \emph{achieved by interest}}

The system should include the further ability to select tasks that are
appropriate to events that it designates as being of interest.  It
should be able to recognise (using the foundation of the
\emph{prepared mind}) those features that are worth shifting focus to,
as distinct from those which are salient in some way but not
sufficiently interesting to warrant a focus shift.

\paragraph{Explanation \emph{of the event}}

Finding correlations in data is achievable via data mining, pattern detection and machine learning---however still more is needed for this aspect of serendipity.  Specifically, causal links need to be established, based on the knowledge in the system's {\em prepared mind}.  These may be expressed in explicit functional or operational terms, or statistical models.
%% Do we require ``Explainable AI''?

\paragraph{Bridge \emph{to a problem}}

Once a theory is established that makes a particular event salient, interesting, and explicable, the system will need some capability to abstract its new theory so that it can be applied.  Moreover, we consider serendipity to not just be problem solving, but to integrally involve problem identification.  That is, it is not enough to create a new theory, but the system must also be able to establish a new context in which that theory is relevant.  It must draw on its \emph{prepared mind} to do so.

\paragraph{Valuation \emph{of the result}}

Serendipity results in discoveries or inventions that are seen to be
interesting, aesthetically valuable, or to have other positive
features.  A system with serendipitous potential should use its
\emph{prepared mind} to evaluate the results (and potentially also the
processes involved), so as to make some determination of their value.

\medskip

Note that while we explicitly require \emph{chance} in the first
phase, perception, it could play a role at any of the further stages
as well.

%\textbf{[Make clearer: new connection is not a bridge.]}



% 1) Is this really a computational model? I'd expect such a model to be formalised in detail, with no ambiguity in respect to individual modules and how they connect. Here however, we take each module and propose how it _could_ be implemented, not settling on one configuration. Maybe just stick with "evaluation framework"? Also shorten title to "A framework for assessing the serendipity potential in computational systems"?
% 2) Bind the following two sections together by expressing that and how they define two components of an overall framework - how one completes the other, and how they're both necessary for evaluation. I'd suggest to motivate this section practically: starting with the goal to evaluate the serendipity potential of existing or planned computational systems, what do we need to support coherent evaluation across systems? 

This section develops cognitively and computationally realistic
definitions for each of the six concepts from our synthesis of
theories in Section \ref{sec:literature-review}. 
%%
%% In outline:
%% % I don't think that we really provide definitions, but heuristics and pointers to possible implementations. Furthermore, this statement only relates to part 2, doesn't it?
%% \begin{itemize}
%% \item[] Perception: An interface to the world
%% \item[] Attention: Directed processing capacity
%% \item[] Focus shift: Evaluation of data via existing fitness functions
%% \item[] Explanation: Building a predictive model
%% \item[] Bridge: Identifying or positing a problem
%% \item[] Valuation: Evaluation of the solution via an existing fitness function
%% \end{itemize}
%%
We begin in Section \ref{sec:modelDefinition} with a high-level schematic diagram
that shows how the six phases might in principle be manifested
together in a computational system.  To demonstrate that the schematic
realistically captures common understandings of serendipity, we use it
to redescribe a famous historical case of serendipity: the invention
of PostIt\textsuperscript{\textregistered} Notes at 3M. This helps us
flesh out the requirements given in Section
\ref{sec:theory-considerations}, and prepares the ground for
Section \ref{sec:modelTerms}, where we present
%%  Definitions
%% \ref{def:perception}--\ref{def:result}, outlined above.
informally-stated but practically-inspired definitions of each of the six terms.
% 1) What is a ``schematic'' supposed to be, compared to what we've done previously? That's what I'd ask myself as the typical CS reader of this article. Can we be more specific here? I think we've been discussing the notion of a "process model", emphasising how everything work together, while adding detail to the individual components. Could this be expressed as some kind of UML  diagram, e.g. a "behaviuour diagram"? This would also allow us to use existing notation (e.g. box shapes, etc.) https://en.wikipedia.org/wiki/Unified_Modeling_Language
% 2) I think you mentioned that this diagram only represents one possible instance of how such a system could look like. Please make this idea of a "possible instance" clear and discuss (e.g. in the end of the section) how instances can vary. This is crucial as we want to know how much flexibility is in the framework that we'd like to use for evaluation.
We support the definitions with existing foundational theories from
philosophy and cognitive science, and, for each, outline a set of
heuristics that could inform future implementation work, inspired by
existing implementations.
%The heuristics are not alone sufficient to inform future implementations; only together with pointers to possible implementations do they allow us to (again, in the practical evaluation task) both recognise existing- and identify missing components in present (!!! not mentioned above) or planned systems. Given the paragraph above, you seem to have a different goal in mind than the one in the title: this should be about evaluation, not designing a new serendipitous system. Please clarify.

Each of the six phases in the model has a wide
horizon, often encompassing both good-old-fashioned AI and contemporary approaches. For example, ``Machine Perception and Artificial Intelligence'' is the title of a book series published by World Scientific that began in 1992 and currently contains 83 volumes.\footnote{\url{https://www.worldscientific.com/series/smpai}}
We must therefore be selective rather than comprehensive in our
approach to the literature. 
%Our ~200 references tell a different story, maybe go through one round of trimming?
Our overall aim is show that how computation might be employed to
produce serendipitous results.  Section \ref{sec:system-analysis} will
then use this model to comprehensively assess the potential for
serendipity in discrete implemented systems.
%1) I'd put these goals in the beginning of this intro, not in the end. Cf. my other comments.
%2) "to indicate in broad terms" is a lot of hedging! You have to justify why we are not more specific. There's clearly two counteracting goals: (i) providing a sufficiently detailed framework for evaluation vs. (ii) being too specific about potential applications so that the evaluation framework doesn't scale to new developments. You could make this explicit? 

\subsection{A process model and rational reconstruction of a historical case study} \label{sec:ww-model}
\label{sec:modelDefinition}
%This is the first time you mention the notion of "process model". Please (i) check if this is a fixed notion (does the reader know instantly what to expect? I don't think so, cf. my UML commentary) and back-propagate into the intro.
%What is a _rational_ reconstruction? Also, I'd suggest to put this differently: "Design and evaluation of a process model for serendipity potential". Then you could sell this as a kind of pre-study to the actual evaluation of computational systems in the next big section, here developing a process model and checking whether is fits a prominent example of serendipity in people. As a baseline, so to say. I definitely want to avoid the reader thinking "golly, not another historical study!" - it needs to come with a purpose.
\begin{figure}[h]
\begin{minipage}[b]{\textwidth}
{\centering
\begin{tikzpicture}[
single/.style={draw, anchor=text, rectangle},
]
\node (discovery) {\textbf{\emph{Discovery:}}};
% ``poet generates poem''


\node[single, right=8mm of discovery.east,text width=1.5cm] (poet) {\emph{generative\\ process}};
\node[single, right=6mm of poet.east] (poem) {$E$};
\draw [->] (poet.east) -- (poem.west);

\node[above left=1mm and -10mm of poet] (perception) {{\sf perception}};
% \draw [->,dotted,shorten >=0.05cm,>=stealth] (perception) -- ([xshift=5mm]poet.north west);
%\node[right=1mm of perception] (plane) {{\large\faPaperPlaneO}};

% ``critic listens to poem and offers feedback''
\node[ellipse, draw, right=9mm of poem.east,text width=1.3cm] (critic) {\emph{feedback}};
\draw [->] (poem.east) -- (critic.west);
\node[single, above=8mm of critic.north,text width=1.4cm] (experience) {\emph{reflective\\ process}};
\node[draw,diamond,inner sep =.3mm, above right=4mm and 3mm of critic] (comment) {\raisebox{2mm}{$p\vphantom{^{\prime}}_1$}} ;
\node[draw,diamond,inner sep =.3mm, above left=4mm and 3mm of critic] (reflection) {\raisebox{2mm}{$p\vphantom{^{\prime}}_2$}} ;

\node[above right=2mm and 1mm of comment] (attention) {{\sf attention}};
% \draw [->,dotted,shorten >=0.05cm,>=stealth] (attention) -- ([xshift=1mm,yshift=1mm]comment.east);

\draw[->,thick] ([yshift=1mm]critic.east) to [out=0,in=270] (comment.south) ;
\draw[->,thick] (comment.north) to [out=90,in=0] (experience.east) ;
\draw[->,thick] (experience.west) to [out=180,in=90] (reflection.north) ;
\draw[->,thick] (reflection.south) to [out=270,in=140] ([yshift=1.5mm]critic.west) ;

% nonprinting point to use to bend curve
\coordinate[below right=3mm and 7mm of critic] (mid1);

\node[single, below left=10mm and 2mm of critic] (feedback) {$T$};
\node[single, left=6mm of feedback] (selection) {$T^{\star}$};

% \node[below left=-9mm and 1mm of selection] (subselection) {{\large\faDatabase}};

\draw [-] (feedback.west) -- node[ fill=white, anchor=center, pos=0.5,font=\bfseries,inner sep=0pt,minimum size=1mm](selectionprocess){\guillemotleft} (selection.east);

\node[above right=3mm and -8mm of selectionprocess] (interest) {{\sf interest}};
%\draw [->,dotted,shorten >=0.1cm,>=stealth] (interest) -- ([xshift=-.01mm,yshift=-.01mm]selectionprocess.north);

\node[below=.65cm of discovery] (focusshift) {{\small \textbf{\emph{[Focus shift]}}}};

% draw the first curve into focus shift
\draw [->] ([yshift=-1mm]critic.east) to[out=0,in=90] (mid1) to[out=270,in=0] (feedback.east);

%%% Next phase
\node[below=2.6cm of discovery] (invention) {\textbf{\emph{Invention:}}};

% ``poet integrates feedback''
\node[ellipse, draw, right=12mm of invention.east,text width=1.71cm] (integrator) {\emph{verification}};



% nonprinting point to use to bend curve
\coordinate[above left=2mm and 9mm of integrator] (mid2);



\draw [->,dashed,shorten >=5pt,>=stealth] (integrator) to[in=270] (selectionprocess);

% draw the second curve out from focus shift
\draw [->] (selection.west) to[out=180,in=90] (mid2) to[out=270,in=160] (integrator.west);

% ``poet asks questions about the feedback''

\node[single, below=9mm of integrator.south,text width=2cm] (explainer) {\emph{experimental\\ process}};

\node[draw,diamond,inner sep =.3mm, below right=4mm and 5mm of integrator] (question) {\raisebox{2mm}{$p^{\prime}_1$}};
\node[draw,diamond, inner sep =.3mm, below left=4mm and 5mm of integrator] (answer) {\raisebox{2mm}{$p^{\prime}_2$}};

\node[below left=1mm and 1mm of answer] (explanation) {{\sf explanation}};
% \draw [->,dotted,shorten >=0.05cm,>=stealth] (explanation) -- ([xshift=-1mm,yshift=-1mm]answer.west);

\draw[->,thick] ([yshift=-1mm]integrator.east) to [out=0,in=90] (question.north) ;
\draw[->,thick] (question.south) to [out=270,in=0] (explainer.east) ;
\draw[->,thick] (explainer.west) to [out=180,in=270] (answer.south) ;
\draw[->,thick] (answer.north) to [out=90,in=200] ([xshift=1mm,yshift=-1.8mm]integrator.west) ;

\node[yshift=1mm,single, right=10mm of integrator.east] (problem) {$M$};

\draw [->] ([yshift=1mm]integrator.east) -- (problem.west);

% ``poet reflects on feedback and updates codebase''

\node[single, right=6mm of problem.east,text width=1.2cm] (pgrammer) {\emph{creative}\\ \emph{process}};
\draw [->] (problem.east) -- (pgrammer.west);

\node[above right=1mm and -10mm of pgrammer] (bridge) {{\sf bridge}};
%\node[left=-1mm of bridge] (compress) {{\large\faCompress}};

% \draw [->,dotted,shorten >=0.05cm,>=stealth] (bridge) -- ([xshift=1mm,yshift=1mm]pgrammer.north);

\node[single, right=10mm of pgrammer.east] (solution) {$P$};
\draw [->] (pgrammer.east) -- (solution.west);

\node[single, below=6mm of solution.south,text width=1.6cm] (eval) {\emph{evaluation}\\ \emph{process}};
\draw [->] (solution.south) -- (eval.north);

\node[below left=1mm and -10mm of eval] (valuation) {{\sf valuation}};
% \draw [->,dotted,shorten >=0.1cm,>=stealth] (valuation) -- (eval.west);

\node[single, right=4mm of eval.east,text width=.3cm] (etc) {...};
\draw [->] (eval.east) -- (etc.west);
\end{tikzpicture}


\par}
\smallskip
\end{minipage}
\caption{A boxes-and-arrows diagram, showing one possible process
  model capable of producing serendipitous results.}\label{fig:model}
\end{figure}
% Is this really consistent with the notation of box-and-arrow diagrams? Do they contain e.g. diamonds?
% Also, please briefly mention what the shapes correspond to in here. I'm not sure if "box-and-arrow" diagrams are really a standard. 

Figure \ref{fig:model} places the six phases discussed above into a
diagram outlining the idealised implementation of a (potentially)
serendipitous system.  Some steps are expanded in more detail than
others.  Other architectures might foreground different kinds
of feedback between the main steps, but to keep things simple we
have not shown all possible ways in which the process might revisit
earlier steps as it runs.
We illustrate how the diagram works in a rational reconstruction
of the invention of Post-Its\textsuperscript{\textregistered} at 3M
(quotes below are from \citet{FT3M}).
The level of detail and specificity is intermediate between the design
requirements outlined in Section \ref{sec:theory-considerations} and
the definitions and heuristics that will be advanced in Section
\ref{sec:modelTerms}.  Before developing definitions of the individual
components, it is useful to have an example that puts
the whole process together, i.e, making the interconnections between
the phases explicit.
%1) _The_ or _one possible_ implementation? Why would there be more than one?
%2) "We illustrate how it works": We have designed this model in an iterative process to successfully reflect the processes in existing examples of serendipity in people. We illustrate its working based on the 3M case. 
%3) The last paragraph is good. Would be even better if you described why developing this model is a necessary step towards the definitions/next section.
%%
One immediate challenge arises in building a rational reconstruction of
the Post-Its\textsuperscript{\textregistered} example: the
story includes several steps that could informally be called
``serendipitous'' in light of the success that follows. Our reconstruction
is focused by this aim: to illustrate how a modular architecture
like the one illustrated can create serendipitous results---in this
case, using a social rather than computational infrastructure.
% 1) That's good, but I'd frame it a bit more as "why have we chosen the 3M example specifically"?
% 2) "Social vs. computational": I think you're talking about different things here. E.g. curious whispers is a social CC system, despite being computational.
% 3) If you point out the "serendipity within serendipity" and "Social" aspects of the 3M case here, you also have to point it out later on and show how the model applies across different levels of granularity (e.g. explaining serendipity in each smaller instance as well as in the big one).
\paragraph{Perception of a chance event}
The first module is a \emph{generative process}.  In an
implementation, this may be based on direct observations of the
world; it may also include system-internal sources of chance,
such as a random number generator.  The output of the generative
module is understood as a chance event, $E$, that has been perceived
by the system. It is passed on to the next stage.
%Maybe you could change the emphasis here? The bold of the "example" is much more present than the italics of the \paragraph. To get the right structure, it would be good to have it the other way round.

\paragraph{\textbf{\upshape Example}}
In the 3M case study the event of interest was generated by Spencer
Silver's work in a team that was carrying out research on
``pressure-sensitive adhesives.''
\begin{quote}
Spencer Silver: ``\emph{As part of an experiment, I added more than
  the recommended amount of the chemical reactant that causes the
  molecules to polymerise. The result was quite astonishing. Instead
  of dissolving, the small particles that were produced dispersed in
  solvents. That was really novel and I began experimenting
  further. Eventually, I developed an adhesive that had high `tack'
  but low `peel' and was reusable.}''
\end{quote}
Here we take $E$ to include not only the bare fact of the adhesive's
creation, but also Silver's preliminary assessment.  Simply put, the
new high-tack, low-peel, adhesive would not have been created had the
reaction not captured Silver's attention and interest.  However, we
certainly cannot explain the serendipitous invention of
Post-Its\textsuperscript{\textregistered} with reference to these acts
alone.  With regard to social infrastructures, as
\citet[p.~23]{society-of-mind} remarked, ``It is not enough to explain
only what each separate agent does.  We must also understand how those
parts are interrelated---that is, how \emph{groups} of agents can
accomplish things.''
% Still, I believe there's a strong element of explanation in this. I'd consider the perception to be only about noticing the stickiness. This is very minimal so far; Only in the second stage does the fact that this kind of stickiness is different from what we've seen before raise attention, as it doesn't fit our predictions. It is unclear why it would cause interest (if not for this difference, which is already the mechanism behind attention), but it is certainly only explained after the interest stage in terms of a change in chemical reactant. This quote alone seems to tackle a lot of factors, but it is not sufficient yet for "serendipity within serendipity", as a bridge is missing. However, this quote is clearly not about perception alone.

\paragraph{Attention to salient detail}
In this stage certain aspects of $E$ will be marked up as being of
potential interest, leading to $T$ in the figure.  This designation
does not in general arise all at once.  $T$ is considered to be the
result of \emph{feedback}, an abstraction over a more complex
\emph{reflective process}.  In Figure \ref{fig:model}, the reflective
process makes use of two primary functions: $p_1$ notices particular
aspects of $E$, and another, $p_2$, applies processing power and
background knowledge to enrich $E$ with additional information.  We
could call $p_1$ \emph{awareness}, and $p_2$ \emph{concentration}.
There may be several rounds of feedback applied (recursively) in order
to construct $T$.  Looking ahead to the next phase, $T$ will serve to
\emph{trigger} subsequent interest: but notice that the system is
explicitly involved in creating $T$, which does not simply arrive
wholly formed.  Nevertheless, at this stage there is little direct
evidence of how it will be used later.

\paragraph{\textbf{\upshape Example}}
In the 3M case study the key aspects of the reflective process were
implemented by Silver (who spread \emph{awareness} of the new
adhesive) together with other
employees (who developed a prototype product and gave the topic further \emph{concentration}).

\begin{quote}
Spencer Silver: ``[T]\emph{he company developed a bulletin board that
  remained permanently tacky so that notes could be stuck and
  removed. But I was frustrated. I felt my adhesive was so obviously
  unique that I began to give seminars throughout 3M in the hope I
  would spark an idea among its product developers.}''
\end{quote}

\begin{quote}
Art Fry: ``\emph{I was at the second hole on the golf course, talking
  to the fellow next to me from the research department when he told
  me about Spencer Silver, a chemist who had developed an interesting
  adhesive. I decided to go to one of Spencer's seminars to learn
  more. I worked in the Tape Division Lab, where my job was to
  identify new products and build those ideas into businesses. I
  listened to the seminar and filed it away in my head.}''
\end{quote}

\paragraph{Focus shift achieved through interest}
The trigger $T$ thus consists of the original event, $E$, together
with a range of newly-added metadata and markup.  A focus shift
selects ({\bf \guillemotleft}) some elements from this complex object,
potentially using them to retrieve additional data.  The result is
``of interest,'' denoted above by $T^{\star}$.

\paragraph{\textbf{\upshape Example}}
In the 3M case study, the information that Fry had filed away
before ($T$) became interesting when he realised that he ``had a
[related] practical problem'' ($T^{\star}$).

\begin{quote}
Art Fry: ``\emph{I used to sing in a church choir and my bookmark
  would always fall out, making me lose my place.  I needed one that
  would stick but not so hard that it would damage the book.  The next
  morning, I went to find Spencer and got a sample of his adhesive.}''
\end{quote}
In this case, the adhesive becomes interesting insofar as it could
potentially used to create a re-stickable bookmark.  3M allowed its
employees to selectively allocate 15\% of their time
\cite{tce-postits}, and Fry decides to initiate his own experiments.

\paragraph{Explanation of the event}
The now-interesting trigger $T^{\star}$ is submitted for
\emph{verification}, which Figure \ref{fig:model} depicts as an
abstraction over an \emph{experimental process}, whose operations here
again consist of two primary functions: \emph{theory generation},
$p^{\prime}_1$, and \emph{theory checking}, $p^{\prime}_2$.  The
result of this process is a \emph{model}, $M$.  The dashed arrow in
the diagram is meant to indicate that the focus shift stage may be
revisited and new selections made as this process progresses, i.e.,
$T$ may become interesting for new or different reasons as the
experimental process progresses.

\paragraph{\textbf{\upshape Example}}
In the 3M case study, Fry already has in mind the theory
($p_1^\prime$) that re-stickable bookmarks can be made using the new
adhesive.  Fry creates and adjusts a working prototype  ($p_2^\prime$)
on the way to verifying his theory.

\begin{quote}
Art Fry: ``\emph{I made a bookmark and tried it out at choir practice; it
didn't tear the pages but it left behind some adhesive. I needed to
find a way to keep the particles of the adhesive anchored to the
bookmark.  After a few experiments, I made a bookmark that didn’t
leave residue and tested it out on people in the company.}''
\end{quote}
Note that in this case the event $E$ has not been explained in terms
of ``how'' but rather, contextually, in terms of ``so what?''  The
nature of the explanation will differ from case to case.  The common
feature is the creation of a causal model of some sort.  In this case,
the causal model $M$ is a \emph{method} for creating re-stickable
bookmarks that don't leave residue.

\paragraph{Bridge to a problem}
Here the system forms a connection (``bridge'') between the
explanation in the form $M$ and some as-yet-unspecified problem, $P$.
The schematic represents this step in one block, a \emph{creative
  process}.  This is clearly underspecified, but we shall describe
different possible implementation strategies shortly, in Section
\ref{sec:modelTerms}.

\paragraph{\textbf{\upshape Example}}
Let's see how this process worked in the 3M case study.  Fry now had a
prototype, but so far it didn't solve a very interesting problem.
(``They liked the product, but they weren’t using them up very
fast.'')  But then:

\begin{quote}
Art Fry: ``[O]\emph{ne day, I was writing a report and I cut out a bit
  of bookmark, wrote a question on it and stuck it on the front. My
  supervisor wrote his answer on the same paper, stuck it back on the
  front, and returned it to me. It was a eureka, head-flapping moment
  -- I can still feel the excitement.  I had my product: a sticky
  note.}''
\end{quote}

It would seem that no one, including Fry, had thought about this
problem before: how can we easily make notes on a document, without
marking up the document itself, and without introducing other separate
sheets of paper that would need to be stapled or paper-clipped to the
document, or that might get lost?

Indeed, without knowing the solution in advance, or having $M$ in mind
and re-stickable bookmarks to hand, the problem might even sound like
a contradiction in terms.  It would probably have been impossible to
solve it very well using conventional methods
\cite[p.~90]{altshuller2007innovation}.  But remember that Fry was
part of the Tape Division.  By cutting off a piece of the bookmark,
and affixing it to the front of the report, he was using the bookmark
like one might have used a piece of tape---which would have been
another semi-conventional solution, different from staples and
paperclips, for affixing a separate sheet of paper.  However, the new
``sticky note'' had several advantages over tape: it could be written
on directly and easily removed later.  Thus, we may rationally
reconstruct the bridge to $P$ via an intermediate virtual solution of a note
taped to the report's cover.

\paragraph{Valuation of the result}
The new problem, $P$, which now conveniently has a solution in the
form of $M$, is passed to an \emph{evaluation process}, and, from
there, to further applications.  One possible class of applications
would be a change to any of the modules that participated in the
workflow, corresponding to the potential for learning from
serendipitous events noted by \citet{lawley2008maximising}.

\paragraph{\textbf{\upshape Example}}
The 3M example shows that evaluation can itself be a complex process:

\begin{quote}
Art Fry: ``\emph{We made samples to test out on the company and the
  results were dramatic.  We had executives walking through knee-deep
  snow to get a replacement pad.  It was going to be bigger than Magic
  Tape, my division’s biggest seller.  In 1977, we launched Post-it
  Notes in four cities.  The results were disappointing and we
  realised we needed samples.  People had to see how useful they
  were. Our first samples were given out in Boise, Idaho and feedback
  was 95 per cent intent to re-purchase.  The Post-it Note was
  born.}''
\end{quote}
Notice that in this case the approach to valuation is itself updated on the fly.

%%%%%%%%%%%%%%%%%%%%%%%%%%%%%%%%%%%%%%%%%%%%%%%%%%%%%%%%%%%%%%%%%%%%%%%%%%%%%%%%%%%%%%%%%%%%%%%%%%%%

The effort incorporates an ``ecological'' \cite{kenyon2013ecological}
perspective on artificial intelligence, in which the system develops
in relationship to its operating environment.  This bears on the
concept of ``self-improving'' \cite{Majot2017} AI systems.  The model
of serendipity potential details one way in which such improvements
can be structured.
%%
Below, we discuss additional related work (Section \ref{sec:related}),
existing research that incorporates or references our model, and 
directions for further work (Section
\ref{sec:incorporating}), as well as three representative application
areas (Section \ref{sec:applications}).
 
\subsection{Related work} \label{sec:related}

%%% Most closely related work from introduction...
%% The framework that we advance was inspired by earlier work of
%% \citet{pease2013discussion}, who explored ways to encourage
%% ``discovery \ldots\ in which chance plays a crucial role'' within
%% computational models of creativity.
%% %%
%% Earlier work by \citet{simonton1999creativity} draws relationships
%% between serendipity, creativity, and evolutionary processes.  He
%% advances several proposals that break down the dichotomy between
%% programs which are ``strictly random'' and discovery systems that make
%% use of ``systematic heuristic search'' (p.~169).  Of particular
%% interest in that treatment were evolutionary processes that are
%% ``independent of the environmental conditions of the occasion of their
%% occurrence'' \cite{campbell1960blind}, including combinatorial
%% processes---a condition of ``blindness'' that we will dispense with here.
%% Perception of the environment is a key criterion for the model we present.
%% %%
%% \citet{corneli2015feedback} took preliminary steps towards the system
%% orientation that is developed here, and also considered how
%% social infrastructures might implement several of the serendipity
%% patterns noted by \citet{van1994anatomy}.  \citet{niu2017framework}
%% and \citet{melo2018} have recently proposed frameworks that can help support
%% the experience of serendipity by system users---a testament to the
%% broader interest of modelling serendipity, but different from our
%% current aim.

%% \citet{Figueiredo2001} presented an early analysis and compared
%% serendipity with more standard problem solving.
%%   For example, Pease et al describe
%% the ``bridge'' as the technique, or set of techniques, that ``enables
%% one to go from the trigger to the result'' (p.~65).  That might
%% realistically encompass any and all of the steps we've described in
%% the current work, and furthermore, is imprecise enough to incorporate
%% both path-dependent solutions and serendipity.

%% \citet[p.~335]{Edmonds1994} pointed out that the
%% relationship between ``understanding the creative process'' and
%% ``supporting or amplifying that process'' is not entirely clear, and
%% that, in computing, ``researchers can largely be divided into those
%% considering one or other of those issues in isolation.''  So it has
%% been for research on serendipity as well.
%%   Research into support
%% systems has largely dominated in the computing disciplines, though we
%% have traced several other threads of work that touch on serendipity in
%% settings where there is little direct human involvement, or where
%% humans are involved as stakeholders rather than as users.  Inspired by
%% user-oriented work \citet{Campos2002} assert that ``it is quite
%% possible to program \emph{for} serendipity, that is, to induce
%% serendipitous insights through the use of computers''---though it is
%% worth noting that they remain uncommitted as to just what sort of
%% agent is capable of having those insights.

%%% I'd remove this whole paragraph, since creativity support is different from serendipity support. If you want to keep it, relate it closer to the creativity aspect as done at some point in Sec. 2. But I'd opt for brevity.
 
% JC: Actually, Edmonds is talking specifically ``serendipitous
% creativity'' and I see it as a nice piece of related work and a good
% way in to the discussion.  I've tried to clarify, but would not be
% keen to remove this.


It is unsurprising, then, to learn from \citet{swanson2016predictive}
that the predictive processing framework, which we pointed to as an
inspiration when introducing our model in Section
\ref{sec:modelTerms}, ``should not be regarded as a new paradigm, but
is more appropriately understood as the latest incarnation of an
approach to perception and cognition initiated by Kant and refined by
Helmholtz.''
%%
Kant had contended that ``reason has insight only 
into what it itself produces according to its own design,''
and disparaged the notion of
learning from accidental observations absent ``a previously thought out
plan'' \cite[p.~20]{kant1929critique}.  One also wonders, just what can be learned from a previously
thought out plan in the absence of accidents?
Van Andel's insistence that pure serendipity
cannot be manifested by a computer program seems to address this question.
And yet, the hard line that he takes on the matter
might be significantly tempered if the program in
question was allowed to implement a learning machine in the sense indicated by Turing.

In fact, Kant was also led to consider something akin to unsupervised learning,
which he called \emph{reflective
judgement}.  This process subsumes objects ``under a law
that is yet to be given\ldots\ under a law which is in fact
only a principle of reflection on objects for which we have no objective
law at all''
\cite[p.~265]{kant1987critique}.
This is not quite as different from the
above-mentioned considerations regarding previously thought out plans
as it may at first seem.
Reflection is
a ``subjective principle governing the purposive use of our cognitive powers'' \cite[p.~266]{kant1987critique}.  As an example along these lines, Eco suggested that,
had Kant had the opportunity to observe the platypus, he
would have concluded that it 
is ``a masterpiece of design, a fantastic example of environmental
adaptation, which permitted the mammal to survive and flourish in rivers''
\cite[p.~93]{eco2000kant}.
%%
%% The difference between a
%% ``driving'' (\emph{bewegende}) and ``building'' (\emph{bildende}) force
%% (\emph{ibid}.)~seems to what makes the
There is quite a difference between this creative
line of abductive reasoning and {\sf HRL}'s reductive approach, traced in
Section \ref{sec:pursuit}.  However, as we saw in that section, given
a somewhat richer background theory, {\sf HRL} was also capable of exercising
something akin to reflective judgement, and thereby reinvent
a famous number-theoretic conjecture.
%%
%%  put much stock in the prepared mind: ``Accidental observations, made in
%% obedience to no previously thought-out plan, can never be made to
%% yield a necessary law, which reason alone is concerned to discover.''
%%
%% Perhaps van Andel's remarks on programming serendipity, quoted in the
%% introduction, should be understood relative to similar caveats.
%% These issues seem to be resolved 


Previous work  described an
information-processing model of \emph{insight} \cite{demystification},
after the outline provided by Wallas \cite{wallas1926art}.
Such ideas seem to described applications of
computational technology that ``facilitate the discovery of
previously unknown cross specialty information of scientific
interest,'' as discussed by
\citet[p.~183]{swanson1997interactive}, i.e.,
``literature-based discovery'' \cite{smalheiser2017rediscovering}.
In the approach of Swanson and Smalheiser, conditions of
\emph{complementarity} and \emph{noninteraction} between
two bodies of literature suggest the presence
of ``unnoticed useful information,'' which may be hinted at through
``indirect linkages'' \cite[pp.~184, 185]{swanson1997interactive}.
%%
One class of explicit indirect links are \emph{bridging terms},
as we noted in Section \ref{sec:modelTerms}.
%%
Surfacing these connections drives at insight,
if that is understood to mean ``an improved representation
of an important previously unsolved problem, which now likely contains
the essence of a correct solution'' \cite[p.~118]{demystification}.

The broader parallels we've drawn between Wallas's model
and the concept of serendipity
(Section \ref{sec:distill}) suggest that the latter concept goes beyond
``insight'' to include aspects of ``evaluation.''
%%
Cases of true serendipity integrally involve what Swanson and Smalheiser refer to as ``problem generating''
\cite[p~.186]{swanson1997interactive}.  But in
serendipity, this happens relatively late in the process, rather
than at the outset as it did in Swanson and Smalheiser's work.  \citet[p.~153]{kulkarni1988processes} suggested a
related heuristic: ``If the outcome of an experiment violates
expectations for it, then make the study of this puzzling phenomenon a
task and add it to the agenda.''
%%%%
By remaining \emph{open} \cite{jurvsivc2012cross} to the
identification and pursuit of new challenges,
potentially-serendipitous processes are able to pose and solve novel, useful,
problems.
%%%%
As we noted in Section \ref{sec:theory-considerations}, this comes
with significant demands for any implementation: our examples have
shown that these can indeed be met.  However, Section \ref{sec:pursuit}
shows that this does not always bring immediate advantages.
%%%%

%% but the bridge, which is linked to the experience
%% of insight (as above) and which in humans seems to be facilitated by
%% ``opportunistic assimilation and long-term memory consolidation'' and
%% rewarded by ``positive affective overtones''
%% \cite[p.~118]{demystification} had been explored in less detail.

\subsection{Further applications} \label{sec:applications}

Accounting for serendipity potential in computational systems is
particularly relevant to the design and development of systems with
autonomy.  Due care to the way it is modelled may provide some
built-in assurances that the behaviour of such systems will be
salutary.  The way the valuation stage is implemented will be
particularly relevant, if social values are to be incorporated in the
system's judgements.  However, requirements that apply at this stage
will push back on the earlier stages as well.  Furthermore, several of
the earlier stages carry out of preliminary evaluations, for which
various possible use-cases will have different requirements.  To
underscore these points we now consider representative application
areas in three clusters.

\paragraph{A.~Automatic programming}
Automatic programming is increasingly important in industrial
applications.  Eric Bant\'egnie, the founder of a company that makes
safety-critical software, remarks in a recent interview for \emph{The
  Atlantic} \cite{saving-the-world-from-code},
\begin{quote}
\emph{``Nobody would build a car by hand.  Code is still, in many
  places, handicraft.  When you're crafting manually 10,000 lines of
  code, that's okay.  But you have systems that have 30 million lines
  of code, like an Airbus, or 100 million lines of code, like your
  Tesla or high-end cars--that's becoming very, very complicated.''}
\end{quote}
Solutions to this problem include variations on the theme of
\emph{model-based design}, in which code is generated from a physical
or mathematically formalised model of the system's behaviour.
Schmidhuber's \cite{schmidhuber2007godel} ``G\"odel Machines'' are one
such example.  However, as the system's complexity grows, we encounter
a problem: \citet{minsky1967programming} had already envisioned
systems that reprogram themselves in ways that are sufficiently
complex that their users simply will not be able to understand the way
they work.  Many modern large-scale machine learning systems are often
similarly opaque, although some current research seeks to address this
issue (e.g., \citet{DBLP:journals/corr/ParkHASDR16}).  The ``right to
explanation'' for people who are subject to automated decision making
is discussed in current legal and policy debates
\cite{wachter2017right}.
%% As in the famous story by Borges, the system's ``map'' may obscure
%% the underlying territory.

In order for complex self-programming systems to realise behaviour
that could be deemed serendipitous, criteria are needed that can make
their products and processes meaningful, at least to the systems
themselves.  This shifts the basic emphasis in automated programming
applications from a mostly local analysis that asks which modules can
be fit together in a plan or service-oriented assembly, to a more
global analysis that seeks to qualitatively characterise the generated
outputs, programmes, or behaviour.  In terms of our model, the
\textbf{explanation} phase will play a particularly crucial role in
future developments.  We also encounter challenges for autonomous
\textbf{valuation}, and indeed, meta-evaluation, insofar as a
self-programming system may come up with new frameworks for evaluating
the programs it creates, and will need some way to evaluate these
\cite{jordanous2014stepping}.  Embedding the system in a context in
which it is tasked with responding meaningfully to
user-submitted prompts might provide a useful training ground
\cite{corneli2016language}.

\paragraph{B.~Recommender systems}
Designers of recommender systems frequently consider serendipity with
the goal of supporting discovery on the part of the end user.
In Section
\ref{sec:related},
we discussed
one such research prototype, {\sf Max}, but related
considerations also apply to tools that are in popular use.  For
instance, an Amazon spokesperson told \emph{Fortune}:
\begin{quote}
\emph{``Our mission is to delight our customers by allowing them to
serendipitously discover great products.  We believe this happens
every single day and that's our biggest metric of success.''}
\cite{amazon-secret}
\end{quote}
We discussed {\sf Max}'s
limitations above, but in fact a few recommender
system architectures do seem to take into account \emph{invention} on
the system's side, e.g., by using Bayesian methods to generate new
ways of making recommendations \cite{shengbo-guo-thesis}.  However,
even in the Bayesian setting, the system has limited autonomy.  In
complex and rapidly evolving domains where hand-tuning is
cost-intensive or infeasible, it may make sense to build recommender
systems that can more flexibly invent new recommendation
strategies, not just fine-tune pre-existing strategies.  Considering
such next-generation recommender systems, we can then ask what it
would mean for a new strategy to be ``serendipitous.''

In effect, such an architecture would invert the flow of control that
is present in current serendipity support tools.  No longer
would the system simply supporting the user's experience.
User behaviour or other underlying changes in the domain would
now support the system's
ongoing development (perhaps enacted through self-programming,
along the lines described above).  Accordingly, in order to bootstrap such a system,
users might be assigned tasks that are designed to trigger serendipity
on the system-side.  Thus, the system would explicitly ``use'' the
user.  Techniques such as active learning---which has been applied to
make serendipity as a service more efficient
\cite{10.1007/978-3-319-58068-5_8}---might be applied to reduce the number
user-supplied ratings that need to be acquired.

This design sketch points to long-term considerations pertaining to
\textbf{valuation} that go beyond the anatomy of a unique
serendipitous event.  In particular, the short-term value of
recommendations to users (as measured by business metrics) should
potentially be allowed to suffer as long as the expected value of
future, more intelligently informed, recommendations remains higher.
Both data gathering and experimentation may come at the cost of
performance.  Optimising the learning process itself becomes a project
for evaluation.

\paragraph{C.~Computational creativity}
\citet{jordanous10} reported on a system using genetic algorithms for
computational jazz improvisation, which was later given the name {\sf
  GAmprovising} \citep{jordanous:12}.  One of the system's key
limitations is its ``fitness bottleneck'' \cite{jordanous10}: a human
evaluator responsible for rating the typicality and value of the
generated compositions (after \citet{ritchie2007some}).  Furthermore,
because this takes place iteratively, at each generation, the user is
essentially responsible for all responsibilities related to
\textbf{valuation}.  This also drove the system's \textbf{interest}
and corresponding \textbf{focus shifts}.  Despite to these
limitations, {\sf GAmprovising} was seen to have reasonably
satisfactory behaviour:
\begin{quote}
\emph{``Over several runs, it was able to produce jazz improvisations which
slowly evolved from what was essentially random noise, to become more
pleasing and sound more like jazz to the human evaluator's ears''}
\mbox{\cite{jordanous10}}.
\end{quote}
Clearly, a future evolutionary music system would be more convincing
as an autonomous creative entity if it could evaluate musical works
without such involved user intervention.  Interaction between the system's tasks and
more dynamism in its influences could help differentiate behaviour
within individual threads or generations, providing a route to quite
sophisticated internal evaluation.  For example, perhaps the
population of evolving agents could be tuned to notice and take an interest in specific
musical patterns.  Consequently, individual Improvisers would be more
selective, and more variable in performance, while the population as a
whole would be more musically sophisticated.  More sophisticated
judgements of quality would open the way for the system to learn
identifiable musical skills and to innovate relative to this skill
set.

For comparison, {\sf Entropica} is a system that seeks to maximise
entropy, and it used this single ability to make discoveries in
several domains \cite{wissner2013causal}; however it is not clear that
any of its discoveries involved \textbf{bridging} one of its skills or
problems into a new domain.\footnote{\url{http://www.entropica.com/}}

Earlier, we proposed concept blending as one possible tool to use in
bridging.  However, \citet{bou2015role} show that while concept
blending can be applied to analyse and retrospectively reconstruct
mathematical examples, more work would be needed to build a mathematical
system that convincingly asks questions that drive the selection of
items to blend.  \citet{kaliakatsos2016argument} used blending in a
music context, but as for {\sf GAmprovising}, their system required a human in
the loop for evaluation purposes.

%% \citet{wopereis2017} remark that
%% modelling serendipity in computational systems is a topic that is
%% growing in interest: ``Seeking serendipity may sound as a paradox,
%% just like controlling it, [however there is] increasing evidence that
%% we can influence and stimulate it.'' \citet{surroca2015} noted that
%% our work was the only instance of ``the formalization and the
%% measurement of this phenomenon'' that they had knowledge of (p.~404).

%% Here we should stress that quantitative measurement of serendipity
%% potential, which we had attempted to deal with in an earlier draft of
%% this paper, gives rise to complications that have since caused us to
%% beat a retreat.  A full picture of serendipitous creativity must take
%% into account both the discoverer and the environment, and in the
%% valuation step, the discovery itself, if not also way it is
%% communicated (cf.~\cite{jordanous2016four}).  Measuring the
%% serendipity potential of a given system is not realistically possible
%% without knowing a great deal about the landscape in which that system
%% operates.  This does not detract from the possibility of
%% operationalising the concept of serendipity potential within specific
%% applications, as our analysis above shows, and as we detail in further
%% examples below.  It might be possible to formalise the concept of
%% serendipity potential in a Solomonoff-style probabilistic treatment
%% \cite{solomonoff1986application},
%% or as a suitably formulated Bayesian reinforcement learning problem \cite{vlassis2012bayesian},
%% or in some other framework, but this must be left for future
%% work.  In addition, while we have been inspired by predictive processing
%% and active inference, the project of formally redescribing the model
%% in terms of the situated, recursive, neural architectures frequently referred to
%% in that line of work is similarly deferred.

%% The existing model's qualitative aspects have informed discussions of
%% the serendipity potential of recommender systems
%% \cite{kotkov2016survey,patel2018} and the reporting of serendipitous
%% events \cite{allen2018}.  The framework was also referenced briefly in
%% connection with research into serendipity in revenue models
%% \cite{bechmann2016}, preference-guided content discovery on the Web
%% \cite{surroca2015}, computational models of curiosity
%% \cite{grace2017}, literary creativity \cite{gervas2016integrating} and
%% musical improvisation \cite{wopereis2017}.  All of these would be
%% interesting topics to develop further, and such investigations would
%% be likely to give rise to additional domain-specific heuristics.  We
%% look at three broad application areas in the next section.

With regard to Harold Cohen's painting program, \citet[p.~340]{Edmonds1994} remarks
``Perhaps the prime restriction on {\sf AARON}'s creativity is that it
cannot see.'' 
Although more recent computer painting programs have
overcome this limitation (e.g., in \citet{colton2015painting}), this
does not immediately translate into richly meaningful behaviour.

%% {\ldots \emph{when Sims initially attempted to evolve locomotion behaviors, things did not go smoothly. In a simulated land environment with gravity and friction, a creature’s fitness was measured as its average ground velocity during its lifetime of ten simulated seconds. Instead of inventing clever limbs or snake-like motions that could push them along (as was hoped for), the creatures evolved to become tall and rigid. When simulated, they would fall over, harnessing their initial potential energy to achieve high velocity.  Some even performed somersaults to extend their horizontal velocity}}
%% Krcah \ldots\ bred creatures to jump as high above the ground as possible. 
%% In the first set of experiments, each organism’s fitness was calculated as the maximum elevation reached by the center of gravity of the creature during the test. This setup resulted in creatures around 15 cm tall that jumped about 7 cm off the ground. However, it occasionally also resulted in creatures that achieved high fitness values by simply having a tall, static tower for a body, reaching high elevation without any movement. 
%% In an attempt to correct this loophole, the next set of experiments calculated fitness as the furthest distance from the ground to the block that was originally closest to the ground, over the course of the simulation. When examining the quantitative output of the experiment, to the scientist’s surprise, some evolved individuals were extremely tall and also scored a nearly tenfold-improvement on their jumps! However, observing the creatures’ behaviors directly revealed that evolution had discovered another cheat: somersaulting without jumping at all.

%Why it matters
% In the year 2000, serendipity was voted Britain's favourite word, in connection with a poll run by the Word Festival of Literature.
% https://www.theguardian.com/uk/2000/sep/19/books.booksnews
% However they are described, products such as penicillin, the Velcro\textsuperscript{\texttrademark} strip, and 3M's ubiquitous Post-it\textsuperscript{\textregistered} Notes have changed everyday lives in both big and small ways, and in the process become household names. 
% However, each of these scientific advances and paramount inventions had an unexpected component.

% It is often recounted how Alexander Fleming discovered the antibiotic properties of \emph{penicillum notatum} whilst cleaning contaminated Petri dishes; how Georges de Mestral had the idea that little hooks like those found in burrs stuck to his dog's fur could be useful for something and not simply a nuisance; how Spencer Silver accidentally created a high-tack, low-peel, adhesive that no one found particularly useful, until his coworker Arthur Fry used it to make restickable bookmarks to mark the pages in a hymnbook.  In each of these familiar examples, we see radical changes in the evaluation of what's interesting, followed by an outcome in which the initially unexpected turns out to be both explicable and useful.  We enjoy stories like this, and in the year 2000, serendipity was even voted Britain's favourite word.

%Serendipity in a computational context so far

%Serendipity as service and our perspective shift
%% Crucially, all of these examples use the concept of serendipity to denote and design systems which stimulate the experience of serendipity in their users.  They realise what we call \emph{serendipity as a service}. 

%%% , possibly to the benefit of the user or other stakeholders in the long run.
% In such systems, serendipity is not programmed in as the primary goal, but is a useful side effect that must be allowed for, even if it is not explicitly planned.
%% By and large, the existing mainstream uses of serendipity in a computing context could be characterised as serendipity support tools \cite{andre2009discovery}.  

%  Indeed, this example indicates that there is more to serendipity than simply  ``making discoveries'' \cite{walpole1937yale}.
%Our framework
%Serendipity potential

%%
%% However---contrary it would seem, to de la Maza's hopes---van Andel has put forward a particularly strong version of Lovelace's Objection, stating that an artificial system could never be independent of a person in leveraging serendipity.
%% \begin{quote}
%% \emph{``Like all intuitive operating, pure serendipity is not amenable to generation by a computer. The very moment I can plan or programme `serendipity' it cannot be called serendipity anymore. All I can programme is, that, if the unforeseen happens, the system alerts the user and incites him to observe and act by himself by trying to make a correct abduction of the surprising fact or relation.''}  \cite{van1994anatomy}
%% \end{quote}

%% In a comparable human context, Louis Pasteur, who is known for his serendipitous discoveries in chemistry and biology \cite{roberts,gaughan2010accidental}, famously remarked: ``Dans les champs de l'observation le hasard ne favorise que les esprits préparés'' (``In the fields of observation chance favours only prepared minds'') \cite[][p. 131]{pasteur-chance}.\footnote{Van Andel pointed out (p.c.) that Pasteur's manuscript actually says ``Dans les champs de l’observation,
%% le hasard ne favorise que \emph{des} esprits pr\'epar\'es'' \cite{bourcier2011serendipite}---``In the fields of observation chance favours only \emph{some} prepared minds.''}  ``Preparedness'' encompasses various ways in which the serendipity potential of an intelligent system can be enhanced.  

%% Broadly, we see this work as a contribution to machine discovery, a
%% topic that has been of interest throughout the history of AI research and
%% is increasingly relevant in contemporary applications.
%% Indeed, Herbert Simon contended that ``a
%% large part of the research effort in the domain of `machine
%% learning' is really directed at `machine discovery'{''}
%% \cite[p.~29]{simon1983should}.  Since serendipity has often played an important role in
%% human discovery, it makes sense to ask whether there is a way of
%% interpreting it in a computational context.  We develop such an
%% interpretation in this paper.

%% The current work makes the following concrete contributions towards the future
%% development of systems with serendipity potential, and towards their
%% rigorous analysis.
%In this paper, we propose a qualitative process model of serendipity, and a definition of the serendipity potential of a system in terms of the components the model is made of and their interplay. We include pointers in this direction in the final sections. We assess.. impact/benefits for application areas.



%% \paragraph{Environmental factors}
%% \begin{itemize}
%% \item Dynamic world: 
%% \item Multiple contexts: Reasoning could operate across different domains, eg one agent in number theory, one in group theory.
%% \item Multiple tasks: 
%% \item Multiple influences: This is a multi-agent system, where
%% multiple software agents with different knowledge and goals
%% interact.
%% \end{itemize}

%% \paragraph{Skills of the discoverer in the computational case}

%% \begin{enumerate}
%% \item Prepared mind: the agent's background knowledge,
%% developed theory, store of unsolved problems, set of
%% production rules, interestingness measure, and revision
%% techniques, and current focus 
%% \item Perception of the new event is when agent receives the
%% conjecture ``all even numbers are the sum of two primes''.
%% Agent pays attention to the conjecture by re-constructing
%% it in order to put it into context with its existing theory
%% \item Bridge: the agent performs the theory formation step to
%% get the new (modified) conjecture. 
%% \item Result: 
%% \end{enumerate}

%% \paragraph{The value of the discovery in the computational case}


%% \begin{table}[ht]
%% \begin{center}
%% \footnotesize

% This might look nicer with black lines between the cells? -JC
% Sorted! - Jc

%% \begin{tabularx}{\textwidth}{l|p{\widthof{Explanat}}|p{\widthof{Explanat}}|p{\widthof{Explanat}}|p{\widthof{Explanation}}|p{\widthof{Explanat}}|p{\widthof{Explanat}}|}
%% \multicolumn{1}{c}{\phantom{System}} &
%% \multicolumn{1}{c}{Perception} &
%% \multicolumn{1}{c}{Attention} & 
%% \multicolumn{1}{c}{Interest\slash F.S.} &
%% \multicolumn{1}{c}{Explanation} &
%% \multicolumn{1}{c}{Bridge} &
%% \multicolumn{1}{c}{Valuation} \\
%% %% \phantom{System} & Percep- & Atten- & Interest/   & Explan- & Bridge & Valua- \\
%% %%                  & tion    & tion   & Focus Shift & ation   &        & tion \\
%% \hline 
%% \multicolumn{1}{l}{}&\multicolumn{1}{c}{}&\multicolumn{1}{c}{}&\multicolumn{1}{c}{}&\multicolumn{1}{c}{}&\multicolumn{1}{c}{}&\multicolumn{1}{c}{}\\
%% \hhline{~------}
%% \multicolumn{1}{l}{\scriptsize {\sf DAYDREAMER}} & \multicolumn{1}{|Y|}{\cellcolor{yellow!25}SOME} & \multicolumn{1}{|Y|}{\cellcolor{green!25}Y}    & \multicolumn{1}{|Y|}{\cellcolor{green!25}Y}    & \multicolumn{1}{|c|}{\cellcolor{green!25}Y}    & \multicolumn{1}{|Y|}{\cellcolor{green!25}Y}    & \multicolumn{1}{|Y|}{\cellcolor{yellow!25}SOME} \\
%% \hhline{~------}
%% \multicolumn{1}{l}{\scriptsize Calculator}            & \multicolumn{1}{|Y|}{\cellcolor{yellow!25}SOME} & \multicolumn{1}{|Y|}{\cellcolor{red!25}N} & \multicolumn{1}{|Y|}{\cellcolor{red!25}N}    & \multicolumn{1}{|c|}{\cellcolor{red!25}N}    & \multicolumn{1}{|Y|}{\cellcolor{red!25}N}    & \multicolumn{1}{|Y|}{\cellcolor{red!25}N}    \\
%% \hhline{~------}
%% \multicolumn{1}{l}{\scriptsize Coll.-Mob. (System)}   & \multicolumn{1}{|Y|}{\cellcolor{yellow!25}SOME} & \multicolumn{1}{|Y|}{\cellcolor{yellow!25}SOME} & \multicolumn{1}{|Y|}{\cellcolor{yellow!25}SOME} & \multicolumn{1}{|c|}{\cellcolor{yellow!25}SOME} & \multicolumn{1}{|Y|}{\cellcolor{red!25}N}    & \multicolumn{1}{|Y|}{\cellcolor{red!25}N}    \\
%% \hhline{~------}
%% \multicolumn{1}{l}{\scriptsize Coll.-Mob. (Audience)} & \multicolumn{1}{|Y|}{\cellcolor{green!25}Y}    & \multicolumn{1}{|Y|}{\cellcolor{green!25}Y}    & \multicolumn{1}{|Y|}{\cellcolor{green!25}Y}    & \multicolumn{1}{|c|}{\cellcolor{green!25}Y}    & \multicolumn{1}{|Y|}{\cellcolor{yellow!25}SOME} & \multicolumn{1}{|Y|}{\cellcolor{yellow!25}SOME} \\
%% \hhline{~------}
%% \multicolumn{1}{l}{\scriptsize {\sf GH}}   & \multicolumn{1}{|Y|}{\cellcolor{yellow!25}SOME} & \multicolumn{1}{|Y|}{\cellcolor{yellow!25}SOME} & \multicolumn{1}{|Y|}{\cellcolor{green!25}Y} & \multicolumn{1}{|c|}{\cellcolor{green!25}Y} & \multicolumn{1}{|Y|}{\cellcolor{red!25}N} & \multicolumn{1}{|Y|}{\cellcolor{yellow!25}SOME}   \\
%% \hhline{~------}
%% \end{tabularx}
%% \normalsize
%% \end{center}
%% \caption{Summary of our analysis of the serendipity potential of example systems \label{tab:systems-analysis}}
%% \end{table}

%% \citet[p.~2]{rothenberg:90} reviewed a
%% collection of international perspectives on creativity and found
%% ``creativity involves thinking that is aimed at producing ideas or
%% products that are relatively novel.''
%%
%In computational creativity the combination of novelty and usefulness is accepted as key
%%

%Several nearby categories involve luck,
%%  are
%% \emph{blind luck}, the \emph{luck of the diligent} (or
%% pseudoserendipity) and \emph{self-induced luck};

%% He elaborated
%% this thought in light of what he termed the {``}`barking up the right
%% tree' phenomenon'':
%% % (p.~50) that relates the ``intersection between the
%% % personal hobby'' of the investigator and ``the scientific problem''
%% % (p.~49):
%% \begin{quote}
%% ``[I]\emph{f you happen to be the kind of person who hunts afield, it
%%     may be, in fact, your dog who leads you up to the correct tree,
%%     and to a desirable conclusion.}'' \cite[p.~50]{austin1978chase}
%% \end{quote}
%% %
%% Thus, some, if not all, forms of creativity combine both chance and
%% choice.

  We
agree with Copeland that a contextual perspective is necessary, and we
will return to this theme in what follows: nevertheless an agent (or
\emph{agency}, per \citet{society-of-mind}) that experiences
serendipity is also necessary, and a natural place to begin modelling
work.

 \citet{copeland2017serendipity} has
argued that ``the insight of the individual is insufficient for
bringing about a serendipitous, scientific discovery,'' and makes a
case for an understanding of serendipity that ``goes beyond the
cognitive.'

%% \citet{turing1950mind} had used a ``skin-of-an-onion'' analogy to
%% suggest that all of the attributes of mind should be realisable in
%% mechanical systems.
% 1) Ah okay; I only notice now that you actually kept the definitions. Well, I'm a bit on the fence here, as I don't think that we are capable of providing an exhaustive definition of any of these components. Instead, I'd have argued to provide a set of heuristics to form a definition inductively. Importantly, we understood heuristics in different ways. I understood a heuristic as something like "if you see this, we're dealing with perception"; you however understand it as "to increase the serendipity potential with respect to this component, do this and that". I guess that's fine.
% 2) Start with our goal again: we want to evaluate the serendipity potential of existing and planned systems; thus we need means to recognise (present/missing) components. We thus provide both heuristics and possible implementations to support this task. The reader needs something to hold on to.
%%
% Be even more transparent about what we can and what we couldn't do: we cannot get an exhaustive definition right; and we cannot list all possible methods and we're no experts in each individual domain, e.g. perception-computer vision, so rather list a number of example systems with interesting implementations of those heuristics.
%%
%% Upon considering these reflections, we cannot subscribe to the view that
%% serendipity is ``a process of discovering with a completely open
%% mind'' \cite{darbellay2014interdisciplinary}.  The mind will in
%% general have been shaped by previous interactions with the world.

%% %%
%% We outline representative directions for future applications of
%% our model in the domains of automated programming, recommender systems, and
%% computational creativity.
%% %% Section 6
%% We conclude that it is feasible to equip computational systems with
%% the potential for serendipity, and that this could be beneficial in
%% varied artificial intelligence applications, particularly those
%% designed to operate responsively in real-world contexts.
%% 1-introduction.tex:\section{Introduction} \label{sec:introduction}
%% 1-introduction.tex:%\subsection{Outline}
%% 2-literature.tex:\section{The structure of serendipitous occurrences: deriving our model from a literature review}
%% 2-literature.tex:\subsection{Etymology and selected definitions}
%% 2-literature.tex:\subsection{Theories of serendipity and creativity} \label{sec:serendipityInvention}
%% 2-literature.tex:\subsection{Distilling the literature into a framework}
%% 2-literature.tex:%\subsubsection{Connections across prior frameworks} \label{sec:connections}
%% 2-literature.tex:\subsection{Design requirements for each phase} \label{sec:theory-considerations}
%% 2-literature.tex:\subsection{Summary} \label{sec:literature-summary}
%% 3-model.tex:\section{A computational model and evaluation framework for assessing the potential for serendipity in computational systems} \label{sec:our-model}
%% 3-model.tex:\subsection{A process model and rational reconstruction of a historical case study} \label{sec:ww-model}
%% 3-model.tex:\subsection{Definitions of the model's component terms} \label{sec:modelTerms}
%% 3-model.tex:\subsection{Summary}
%% 4-case-studies.tex:\section{Testing the effectiveness of the model: Can it discriminate between systems that have serendipity potential and those that do not?}\label{sec:system-analysis}
%% 4-case-studies.tex:%\subsection{The Serendipity Engine}
%% 4-case-studies.tex:\subsection{{\sf DAYDREAMER}}
%% 4-case-studies.tex:  \@startsection{paragraph}{4}%
%% 4-case-studies.tex:\subsection{Calculator}
%% 4-case-studies.tex:\subsection{{\sf Colloquy of Mobiles}}
%% 4-case-studies.tex:\subsection{The {\sf GH} System}
%% 4-case-studies.tex:\subsection{Summary}
%% 5-discussion.tex:\section{Discussion} \label{sec:discussion}
%% 5-discussion.tex:\subsection{Work incorporating the model and potential for further development}
%% 5-discussion.tex:\subsection{Further applications}
%% 6-conc-future-work.tex:\section{Conclusions} \label{sec:conclusion} 
%% 6-conc-future-work.tex:%% \subsection{Serendipity in computational systems: Potential future applications} \label{sec:computational-serendipity}
%% 6-conc-future-work.tex:%\subsection{Overall summary}

%% %% Section 3
%% We then use this framework to organise a survey 
%% of literature in cognitive science, philosophy, and computing,
%% which yields practical definitions
%% of the six phases, along with heuristics for implementation.
%% %% Section 4
%% We use the resulting model
%% to evaluate the serendipity potential of four existing systems
%% developed by others, and two systems previously developed by two of us.
%% Section 5




%% Note that in addition to the requirements described in Section
%% \ref{sec:theory-considerations} regarding the explanation step, it may
%% be possible for the system to explain its own processing: for example,
%% to explain why the event was of interest in the first place, or
%% to explain how it developed the bridge.  However, from the point of
%% view of our model, such features of so-called ``\emph{Explainable AI}''
%% \cite{lane2005explainable} are only nice-to-have.

%% We turn now to a deeper exploration of the multi-faceted foundations
%% of serendipity from an explicitly computational perspective.

%%%%%%%%%%%%%%%%%%%%%%%%%%%%%%%%%%%%%%%%%%%%%%%%%%%%%%%%%%%%%%%%%%%%%%%%%%%%%%%%%%%%%%%%%%%%%%%%%%%% WE COULD MAKE ANOTHER PAPER OUT OF THIS SURVEY
                                         
% Overall a much better chapter, but still feels (i) not precise enough in many places regarding potential implementation details and (ii) is sometimes too specific, e.g. talking about aesthetics when talking about interest. Furthermore, it would benefit from (iii) more hands-on motivation .
% Re: (i) I think it has helped in some places to mention how existing projects implement these ideas, instead of just referencing them. Maybe check again if this is done for every component? It has not been done for the latter, for instance. 
% Re: (iii) You could first remind the reader of the goal to develop an evaluation tool. And then motivate the necessary steps towards there. Process model, pre-study evaluated on historical example. Then heuristics for each component and potential applications to flesh this out.

\subsection{Examples of systems implementing the phases} \label{sec:modelExamples}

In this section we give some examples of systems implementing the
individual phases.

\paragraph{\textbf{\upshape Perception: Heuristics}}

\begin{description}
\item[To create the possibility for varied patterns of inference to arise, support rich interfaces.] 
Computer support for natural
  language interaction remains limited; Human-Computer Interaction
  researchers have experimented with a much wider range of interface
  designs (e.g., ranging from 
head tracking and gesture tracking \cite{turk2000perceptive} to
interaction through dance \cite{jacob2015viewpoints}
  and with physical models \cite{stopher2017technology}).
  %This is irritating: you talk about ways to interact with the world; how does this relate to perception? This has to be clarified first. Then still, I don't get why we're talking about natural language here, or about head/gesture tracking, ...
  %A point that you could make though is that perception for serendipity in artificial systems should not be restricted to the human senses; and any form of perception could prove valuable.
\item[To reduce constraints, allow features to be defined inductively.]
Rather than building systems that simply notice
  pre-conceived features of the environment, recent research has dealt
  with systems that independently discover perceptible features
  \cite{inceptionism}.
\item[Organise and process perceptions differently depending on the tasks undertaken.]
 Humans have \emph{head direction} and \emph{grid cells} that
  help define our relationship to the environment, and that support
  spatial navigation tasks.  Similar phenomena have been reproduced in
  machine learning programs for similar tasks \cite{Banino2018,cueva2018emergence}.
  However, AI systems often operate in environments that are
  structured very differently from their human analogues, e.g., by
  machine learning over text corpora.  Rather than adjusting the
  perceptions, it may be be preferable to build constraints on action
  that give an ``explicit characterization of acceptable behavior''
  \cite[p.~356]{caliskan2017semantics}.
  %I don't get what the last sentence is supposed to mean, and how it related to our perception problem.
\end{description}

\paragraph{\textbf{\upshape Attention: Heuristics}}

\begin{description}
\item[Attention can be understood as competition for scarce processing resources.]
   For example, visual attention has been
  described this way \cite{helgason2012attention}, and parallels can
  be seen in grammar-inducing processes \cite{wolff1988learning}.
  Taken as a metaphor, this extends to ``the mental grammar of the
  investigator'' and the way they ``parse their conceptual domain''
  \cite{doi:10.1080/10400410409534554}.
  % Do we really have space for "metaphorical thinking" here? We should be as close to the point as possible. 
\item[Attention can be time-delineated.]  In his design of the
  discovery system {\sf AM}, Doug Lenat assigned ``a small
  interestingness bonus'' \cite[p.~281]{lenat1984and} to each new
  concept the system created.  The bonus decayed rapidly with each new
  task undertaken, but in the mean time made the new concept more
  likely to be used.  This was inspired by a similar but more complex
  ``Focus of Attention'' facility in the blackboard system {\sf
    Hearsay-II} \cite{lesser1977retrospective}.
\item[Competition may be less natural when we can take advantage of
  parallelism.]  Humans have the ability to process complex activities
  in parallel \cite[pp.~40--42]{blackmore2005consciousness}; as we saw
  in Section \ref{sec:ww-model}, social infrastructures can distribute
  features of attention such as awareness and concentration.
  \emph{Joint attention} is one such important social phenomenon.  In
  related computational work \citet{zhuang2017parallel} describe a
  system for parallel attention that recurrently identifies objects in
  images.  It makes use of both image-level attention and text-based
  proposals (the latter directed to image regions), allowing image
  contents to be identified in a dialogue format.  \citet{xu2015show}
  also worked on image captioning, this time using a long short-term memory
  (LSTM) network that independently selected image regions.
  LSTMs are detailed computational models of neurons that are
  capable of learning long-term dependencies.  Xu et al trained their networks
  using models of ``soft'' and ``hard'' attention: the latter did somewhat better for the metrics
  considered.  For a navigation task, \citet{vemula2017social} had success using ``soft
  attention over all humans in the crowd,'' 
  i.e., not simply the people who are nearest.
\end{description}                        

\paragraph{\textbf{\upshape Focus Shift: Heuristics}} 
                                         
\begin{description}                      
\item[Interest can be linked to novelty in order to inspire learning.]
  The {\sf Curious Design Agents} developed by Saunders
  \cite{Saunders2007} evolve artworks in respect to a sophisticated
  measure of interestingness.  These agents cluster artworks together,
  and assess the novelty of new inputs by means of classification
  error.  They then determine a new artwork's interestingness by
  mapping its novelty to an inverse-U-shaped curve, inspired by the
  Wundt curve (cf.~\citet[pp.~17--19]{berlyne2013pleasure}).  This
  model is useful ``in modelling autonomous creative behaviour'' and
  can ``promote life-long learning in novel environments''
  \cite{saunders2010curious}.
%%
  A similar conception of interest is has been applied to ``generate
  art with increased levels of arousal potential in a constrained way
  without activating the aversion system,'' using a variant of
  Generative Adversarial Networks to motivate the creation of
  visual artworks that exhibit ``stylistic ambiguity'' \cite[p.~97]{elgammal2017can}.
%%
  Mathematicians, such as
  \citet{birkhoff1933aesthetic}, have proposed many mathematical
  theories of aesthetics, though philosophers have just as often
  refuted them \cite[p.~4]{hyman2006objective}.  In J\"urgen
  Schmidhuber's work, interestingness is positioned as the ``first
  derivative of subjective beauty'' \cite{schmidhuber2009art}---where
  beauty is understood to mean compressibility.  Here, phenomena that
  maximise prediction error drive curiosity.
  \citet{javaheri2016analysis} apply related measures of
  \emph{information gain} and \emph{Komolgorov complexity} to evaluate
  and drive the evolution of 2D patterns generated by cellular
  automata.
  %I think Birkhoff too specifically talks about aesthetics, but that's not our focus here. Trim?
  %Here's a recent, well-known paper implementing curiosity with deep learning: https://arxiv.org/abs/1606.01868
\item[Interest can be linked to aesthetics in order to capture varied notions of fitness.]
  \citet{dhar2011high} describe an ``aesthetics
  classifier'' that can determine the potential interestingness of
  images in terms of \emph{high level content} and \emph{compositional
    attributes} such as ``people present'', ``opposing colors'', and
  ``follows rule of thirds.''
  \citet{DBLP:journals/corr/abs-1802-10240} applied machine learning
  to a corpus of digial photographs with ratings and reviews, and
  generated new textual descriptions and rating predictions based on
  the crowdsourced descriptors.  {\sf DARCI} (short for Digital ARtist
  Communicating Intention) is a generative program which similarly
  links crowdsourced image descriptions to extracted features
  \cite{norton2013finding}.  It evolves input images using a fitness
  function that optimises for a combination of \emph{appreciation},
  defined in terms of describability, and \emph{interest}, which is,
  as above, an inverse-U-shaped measure of similarity to the input
  image.
\item[Beauty is in the eye of the beholder.] \citet{corneli2016x575}
  follow \citet{waugh1980poetic} in describing \emph{complexity} and
  \emph{coherence} as two key aspects of poetic beauty.  With regard
  to their implemented system that generates linked verse: ``A reader
  may identify some fortuitous resonances [in the system-generated
    poems] but the system itself does not yet recognise these
  features.''  \citet{veale2015game} discusses a related \emph{placebo
    effect} among readers of computer-generated tweets, and the
  broader role that ``an active and receptive mind'' plays in our
  interactions with the world.
\end{description}   

\paragraph{\textbf{\upshape Explanation: Heuristics}} 
                                         
\begin{description}                      
\item[Experiments can have limited scope and still be useful.]  For
  example, \citet{delamaza1994generate} describes two implementations
  of a ``Generate, Test, and Explain'' architecture.  The programs
  involved used decision trees to connect secondary contextual
  information (e.g., macroeconomic indicators) to more elementary
  data-driven predictions (e.g., of stock market behaviour).  This
  work did not, however, ``refine domain theories'': the aim is solely
  to ``connect the `correlations' uncovered by the generate and test
  module to the causal model provided by the domain theory''
  (\emph{ibid.}, p.~50).  This is permissible: a strategic use could in principle be
  found later.
  Kulkarni and Simon's \cite{kulkarni1988processes} {\sf
    KEKADA} is cited as an example of a system that can
  refine the domain theory.
\item[Given a sufficiently rich background, only a small amount of new data is needed.]
  The term \emph{explanation-based learning}
  \cite{ellman1989explanation,cohen1992abductive} denotes a process in
  which an explanation of one event leads to a rule that can be
  applied to similar events in the future.  This typically requires
  significant background knowledge.  Imitation learning, learning from
  demonstrations, learning by example, and one-shot learning are
  related concepts (see, for example, \cite{cypher1993watch}).
  \emph{Case-based reasoning} formulates background knowledge as an
  extensive catalogue of somewhat-similar ``cases'': here explanation
  may play a role in determining how two cases match
  \cite[p.~11]{aamodt1994case}.
\item[Learning is less efficient, but more widely applicable, than knowing.]
  Sussman's \cite{sussman1973computational} {\sf Hacker} was
  able to ``diagnose five classes of mistake and adapt differentially
  to them, generalizing its adaptive insights so that they can be
  applied to many problems of the same structural form''
  \cite{boden1984failure}.  However, ``Hacker is not as good at
  solving blocks world problems as would be a much simpler program
  that just goes about it directly with some good heuristics and a
  minimum of exploration.  Hacker's justification is as an
  epistemological model, not as a real problem solver''
  \cite{levin1975bateson}.  Sussman-style ``critics''---that find,
  fix, and in future avoid planning bugs---have been widely used
  \cite{Sacerdoti:1975:SPB:907010,Young1994,erol1995critical,singh2005alternate,kaelbling2011hierarchical}.
  For example, this approach has been applied to help build video game
  characters that make situationally-appropriate plans in complex,
  changing, environments \cite{hawes2001anytime}.
\item[Communication between agents can transfer causal information.]
  \citet{moore1995participating} and \citet{cawsey1992explanation}
  describe systems that provide explanations to the user in
  interactive dialogues.  Subsequent research compared
  ``mixed-initiative'' and ``non-mixed-initiative'' dialogues using
  computer simulations \cite{ishizaki1999exploring}.  However, there
  are other ways to share and integrate causal information when it has
  formal representations \cite{GeiHofSch16}.  As is well known from
  research on social dilemmas, thin communication protocols constrain
  agents' ability to cooperate; however, sufficiently complex agents
  can learn to cooperate even with limited communication bandwidth
  \cite{leibo2017multi}.
\end{description}                        

\paragraph{\textbf{\upshape Bridge: Heuristics}} 
                                         
\begin{description}                      
\item[Similarity, analogy, and metaphor can be used to retrieve known problems.]
  In case (i), the problem was either known to the system,
  or can be retrieved, e.g., via a search process based on analogy
  between the explanation and a catalogue of existing problems.
  \citet{sowa2003analogical} describe three kinds of analogies that
  apply to graphical knowledge structures: matching types with a
  common supertype, matching isomorphic subgraphs, and identifying
  transformations that can change the subgraphs of one graph into
  another.  They give as an example an analogy between a cat and a
  car, found using WordNet data.  In one real-world example, designers
  at Speedo developed a new material to make swimmers faster by
  incorporating a tiny tooth-like network similar to the denticles
  found in the surface of a shark's skin \cite{ingledew2016how}.  The
  related concept of ``metaphor'' emphasises the role of a
  representational system in expressing an analogy.
  \citet{xiao2016meta4meaning} describe one way in which the relevant
  background that is needed to interpret (or create) metaphors might
  be acquired.  Structure-based retrieval of source domains may give a
  significant boost to the creativity of the analogies that can be
  constructed \cite{Donoghue2002}.
\item[Concept blending may, but does not necessarily, help identify new problems.]
  The bridge might be established by \emph{concept
  blending}, otherwise known as \emph{conceptual integration}
  \cite{fauconnier2008way,fauconnier1998conceptual}.  This approach
  from cognitive science that has recently received increased
  attention in computer science
  \cite{confalonieri2018concepts,besold2015analogy,EPPE2018105}.  The
  method forms new combinations of existing concepts, but
  \citet{fauconnier1998conceptual} advise that ``the most suitable
  analog for conceptual integration is not chemical composition but
  biological evolution.''  Nevertheless, blending can also be
  contrasted with simple models of genetic crossover, where the only
  commonalities that are guaranteed to be preserved are those at at
  the level of individual matching alleles.  In blending, commonalities are
  potentially more abstract.  Finding analogies can be seen as the
  first step in the process of concept blending: for example, given
  the analogy identified by Sowa and Majumdar, multiple different
  cat-car hybrids could be devised, some suitable for nightmares, some
  for children's toys.  Like biological evolution, the blending
  process can involve the outside world in the specification and
  evaluation of blends, and it can do this in ways that combinatorial
  search does not.  \citet{EPPE2018105} have implemented several
  standard-use heuristics that can be used to give basic assessments
  to various blends, but in general blends are evaluated contextually.
  Thagard and Stewart evaluate blends using an abstract simulated
  model of ``cognitive appraisal and physiological perception'' which
  stands for an overall emotional reaction
  \cite[p.~11]{thagard2011aha}. The emotions themselves represent
  circumstances which might be in some sense novel, however they might just as
  well represent a known problem.  Thagard and Stewart
  focus on ``problem solving'' rather than
  problem specification.  They consider an ``aha moment'' to occur when there is a
  good match between the newly-generated combination and the
  background emotions.
  %% ; however, the underlying causes of such an
  %% alignment are not given much attention.
  Returning to the 3M
  example, sticky notes appeared as a particularly satisfactory blend
  between re-stickable bookmarks and the known problem
  of affixing notes to documents.  Indeed, the presence
  of the bookmark prototype allowed a new problem to be
  specified---how to attach a note in a way that would not damage the
  document, and would not require a separate fastener.
  This problem likely would never have been considered if
  the only solutions to hand were the existing conventional
  technologies.  It was an eureka moment for Arthur Fry because he
  had in mind the problem of coming up with a new product: but the
  product itself appeared hand-in-hand with a new problem.  The
  invention of Velcro\textsuperscript{\texttrademark} can similarly be
  reconstructed as such an example, in which the biological problem of
  seed propagation, and its solution of tiny hooks, is blended with
  the domain of fashion to bridge to a new problem: could clothes be
  conveniently fastened using a hook-and-loop mechanism?  We note
  that de Mestral had to expend considerable further effort before he
  was able to answer this question in the affirmative.  This example
  serves to illustrate that a full solution does not always emerge at the same time
  as the problem.
\item[Working across domains can give rise to intriguing ideas.] Text mining has
  been used to generate hypotheses by first identifying \emph{bridging
    terms} between different bodies of literature \cite{swanson1997interactive,weeber2001using,jursic2012,jurvsivc2012cross}.
   These methods may be employed in
 \emph{closed discovery} models where
  the ``two domains of interest \ldots\ are identified by the expert prior to starting
the knowledge discovery process'' or
 \emph{open discovery}
   models where the process works
   ``from a given starting domain towards a yet unknown second domain''
  \cite{jurvsivc2012cross}.
  These correspond, more or less, to the two cases of Definition \ref{def:bridge}.
\item[Experiments can give surprising insights.]  Experiments have
  been designed using both classic expert system methods
  \cite{Lorenzen1992} as well as modern reinforcement learning
  techniques \cite{melnikov2018active}.  However, it is not clear if
  any software systems are yet looking for bridges between
  experiments, which would allow them to make use of the fact that
  interesting things can be learned when a method is applied ``in just
  a slightly different way'' \cite[p.~28]{austin1978chase}, and
  specialisations of this, e.g., ``the unexpected yield from a control
  experiment may be more fruitful than that from the main experiment''
  (p.~32).
\end{description}

\paragraph{\textbf{\upshape Valuation: Heuristics}} 
                                         
\begin{description}                      
\item[Model a sense of taste.] The system's taste is explicitly
  modelled in the case of the artworks evolved by the {\sf Curious
    Design Agents} described by \citet{Saunders2007}.
    %I'd try to avoid the notion of "taste". You could talk about preferences, e.g. as in Michael Cook's Cook, Michael, and Simon Colton. "Generating Code For Expressing Simple Preferences: Moving On From Hardcoding And Randomness." ICCC. 2015.
    %Also, Ventura points at various means to model objective functions in Ventura, Dan. "Mere generation: Essential barometer or dated concept." Proceedings of the Seventh International Conference on Computational Creativity, ICCC. 2016.
\item[Allow the system to use the world.] As an alternative route to
  working with affect, a system might outsource emotional processing
  to a human user, ``recognise'' the user's affective expression
  \cite[p.~15]{picard1995affective}, and use that as the basis of an
  evaluation.
  %That's the whole idea behind interative evolutionary processes, I believe.
  %A recent paper bringing this idea to deep learning: Deep Interactive Evolution https://arxiv.org/pdf/1801.08230.pdf
\item[Allow the system to shape its own goals.] 
Whether or not the user
  is given a role in the evaluation process, systems may be designed
  to shape their own goals
  \cite{kaplan2007intrinsically,singh2010intrinsically}.
\end{description}


This is the main tenet of \emph{predictive
  processing} theories of mental organisation, which for embodied
agents extends to \emph{active inference}, whereby the results of
action inform the update of beliefs \cite{friston2016active}.  ``Avoiding surprises
means that one has to model and anticipate a changing and itinerant
world,'' and importantly, ``average surprise or entropy
$\mathbf{H}(s\,\vert\,m)$ is a function of sensations and the agent
(model) predicting them'' \cite{Friston2012}.  
Note that reducing such measures over the long term
can encourage exploration over shorter time scales.
Applications of these
ideas in AI and robotics have received recent attention.
We allude to
this work in framing our model of serendipity as a process that
itself creates models on the fly, though we aim here at a practical
rather than a formal treatment.

%% Simonton makes use of a somewhat related concept of fitness,
  %% distinguishing between \emph{blind} and \emph{sighted} selection
  %% \cite[p.~159]{simonton2010creative}; however, he separates
  %% out the ``true'' fitness of selected items, which is understood as a
  %% measure of their utility for the agent (which is what what the agent
  %% or may not may be blind to) from the selection filter.
  %% Our definition above makes no assumptions about the
%% actual subsequent utility of selected items.


%% Great strides have been made already in applications of machine
%% discovery, including both autonomous and mixed-initiative software
%% systems \cite{lobo2015inferring,better-drugs}.  Funding initiatives,
%% such as feasibility studies proposed by the UK's EPSRC
%% \cite{AutomatingScienceEPSRC, AutomatingScienceEPSRCNetwork},
%% highlight the potential for a step change in the physical sciences
%% brought about by the application of AI technologies.
%% %% , which
%% %% must deal with small and medium-sized collections of data as well as
%% %% big data \cite{mansinghka2015bayesdb,kolb2017learning}.
%% Writing in \emph{Nature}, Nic Fleming \cite{fleming:18} suggests that
%% ``AI and machine learning will usher in an era of quicker, cheaper and
%% more-effective drug discovery.''  Artificial intelligence is also
%% increasingly applied in other fields, such as business analytics.
%% %%
%% The question as to whether or not serendipity can play a useful role
%% in automated discovery is ultimately an empirical one.  In this paper
%% we supply groundwork for a rigorous investigation of whether and how
%% serendipity potential could improve performance in automated discovery
%% tasks.  We suspect that designing systems with serendipity potential
%% will bring about more and better discoveries, enhance usability, and
%% allow future systems to operate effectively in more complex and
%% dynamically-changing environments---but for now, these are
%% speculations, not claims.


%% %%
%% %%% 1) widely applicable: feasible, applicable and benefitial ...
%% %%% 2) "amenable to" is already quite a stretch: which could provide a starting point for an actual implementation would be more adequate. 
%% %%% 3) As part of the conclusions, please state once again the difference between serendipity in the user/in the system/etc., and mark this as one of our contributions. Generally, you can go along the contributions list in the intro and check if everything's present. 
%% %%% 4) Other than that, I'd keep this short and avoid any additional philosophical discussions. I'd therefore opt to remove everything from "We cannot be sure how AI ... to "of Walpole himself".
%% %%
%% %%%%%%%%%%%%%%%%%%%%%%%%%%%%%%%%%%%%%%%%%%%%%%%%%%%%%%%%%%%%%%%%%%%%%%%%%%%%
%% %!%! As a sanity check can we answer any the questions from the précis? !%!%
%% %%%%%%%%%%%%%%%%%%%%%%%%%%%%%%%%%%%%%%%%%%%%%%%%%%%%%%%%%%%%%%%%%%%%%%%%%%%%
%% %% Where does the initial “vocabulary” come from, and how does it relate to the system’s operating environment?
%% % - ``Vocabulary'' and even grammar might be induced through clustering.
%% Our model takes into account the following features which we have
%% synthesised from existing literature on serendipity and creativity.
%% The model concerns a system that develops a set of perceptions, using
%% an interface to the world.
%% % How does “concept formation” work?
%% % - As we've noticed, *attention* plays a role in what's noticed
%% % - Interest may also play a role in forming the boundaries of concepts
%% % - New ``vocabulary'' would presumably be evaluated in a preliminary way through their *explanatory* power
%% These perceptions are then subject to selective processing
%% that induces further
%% structure.  Certain observed data that have appeared by chance are
%% selected as being of interest.  A predictive model that can explain
%% their contextual relevance is then formed.
%% %In the process, new concepts can be introduced.
%% The resulting explanatory model is then bridged to a new---or, in the
%% pseudoserendipitous case, existing---problem.
%% % \cite{deleuze1994philosophy}:
%% % How do the outcomes of the concept formation process relate to the system’s operating environment?
%% % - Anything worthy of being called a concept is presumably also related to a problem that is being identified
%% The new unexpected solution to this problem is then given a positive
%% valuation by the system, possibly acting as a proxy for a user or
%% other stakeholders.
%% % Can they be evaluated?
%% % - Going beyond explanation to practical evaluation
%% % How do we know when to stop running the process, if ever?
%% % - Noting that there are feedback loops as we gather more data, there's no inherent need to stop the process as long as new things are being explained or new practical applications are being found.  I suppose that there's a sort of thermodynamic limit related to Somolonoff kinds of theories.  Relationships with AIXI would be interesting to develop in future work.


%% %% \begin{quote}
%% %%  ``\emph{That Turing's 1950 essay in the prestigious
%% %% journal \emph{Mind} contains so much that science fiction writers invented
%% %% during the preceding decade and a half testifies to the fruitfulness
%% %% of pulp-magazine science fiction's long-continued conversation as well as to
%% %% the acumen and inventiveness of its finest writers.}''  
%% %% \end{quote}
%% To relate this historical fact to future work that might follow along
%% similar lines, we note that the Turing family motto is ``fortuna
%% audentes juvat''---fortune favours the bold
%% \cite[p.~3]{turing2012alan}.

%% Austin \cite[p.~78]{austin1978chase} had advanced two further
%% variations on Pasteur's famous principle, namely: \emph{Chance favours
%%   those in motion} and \emph{chance favours the individualised
%%   action}.  This latter is almost a rewording of the famous latin
%% phrase 

%%   What we can consider now is a
%% simpler question: what are the next step?
%% %%
%% Looking at the model described here from the perspective of
%% implementation, we can see plenty that needs to be done, particularly
%% at the interfaces between the modules.


%As in the previous examples, the issues discussed here involve a
%certain inversion of roles: it is no longer sufficient to think of
%stakeholders as simply external to the system, rather, they need to be
%modelled and simulated in a potentially quite elaborate way within the
%system.

%% The process
%% ``open[s] new lines of
%% investigation in the growth of knowledge''
%% \cite{swanson1997interactive}.

The framework that we advance was inspired by earlier work of
\citet{pease2013discussion}, who explored ways to encourage
processes of discovery ``in which chance plays a crucial role'' within
computational models of creativity.
%%
\citet{simonton2010creative} had previously drawn relationships
between serendipity, creativity, and evolutionary processes. Of
particular interest for his analysis were generative processes which
are ``independent of the environmental conditions of the occasion of
their occurrence'' \cite{campbell1960blind}, including combinatorial
as well as random processes---a condition understood to imply
teleological ``blindness.''  In a creativity setting, this condition
means that one cannot accurately predict the underlying ``fitness'' of
different ideational variants \cite[p.~159]{simonton2010creative}.
Simonton cites {\sf BACON} \cite{langley1987scientific} as an example
of a blind but nonetheless ``systematic'' search program, based on
``heuristic methods in which a solution is no longer guaranteed''
\cite[p.~169]{simonton2010creative}.
