\section{Integrating the features}\label{sec:system-analysis}
%%% Find a shorter title. E.g. Evaluating our model: discriminating serendipity potential

Here we discuss three historical systems in relation to our model. 

\subsection{{\sf DAYDREAMER}: Serendipity in the System}
%%% Btw, these experiments might be related: https://deepmind.com/blog/agents-imagine-and-plan/
The {\sf DAYDREAMER} system \cite{mueller1990} was intended to provide a computational model of daydreaming.  An agent is guided to use its `imagination' to develop ideas and construct short narratives.  The principle behind {\sf DAYDREAMER} is that a planning agent can operate in a `relaxed' manner to explore possibilities in unusual ways, where the relaxation state is achieved by removing or reducing constraints on the search process that guides the exploration.  {\sf DAYDREAMER}'s exploration is driven by loosely constrained planning mechanisms which are given a pre-determined goal.
%% The agent to construct a plan in a less-constrained manner. 
The generated plan then becomes the basis of a fictional narrative.
Mueller identifies a distinction between {\sf DAYDREAMER} and other comparable systems:
\begin{quote}
\emph{``There are certain needless limitations of most present-day
  artificial intelligence programs which make creativity difficult or
  impossible: They are unable to consider bizarre possibilities and
  they are unable to exploit accidents.''} \cite[p. 14]{mueller1990}
\end{quote}
In other words, the {\sf DAYDREAMER} system was designed to capitalise
on the unusual or accidental {\em non-obvious} options available to
it, which gives intuitive support for Mueller's case that it can act
serendipitously.
Here is a episode that illustrates the first five steps of our model.
\begin{quote}
\textbf{Perception \& Attention}: \emph{``DAYDREAMER receives an alumni directory from the college she attended which happens to contain the number of Carol Burnett.''}\\
\textbf{Focus shift \& Explanation}: \emph{``DAYDREAMER had previously been daydreaming about contacting Harrison Ford in order to ask him out again. Serendipitously, DAYDREAMER realizes that the alumni directory is applicable to the problem of finding out the unlisted telephone number of Harrison Ford.''}\\
  \textbf{Bridge}: \emph{``DAYDREAMER could possibly find out Harrison's telephone number by obtaining a copy of the alumni directory from the college Harrison Ford attended, if any.''} \cite[p. 125]{mueller1990}
\end{quote}

\subsection{{\sf Colloquy of Mobiles}: Serendipity as a Service}

The 1968 Cybernetic Serendipity exhibition at the ICA in London proved popular with the museum-going public, and has been extensively discussed in subsequent literature
\cite{Edmonds1994,macgregor2002cybernetic,usselmann2003dilemma}.  For
our purposes the interesting question is whether, and how, the concept
of ``serendipity'' relates to one of the more famous artworks that was
exhibited there.
%%% I find this a very weak motivation. Not even Pask himself has related his artwork to serendipity, has he? If you want to keep this, I'd (i) stress the artificiality of the system without being classic-computational; (ii) the social aspect leading to serendipitous encounters with people. This is currently still too implicit.
%%
Here we examine the experience of serendipity by gallery attendees.
In an essay that describes the details of his installation, composed
before the exhibition took place, Pask wrote:

\begin{quote}
\textbf{Perception}: ``[T]\emph{he mobiles produce a complex auditory
  and visual effect by dint of their interaction.}\\
%%
\textbf{Attention}: ``\emph{They cannot, of course, interpret these
  light and sound patterns.  But human beings can}''\\
%%
\textbf{Focus shift}: ``\emph{and it seems reasonable that they will
  also aim to achieve patterns that they deem pleasing by interacting
  with the system at a higher level of discourse. I do not know.  But
  I believe it may work out that way.}''
%% \textbf{Bridge}: 
\cite[p.~91]{pask1971comment} 
\end{quote}
This prediction proved true \cite[p.~98]{pask1971comment}.  Subsequent
commentators then pick up the story.
\begin{quote}
\textbf{Explanation}: ``\emph{actions lead to impacts on the environment that lead to sensing and further modification of actions}'' \cite{haque2007architectural}.\\
\textbf{Bridge:} 
``\emph{one can understand the Colloquy as at once an unusual art
  object and a demonstration model of the cybernetic ontology}''
\cite[p.~48]{pickering2007ontological}.\\
\textbf{Valuation:} 
``\emph{The work introduced machinic attributes that even today still sound
very advanced to museum audiences}''
\cite[p.~5]{gemeinboeck2015performance}.
\end{quote}
%%
There was no possibility for serendipity on the system side, since it
missed the bridge and valuation steps, nevertheless there was a
possibility for serendipity in the ``wider'' system that included
human actors.
%% The
%% historical evidence is that there was a certain serendipity to the
%% exhibition as a whole:

%% \begin{quote}
%% ``\emph{Where in London could you take a hippy, a computer programmer,
%%     a ten-year-old schoolboy and guarantee that each would be
%%     perfectly happy for an hour without you having to lift a finger to
%%     entertain them?}'' (2 August 1968, \emph{Evening Standard})
%% \end{quote}


\subsection{{\sf HR} and {\sf LHR}: On the trail of serendipity} \label{sec:pursuit}
In this section we give an account of several episodes in a historical sequence of development of the related discovery systems {\sf HR} and {\sf LHR}. We use our model to discern the serendipity potential (if any) for the system described at that stage of development.

{\bf Perception:} an interface between the system and the world which selectively allows
evidence of events to enter the system.

{\bf Attention:} directs the system’s processing power to the perceived event or certain aspects
thereof.

{\bf Focus shift:} occurs if processing leads to a functional hypothesis related to the event.

{\bf Explanation} uses reasoning to extend the hypothesis about the observed event to other
events in the context within which the system operates.

{\bf Bridge:}  generalises and reworks
the explanation as a solution strategy for a problem in the system's
operating domain.

{\bf Evaluation:} The solution is
\textbf{evaluated} according to some pre-existing objective function.

We define the {\em Serendipity Potential} of a system as a tuple of the 6 possible phases and 10 possible transitions identified in []. 




[HERE]
\subsection{{\sf HR} and {\sf LHR}: OLD}
In this section we give an account of several episodes in a historical sequence of development of the related discovery systems {\sf HR} and {\sf LHR}. We use our model to discern the serendipity potential (if any) for the system described at that stage of development.

\begin{ep}[HR constructs the concept of the central elements in a group]\label{ex:central}
The {\sf HR} system\footnote{Named after mathematicians Hardy (1877 - 1947) and Ramanujan (1887 - 1920).} \citet{colton2002automated} is a machine learning tool which performs automated discovery in a variety of domains. 
One early success was in the domain of abstract algebra, in which {\sf HR} re-discovered the concept of \emph{the central elements of a group} (the set of elements in a group that commute with every element in the group) \cite{colton2002automated}.  This is a core concept in Group Theory that appears in most if not all basic textbooks on the subject.  HR discovered this concept by applying its {\em compose}, {\em exists} and {\em forall} production rules in the following way:  

\begin{align*}
\left.
\begin{array}{c}
{[}a,b,c{]} : a*b=c\\
{[}a,b,c{]} : a*b=c
\end{array}
\right\}\:\:
&\mathbf{compose}\rightarrow [a,b,c] : a*b=c \:\text{\emph{\&\&}}\: b*a=c\\
&\mathbf{exists}\rightarrow [a,b] : \text{exists}\ c\ (a*b=c \:\text{\emph{\&\&}}\: b*a=c)\\
&\mathbf{forall}\rightarrow [a] : \text{all}\ b\ (\text{exists}\ c\ (a*b=c \:\text{\emph{\&\&}}\: b*a=c))
\end{align*}
\end{ep}

In addition to perceiving new conjectures (which it generates), we allow that these are given further attention (by testing them with examples and checking them with external systems), but in the context of our model, that is where things stop.

The {\em Serendipity Potential} of HR$_{ep1}$ is 
$<P, A_{blind}, T_{weak}(P, A)>$

\begin{ep}[HR refutes a boring conjecture in monoid theory]\label{ex:monoid}
Colton 
subsequently enhanced the system so that whenever it finds a
counterexample to a new conjecture, it tests to see whether the
counterexample also breaks any previously unsolved open conjecture.
In this case, the system's ``prepared mind'' takes the form of previous
experiences, background knowledge, a store of unsolved problems, as
well as skills and a current focus.  The new counterexample arises
due partly to factors beyond the system's control, in particular, the
built-in structure of the
domain.
%% dynamic complexity, or some other process independent of the
%% current focus.
This version of the system was tested in three test domains:  group theory (associativity, identity and inverse axioms), monoid theory (associativity, identity) and semigroup theory (associativity). When run in breadth first mode, i.e., with no heuristic search, during sessions with tens of thousands of production rule steps, there were no instances of open problems which were solved in this way. Amending the search strategy to random led to one instance of a newly generated counterexample solving a pre-existing conjecture in monoid theory, none at all in group theory and a handful of times in semi-group theory (there were three times when a new counterexample dispatched an open conjecture, and on one occasion, ten open conjectures were dispatched by one counterexample).  Not only was it rare, the conjectures which were disproved in this way could not be considered interesting: for instance, the monoidal conjecture which was an open problem disproved by a later counterexample was the following:
\begin{align*}
\forall b, c, d &\hspace{.2cm}(((b * c = d \wedge c * b = d \wedge c * d = b \wedge (\exists(e * c = d \wedge e * d = c)))\\
&\hspace{.2cm}\leftrightarrow(b * c = d \wedge (\exists f(b * c = f)) \wedge (\exists g(g * c = b)) \wedge d * b = c \wedge c * d = b)))
\end{align*}

\noindent This conjecture does {\em not} appear in textbooks on Monoid Theory.
\end{ep}

Alongside the attributes of perception and attention that were present
in Episode \ref{ex:central}, we now have evidence of a focus shift,
since every newly-disproved conjecture is considered in the context of
each open conjecture, a potentially long series of (rather limited)
recontextualisations.  However, this evidence must be seen as rather
weak, since the same basic routine happens with each disproved
conjecture, whereas one might expect a true recontextualisation would
be data-specific.
%%
The explanation phase is somewhat better represented (namely via the
demonstration that the counterexample refutes the earlier conjecture)
but the bridge is wholly missing, so the system's findings can only be
regarded as path-dependent rather than serendipitous.
Evidence for the valuation phase is also be
quite weak, if everyday standards of mathematical value were applied.

The {\em Serendipity Potential} of HR$_{ep2-breadth}$ is 
$<P, A, FS,
[T(P, FS)..]>$

The {\em Serendipity Potential} of HR$_{ep2-random}$ is 
$<P, A, FS, E, B, V_{poor}
[T(P, A)..]>$

\begin{ep}[HRL undiscovers the platypus]\label{ex:platypus}
{\sf HRL} was an adaptation of {\sf HR}, developed by \citet{pease07} and 
based on a theory of argumentation that acknowledges the role of conflict and ambiguity in mathematical discovery.  The theory, based on the work of
\citet{lakatos}, can also be used to describe (some) real-world
discoveries in mathematics.  {\sf HRL} is a distributed system,
comprised of ``student'' and ``teacher'' agents, each running a copy
of Colton's {\sf HR}.  The systems have different input knowledge,
measures of interestingness, and different ways of producing concepts.
The overall system is organised into independent work phases, and
discussion phases, in which conjectures, concepts, and counterexamples
are communicated.  Students react to counterexamples using Lakatos's
methods.  One such discussion, developed around a simple theory of
animals, progressed as follows:
\begin{itemize}
\item[\emph{A}:] ``There does not exist an animal which produces milk and lays eggs.''
\item[\emph{B}:] ``The platypus does''
\item[\emph{A}:] {[}Checks new object against current theory. Finds it breaks 11\% of its conjectures{]}\newline ``The platypus is not an animal''
\item[\emph{B}:] {[}Finds that the platypus breaks 31\% of its own conjectures.{]}\newline ``Okay - I'll accept that.''
\end{itemize}
\end{ep}

We will discuss this example together with the following:

\begin{ep}[HRL formulates Goldbach's Conjecture]\label{ex:goldbach}
The same system could also do theory formation in basic number theory.
Here is another dialogue:
\begin{itemize}
\item[\emph{A}:] {[}Knows: numbers 10-20, integer, div, mult{]}\newline ``All even numbers are the sum of two primes.''
\item[\emph{B}:] {[}Knows: numbers 0-10, integer, div, mult{]}\newline ``2 is not the sum of two primes.''
\item[\emph{A}:] {[}Checks new object against current theory. It fits well and doesn't break any further conjectures{]}\newline ``Okay - I'll accept that 2 is a number. Then my conjecture is `All even numbers except 2 are the sum of two primes'.''
\end{itemize}
\end{ep}

Let's consider whether either of examples 3 and 4 meet our criteria.
%%
The system's basic perceptual abilities again rely on its generative
methods, drawing where relevant on external systems.  Agents develop
concepts, conjectures, theorems, and examples that are given
preliminary assessments: the most interesting findings are shared
during the ``discussion phase''.  This is reasonable evidence of
attention.  We saw evidence of a limited focus shift even for {\sf HR}
in Episode \ref{ex:monoid}; somewhat more convincing context- and
data-specific behaviour is illustrated with {\sf HRL}'s agent model,
since each agent is working with its own theory, and can independently
decide what to do with the evidence shared by the other agents.
Causal models are established using external systems, which produce
proofs or check for counterexamples.  This allows the system to
generate simple explanations, for example, Agent \emph{A}'s assertion
that ``The platypus is not an animal.''
%%
In Episode \ref{ex:platypus}, this explanation is simply accepted, but
in Episode \ref{ex:goldbach}, Agent \emph{A} uses \emph{B}'s
counterexample to produce a new conjecture.  The reasoning can be seen
as a bridge to a new problem:
\begin{quote}
``All even numbers are the sum of
two primes.''\:$\rightarrow$\:``2 is not the sum of two
primes.''\:$\rightarrow$\:``All even numbers except 2 are the sum of two
primes.'' 
\end{quote}
This conjecture is considered interesting by the system for roughly
the same reason it is historically interesting: it is succinctly
stated but continues to evade proof.  However, since the conjecture
is already well known (and remains unproved), the simple
fact of its reformulation by {\sf HRL} has little chance of 
receiving the kind of recognition given to original mathematical
discoveries---of the sort that have in fact been made with {\sf HR}
\cite{colton2007computational}.

The {\em Serendipity Potential} of HRL$_{ep3}$ is 
$<P, A, FS, E;
T(P, A), T(A, FS), T(FS, E)>$

The {\em Serendipity Potential} of HRL$_{ep4}$ is 

$<P, A, FS, E, B, V;
T(P, A), T(A, FS), T(FS, E), T(E, B), T(B, V)>$

Eco suggested that if Kant had had the opportunity to observe the platypus, he would have concluded that it is ``a masterpiece of design, a fantastic example of environmental adaptation, which permitted the mammal to survive and flourish in rivers'' \cite[p.~93]{eco2000kant}.  There is quite a difference between this creative line of abductive reasoning and {\sf HRL}'s reductive approach.  However, as we have seen, given a somewhat richer background theory, {\sf HRL} was also capable of exercising something akin to reflective judgement, which it used to reinvent a famous number-theoretic conjecture.
