\section{A model of serendipity potential} \label{sec:our-model}

%\subsection{Definitions of the model's phases and links between them} \label{sec:modelTerms}

We now present short definitions of each of the six phases identified
in the previous section, along with links between them.  As mentioned
earlier, our thinking is informed by the predictive processing
framework, advocated in recent work of \citet{clark2013whatever},
building on the work of \citet{friston2009free} and others.  A central
idea in such theories is that perceived events are only passed forward
to higher cognitive layers if they do not conform with our prior
expectations.  A response to prediction error can either motivate
action, which ameliorates the error by bringing the world into
alignment with our predictions, or else motivate adaptation of the
predictive models.
%%
%% %% Maybe also cite:
%% Smart, P. R. (forthcoming) Predicting Me: The Route to Digital
%% Immortality? In R. W. Clowes, K. Gärtner & I. Hipólito (Eds.), The
%% Mind-Technology Problem: Investigating Minds, Selves and 21st Century
%% Artifacts. Springer, Berlin, Germany.

This perspective highlights the fact that, going beyond Pasteur's
famous idiom, chance not only \emph{favours}, but also \emph{shapes}
the prepared mind.  For example, ``neural networks learn to associate
(combine) patterns without being explicitly programmed in respect of
those patterns'' \cite[p.~137]{boden}.  Whereas predictive processing
accounts are ``currently computationally fleshed out predominantly at
the basic perception and motor control level'' \cite{KWISTHOUT201784},
here, we will consider a range of higher cognitive functions.  Current
research in cognitive science observes that the functional
architecture of the human brain ``allow[s] the confluence of
information related to perception, cognition, emotion, motivation, and
action'' \cite[p.~357]{Pessoa2017}.  Our model is, similarly, an
abstraction of a functional architecture.



%% We will have to deal with the topic raised by the AIJ editor
%% -- Are there different places to begin in the model?  Do
%% we always start with perception?

The six phases in our model exhibit clear logical dependencies.%% , which
%% we will illustrate using notation from constructive type theory---a
%% notational strategy inspired by work in computational semantics
%% \cite{Chatzikyriakidis2018}.  In particular, it has been applied to
%% model linguistic phenomena of anaphora and presupposition
%% \cite{10.1007/978-3-662-43742-1_2,krahmer1999presupposition,piwek2000presuppositions}.
We assume that earlier steps can be returned to from later ones, and
also that anticipation of later steps plays a role in the way the
process runs.  Anticipation is widely recognised to play a role in
pseudoserendipitous discovery: for example, improved ways to process
rubber and safe antibiotics were pursued in broad outline long before
the specific details became clear \cite{fleming,goodyear1855gum}.
Similarly, Pasteur's research has been retroactively described as
``use-inspired'' \cite{stokes1997pasteur}.  Anticipation plays an
subtle role in serendipity: the unexpected event does not emerge
relative to a blank slate, but to existing top-down predictions,
including our anticipation of ourselves, our customs, and our
\emph{Umwelt} \cite{dennett_2013}.

The model remains relatively open interpretation, which we regard as a
strength.  For example, some of the key stages involve synthesis,
e.g., of a new hypothesis, a new method, or a new application, and we
do not say here how such syntheses would be implemented.  This allows
anyone interested in exploring the model in practical settings the
opportunity to make an argument for
or against the presence of the phases and transitions between them
within their system.  In Section \ref{sec:system-analysis}, we will
give examples of historical systems that match the outlines of the
model, and we comment briefly on the prospects of more creating robust
demonstration systems using contemporary technologies in Section
\ref{sec:discussion}.

% \subsection{Six phases: {perception}, {attention}, {focus shift}, {explanation}, {bridge}, and {evaluation}}

%% In the rest of the section we will take the following story as a running example.

%% \begin{quote}
%% % Some 40 years ago 
%% ``\emph{Symcha Blass, an Israeli engineer, observed that a large tree near a leaking faucet exhibited a more vigorous growth than the other trees in the area, which were not reached by the water from the faucet.  Blass knew that conventional methods of irrigation waste much of the water that is applied to the crop, and so the example of the leaking faucet led him to the concept of an irrigation system that would apply water in small amounts, literally drop by drop. Eventually he devised and patented a low-pressure system for delivering small amounts of water to the roots of plants at frequent intervals.}'' \cite{shoji1977drip} 
%% \end{quote}

% http://anticipationconference.org/call-for-participation/

%% In this respect we note that Friston's model of predictive processing
%% makes more specific and detailed assumptions about structure and
%% interconnection than we will adhere to here, namely that ``error-units
%% receive messages from the states in the same level and the level
%% above; whereas state-units are driven by error-units in the same level
%% and the level below'' \cite[p.~297]{friston2009free}.  In simpler
%% biologically-inspired terms, ``the brain generates top-down
%% predictions that are matched bottom-up with sensory information''
%% \cite[p.~2]{Bruineberg2018}.  The mismatch between sense data and
%% existing ubiquitously generative models is how prediction errors are
%% said to arise, which the system then strives to correct.  Our model
%% also has integral generative aspects, but they differ at the different
%% phases.

\begin{defn}%% [\textbf{Perception: An interface to the world}]
\label{def:perception}
\hypertarget{def:perception}{}\textbf{Perception} is an interface
between the system and the world which selectively allows evidence of events to
enter the system.
%% , whenever the event's occurrence aligns with the
%% system's sensors.
\end{defn}

\setlist[description]{font=\normalfont\itshape,itemsep=-2pt}

\begin{defn}\label{def:attention}
\hypertarget{def:attention}{}\textbf{Attention} directs the system's  processing power to a perceived event or certain aspects thereof.
\end{defn}

\begin{defn}\label{def:interest}
  \hypertarget{def:interest}{} A \textbf{focus shift} occurs if processing leads to a functional hypothesis related to the event.
\end{defn}

\begin{defn}\label{def:explanation}
  \hypertarget{def:explanation}{} An \textbf{explanation} uses
  reasoning to extend the hypothesis about the observed event to other
  events in the context within which the system operates.
\end{defn}

\begin{defn}\label{def:bridge}
\hypertarget{def:bridge}{}A \textbf{bridge} generalises and reworks
the explanation as a solution strategy for a problem in the system's
operating domain.
\end{defn}

\begin{defn}\label{def:result}
\hypertarget{def:result}The solution is
\textbf{evaluated} according to some pre-existing objective function.
%% This may be system-intrinsic, or specified by the system's user or a third party.
\end{defn}

\noindent Links between phases can be described with the logic of
predictive processing: errors are propagated bottom-up
($\mathrm{Phase}\ n\rightarrow\mathrm{Phase}\ n+1$) relative to predictions which are made top-down
($\mathrm{Phase}\ n+1\rightarrow\mathrm{Phase}\ n$).


%% Popper Objective Knowledge 1972 - cited in The effect of Input Knowledge on Creativity.
\medskip
\noindent
\begin{tabular}{lp{.4\textwidth}lp{.4\textwidth}}
\multicolumn{2}{l}{\textbf{Top-down Predictions}}   & \multicolumn{2}{l}{\textbf{Bottom-up Errors}}\\[.1cm]
 2$\rightarrow$1 & Attention tells the system what \emph{perceptions} to expect.
 &1$\rightarrow$2 & A surprising perception elicits \emph{attention}.\\
 3$\rightarrow$2 & Existing functional hypotheses tell the system how and where to direct its \emph{attention}.
 &2$\rightarrow$3 & When attention is directed to something in a surprising way, e.g., by recontextualisation, this can provoke a \emph{focus shift}.\\
 4$\rightarrow$3 & Existing explanations tells the system how to perform \emph{focus shifts}, i.e., what methods to use and what hypotheses are likely to be supported.
 &3$\rightarrow$4 & An overly specific or poorly supported hypothesis stimulates effort towards a broader \emph{explanation}. \\
 5$\rightarrow$4 & Existing bridging strategies tell the system how to form applicable \emph{explanations}.
 & 4$\rightarrow$5 & An explanation that doesn't solve an interesting problem, or a problem missing an explanation, can provoke the search for a \emph{bridge}.\\
 6$\rightarrow$5 & The evaluation a solution guides \emph{bridge} creation, e.g., by modifying the solution or the problem.
 &5$\rightarrow$6 & A surprising solution elicits \emph{evaluation}.\\
\end{tabular}
\medskip

%% \begin{itemize}
%% \item[1$\rightarrow$2] A surprising perception elicits \emph{attention}.
%% \item[2$\rightarrow$1] Attention tells the system what perceptions to expect.
%% \item[2$\rightarrow$3] When attention is directed to something in a surprising way, this can provoke a \emph{focus shift}.
%% \item[3$\rightarrow$2] Existing functional hypotheses tell the system how to direct its attention.
%% \item[3$\rightarrow$4] An overly specific or poorly supported hypothesis stimulates effort towards a broader \emph{explanation}.
%% \item[4$\rightarrow$3] Explanation tells the system what hypotheses are likely to be supported.
%% \item[4$\rightarrow$5] Explanations that don't solve an interesting problem provoke the search for a \emph{bridge}.
%% \item[5$\rightarrow$4] Existing bridging strategies tell the system how an explanation is likely to be applicable.
%% \item[5$\rightarrow$6] A novel solution elicits \emph{evaluation}.
%% \item[6$\rightarrow$5] An unsatisfying solution provokes the search for a new solution or a new problem to work on.
%% \end{itemize}

% \subsection{Summary}

Certainly it is possible to speak about serendipity, for example as
accidents plus sagacity, without such an elaborate framework.
However, the model provides guidelines for implementors and analysts
that simpler definitions would not, and more precision about the
phases and their relationships than other recent models.  Illustrative
examples of each of the phases can be readily found in the literature:
\begin{itemize}
\item Spencer Silver discovered a new high-tack, low-peel, adhesive
  that was not at all useful for his original task of gluing
  aeroplanes together \cite{tce-postits}.
\item Comparing different reactions to the same basic observation in
  two different labs, \citet[p.~131]{floppyearedrabbits1958barber}
  note that ``the methods and assumptions on which a systematic
  investigation is built selectively focus the researcher's
  attention.''
%%
\item De Mestral perfomed a focus shift in a literal fashion, when he
  examined burrs under a microscope and noticed how their hooks work
  as a fastener in connection with certain fibres.
%%
 %%
\item Nearly 60 years before Alexander Fleming, Eugene Semmer both
  discovered and also cursorily explained the curious effects of
  \emph{penicillium notatum}---but he failed to grasp the vital
  problem his discovery could have solved
  \cite[p.~75]{cropley2013creativity}.
%%
\item
  Heavier-than-air flight had been imagined long before it was a
  reality, but solving practical problems in aviation depended on
  dramatic contemporaneous progress in human interaction
  \cite[p.~292]{spenser2008airplane} that could not have been planned
  in advance.
%%
\item All of these examples, once all of the steps were eventually in
  place, led to positive evaluations.
\end{itemize}

%% We then defined each component with reference
%% to theoretical literature.  Table \ref{tab:model-summary-table}
%% summarises the model that results from this analysis.

%% %\begin{landscape}
\tikzset{
    boxComponent/.style={
    rectangle,
    draw=black, very thick,
    minimum height=2em,
    minimum width=10em,
    inner sep=2pt,
    text centered,
    },
    boxText/.style={
    rectangle,
    draw=black,
    minimum height=14em,
    minimum width=10em,
    inner sep=0em,
    outer sep=0em,
    },
}

%\thispagestyle{empty}

\begin{table}[h]
\captionsetup{width=1\linewidth}
\vspace{1cm}\hspace{0cm}
\resizebox{1\textwidth}{!}{
%% Wow, I didn't know how to use ``node distance'' for auto-layout -JC.
%% Nice!
\begin{tikzpicture}[->,>=stealth',node distance=4cm]

 % Position of QUERY
 % Use previously defined 'state' as layout (see above)
 % use tabular for content to get columns/rows
 % parbox to limit width of the listing
 \node[boxComponent] (Perception) {\hyperlink{def:perception}{\textbf{Perception}}};
 \node[boxComponent, right of = Perception] (Attention) {\hyperlink{def:attention}{\textbf{Attention}}};
 \node[boxComponent, right of = Attention] (Interest) {\hyperlink{def:interest}{\textbf{Focus Shift}}};
 \node[boxComponent, right of = Interest] (Explanation) {\hyperlink{def:interest}{\textbf{Explanation}}};
 \node[boxComponent, right of = Explanation] (Bridge) {\hyperlink{def:bridge}{\textbf{Bridge}}};
 \node[boxComponent, right of = Bridge] (Valuation) {\hyperlink{def:result}{\textbf{Valuation}}};

% \node[above left= -1.5mm and 9.2mm of Perception.center] {\large{\pmglyph{\HDiv}}};
% \node[above right= -1.5mm and 9.2mm of Perception.center] {\reflectbox{\large{\pmglyph{\HDiv}}}};

 \coordinate[above= 9mm of Perception  ] (PerceptionPrime) ;
 \coordinate[above= 9mm of Attention    ] (AttentionPrime) ;
 \coordinate[above= 9mm of Interest     ] (InterestPrime) ;
 \coordinate[above= 9mm of Explanation  ] (ExplanationPrime) ;
 \coordinate[above= 9mm of Bridge       ] (BridgePrime) ;
 \coordinate[above= 9mm of Valuation    ] (ValuationPrime) ;

\begin{scope}[thick,decoration={
    markings,
    mark=at position 0.52 with {\arrow{latex}}}
    ]
\draw[postaction={decorate},-] (AttentionPrime) -- (PerceptionPrime);                   % L
\draw[postaction={decorate},-] (InterestPrime) -- (AttentionPrime);                     % L
\draw[postaction={decorate},-] (ExplanationPrime) -- (InterestPrime);                   % L
\draw[postaction={decorate},-] (BridgePrime) -- (ExplanationPrime);                     % L
\draw[postaction={decorate},-] (ValuationPrime) -- (BridgePrime);                      % L
\end{scope}
%%
\begin{scope}[thick,decoration={
    markings,
    mark=at position 0.65 with {\arrow{latex}}}
    ]
 \path
 (PerceptionPrime) edge[postaction={decorate},-] (Perception)                                   % D
 (Attention) edge[postaction={decorate},-,transform canvas={xshift=-9mm}] (AttentionPrime)     % U
 (AttentionPrime) edge[postaction={decorate},-,transform canvas={xshift=9mm}] (Attention)      % D
 (Interest) edge[postaction={decorate},-,transform canvas={xshift=-9mm}] (InterestPrime)       % U
 (InterestPrime) edge[postaction={decorate},-,transform canvas={xshift=9mm}] (Interest)        % D
 (Explanation) edge[postaction={decorate},-,transform canvas={xshift=-9mm}] (ExplanationPrime) % U
 (ExplanationPrime) edge[postaction={decorate},-,transform canvas={xshift=9mm}] (Explanation)  % D
 (Bridge) edge[postaction={decorate},-,transform canvas={xshift=-9mm}] (BridgePrime)           % U
 (BridgePrime) edge[postaction={decorate},-,transform canvas={xshift=9mm}] (Bridge)            % D
 (Valuation) edge[postaction={decorate},-] (ValuationPrime);                                     % U
\end{scope}


%\draw[decorate,decoration={triangles,segment length=12mm}] (ValuationPrime) -- (PerceptionPrime);

%Boxes for definitions
 \node[boxText, below of = Perception,yshift=2em] (PerceptionDef)
 {\small
 \begin{minipage}[t][14em]{10em}
 \begin{flushleft}
  \textbf{Interface to world}\\[.2cm]
  \textbullet\ \citeauthor{russel2003artificial} - different kinds of environments\\
  \textbullet\ \citeauthor{hume1904enquiry}/\citeauthor{peirce1931necessity} - chance is negative/fundamental\\
  \textbullet\ \citeauthor{hoffman2015interface} - adaptivity of not seeing reality as it is\\
  \textbullet\ \citeauthor{friston2009free} - we sense what we can predict
 \end{flushleft}
 \end{minipage}
 };

 \node[boxText, below of = Attention,yshift=2em] (AttentionDef){\small
 \begin{minipage}[t][14em]{10em}
  \begin{flushleft}
  \textbf{Directed processing power}\\[.2cm]
  \textbullet\ \citeauthor{clark2013whatever} - prediction error\\
  \textbullet\ \citeauthor{singh2005architecture} - layered architecture\\
  \textbullet\ \citeauthor{bateson-logical-categories} - changing behaviour\\
  \textbullet\ \citeauthor{rowley2007wisdom} - meaning making
  \end{flushleft}
 \end{minipage}
 };

 \node[boxText, below of = Interest,yshift=2em] (InterestDef) {\small
 \begin{minipage}[t][14em]{10em}
\begin{flushleft}
  \textbf{Hypothesis generation}\\[.2cm]
  \textbullet\ Wundt curve\\
  \textbullet\ \citeauthor{berlyne1954theory} - epistemic and perceptual curiosity\\
  \textbullet\ \citeauthor{logan1994modelling} - belief revision in information seeking\\
  \textbullet\ \citeauthor{patalano1993predictive} - predictive encodings
\end{flushleft}
 \end{minipage}
 };

 \node[boxText, below of = Explanation,yshift=2em] (ExplanationDef){\small
 \begin{minipage}[t][14em]{10em}
\begin{flushleft}
  \textbf{A predictive model}\\[.2cm]
  \textbullet\ \citeauthor{lawson1998metaphysics} - principles and causes\\
  \textbullet\ \citeauthor{pease2011computational} - framing\\
  \textbullet\ \citeauthor{bateson-logical-categories} - change of pattern\\
\end{flushleft}
 \end{minipage}
 };

 \node[boxText, below of = Bridge,yshift=2em] (BridgeDef) {\small
 \begin{minipage}[t][14em]{10em}
\begin{flushleft}
  \textbf{Identifying or positing a problem}\\[.2cm]
  \textbullet\ \citeauthor{bergson1946creative} - creativity of problem statement\\
%  \textbullet\ \citeauthor{thagard2011aha} - ``aha moment''\\
  \textbullet\ \citeauthor{boden1998creativity} - transform the space\\
  \textbullet\ \citeauthor{pease2011computational} - new aesthetic
\end{flushleft}
 \end{minipage}
 };

 \node[boxText, below of = Valuation,yshift=2em] (ValuationDef) {\small
 \begin{minipage}[t][14em]{10em}
\begin{flushleft}
  \textbf{Evaluation of solution}\\[.2cm]
  \textbullet\ \citeauthor{bergson1991matter} - affection\\
  \textbullet\ \citeauthor{campbell2005serendipity} - rational exploitation\\
\end{flushleft}
 \end{minipage}
 };


%%  % Boxes for implementation details
%%  \node[boxText, below of = PerceptionDef, yshift = -4em] ()
%%  {\small
%%  \begin{minipage}[t][14em]{10em}
%% \begin{flushleft}
%%   \textbf{HCI, automated feature finding, emergence of grid cells} \\[.2cm]
%%   \textbullet\ \citeauthor{turk2000perceptive}\\
%%   \textbullet\ \citeauthor{jacob2015viewpoints}\\
%%   \textbullet\ \citeauthor{stopher2017technology}\\
%%   \textbullet\ DeepDream\\
%%   \textbullet\ \citeauthor{Banino2018}
%% \end{flushleft}
%%  \end{minipage}
%%  };

%%  \node[boxText, below of = AttentionDef, yshift = -4em] ()
%%  {\small
%%  \begin{minipage}[t][14em]{10em}
%% \begin{flushleft}
%%   \textbf{Visual attention, competition for resources, temporal bonus, soft attention}\\[.2cm]
%%   \textbullet\ \citeauthor{sun2003object}\\
%%   \textbullet\ \citeauthor{tsotsos1995modeling}\\
%%   \textbullet\ \citeauthor{baars1997theatre}\\
%%   \textbullet\ \citeauthor{lesser1977retrospective}\\
%%   \textbullet\ \citeauthor{vemula2017social}
%% \end{flushleft}
%%  \end{minipage}
%%  };

%%  \node[boxText, below of = InterestDef, yshift = -4em] ()
%%  {\small
%%  \begin{minipage}[t][14em]{10em}
%% \begin{flushleft}
%%   \textbf{Autonomous creative behaviour, aesthetics classifier, compression, information gain}\\[.2cm]
%%   \textbullet\ \citeauthor{Saunders2007}\\
%%   \textbullet\ \citeauthor{dhar2011high}\\
%%   \textbullet\ \citeauthor{schmidhuber2009art}\\
%%   \textbullet\ \citeauthor{javaheri2016analysis}
%% \end{flushleft}
%%  \end{minipage}
%%  };

%%  \node[boxText, below of = ExplanationDef, yshift = -4em] ()
%%  {\small
%%  \begin{minipage}[t][14em]{10em}
%% \begin{flushleft}
%%   \textbf{Explanation-based learning, epistemic modelling, critics, dialogue, integration of causal models}\\[.2cm]
%%   \textbullet\ \citeauthor{ellman1989explanation}\\
%% %  \textbullet\ \citeauthor{cohen1992abductive}\\
%%   \textbullet\ \citeauthor{delamaza1994generate}\\
%%   \textbullet\ \citeauthor{sussman1973computational}\\
%% %  \textbullet\ \citeauthor{Sacerdoti:1975:SPB:907010}\\
%%   \textbullet\ \citeauthor{singh2005alternate}\\
%%   \textbullet\ \citeauthor{moore1995participating}\\
%%   \textbullet\ \citeauthor{GeiHofSch16}
%% \end{flushleft}
%%  \end{minipage}
%%  };

%%  \node[boxText, below of = BridgeDef, yshift = -4em] ()
%%  {\small
%%  \begin{minipage}[t][14em]{10em}
%% \begin{flushleft}
%%   \textbf{Analogy, metaphor, concept blending, bridging terms}\\[.2cm]
%%   \textbullet\ \citeauthor{sowa2003analogical}\\
%%   \textbullet\ \citeauthor{xiao2016meta4meaning}\\
%%   \textbullet\ \citeauthor{confalonieri2018concepts}\\
%%   \textbullet\ \citeauthor{EPPE2018105}\\
%%   \textbullet\ \citeauthor{swanson1997interactive}\\
%%   \textbullet\ \citeauthor{jursic2012}
%% \end{flushleft}
%%  \end{minipage}
%%  };

%%  \node[boxText, below of = ValuationDef, yshift = -4em] ()
%%  {\small
%%  \begin{minipage}[t][14em]{10em}
%% \begin{flushleft}
%%   \textbf{Modelling taste, affect, intrinsic motivation}\\[.2cm]
%%   \textbullet\ \citeauthor{Saunders2007}\\
%%   \textbullet\ \citeauthor{picard1995affective}\\
%%   \textbullet\ \citeauthor{kaplan2007intrinsically}\\
%%   \textbullet\ \citeauthor{singh2010intrinsically}
%% \end{flushleft}
%%  \end{minipage}
%%  };

% \node [rotate=90, left of = Perception, xshift=2.5em,yshift=6em] {\textbf{Definitions}};

% \node [rotate=90, left of = Perception, xshift=-13em,yshift=6em] {\textbf{Implementations}};

%% \node[ below of = PerceptionDef,yshift=3em] (PreceptionNote) {$\Gamma(t,\ldots)$};
%% \node[ below of = AttentionDef,yshift=3em] (AttentionNote) {$\Gamma_1\subset\Gamma$};
%% \node[ below of = InterestDef,yshift=3em] (InterestNote) {$\Gamma_1\vdash f\rightarrow g\rightarrow\ldots$};
%% \node[ below of = ExplanationDef,yshift=3em] (ExplanationNote) {$\Gamma_1 \vdash \phi : \Pi x:X \mathbin{.} Y(x)$};
%% \node[ below of = BridgeDef,yshift=3em] (BridgeNote) {$\beth : \Gamma_1 \rightarrow \Gamma_2, \phi \mapsto \psi$};
%% \node[ below of = ValuationDef,yshift=3em] (ValuationNote) {$\psi(t,\ldots)$};

\begin{scope}[thick]
  % draw the connecting arrows
 \path
 (Perception) edge (Attention)
 (Attention) edge (Interest)
 (Interest) edge (Explanation)
 (Explanation) edge (Bridge)
 (Bridge) edge (Valuation)
;
\end{scope}
\end{tikzpicture}}

\begin{tabular}{p{.1\textwidth}p{.\textwidth}}
\caption{Our model for systems with serendipity potential. The flowchart at top provides a visual key, showing that previous phases can be returned to at any point.  The body of the table summarises Definitions \ref{def:perception}--\ref{def:result}.\label{tab:model-summary-table}}
\end{tabular}
\end{table}
%\end{landscape}

%Is this table stil up-to-date?
