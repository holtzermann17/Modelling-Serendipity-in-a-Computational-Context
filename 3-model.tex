\section{A computational model and evaluation framework for assessing the potential for serendipity in computational systems} \label{sec:our-model}

%\subsection{Definitions of the model's phases and links between them} \label{sec:modelTerms}

We now present short definitions of each of the six phases identified
in the previous section, along with links between them.  As mentioned
earlier, our thinking is informed by the predictive processing
framework, advocated in recent work of \citet{clark2013whatever},
building on the work of \citet{friston2009free} and others.  A central
idea in such theories is that perceived events are only passed forward
to higher cognitive layers if they do not conform with our prior
expectations.  A response to prediction error can either motivate
action, which ameliorates the error by bringing the world into
alignment with our predictions, or else motivate adaptation of the
predictive models.
%%
%% %% Maybe also cite:
%% Smart, P. R. (forthcoming) Predicting Me: The Route to Digital
%% Immortality? In R. W. Clowes, K. Gärtner & I. Hipólito (Eds.), The
%% Mind-Technology Problem: Investigating Minds, Selves and 21st Century
%% Artifacts. Springer, Berlin, Germany.

This perspective highlights the fact that, going beyond Pasteur's
famous idiom, chance not only \emph{favours}, but also \emph{shapes}
the prepared mind.  For example, ``neural networks learn to associate
(combine) patterns without being explicitly programmed in respect of
those patterns'' \cite[p.~137]{boden}.  Whereas predictive processing
accounts are ``currently computationally fleshed out predominantly at
the basic perception and motor control level'' \cite{KWISTHOUT201784},
here, we will consider a range of higher cognitive functions.  Current
research in cognitive science observes that the functional
architecture of the human brain ``allow[s] the confluence of
information related to perception, cognition, emotion, motivation, and
action'' \cite[p.~357]{Pessoa2017}.  Our model is, similarly, an
abstraction of a functional architecture.



%% We will have to deal with the topic raised by the AIJ editor
%% -- Are there different places to begin in the model?  Do
%% we always start with perception?

The six phases in our model exhibit clear logical dependencies.%% , which
%% we will illustrate using notation from constructive type theory---a
%% notational strategy inspired by work in computational semantics
%% \cite{Chatzikyriakidis2018}.  In particular, it has been applied to
%% model linguistic phenomena of anaphora and presupposition
%% \cite{10.1007/978-3-662-43742-1_2,krahmer1999presupposition,piwek2000presuppositions}.
We assume that earlier steps can be returned to from later ones, and
also that anticipation of later steps plays a role in the way the
process runs.  Anticipation is widely recognised to play a role in
pseudoserendipitous discovery: for example, improved ways to process
rubber and safe antibiotics were pursued in broad outline long before
the specific details became clear \cite{fleming,goodyear1855gum}.
Similarly, Pasteur's research has been retroactively described as
``use-inspired'' \cite{stokes1997pasteur}.  Anticipation plays an
subtle role in serendipity: the unexpected event does not emerge
relative to an blank slate, but to existing top-down predictions,
including our anticipation of ourselves, our customs, and our
\emph{Umwelt} \cite{dennett_2013}.

Some of the key stages involve synthesis, e.g., of a new hypothesis, a
new method, or a new application, and we only give the briefest
indication here of how such syntheses may be implemented.  In Section
\ref{sec:system-analysis}, we give examples of historical systems that
match the outlines of the model and we comment on the prospects of
more robust implementations using use contemporary synthesis
technologies in Section \ref{sec:discussion}.

% \subsection{Six phases: {perception}, {attention}, {focus shift}, {explanation}, {bridge}, and {evaluation}}

%% In the rest of the section we will take the following story as a running example.

%% \begin{quote}
%% % Some 40 years ago 
%% ``\emph{Symcha Blass, an Israeli engineer, observed that a large tree near a leaking faucet exhibited a more vigorous growth than the other trees in the area, which were not reached by the water from the faucet.  Blass knew that conventional methods of irrigation waste much of the water that is applied to the crop, and so the example of the leaking faucet led him to the concept of an irrigation system that would apply water in small amounts, literally drop by drop. Eventually he devised and patented a low-pressure system for delivering small amounts of water to the roots of plants at frequent intervals.}'' \cite{shoji1977drip} 
%% \end{quote}

% http://anticipationconference.org/call-for-participation/

%% In this respect we note that Friston's model of predictive processing
%% makes more specific and detailed assumptions about structure and
%% interconnection than we will adhere to here, namely that ``error-units
%% receive messages from the states in the same level and the level
%% above; whereas state-units are driven by error-units in the same level
%% and the level below'' \cite[p.~297]{friston2009free}.  In simpler
%% biologically-inspired terms, ``the brain generates top-down
%% predictions that are matched bottom-up with sensory information''
%% \cite[p.~2]{Bruineberg2018}.  The mismatch between sense data and
%% existing ubiquitously generative models is how prediction errors are
%% said to arise, which the system then strives to correct.  Our model
%% also has integral generative aspects, but they differ at the different
%% phases.

\begin{defn}%% [\textbf{Perception: An interface to the world}]
\label{def:perception}
\hypertarget{def:perception}{}\textbf{Perception} is an interface
between the system and the world which selectively allows evidence of events to
enter the system.
%% , whenever the event's occurrence aligns with the
%% system's sensors.
\end{defn}

\setlist[description]{font=\normalfont\itshape,itemsep=-2pt}

%% \paragraph{\textbf{\upshape Foundations}}

%% \begin{description}
%% \item[System-environment relationships differ widely, and develop
%%   differently.]  The environment may be more or less observable;
%%   events may appear to be more deterministic or more stochastic in
%%   nature \cite[pp.~42--44]{russel2003artificial}. The system may be
%%   able to self-program using the environment, possibly via some form
%%   of interaction with other systems
%%   \cite[esp.~p.~234]{clark1998being}.  The system's perceptual
%%   features and limitations can vary with time, location, the state of
%%   development of the system, and other factors.  We do make any
%%   specific assumptions about how the system must behave in order to
%%   perceive.
%% \item[Chance can play various roles in shaping perception.] For
%%   \citet[p.~99]{hume1904enquiry} \emph{chance} denotes the absence of
%%   an explanation; for \citet{peirce1931necessity} it is one of several
%%   fundamental aspects of reality; for Bergson
%%   \cite[p.~234]{bergson1983creative}, it ``objectifies the state of
%%   mind'' of one whose expectations are confounded.  By any
%%   understanding:
%%   % 1) I don't understand how these references to chance contribute to the understanding of perception.
%%   % 2) The last sentence appears incomplete.
%% \item[The system has limited control.] The world is not entirely under
%%   the control of the system:
%% %  \cite{burroughs1978limits}
%%   furthermore,
%%   perceptions necessarily constitute an incomplete picture of reality
%%   \cite{hoffman2015interface}.
%% %% Perception is often theorised within a \emph{perception-action
%% %% cycle} \cite{cutsuridis2011perception}, but we do not make strong
%% %% assumptions about the way systems act in the world.
%% Taking a view grounded in predictive processing, \citet[pp.~2, 17--18]{10.3389/frobt.2018.00021} emphasise 
%% the epistemic and existential salience of generative models
%% (and continuous action/perception loops, including proprio- and intero-ception)
%% for both organisms and future robots.  The basic view is
%% that 
%% ``we harvest sensory signals that we can predict'' \cite{friston2009free},
%% though such predictions are fallible.  
%% % further details fall outside the scope of our effort.
%% \end{description}

%% \paragraph{\textbf{\upshape Notation}}

%% $\Gamma$ is a \emph{context}.  We understand its contents to depend on
%% time and other variables, for instance, location, sensors used, and so
%% on.  Perception includes objects of various types, their attributes,
%% and relationships between them.

%% \paragraph{\textbf{\upshape Example}}

%% One of the examples collected by \citet{van1994anatomy} is the story
%% of Simcha Blass (cf.~\cite{shoji1977drip}).
%% The story begins at some time $t$, with some configuration of sensors
%% $s$, in which
%% %%%%%%%%%%%%%%%%%%%%%%%%%%%%%%%%%%%%%%%%%%%%%%%%%%%%%%%%%%%%%%%%%%%%%%%%%%%%%%%%%%%%%%%%%%%%%%%%%%%%
%% $$\Gamma(t,s) \equiv\: a_1, a_2, a_3, \ldots, a_n : A,\enspace
%% f:A\rightarrow \mathbb{R},\enspace
%% m:\mathbb{R},\enspace
%% >:\mathbb{R}\rightarrow\mathbb{R}\rightarrow \mathbb{B}, \ldots$$
%% %%%%%%%%%%%%%%%%%%%%%%%%%%%%%%%%%%%%%%%%%%%%%%%%%%%%%%%%%%%%%%%%%%%%%%%%%%%%%%%%%%%%%%%%%%%%%%%%%%%%
%% Here, $A$ denotes `tree', and $a_1,\ldots,a_n$ are specific trees.
%% $f$ measures the height of each tree,
%% $m$ is the expected height of trees based on previous observations,
%% and  $>$ is used to compare their heights.
%% In our examples, one of the trees was surprisingly tall, in other words, $f(a_1)>m$
%% was observed.


%% \begin{center}
%% \asterism
%% \end{center}

\begin{defn}\label{def:attention}
\hypertarget{def:attention}{}\textbf{Attention} directs the system's  processing power to the perceived event or certain aspects thereof.
\end{defn}



%% \paragraph{\textbf{\upshape Foundations}}

%% \begin{description}
%% \item[Adaptive attention is related to surprise.] Each perceived event causes a prediction error with respect to the agent's predictive model. Only if that error crosses a certain threshold is the event given further attention.
%% \item[Learning, context, and meaning arise together with attention.]
%%   ``Punctuating events'' \cite[p.~301]{bateson-logical-categories}
%%   from a stream of data is one form of attention.  Identifying
%%   patterns that are stable over time begins to give the data
%%   ``context and interpretation'' \cite{rowley2007wisdom}.
%% \item[To some approximation, features of the environment will be attended to.] This is a more specific version of the hypothesis
%%   that hierarchical structures in the environment will be mirrored by
%%   \emph{adaptive} agents \cite{simon1962architecture,simon1995near}.
%%   Outside intervention may be needed to optimise learning about tasks
%%   with complicated problem/subproblem structure
%%   \cite{goldenberg2004may}.
%%   %% Agents tend to see most clearly the
%%   %% distinctions that they themselves draw
%%   %% \cite{keeney1982epistemology}.
%% \end{description}

%% \paragraph{\textbf{\upshape Notation}}

%% $\Gamma_1 \subset \Gamma$ is a \emph{restricted context}, obtained by
%% constraining some of the variables on which $\Gamma$ depends.  Again,
%% the contents of $\Gamma_1$ vary with time.  Attention generates
%% changes in the way new perceptions come online.

%% \paragraph{\textbf{\upshape Example}} Looking closer, Blass observed that there was a pipe near the tall tree:
%%  $\Gamma_1(t_1,s_1)\vdash p:\mathit{Pipe}$.  Looking closer still,
%% he observed that the faucet was leaking:
%% $\Gamma_1(t_1^\prime,s_1^\prime)\vdash p:\mathit{Pipe}, l:\mathit{Leak}$.
%% In this case, $\Gamma_1$ also includes the perceptions from the original context $\Gamma$.

%% \begin{center}
%% \asterism
%% \end{center}

\begin{defn}\label{def:interest}
  \hypertarget{def:interest}{} A \textbf{focus shift} occurs if processing leads to a functional hypothesis related to the event.
\end{defn}
% I'm not sure if "fitness function" is generic enough here. Rather "objective function", expressing a system's goals? You write yourself that our use is at odds with the evolutionary literature; so why do we use the term then, still?

% Evaluation of data via existing objective functions

%% \paragraph{\textbf{\upshape Foundations}}
                                         
%% \begin{description}                      
%% \item[Context change is a possible basis for belief revision.]
%%   \citet{logan1994modelling} use the notion of \emph{belief revision}
%%   to model situations of collaborative information-seeking.  Ground
%%   assumptions are shared in the context of such dialogues, and can
%%   change as conversations progress.  In our model, the focus shift
%%   similarly causes the context to evolve, so that the ground
%%   assumptions are no longer
%%   the same.  \citet{harman1986change} treated the implications of
%%   changing circumstances for bringing about a ``reasoned change of
%%   view'' (p.~3); he described previous work by
%%   \citet{Doyle:1980:MDA:889488} on the system {\sf SEAN}, which
%%   incorporated defeasible reasoning, as one of only a few earlier
%%   efforts in this area.  More recently, \citet{clarke2017assertion}
%%   argues that \emph{belief} is context-sensitive, depending for
%%   example on purpose, and on the stakes involved.  In particular, in a
%%   dialogue, the sincerety of a given remark is linked to the context,
%%   not just to the remark's propositional content.  An agent's
%%   interest is often what motivates bringing the event into a new context.
%%   In the case of Velcro\textsuperscript{\texttrademark}, the focus
%%   shift occurred in quite a literal fashion, when de Mestral examined
%%   burrs under a microscope.  This example provides another useful
%%   mnemonic: burrs' hooks allow them to ``hitchhike'' into new contexts
%%   \cite[\textsection1.1]{jenkins2011bio}.
%%   \citet{patalano1993predictive} describe the related mental
%%   phenomenon of \emph{predictive encodings} that record ``blocked
%%   goals in memory in such a way that they will be recalled by
%%   conditions favorable for their solution.''
%%   %%   For example, Fry was sufficiently interested in the adhesive to
%%   %% recall it in connection with a familiar problem at choir
%%   %% practice.
%% \item[Interest is related to curiosity.]  Berlyne distinguished
%%   between \emph{perceptual} and \emph{epistemic} curiosity, while
%%   positing a relationship between them: one ``leads to increased
%%   perception of stimuli'' and the other to ``knowledge''
%%   \cite[p.~180]{berlyne1954theory}.  He posited that responses would
%%   be strongest in an ``intermediate state of familiarity'' which
%%   triggered conflict, whereas ``too much familiarity will have removed
%%   conflict by making the particular combination an expected one''
%%   (p.~189).  Accordingly, such curiosity depends on prior
%%   preparations.  In some reinforcement learning models, a
%%   \emph{novelty bonus} ``acts like a surrogate reward'' and ``distorts
%%   the landscape of predictions and actions, as states predictive of
%%   future novelty come to be treated as if they are rewarding''
%%   \cite[p.~554]{kakade2002dopamine}.  Whether or not novelty is
%%   interesting in and of itself, the system's initial assessment
%%   motivates it to look for further information or for ``new
%%   connections,'' as per \citet{Makri2012a}.  This effort is expected
%%   to yield a future payoff, whether in terms of additional novelty,
%%   more efficient organisation of the system's knowledge base, or some
%%   other reward.
%% % You could make two claims here: objective functions could either be extrinsic (e.g. a task specified by the system designer) or intrinsic (pursuing the activity has only inherent, but not instrumental value). In the second case, we can talk about motivations such as curiosity or learning progress. Cf. Oudeyer and Kaplan, 2007/2008 on intrinsic motivations. Crucially, interest is then not exclusively related to curiosity, but to a whole set of intrinsic motivations.
%% \end{description}

%% \paragraph{\textbf{\upshape Notation}}

%% Within $\Gamma_1$, features are identified that seem to have causal
%% relationships with each other.  These relationships might be
%% stochastic.  At this stage, the relationships identified have the
%% status of hypotheses.

%% \paragraph{\textbf{\upshape Example}}
%%  $\Gamma_1 \vdash \mathit{Time\ passes} \rightarrow
%% \mathit{Valve\ degrades} \rightarrow \mathit{Pipe\ Leaks} \rightarrow \mathit{Water\ spreads} \rightarrow \mathit{Tree\ grows}$.

% I don't think that it is helpful to talk about aesthetics here, as it only narrows our scope. 
% I would remove the last item / paragraph entirely. Clearly, judgements of interest depend on the person, but we've discussed that in the "prepared mind" reference earlier in this section. The focus on poetry is too narrow.
% I'd stick with only two items: one on extrinsic objective functions and one on intrinsic ones, where you could talk about interestingness and other things. 

%% \begin{center}
%% \asterism
%% \end{center}

\begin{defn}\label{def:explanation}
  \hypertarget{def:explanation}{} An \textbf{explanation} uses
  reasoning to extend the hypothesis about the observed event to other
  events in the context within which the system operates.
\end{defn}
% There's two fundamental mechanism to build a predictive model: a maximum likelihood estimator in classic inferential statistics, or Bayesian inference where the goal is to find the posterior p(theta|evidence) using the prior p(theta) and the likelihood p(evidence|theta). Here, theta is a (set of) parameter which characterises our model. Bayesian inference easily becomes intractable and can be approximated with Gibbs Sampling or Variational Inference, amongst others. It seems like you completely left related work out here. All of these approaches are explaines in C. M. Bishop's Pattern Recognition and Machine-Learning book. 
%% \paragraph{\textbf{\upshape Foundations}}
                                         
%% \begin{description}
%% \item[A new model yields an improved ability to make a prediction.]
%%   Our assumptions about chance, described earlier, insist that the
%%   perceiving agent has at best a limited ability to predict the events
%%   it perceives.  The explanation stage now enables the agent to make
%%   new predictions \cite[p.~389]{sowa2000knowledge}.
%%   \citet[p.~101]{swirski2000between} points out that to be effective,
%%   explanation needs ``a stopping rule'', e.g., ``the standard causal
%%   pattern in the social sciences'' requires only ``a description of
%%   the actions and the motivations behind them that were sufficient to
%%   produce a change in the circumstances.''
%% \item[There are different kinds of viable explanations.] We do not
%%   impose a practical requirement at this stage.  The explanation can
%%   establish ``how'' rather than ``so what.''  For instance,
%%   \citet{van1994anatomy} describes Blass's explanation of as follows:
%%   ``On investigation he found that, although the soil around the tree
%%   was dry, water was continually dripping from a nearby leaking
%%   connection in a water pipe.''  This is a perfectly suitable ``how''
%%   explanation.  According to this interpretation of the story, the
%%   usefulness of Blass's model only arose later.  According to
%%   Aristotle, the fundamental question that must be addressed is
%%   ``why?''  \cite{sep-aristotle-causality}: answers are to be
%%   demonstrated in terms of ``principles and causes'' \cite[Book Gamma,
%%     p.~81]{lawson1998metaphysics}.  Explanations may only be valid
%%   within circumscribed regimes.
%% \item[The system creates an explanation of the event for itself.]  At
%%   this stage the system is not, in general, aiming to explain its
%%   behaviour to someone else, or otherwise make its behaviour transparent
%%   (in the sense of \emph{Explainable AI}
%%   \cite{lane2005explainable}).  Nevertheless we may think of
%%   explanation as an expository device or ``framing''
%%   \cite{pease2011computational} that relies on the system's ability to
%%   retrieve a suitable context, and to establish relationships between
%%   elements of this wider context.  Explanatory prowess is not simply a
%%   matter of paying attention, but depends in particular on having
%%   learned ``what to pay attention to'' \cite[p.~4]{levin1975bateson}.
%%   Notice, then, that requirements arising in this stage can impose
%%   constraints on earlier stages: ``the methods and assumptions
%%   on which a systematic investigation is built selectively focus the
%%   researcher's attention''
%%   \cite[p.~131]{floppyearedrabbits1958barber}.
%% \end{description}
                                         
%% \paragraph{\textbf{\upshape Notation}}
%% The function will at this stage be generalised to cover other similar
%% cases, and an \emph{inhabitant} $\phi$ of the suitably generalised
%% function type is constructed.  Whereas the focus shift led to the
%% generation of hypotheses, this step generates a reproducible working
%% method.  We do not insist that this method functions the same way in
%% all possible contexts.

%% \paragraph{\textbf{\upshape Example}}
%% $\Gamma_1 \vdash \phi : (\Pi l:\mathrm{Leak}, d:\mathrm{Rate},
%% r:\mathrm{Radius}, p:\mathrm{Plant} \mathbin{.}
%% \mathrm{Growth}(l,d,r,p))$, i.e., in general when water leaks a
%% sufficient rate near a growing plant, that plant's growth is likely to
%% be increased.

%% \begin{center}
%% \asterism
%% \end{center}

\begin{defn}\label{def:bridge}
\hypertarget{def:bridge}{}A \textbf{bridge} generalises and reworks
the explanation as a solution strategy for a problem in the system's
operating domain.
\end{defn}
%% \paragraph{\textbf{\upshape Foundations}}
                                         
%% \begin{description}
%% \item[Application lies beyond explanation.]
%%   The bridging process can be
%%   conveniently outlined by comparing a positive example with a
%%   corresponding counterexample.  Nearly 60 years before Alexander Fleming,
%%   Eugene Semmer both discovered and also cursorily explained the
%%   curious effects of \emph{penicillium notatum}---but he did not find
%%   a bridge to the vital problem his discovery could have solved
%%   \cite[p.~75]{cropley2013creativity}.  His ``methods and
%%   assumptions'' \cite[p.~131]{floppyearedrabbits1958barber}
%%   constrained his thinking.
%% \item[Pseudoserendipity versus true serendipity]
%%   \citet[p.~3]{Figueiredo2001} made the distinction between
%%   serendipity and pseudoserendipity  crisp by introducing
%%   the ``serendipity equations'':
%% \begin{quote}
%% \begin{tabular}{cc}
%% \emph{pseudoserendipity} & \emph{serendipity}\\
%% $\begin{array}{c}
%% P1 \subset (\mathit{KP}1)\\
%% M \subset (\mathit{KM})
%% \end{array} \Rightarrow S\hspace{-.12em}1 \subset (\mathit{KP}1, \mathit{KM}, \mathit{KN})$
%% &
%% $\begin{array}{c}
%% P1 \subset (\mathit{KP}1)\\
%% M \subset (\mathit{KM})
%% \end{array} \Rightarrow
%% \begin{array}{c}
%% P2 \subset (\mathit{KP}2)\\
%% S\hspace{-.1em}2 \subset (\mathit{KP}2, \mathit{KM}, \mathit{KN})
%% \end{array}$
%% \end{tabular}
%% \end{quote}
%% In the pseudoserendipitous case, a given problem $P1$ in the knowledge
%% domain $\mathit{KP}1$ becomes solveable (whence, $S1$) by the addition of
%% additional knowledge, supplied by $M$.  In the serendipitous case, the
%% initial set up is similar, but the result is not a solution to the
%% original problem: rather, it is a new problem, $P2$, together with its
%% solution.
%% \item[The bridge is transformational.]  Although the notation above
%%   makes the distinction between the two cases clear, it somewhat
%%   disguises the principle common to both.  Thus, in case (i), there is
%%   must be more going on than just new information coming online which
%%   happens to make a problem solveable.  Otherwise any online
%%   problem-solving system could be seen as pseudoserendipitous.  For
%%   example, when putting together a model aeroplane, this is done piece
%%   by piece, and even the order in which the pieces are put into place
%%   is more or less predictable.  It would not be said that either the
%%   last piece added, nor any of the other pieces that were added along
%%   the way, was the result of pseudoserendipitous creativity.  By
%%   contrast, historical progress in aviation depended on dramatic
%%   contemporaneous progress in human interaction
%%   \cite[p.~292]{spenser2008airplane}, which suggests that this process
%%   could not have been planned in advance.  To consider another simple
%%   example, assembling a jigsaw puzzle is not a fully predictable
%%   process.  However, even if a previously missing piece was suddenly
%%   supplied which made the puzzle solveable, this would not be a bridge
%%   according to our definition.  In short, both pseudoserendipitous and
%%   serendipitous creativity involve ``the transformation of some (one
%%   or more) dimension of the space so that new structures can be
%%   generated which could not have arisen before''
%%   \cite[p.~348]{boden1998creativity}.
%% \item[Problem identification is meta-level.]  Constructing a bridge
%%   involves a meta-problem, in other words, a fitness function or
%%   ``aesthetic'' \cite{pease2011computational}, through which an entire
%%   class of problems is surveyed, and the most suitable one (i)
%%   selected or (ii) constructed.  One way to develop this aesthetic is
%%   through experience with previous problems: \emph{derivational
%%     analogy} is ``a process that draws analogies from the experiences
%%   of the past reasoning process'' \cite{Melis98anargument}. Herbert
%%   Clark \cite[p.~169]{Clark:1975:BRI:980190.980237} used the term
%%   `bridging', in a linguistics setting, to refer to a certain class of
%%   inferences: ``ones the speaker intends the listener to draw as an
%%   integral part of the message.''  Our setting is different: people
%%   frequently attribute `meaning' to events, but it is the observer's
%%   role in interpreting the events that we focus on
%%   (cf.~\cite{dennett_2013}).  Interpretations tend to be conditioned
%%   by previous experience.
%% \end{description}

%% \paragraph{\textbf{\upshape Notation}}

%% $\Gamma_2$ is a context and $\beth : \Gamma_1 \rightarrow \Gamma_2$ is
%% a map between contexts such that $\phi \mapsto \psi$.  In general,
%% $\Gamma_2$ and $\psi$ include structure that is not present in
%% $\Gamma_1$.

%% \paragraph{\textbf{\upshape Example}}
%% Blass synthesized a design along the lines of $$\psi : (\Pi
%% h:\mathrm{Hose}, d:\mathrm{Rate}, r:\mathrm{Radius},
%% p_1,\ldots,p_k:\mathrm{Plant} \mathbin{.}
%% \mathrm{Growth}(h,d,r,p_1,\ldots,p_k)).$$ In other words, one hose
%% could be constructed in such a way that it incorporates multiple
%% leaks, controlled by plastic emmitters, and helps multiple plants grow
%% without wasting water.  In practice, actually developing the solution
%% was more complex, and depended on contemporary innovations in plastic
%% manufacturing and new social relationships to fund and deploy the
%% product.

%% However, there would have been ample room for pseudoserendipity in
%% the initial search for methods to build powered aeroplanes.
%% \begin{center}
%% \asterism
%% \end{center}

\begin{defn}\label{def:result}
\hypertarget{def:result}The solution is
\textbf{evaluated} according to some pre-existing objective function.
%% This may be system-intrinsic, or specified by the system's user or a third party.
\end{defn}
%Ah okay, here you mention the extrinsic vs. intrinsic distinction. Anyways, my earlier suggestions still apply.
%Is this the same objective/fitness function that we've been discussed earlier?

%% \paragraph{\textbf{\upshape Foundations}}
                                         
%% \begin{description}                      
%% \item[Affection is based on reflection.]
%% \citet{campbell2005serendipity} highlights the idea of ``rational
%% exploitation'' and the ``discovery of something useful or
%% beneficial'' as key aspects of serendipity.
%% Some processing may be required
%% to get to that point. Here we may refer to the Bergsonian
%% distinction
%% between ``perceptions'' and ``affections''
%% \cite[p.~23]{deleuze1991bergsonism}.
%% Affection is the ``feeling in the instant'', which is {``}`alloyed'
%% to other subjectivities [\ldots] as we understand what we feel and
%% act upon it'' \cite[p.~141]{sutton2008deleuze}.
%% In particular,
%% \citet[p.~17]{bergson1991matter} considers affections to be directly
%% linked to the self-knowledge a being has of its body.  A system's
%% evaluation of the new state of affairs brought about by the processing
%% stages outlined in Definitions \ref{def:perception}--\ref{def:bridge}
%% might be described as ``affective''
%% when a new system configuration is brought about that is then assessed
%% in some reflexive way.  Raw somesthetic sense---e.g., an architecture inspired by
%% the instrumentation of robotic joints with hardwired position
%% sensors---might be alloyed with ``reflective thinking'' \cite{singh2005architecture}
%% that considers global aspects of the configuration and course of action
%% that led to this point.
%% \end{description}                        

%% \paragraph{\textbf{\upshape Notation}}

%% $\psi$ is run, selectively, and generates tangible outcomes.

%% \paragraph{\textbf{\upshape Example}}
%% Although various forms of drip irrigation have existed in the past,
%% Blass and colleagues created the first drip irrigation company.  Their
%% methods continue to be used.

\noindent Links between phases are described can be described with the logic of predictive processing:

%% Popper Objective Knowledge 1972 - cited in The effect of Input Knowledge on Creativity.

\begin{itemize}
\item[1$\rightarrow$2] A surprising perception elicits \emph{attention}.
\item[2$\rightarrow$1] Attention tells the system what perceptions to expect.
\item[2$\rightarrow$3] When attention is directed to something unexpected, this provokes a \emph{focus shift}.
\item[3$\rightarrow$2] Existing functional hypotheses tell the system where to direct its attention.
\item[3$\rightarrow$4] An overly specific or poorly supported hypothesis stimulates effort towards a broader \emph{explanation}.
\item[4$\rightarrow$3] Explanation tells the system what hypotheses are likely to be supported.
\item[4$\rightarrow$5] Explanations that don't solve an interesting problem provoke the search for a \emph{bridge}.
\item[5$\rightarrow$4] Existing bridging strategies tell the system how an explanation is likely to be applicable.
\item[5$\rightarrow$6] A novel solution elicits \emph{evaluation}.
\item[6$\rightarrow$5] An unsatisfying solution provokes the search for a new solution or a new problem to work on.
\end{itemize}

% \subsection{Summary}

Without all of the phases, we would not talk about `serendipity'. This
can be illustrated by example.  Nearly 60 years before Alexander 
Fleming, Eugene Semmer both discovered and also cursorily explained
the curious effects of \emph{penicillium notatum}---but he failed to
grasp the vital problem his discovery could have solved
\cite[p.~75]{cropley2013creativity}.  His ``methods and assumptions''
\cite[p.~131]{floppyearedrabbits1958barber} constrained his thinking.
Failures at the other stages can readily be identified.

%% We then defined each component with reference
%% to theoretical literature.  Table \ref{tab:model-summary-table}
%% summarises the model that results from this analysis.

%% %\begin{landscape}
\tikzset{
    boxComponent/.style={
    rectangle,
    draw=black, very thick,
    minimum height=2em,
    minimum width=10em,
    inner sep=2pt,
    text centered,
    },
    boxText/.style={
    rectangle,
    draw=black,
    minimum height=14em,
    minimum width=10em,
    inner sep=0em,
    outer sep=0em,
    },
}

%\thispagestyle{empty}

\begin{table}[h]
\captionsetup{width=1\linewidth}
\vspace{1cm}\hspace{0cm}
\resizebox{1\textwidth}{!}{
%% Wow, I didn't know how to use ``node distance'' for auto-layout -JC.
%% Nice!
\begin{tikzpicture}[->,>=stealth',node distance=4cm]

 % Position of QUERY
 % Use previously defined 'state' as layout (see above)
 % use tabular for content to get columns/rows
 % parbox to limit width of the listing
 \node[boxComponent] (Perception) {\hyperlink{def:perception}{\textbf{Perception}}};
 \node[boxComponent, right of = Perception] (Attention) {\hyperlink{def:attention}{\textbf{Attention}}};
 \node[boxComponent, right of = Attention] (Interest) {\hyperlink{def:interest}{\textbf{Focus Shift}}};
 \node[boxComponent, right of = Interest] (Explanation) {\hyperlink{def:interest}{\textbf{Explanation}}};
 \node[boxComponent, right of = Explanation] (Bridge) {\hyperlink{def:bridge}{\textbf{Bridge}}};
 \node[boxComponent, right of = Bridge] (Valuation) {\hyperlink{def:result}{\textbf{Valuation}}};

% \node[above left= -1.5mm and 9.2mm of Perception.center] {\large{\pmglyph{\HDiv}}};
% \node[above right= -1.5mm and 9.2mm of Perception.center] {\reflectbox{\large{\pmglyph{\HDiv}}}};

 \coordinate[above= 9mm of Perception  ] (PerceptionPrime) ;
 \coordinate[above= 9mm of Attention    ] (AttentionPrime) ;
 \coordinate[above= 9mm of Interest     ] (InterestPrime) ;
 \coordinate[above= 9mm of Explanation  ] (ExplanationPrime) ;
 \coordinate[above= 9mm of Bridge       ] (BridgePrime) ;
 \coordinate[above= 9mm of Valuation    ] (ValuationPrime) ;

\begin{scope}[thick,decoration={
    markings,
    mark=at position 0.52 with {\arrow{latex}}}
    ]
\draw[postaction={decorate},-] (AttentionPrime) -- (PerceptionPrime);                   % L
\draw[postaction={decorate},-] (InterestPrime) -- (AttentionPrime);                     % L
\draw[postaction={decorate},-] (ExplanationPrime) -- (InterestPrime);                   % L
\draw[postaction={decorate},-] (BridgePrime) -- (ExplanationPrime);                     % L
\draw[postaction={decorate},-] (ValuationPrime) -- (BridgePrime);                      % L
\end{scope}
%%
\begin{scope}[thick,decoration={
    markings,
    mark=at position 0.65 with {\arrow{latex}}}
    ]
 \path
 (PerceptionPrime) edge[postaction={decorate},-] (Perception)                                   % D
 (Attention) edge[postaction={decorate},-,transform canvas={xshift=-9mm}] (AttentionPrime)     % U
 (AttentionPrime) edge[postaction={decorate},-,transform canvas={xshift=9mm}] (Attention)      % D
 (Interest) edge[postaction={decorate},-,transform canvas={xshift=-9mm}] (InterestPrime)       % U
 (InterestPrime) edge[postaction={decorate},-,transform canvas={xshift=9mm}] (Interest)        % D
 (Explanation) edge[postaction={decorate},-,transform canvas={xshift=-9mm}] (ExplanationPrime) % U
 (ExplanationPrime) edge[postaction={decorate},-,transform canvas={xshift=9mm}] (Explanation)  % D
 (Bridge) edge[postaction={decorate},-,transform canvas={xshift=-9mm}] (BridgePrime)           % U
 (BridgePrime) edge[postaction={decorate},-,transform canvas={xshift=9mm}] (Bridge)            % D
 (Valuation) edge[postaction={decorate},-] (ValuationPrime);                                     % U
\end{scope}


%\draw[decorate,decoration={triangles,segment length=12mm}] (ValuationPrime) -- (PerceptionPrime);

%Boxes for definitions
 \node[boxText, below of = Perception,yshift=2em] (PerceptionDef)
 {\small
 \begin{minipage}[t][14em]{10em}
 \begin{flushleft}
  \textbf{Interface to world}\\[.2cm]
  \textbullet\ \citeauthor{russel2003artificial} - different kinds of environments\\
  \textbullet\ \citeauthor{hume1904enquiry}/\citeauthor{peirce1931necessity} - chance is negative/fundamental\\
  \textbullet\ \citeauthor{hoffman2015interface} - adaptivity of not seeing reality as it is\\
  \textbullet\ \citeauthor{friston2009free} - we sense what we can predict
 \end{flushleft}
 \end{minipage}
 };

 \node[boxText, below of = Attention,yshift=2em] (AttentionDef){\small
 \begin{minipage}[t][14em]{10em}
  \begin{flushleft}
  \textbf{Directed processing power}\\[.2cm]
  \textbullet\ \citeauthor{clark2013whatever} - prediction error\\
  \textbullet\ \citeauthor{singh2005architecture} - layered architecture\\
  \textbullet\ \citeauthor{bateson-logical-categories} - changing behaviour\\
  \textbullet\ \citeauthor{rowley2007wisdom} - meaning making
  \end{flushleft}
 \end{minipage}
 };

 \node[boxText, below of = Interest,yshift=2em] (InterestDef) {\small
 \begin{minipage}[t][14em]{10em}
\begin{flushleft}
  \textbf{Hypothesis generation}\\[.2cm]
  \textbullet\ Wundt curve\\
  \textbullet\ \citeauthor{berlyne1954theory} - epistemic and perceptual curiosity\\
  \textbullet\ \citeauthor{logan1994modelling} - belief revision in information seeking\\
  \textbullet\ \citeauthor{patalano1993predictive} - predictive encodings
\end{flushleft}
 \end{minipage}
 };

 \node[boxText, below of = Explanation,yshift=2em] (ExplanationDef){\small
 \begin{minipage}[t][14em]{10em}
\begin{flushleft}
  \textbf{A predictive model}\\[.2cm]
  \textbullet\ \citeauthor{lawson1998metaphysics} - principles and causes\\
  \textbullet\ \citeauthor{pease2011computational} - framing\\
  \textbullet\ \citeauthor{bateson-logical-categories} - change of pattern\\
\end{flushleft}
 \end{minipage}
 };

 \node[boxText, below of = Bridge,yshift=2em] (BridgeDef) {\small
 \begin{minipage}[t][14em]{10em}
\begin{flushleft}
  \textbf{Identifying or positing a problem}\\[.2cm]
  \textbullet\ \citeauthor{bergson1946creative} - creativity of problem statement\\
%  \textbullet\ \citeauthor{thagard2011aha} - ``aha moment''\\
  \textbullet\ \citeauthor{boden1998creativity} - transform the space\\
  \textbullet\ \citeauthor{pease2011computational} - new aesthetic
\end{flushleft}
 \end{minipage}
 };

 \node[boxText, below of = Valuation,yshift=2em] (ValuationDef) {\small
 \begin{minipage}[t][14em]{10em}
\begin{flushleft}
  \textbf{Evaluation of solution}\\[.2cm]
  \textbullet\ \citeauthor{bergson1991matter} - affection\\
  \textbullet\ \citeauthor{campbell2005serendipity} - rational exploitation\\
\end{flushleft}
 \end{minipage}
 };


%%  % Boxes for implementation details
%%  \node[boxText, below of = PerceptionDef, yshift = -4em] ()
%%  {\small
%%  \begin{minipage}[t][14em]{10em}
%% \begin{flushleft}
%%   \textbf{HCI, automated feature finding, emergence of grid cells} \\[.2cm]
%%   \textbullet\ \citeauthor{turk2000perceptive}\\
%%   \textbullet\ \citeauthor{jacob2015viewpoints}\\
%%   \textbullet\ \citeauthor{stopher2017technology}\\
%%   \textbullet\ DeepDream\\
%%   \textbullet\ \citeauthor{Banino2018}
%% \end{flushleft}
%%  \end{minipage}
%%  };

%%  \node[boxText, below of = AttentionDef, yshift = -4em] ()
%%  {\small
%%  \begin{minipage}[t][14em]{10em}
%% \begin{flushleft}
%%   \textbf{Visual attention, competition for resources, temporal bonus, soft attention}\\[.2cm]
%%   \textbullet\ \citeauthor{sun2003object}\\
%%   \textbullet\ \citeauthor{tsotsos1995modeling}\\
%%   \textbullet\ \citeauthor{baars1997theatre}\\
%%   \textbullet\ \citeauthor{lesser1977retrospective}\\
%%   \textbullet\ \citeauthor{vemula2017social}
%% \end{flushleft}
%%  \end{minipage}
%%  };

%%  \node[boxText, below of = InterestDef, yshift = -4em] ()
%%  {\small
%%  \begin{minipage}[t][14em]{10em}
%% \begin{flushleft}
%%   \textbf{Autonomous creative behaviour, aesthetics classifier, compression, information gain}\\[.2cm]
%%   \textbullet\ \citeauthor{Saunders2007}\\
%%   \textbullet\ \citeauthor{dhar2011high}\\
%%   \textbullet\ \citeauthor{schmidhuber2009art}\\
%%   \textbullet\ \citeauthor{javaheri2016analysis}
%% \end{flushleft}
%%  \end{minipage}
%%  };

%%  \node[boxText, below of = ExplanationDef, yshift = -4em] ()
%%  {\small
%%  \begin{minipage}[t][14em]{10em}
%% \begin{flushleft}
%%   \textbf{Explanation-based learning, epistemic modelling, critics, dialogue, integration of causal models}\\[.2cm]
%%   \textbullet\ \citeauthor{ellman1989explanation}\\
%% %  \textbullet\ \citeauthor{cohen1992abductive}\\
%%   \textbullet\ \citeauthor{delamaza1994generate}\\
%%   \textbullet\ \citeauthor{sussman1973computational}\\
%% %  \textbullet\ \citeauthor{Sacerdoti:1975:SPB:907010}\\
%%   \textbullet\ \citeauthor{singh2005alternate}\\
%%   \textbullet\ \citeauthor{moore1995participating}\\
%%   \textbullet\ \citeauthor{GeiHofSch16}
%% \end{flushleft}
%%  \end{minipage}
%%  };

%%  \node[boxText, below of = BridgeDef, yshift = -4em] ()
%%  {\small
%%  \begin{minipage}[t][14em]{10em}
%% \begin{flushleft}
%%   \textbf{Analogy, metaphor, concept blending, bridging terms}\\[.2cm]
%%   \textbullet\ \citeauthor{sowa2003analogical}\\
%%   \textbullet\ \citeauthor{xiao2016meta4meaning}\\
%%   \textbullet\ \citeauthor{confalonieri2018concepts}\\
%%   \textbullet\ \citeauthor{EPPE2018105}\\
%%   \textbullet\ \citeauthor{swanson1997interactive}\\
%%   \textbullet\ \citeauthor{jursic2012}
%% \end{flushleft}
%%  \end{minipage}
%%  };

%%  \node[boxText, below of = ValuationDef, yshift = -4em] ()
%%  {\small
%%  \begin{minipage}[t][14em]{10em}
%% \begin{flushleft}
%%   \textbf{Modelling taste, affect, intrinsic motivation}\\[.2cm]
%%   \textbullet\ \citeauthor{Saunders2007}\\
%%   \textbullet\ \citeauthor{picard1995affective}\\
%%   \textbullet\ \citeauthor{kaplan2007intrinsically}\\
%%   \textbullet\ \citeauthor{singh2010intrinsically}
%% \end{flushleft}
%%  \end{minipage}
%%  };

% \node [rotate=90, left of = Perception, xshift=2.5em,yshift=6em] {\textbf{Definitions}};

% \node [rotate=90, left of = Perception, xshift=-13em,yshift=6em] {\textbf{Implementations}};

%% \node[ below of = PerceptionDef,yshift=3em] (PreceptionNote) {$\Gamma(t,\ldots)$};
%% \node[ below of = AttentionDef,yshift=3em] (AttentionNote) {$\Gamma_1\subset\Gamma$};
%% \node[ below of = InterestDef,yshift=3em] (InterestNote) {$\Gamma_1\vdash f\rightarrow g\rightarrow\ldots$};
%% \node[ below of = ExplanationDef,yshift=3em] (ExplanationNote) {$\Gamma_1 \vdash \phi : \Pi x:X \mathbin{.} Y(x)$};
%% \node[ below of = BridgeDef,yshift=3em] (BridgeNote) {$\beth : \Gamma_1 \rightarrow \Gamma_2, \phi \mapsto \psi$};
%% \node[ below of = ValuationDef,yshift=3em] (ValuationNote) {$\psi(t,\ldots)$};

\begin{scope}[thick]
  % draw the connecting arrows
 \path
 (Perception) edge (Attention)
 (Attention) edge (Interest)
 (Interest) edge (Explanation)
 (Explanation) edge (Bridge)
 (Bridge) edge (Valuation)
;
\end{scope}
\end{tikzpicture}}

\begin{tabular}{p{.1\textwidth}p{.\textwidth}}
\caption{Our model for systems with serendipity potential. The flowchart at top provides a visual key, showing that previous phases can be returned to at any point.  The body of the table summarises Definitions \ref{def:perception}--\ref{def:result}.\label{tab:model-summary-table}}
\end{tabular}
\end{table}
%\end{landscape}

%Is this table stil up-to-date?
