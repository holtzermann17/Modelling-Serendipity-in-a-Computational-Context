\section{Serendipity by Example: The condition, components, dimensions, and factors of serendiptious occurrences} \label{sec:by-example}

This section introduces the concepts that inform our model and
evaluation procedure in Section \ref{sec:our-model}.  We illustrate
each concept by means of historical examples.  The several
process-based theories discussed in Section
\ref{sec:literature-review} serve as prototypes.
%%
Most fundamentally, what we've seen in Section
\ref{sec:literature-review} is that serendipity is, indeed, a process:
it is not a pure event or happenstance that can only be decomposed
into a ``before'' and an ``after.''  However, there does appear to be
a binary aspect to serendipitous occurrences, whether the process is
analysed into exactly two phases, or more (see Table
\ref{tab:theory-summary}).  Thus, for \citet{merton1948bearing}, the
unanticipated datum needs to be observed and seen to be anomalous
before it can become strategic.  For \citet{Makri2012a}, a new
connection is first identified, and only later exploited.
%%
Accordingly, we assert that at the centre of serendipitous processes
there is a sub-event that we will call the \emph{focus shift}.
Without it, a discovery may be made but then abandoned, with its significance unrecognised.
In such a circumstance, a beneficial outcome is quite unlikely,
unless, as sometimes happens, the discovery is later revisited and its
value explored.  The concept of a ``focus shift'' is considerably less
value-laden and metaphysically complex than the notion of ``insight.''
We think of this as the key (necessary but not sufficient)
\emph{condition} for serendipity.

Like \citet{lawley2008maximising}, we theorise serendipity as a sequence.
We think of this sequence in objective terms, as made up of \emph{components}. 
Unlike the focus shift -- which we take to be simply present or absent -- the
four components may be active to a greater or lesser degree.
Accordingly, we outline four corresponding \emph{dimensions}.  These
form the basis of an evaluation rubric in Section \ref{sec:our-model}.
We also put forward four \emph{environmental factors} that appear to be
conducive to processes and outcomes that could be described as
serendipitous.  We illustrate all of these concepts with examples,
below.
%%
The section concludes with some reflections on the status of these
concepts, and, in particular, their suitability for inclusion in a
computational theory.

%We adapt the conceptual framework proposed by \citeA{pease2013discussion}.

\subsection{Key condition for serendipity: focus shift}

Serendipity hinges on an assessment or reassessment -- a
\textbf{focus shift} in which something that was previously unnoticed
or uninteresting, of neutral, or even negative value, becomes
interesting.  In one famous and striking example, George de Mestral,
an electrical engineer by training, and an experienced inventor who
worked in a machine shop, returned from a hunting trip in the Alps. He
removed numerous burdock burrs from his clothes and his dog's fur and
became curious about how they worked. After examining them under a
microscope, he realised the possibility of creating a new kind of
fastener that worked similarly, laying the foundations for
the hook-and-loop mechanism in Velcro\texttrademark\ \cite{roberts}.
% \cite[p. x]{roberts}

\subsection{Components of serendipity}

A focus shift is brought about by the meeting of a \emph{prepared mind} and a \emph{trigger}.  The next step involves building a \emph{bridge} to a valuable \emph{result}. We illustrate these terms by means of the following examples:

\begin{itemize}
\item \textbf{Prepared mind}: 
Alexander Fleming's ``prepared mind'' included his focus
on carrying out experiments to investigate influenza as well as his
previous experience that showed that foreign substances in petri dishes can kill
bacteria.  He was concerned above all with the question ``Is there a
substance which is harmful to harmful bacteria but harmless to human
tissue?''  \cite[p. 161]{roberts}.
\end{itemize}

\begin{itemize}
\item \textbf{Trigger}: The trigger does not directly cause the
  outcome, but rather, inspires a potentially new insight.  It was
  long known by Quechua medics that cinchona bark stops shivering.  In
  particular, it worked well to stop shivering in malaria patients, as
  was observed when malarial Europeans first arrived in Peru
  \cite[pp.~75--77]{desowitz1997gave}.  The joint appearance of shivering
  Europeans and a South American remedy was the trigger.  That an
  extract from cinchona bark can cure and can even prevent malaria was
  learned subsequently.
\end{itemize}

\begin{itemize}
\item \textbf{Bridge}: The bridge often includes reasoning techniques,
  such as abductive inference (what might cause a clear patch in a
  petri dish?), analogical reasoning (de Mestral constructed a target
  domain from the source domain of burrs hooked onto fabric), or
  conceptual blending (Kekul\'e, discoverer of the benzene ring
  structure, is said to have blended his knowledge of molecule
  structure with his dream image of a snake biting its tail, although
  this story is probably apocryphal
  \cite[p.~xv]{kennedy2016inventology}).  The bridge may rely on new
  social arrangements or physical prototypes.  It may have many steps,
  and may feature chance elements.  Several serendipitous episodes may
  be chained together in sequence, so that the result of one episode
  serves as the trigger for the next, and this sequence as a whole
  forms the bridge to a truly unprecedented result.  For example,
  C\'edric Villani \cite[pp.~15--16]{birth-of-a-theorem} describes
  two surprising conversations that happened in one day early on in
  his work on the Boltzmann equation.  The first conversation was
  with Freddy Bouchet, about the way galaxies stabilise,
%-- ``I was thrilled to see Landau damping suddenly make another
%-- appearance, scarcely more than a week after my discussion with
%-- Cl\'ement [Mouhot]''--
   and the second with his colleague \'Etienne Ghys, who provided an
   unexpected link from the content of the first conversation to ideas
   in Komolgorov-Arnold-Moser theory: ``I didn't really want to say
   anything, C\'edric, but those figures there on the board -- I've
   seen them before.''
\end{itemize}

\begin{itemize}
\item \textbf{Result}: This is the new product, artefact, process,
  theory, use for a material substance, or other outcome.  The outcome
  may contribute evidence in support of a known hypothesis, or a
  solution to a known problem.  Alternatively, the result may itself
  {\em be} a new hypothesis or problem.  The result may be
  ``pseudoserendipitous'' in the sense that it was {\em sought}, while
  nevertheless arising from an unexpected source.  More classically,
  it is an \emph{unsought} finding, such as the discovery of the
  Rosetta stone.
\end{itemize}

\subsection{Dimensions of serendipity}\label{sec:dimensions}

The four components described above have attributes that may be present to a greater or lesser degree.  These are: \emph{Chance} -- how likely was the trigger to appear?; \emph{Curiosity} -- how likely was this trigger to be identified as interesting?; \emph{Sagacity} -- how likely was it that the interesting trigger would be turned into a result?; -- and \emph{Value} (how valuable is the result that is ultimately produced?).

\begin{itemize}
\item \textbf{Chance}: Fleming \cite{fleming} noted: ``There are
  thousands of different moulds'' -- and ``that chance put the mould
  in the right spot at the right time was like winning the Irish
  sweep.''  It is important to notice that \emph{he} was in the right
  spot at the right time as well -- this was not a complete
  coincidence.
\end{itemize}

\begin{itemize}
\item \textbf{Curiosity}: Curiosity can dispose a creative person to
  begin or to continue a search into unfamiliar territory.  We use
  this word to describe both simple curiosity and related deeper
  drives.  Charles Goodyear \cite{goodyear1855gum}, discoverer of
  the process for vulcanising rubber, reflects on the role this kind
  of deep curiosity played in shaping his career: ``[F]rom the time
  his attention was first given to the subject, a strong and abiding
  impression was made upon his mind, that an object so desirable and
  important, and so necessary to man's comfort, as the making of
  gum-elastic available to his use, was most certainly placed within
  his reach.  Having this presentiment, of which he could not divest
  himself, under the most trying adversity, he was stimulated with the
  hope of ultimately attaining this object.''
\end{itemize}

\begin{itemize}
\item \textbf{Sagacity}: This old-fashioned word is related to
  ``wisdom,'' ``insight,'' and especially to ``taste'' -- and
  describes the attributes, or skill, of the discoverer that
  contribute to forming the bridge between the trigger and the result.
  Merton \cite[p.~507]{merton1948bearing} writes: ``{[}M{]}en
  had for centuries noticed such `trivial' occurrences as slips of the
  tongue, slips of the pen, typographical errors, and lapses of
  memory, but it required the theoretic sensitivity of a Freud to see
  these as strategic data through which he could extend his theory of
  repression and symptomatic acts.''  The degree to which such data
  are \emph{prima facie} unanticipated and anomalous is clear.  Merton
  would be prepared to take in stride that Freud's claims
  surrounding this data are part of ``an idealized story''
  \cite{freudtheory}.  For Merton ``what the observer brings to the
  datum'' is an essential aspect of strategy; his key criterion is
  that the result ``must permit of implications which bear upon
  generalised theory'' -- not that it be correct.
\end{itemize}

%% Note that the chance ``discovery'' of, say, a \pounds 10 note may
%% be seen as happy by the person who finds it, whereas the loss of
%% the same note would generally be regarded as unhappy.

\begin{itemize}
\item \textbf{Value}: Serendipity concerns happy surprises, but
  different parties may have different judgements as to whether a
  given situation is ``happy'' or ``surprising''.  A third party
  judgement of value can help to discriminate between luck, sleight of
  hand, and \emph{bona fide} value creation.  An unconcerned third party is more likely to see
  serendipity when ``One man's trash is another man's treasure''
  than when ``One man's loss is another man's gain.''
  In order to clearly distinguish between these two cases, wherever possible we prefer to
  make use of independent judgements of value.  A literal example of
  the trash-to-treasure scenario is provided by the Swiss company
  Freitag, which was started by design students who built a business
  around ``upcycling'' used truck tarpaulins into bags and backpacks.
  Thanks in part to clever marketing \cite[pp. 54--55,
    68--69]{russo2010companies}, their product is now a global brand.
\end{itemize}

\subsection{Environmental factors} \label{sec:environmental-factors}

Finally, serendipity seems to be more likely for agents who experience and participate in a \emph{dynamic world}, who are active in \emph{multiple contexts}, occupied with \emph{multiple tasks}, and who avail themselves of \emph{multiple influences}.

\begin{itemize}
\item \textbf{Dynamic world}: Information about the world develops
  over time, and is not presented as a complete, consistent whole.  In
  particular, \emph{value} may come later.
  \Citet[p. 643]{van1994anatomy} estimates that in twenty percent of
  innovations ``something was discovered before there was a demand for
  it.''  To illustrate the role of this factor, it may be most
  revealing to examine a remarkable ``near miss,'' in which the state
  of the world changed, but the dynamics and concomitant implications
  were not attended to carefully.
%%
  \citet[pp.~75-76]{cropley2013creativity} describe the case of Eugen
  Semmer, a veterinary pathologist who intended to carry out a post
  mortem analysis on two unwell horses.  As it turned out, there was
  just one problem with his plan: ``when he arrived in the morning he
  discovered that the animals had unexpectedly and inexplicably
  recovered.''  %% With some consternation, he then determined that
%%   ``their recovery was linked to the unintended presence of spores of
%%   the fungus \emph{penicillium notatum} in his laboratory.''  He
%%   proceeded to test this theory with further \emph{in vivo}
%%   experiments on other animals.
%% \begin{quote}  ``\emph{However, apparently blinded by the
%%   narrow nature of his special knowledge {\upshape\ldots}\ he did not
%%   recognise that he had stumbled on an important life-saver (what we
%%   now call `antibiotics'), and instead went to considerable lengths to
%%   eradicate the spores from his laboratory.}''
%% \end{quote}
%% In this example it appears that a focus shift was effected, since
%% Semmer did give the perplexing fungus further attention: however a
%% bridge to a potentially valuable result was not formed, despite of a
%% growing body of evidence.  This example reveals that the mind needs to
%% be prepared to cope effectively with change.
Semmer effected a focus shift and figured out the cause of this change
of affairs -- \emph{penicillium notatum} spores -- but he failed to
establish a bridge to a valuable result because he was missing the
important ability \cite{bereiter1997situated} to revise his basic
approach in response to changes in the underlying situation.
\end{itemize}

\begin{itemize}
\item \textbf{Multiple contexts}: One of the dynamical aspects at play
  may be the discoverer/inventor going back and forth between
  different contexts with different stimuli and affordances.  3M
  employee Arthur Fry sang in a church choir and needed a good way to
  mark pages in his hymn book -- and happened to have been recently
  attending internal seminars offered by his colleague Spencer Silver
  about restickable glue \cite{tce-postits}.
\end{itemize}

%% Einstein's work at the patent office seems to have been fortuitous
%% not because it gave him ideas, but because it gave him time to work
%% on his ideas.  Famously, this resulted in four fundamental papers
%% in the year 1905.

\begin{itemize}
\item \textbf{Multiple tasks}: Two decades after his \emph{annus
  mirabilis}, Einstein's fame drew letters and a copy of a rejected
  paper from Indian physicist Satyendra Nath Bose.  As a recent Nobel
  Prize winner and an institute director, Einstein was in no way
  obliged to take on the extra task of translating Bose's paper from
  English to German.  Nevertheless, he did so out of interest, and in
  the process learned a calculation method that produced accurate
  physical results, despite making nonstandard physical assumptions
  \cite{delbruck1980bose}.  Einstein's subsequent examination of this
  work led to further papers and the idea of Bose-Einstein statistics,
  which describes fundamental particles, now known as bosons, which do
  not obey the Pauli exclusion principle.
\end{itemize}

\begin{itemize}
\item \textbf{Multiple influences}: The bridge from trigger to result
  is often found by availing oneself of others' points of view.  For
  example, Arno Penzias and Robert Wilson, working at Bell Labs, used
  a large antenna to detect radio waves that were relayed by bouncing
  off satellites.  After they had carefully removed interference
  effects due to radar, radio, and heat, they found residual ambient
  noise that couldn't be eliminated.  They then ruled out directional
  effects that would suggest either a terrestrial or galactic origin
  \cite[p.~3]{lachieze1999cosmological}.  They were mystified, and
  only understood the significance of their work after a friend at MIT
  told them about a preprint on the subject written by astrophysicists
  from nearby Princeton University, who had hypothesised the
  possibility of measuring radiation released by the big bang
  \cite[p.~385]{shu1982physical}.
\end{itemize}
% \ednote{It would help to have a bibliographic citation for the anecdote on p.10 about the Bell Labs antenna.}

\subsection{A step-by-step example illustrating how these concepts can be applied to computational systems} \label{sec:by-example-summary}

In Sections \ref{sec:our-model} and
\ref{sec:computational-serendipity}, we will show how the key
condition, components, dimensions and environmental factors of
serendipity discussed above can be used to model and assess the
potential for serendipity in computational systems.
%%
Here, we develop a preliminary illustration showing how these criteria can be
applied to separate computational examples into instances, non-instances, and
weak instances.
We will consider three systems in the tradition of ``computational
discovery in mathematics'' \cite{colton2007computational}.  Broadly speaking, as richer and more robust system components come online, the potential for serendipity increases.  We consider the following cases:
%%   These system descriptions illustrate two central points: namely that
%% all of the components are necessary in order for a system to have
%% potential for serendipity, and that they can be active to a greater or
%% lesser degree in different systems.  In addition, systems do not
%% require all (or even any) of the supportive environmental factors to
%% achieve serendipitous results, but that the presence of these factors
%% can go along with several advantages.

\begin{itemize}
\item Zero potential for serendipity: Automatic theorem proving
\item Low potential for serendipity: Conjecture generation
\item Moderate potential for serendipity: Conjecture and proof generation
\item High potential for serendipity: Mining an online domain model
\end{itemize}

\paragraph{System A.~Zero potential for serendipity -- Automatic theorem proving} 
A user of an automatic theorem proving system typically has in mind
the theorem for which he or she wishes to establish a formal proof.
That is, an informal proof already exists, and when translating this
into the formal language, only minor logical and syntax errors stand
in the way.  These can be straightforwardly debugged.  Once the proof
has been fully specified, the theorem prover will return a
certification.  There seems to be no chance for serendipity here, on
either the user or the system side.  Even if we were to construe an
erroneous formal proof as a \textbf{trigger} and a corresponding error
message as a \textbf{result}, the other components all fail to
materialise.  Nothing can be generated from this trigger except for
the (to-be-expected) error message.
%% Furthermore, if the user
%% happens to be surprised by the error message, they are unlikely to be
%% particularly happy about it.

\paragraph{System $B$.~Low potential for serendipity -- Conjecture generation}

Here we focus in on a part of the mathematical process that generates
conjectures, without considering proof attempts.  This was the
historical course initially taken by the {\sf HR} project with the
{\sf NumbersWithNames} program \cite{colton2002numberswithnames}.
%%  --
%% although subsequent versions of the system worked with third-party theorem
%% provers to generate proofs \cite{colton2002hr}.
%%
Intuitively, {\sf NumbersWithNames} can help with ``the discovery
part'' of mathematics \cite[p.~7]{colton2002numberswithnames}.
%%
The \textbf{trigger} for this system was a given integer sequence,
which may have been chosen at random or hand-selected by a user.  The system is sometimes able to construct a \textbf{bridge} from the sequence to an
interesting conjecture (\emph{sans} proof) about the sequence, which is
considered to be a valuable, albeit preliminary, \textbf{result}.  A
case can be made for the system possessing an entirely nonexceptional form of
\textbf{curiosity}: each trigger is submitted for further processing,
in this case via a range of transformation rules that explore outwards
from the triggering sequence to discover potential statements that can
be made about it.  However, examining the algorithms used by {\sf
  NumbersWithNames}, the case for \textbf{sagacity} initially seems
rather weak.  In the first place: ``Even after pruning, the program
often produces a plethora of conjectures''
\cite[p.~4]{colton2002numberswithnames}.  Identifying the
\emph{plausible} conjectures among these requires some further
common-sense ideas and straightforward numerical processing.
Naturally, filtering the results list cannot guarantee that any of the
generated plausible results will actually be interesting.  Here it is
worth emphasising that, in practice, many of the interesting results
from {\sf NumbersWithNames} were found based on intelligent problem
selection on the part of the system's users, who were able to supply
a preselected sequence of interest.  Ultimately, the fact that {\sf
  NumbersWithNames} could surface some interesting conjectures about these
sequences suggests that it is sufficiently, if minimally, sagacious,
after all.  Although its preparations are mathematically non-sophisticated, its rule-based processing can be described as an elementary
\textbf{prepared mind}.
%% Moreover, work with this system led to publishable mathematical
%% results from \citet{colton1999refactorable} and others.

\paragraph{System $B^{\prime}$.~Moderate potential for serendipity -- Conjecture and proof generation}

This system has two additional abilities: it is able assess generated
conjectures using a plausibility measure, and, on this basis, to selectively construct proof-attempts.  %Automated problem selection is a key advantage that \emph{System
 % $B^{\prime}$} obtains by using \emph{System $B$} as a submodule.   This makes 
\emph{System $B^{\prime}$} is more discriminating in its
\textbf{curiosity} -- whereas \emph{System $B$} relied
on outside help or on luck to steer its initial choices.  Assuming that proof attempts are frequently successful, the system will be convincing in its \textbf{sagacity} as well.  Its \textbf{results} will be strictly more informative than those of \emph{System $B$}.
%% :\emph{System B}'s ``hit rate'' of 3\% would be much worse if the
%% conjectures did were not filtered.

\paragraph{System C.~High potential for serendipity -- Mining an online domain model} 
In a more futuristic and entirely hypothetical example, we can imagine a system has at its
disposal a large database of formalised proofs, assorted mathematical
concepts, and informal heuristics.   
Additionally, suppose that new data in a machine-accessible format is coming online
all the time -- perhaps the system deploys a next-generation parser on
new papers as they are added to the mathematics Arxiv
(cf.~\citet{ginev2009architecture}).  Such a system could have
a large collection of open problems that it is working on at any given
moment, which, together with the aforementioned facts and heuristic patterns,
constitute a considerably more robust \textbf{prepared mind} than in
the previous system.  This system could take a discriminating
approach to generating conjectures, and apply a range of mathematical
techniques to find a \textbf{bridge} from a conjecture to a proof.
%%
Each new paper or fragment of user interaction it encounters would constitute
a potential \textbf{trigger} for discovery.  Some of these contributions will have
more generative potential than others.  Importantly, the system would
be able to judge for itself whether a given \textbf{result} is globally
new.  The new data expresses highly unpredictable mathematical
content, so \textbf{chance} plays a prominent role in this 
system.  However, like the previous system, this one is fallible: for
all its background knowledge, there is no guarantee that it will find
any worthwhile results on any given day.  Its ``hit rate'' will depend partly on the quality of
the search strategies it uses.  It would be straightforward to
characterise the system's search priorities using the dimension of
\textbf{curiosity}.  Again, the system could afford to be
discriminating, with its allocation of attention driven by an interest in
specific problems.
%%
The system's heuristics for solving these problems would
straightforwardly connect with the dimension of \textbf{sagacity}.
Adding a further layer on top of this, higher-order programming could
be applied that would operationalise the search for new strategies
and heuristics.  The system is clearly situated in a \textbf{dynamic
  world}.  It can avail itself of \textbf{multiple influences} by
reading papers from different mathematical domains.  Switching
attention between proving new theorems and developing new search
strategies and problem solving heuristics would give the system
\textbf{multiple tasks} and \textbf{multiple contexts} for creativity.

\bigskip

The foregoing system descriptions illustrate two central points:
namely that all of the components are necessary in order for a system
to have potential for serendipity, and that they can be active to a
greater or lesser degree.  We will make these assertions more precise
in the following section.

