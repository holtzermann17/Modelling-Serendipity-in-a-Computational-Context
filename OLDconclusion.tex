\section{Discussion and Conclusions} \label{sec:discussion}

In Section \ref{sec:computational-serendipity}, we applied our model to evaluate the serendipity of an evolutionary music improvisation
system, a system for automatically assembling flowcharts, and
a hypothetical class of next-generation recommender systems.

The model has helped to highlight directions for development that
would increase a system's potential for serendipity, either
incrementally or more transformatively.  Our model outlines a path
towards the development of systems that can observe events that would
otherwise not be observed, take an interest in them, and transform the
observations into artefacts with value.

%% In this section, we will show how the model allows for more precise
%% thinking than other existing work touching on this area.  We then
%% discuss implications from our findings for future research.

%\input{12a-recommendations}
%\input{12b-future-work-intro}
%\input{12c-future-work-conclusion}

%\input{11related}

\subsection{Recommendations and Related work} \label{sec:related}

%% Let's focus on applications.

Inspired by social systems that capitalise on this effect, we have investigated the feasibility
of building multi-agent systems that learn by sharing and discussing
partial understandings \cite{corneli2015computational,corneli2015feedback}.

\ednote{Christian: remove? Not immediately related, and we didn't talk enough about the invention aspect. Grace and Maher use intrinsic motivation which is more domain-independent, but also not more about invention than many other CC examples.} Recent work has examined the related topics of \emph{curiosity}
\cite{wu2013curiosity} and \emph{surprise} \cite{grace2014using} in
computing.  The latter example seeks to ``adopt methods from the field
of computational creativity [$\ldots$] to the generation of scientific
hypotheses.''  This provides a useful example of an effort focused on
computational \emph{invention}.  Another related area of contemporary
computing in which serendipitous events may be found is
bioinformatics: ``Instead of waiting for the happy accidents in the
lab, you might be able to find them in the data''
\cite[p.~70]{kennedy2016inventology}.

\ednote{Christian: Either erase or move shorter version into section 3 (Serendipity and Creativity)}As we indicated earlier, creativity and serendipity are often
discussed in related ways.  A further terminological clarification is
warranted.  The word \emph{creative} can be used to describe a
``creative product'', a ``creative person'', a ``creative process''
and even the broader ``creative milieu.''  Computational creativity
must take acount all of these aspects \cite{jordanous2016four}.  In
contrast, the model we have presented focuses only on serendipity as
an attribute of a particular kind of process.  Most often, we speak of
a system's \emph{potential} for serendipity.  In the current work, we
do not use the term to describe an artefactual property (like novelty
or usefulness), or a system trait (like skill).

Figueiredo and Campos \cite{Figueiredo2001} describe serendipitous ``moves'' from one
problem to another, which transform a problem that cannot be solved
into one that can.  
However, it is important to notice that progress with problems does not always mean transforming a
problem that cannot be solved into one that can.  Progress may also
apply to growth in the ability to \emph{posit} problems.  In keeping
track of progress, it would be useful for system designers to record
(or get their systems to record) what problem a given system solves,
and the degree to which the computer was responsible for coming up
with this problem.
%
As Pease et al. \cite[p. 69]{pease2013discussion} remark,
anomaly detection and outlier analysis are part of the standard
machine learning toolkit -- but recognising \emph{new} patterns and
defining \emph{new} problems is more ambitious (compare von Foerster's
\cite{von2003cybernetics} second-order cybernetics). \ednote{Christian: Remove rest of paragraph? Goes a bit too far..}
Establishing complex analogies between evolving problems and solutions
is one of the key strategies used by teams of human designers
\cite{Analogical-problem-evolution-DCC}.  In computational research to
date, the creation of new patterns and higher-order analogies is
typically restricted to a simple and fairly abstract ``microdomain''
\cite{hofstadter1994copycat,DBLP:journals/jetai/Marshall06}.
%
Turning over increased responsibility to the machine will be important
if we want to foster the possibility of genuine surprises.

The {\sf SerenA} system developed by Deborah Maxwell et
al.~\cite{maxwell2012designing} offers a case study in some
of these concepts.  This system is designed to support
serendipitous discovery for its (human) users
\cite{forth2013serena}.  The authors rely on a process-based
model of serendipity \cite{Makri2012,Makri2012a} that is derived
from user studies which draw on interviews with 28 researchers.
Study participants  were asked to look for instances of
serendipity from both
their personal and professional lives.  The research aims to
support the formation of bridging connections from an unexpected
encounter to a previously unanticipated but valuable outcome.
The theory focuses on the acts of reflection that support both
the creation of a bridge, and the estimation of the potential
value of the result.
%
While this description touches on all of the features of our model, {\sf
  SerenA} largely matches the description offered by \citet{andre2009discovery} of discovery-focused systems, in which
the user experiences an ``aha'' moment and takes the
creative steps to realise the result.  {\sf SerenA}'s primary computational method is to
search outside of the normal search parameters in order to engineer
potentially serendipitous (or at least pseudo-serendipitous)
encounters.

In sum, computer-supported serendipity, i.e. ``serendipity as a service'', has been well-studied, but purely computational serendipity has been much more constrained.\ednote{\textbf{R1}: a tendency to ramble off the topic slightly, pulling other interesting issues into the paper, such as the last paragraph on p.24, the paragraph starting "In sum..."}
This may partly be
due to the absence of clear criteria for serendipity, which we address
in the current paper.  Another issue is the widespread reliance on microdomains.  However, there are other underlying factors.
Existing standards for assessing computational creativity have
historically focused on product evaluations.
\citeA{ritchie07} uses metrics that depend on properties that a reasonably sophisticated judge can ascribe to generated artefacts:  ``typicality'', i.e., the extent to which an artefact belongs to a certain genre, and ``quality''.  These are used as atomic measures from which more complex metrics, including ``novelty,'' can be derived.  Most often, the judge is assumed to be a human.
%
In recent years, artefact-centred evaluations are increasingly
complemented by methods that consider process
\cite{colton2008creativity} or a combination of product and process
\cite{jordanous:12,colton-assessingprogress}.  However, processes that
arise outside of the control of the system (and ultimately, outside of
the control of the researcher) may still be deemed out of scope for
computational creativity \emph{per se}.  Unexpected external effects
may even be seen to ``invalidate'' research into computational
creativity.

We would argue that the concept of serendipity brings autonomous
creative systems into clearer focus: not with an abstract notion of
creativity \emph{sui generis}, but creativity in
interaction with the world.  This often requires a different mindset,
and a different approach to system building and evaluation.
%% \begin{quote}
%% ``\emph{Tinkering is a process of serendipity-seeking that does not
%%     just tolerate uncertainty and ambiguity, it requires it.  When
%%     conditions for it are right, the result is a snowballing effect
%%     where pleasant surprises lead to more pleasant surprises.}''
%%   \cite[``Tinkering versus Goals'']{rao2015breaking}
%% %% What makes this a problem-solving mechanism is diversity of individual perspectives coupled with the law of large numbers (the statistical idea that rare events can become highly probable if there are enough trials going on). If an increasing number of highly diverse individuals operate this way, the chances of any given problem getting solved via a serendipitous new idea slowly rises. This is the luck of networks.

%% %% Serendipitous solutions are not just cheaper than goal-directed ones. They are typically more creative and elegant, and require much less conflict. Sometimes they are so creative, the fact that they even solve a particular problem becomes hard to recognize. For example, telecommuting and video-conferencing do more to “solve” the problem of fossil-fuel dependence than many alternative energy technologies, but are usually understood as technologies for flex-work rather than energy savings.

%% %% Ideas born of tinkering are not targeted solutions aimed at specific problems, such as “climate change” or “save the middle class,” so they can be applied more broadly. As a result, not only do current problems get solved in unexpected ways, but new value is created through surplus and spillover. The clearest early sign of such serendipity at work is unexpectedly rapid growth in the adoption of a new capability. This indicates that it is being used in many unanticipated ways, solving both seen and unseen problems, by both design and “luck”.
%% \end{quote}
%% If we control the system, at bottom the best we can hope for is
%% ``pleasant unsurprises.''  At the same time, understanding serendipity
%% may help build autonomous systems that produce fewer ``unpleasant surprises,'' a
%% serious contemporary concern
%% \cite{philosophy-machine-morality,machine-ethics-status}.


\ednote{Christian: Remove paragraph. I agree with the reviewer: this does not relate closely enough to the main topic, and disrupts the flow.} Thus, serendipity is particularly relevant for thinking about
\emph{autonomous systems}.\ednote{\textbf{R1}: a tendency to ramble off the topic slightly, pulling other interesting issues into the paper
on p.25, the final paragraph of Sect 6.1}  There is a certain amount of apprehension
and concern in circulation around the idea of autonomous systems.
\citeA{machine-ethics-status} suggest that these concerns ultimately
come back to the question: will these systems behave in an ethical
manner?  The more we constrain the system's operation, the less chance
there is of it ``running off the rails.''  However, constraints come
with a serious downside.  Highly constained systems will not be able
to \emph{learn} anything very new while they operate.  If this means
that the system's ethical judgement is fixed once and for all, then we
cannot trust it to behave ethically when circumstances change
\cite{powers2005deontological}.  Highly constrained systems are
unlikely to be convincingly \emph{social}, inasmuch as the constraints
rule out emergent behaviour in advance.  Systems that only act
normatively (that is, pursuing purposes for which they have been
pre-programmed) serve as proxies for their creator's judgements, and
do not make \emph{evaluations} that are in any way ``their own.''
Adapting qualitative artefact-oriented measures (like Ritchie's
\cite{ritchie07}) may be necessary in order to build systems
that are capable of carrying out the necessary formative evaluation
steps that effect a focus shift -- as well as a final summative
evaluation of the result.  We return to this constellation of issues
related to system autonomy below.

%
% Ritchie initially bases his metrics on human judgment, but points out different ways to compute them automatically, arising from practical study.  For instance, quality could be computed using a fitness score of the assessed artifacts, which should highly correlate with human-perceived quality.  The typicality of produced artifacts was calculated as their similarity to the artifacts inspiring the generative process.  Nevertheless, this requires a good distant metric.  Both fitness functions and distance metrics are subject to an ongoing debate in computational aesthetics.

%% Although the notion of serendipity that we have developed is
%% process-focused, value is a crucial dimension of serendipity, and
%% evaluations of an outcome (often an artefact) continue to be relevant.
%% Furthermore, 


\subsection{Challenges for future research} \label{sec:recommendations}

\ednote{Christian: There's a lot of potential to cut in this section. I'd remove it entirely, if it wasn't also about the relationship of serendipity and CC. I'd suggest to trim this down to 1-2 paragraphs, and connect it with sec. 2 on serendipity, creativity and invention.} Reviewing the components of serendipity introduced in Section \ref{sec:by-example} and crystalised in the definition of serendipity
presented in Section \ref{sec:our-model} in light of the practice
scenarios discussed in Section \ref{sec:computational-serendipity}, we
can describe the following challenges for research in computational
serendipity.  The essential issues were drawn out in Section \ref{sec:related}, and are expanded here.\ednote{\textbf{R1}: a tendency to ramble off the topic slightly  …  much of the verbiage in Sect 6.2}

\paragraph{\textbf{Autonomy}.} Our case studies in Section
  \ref{sec:computational-serendipity} highlight the potential value of
  increased autonomy on the system side.  The search for connections
  that make raw data into ``strategic data'' is an appropriate theme
  for research in computational intelligence and machine learning to
  grapple with.  In the standard cybernetic model, we control
  computers, and we also control the computer's operating context.
  There is little room for serendipity if there is nothing outside of
  our direct control.  This mainstream model stands in contrast to von
  Foerster's \cite{von2003cybernetics} second-order
  cybernetics.  \citeA{research-priorities} advise researchers to
  consider \emph{verification}, \emph{validity} and \emph{security} as
  well as \emph{control}.  While we wouldn't advocate an uncautious
  approach, it must be pointed out that the dystopic scenarios
  surrounding loss of control may have corresponding utopic
  counter-scenarios; and in any event, we believe there is a more
  fundamental research problem.  \emph{A primary challenge for
    the serendipitous operation of computers is developing
    computational agents that specify their own problems.}

\paragraph{\textbf{Learning}.} Each of the case studies considered in
  Section \ref{sec:computational-serendipity} describes a system that
  is able, in one way or another, to learn from experience.  As we
  considered ways to enhance the measure of serendipity in these
  examples, we were led to consider computational agents that
  participate more meaningfully in ``our world'' rather than in a
  circumscribed microdomain.  Knowledge-intensive development work may
  often be unavoidable.  Understanding how to foster serendipity is a
  particularly important step, because it points to the potential of
  systems learning on their own.  \emph{A second challenge is for
    computational agents to learn more and more about the world we
    live in.}

\paragraph{\textbf{Sociality}.} We may be aided in our pursuit of the
  ``smart mind'' \cite{campbell2005serendipity} required for
serendipity by recalling Turing's proposal that computers should ``be
able to converse with each other to sharpen their wits''
\cite{turing-intelligent}.    The four
supportive factors for serendipity described in this paper -- a
\emph{dynamic world}, \emph{multiple contexts},
\emph{multiple tasks}, and \emph{multiple
  influences} -- resemble nothing more than social
reality.  In our analysis of {\sf GAmprovising} we suggested
that a future version of the system should interact more with the listener, and that individual
Improvisers should allow themselves to be influenced by each other,
rather than working in a digital silo.  \emph{A third challenge is for
  computational agents to interact in a recognisably social way with
  us and with each other, resulting in emergent effects.}

\paragraph{\textbf{Embedded evaluation}.}  \citeA{stakeholder-groups-bookchapter} outline a general programme
  for computational creativity, and examine perceptions of
  computational creativity among members of the general public,
  computational creativity researchers, and existing creative
  communities.  We should now add a fourth important ``stakeholder''
  group in computational creativity research: computer systems
  themselves.  System designers need to teach their systems how to
  make evaluations.  We saw that this is a crucial issue in designing
  an automated programming experiment.  The guiding light is a
  ``non-zero sum'' conception of value.  This can be extended to
  situations in which the the ``product'' is a new process or action.
  Within a Kantian framework ``an agent's moral maxims are instances
  of universally-quantified propositions which could serve as moral
  laws -- ones holding for any agent'' \cite{powers2005deontological}.
  Embedded evaluation has pragmatic as well as philosophical
  implications; thus, for example, the latest implementation of {\sf
    GAmprovising} is limited because it is ``poor at using reasoned
  self-evaluation'' and ``does not generate novel aesthetic measures''
  \cite[pp.~189, 288]{jordanous2012evaluating}.  \emph{A fourth
    challenge is for computational agents to evaluate their own
    creative process and products.}

