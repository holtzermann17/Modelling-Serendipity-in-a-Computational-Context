\section{Conclusions} \label{sec:conclusion} 

%% In this paper, drawing on previous literature on serendipity and
%% creativity (Section \ref{sec:literature-review}) we have assembled a
%% process-oriented model of serendipity in computational systems
%% (Section \ref{sec:our-model}). Table \ref{tab:model-summary-table} summarises the model.  We grounded each of the components in
%% existing cognitive science research, and pointed to existing work that
%% can plausibly be said to implement each of the individual components
%% (summarised in Table \ref{tab:model-summary-table}).  While it does
%% not seem that any one system yet combines all of the features, in
%% Section \ref{sec:computational-serendipity}, we show how our model
%% provides a roadmap for further implementation of systems with serendipitous potential.

%% \subsection{Serendipity in computational systems: Potential future applications} \label{sec:computational-serendipity}

%% We envisage various potential application areas for computational systems with serendipitous potential. 

%% %In this section, we first describe several areas in, or adjacent to,
%% %the field of computing, in which the phenomenon of ``serendipity''
%% %already plays a role.  We then focus on some specific directions for
%% %development of such systems.  We consider the system's autonomy and
%% %the user's involvement as more or less complementary variables.

%% Areas for serendipity of interest to artificial intelligence researchers
%%  include the following
%% \cite{aisbq-serendipity-symposium}, clustered here according to the
%% typical role assumed by a user, if any:

%% \[
%% \hspace{.12em}
%% \text{(A)}
%% \hspace{1em}
%% \begin{rcases*}
%% \text{AI planning} \\
%% \text{service-oriented computing} \\
%% \end{rcases*} \text{Little direct human involvement}
%% \]

%% \[
%% \text{(B)}
%% \hspace{1em}
%% \begin{rcases*}
%% \text{information encountering} \\
%% \text{interdisciplinary collaboration} \\
%% \text{information retrieval} \\
%% \text{physical computing} \\
%% \text{social platforms} \\
%% \text{bibliometrics} \\
%% \end{rcases*} \text{User or users at the centre\phantom{(s)}}
%% \]

%% \[
%% \hspace{-1.4em}
%% \text{(C)}
%% \hspace{1em}
%% \begin{rcases*}
%% \text{mining scientific literature} \\
%% \text{computer art} \\
%% \text{models of the research process} \\
%% %\end{rcases*} \begin{array}{c}\text{User or users as stakeholders}\\ \text{for the program} \end{array}
%% \end{rcases*} \text{User or users as stakeholders} 
%% \]

%% In Cluster (A), the user plays at most a minor role.  Serendipity, if
%% it occurs, is in some sense fundamentally within the computational
%% system.

%% In Cluster (B), the user, or users, tend to play an important role in
%% perceiving, or indeed in bringing about serendipity, but they are
%% supported by collaboration with the computer system as this process unfolds.  The
%% responsibilities associated with the components of our model are
%% distributed across both user and system.  In some cases it may be
%% possible to simulate user behaviour with an additional peripheral
%% system or systems, e.g., a population of computational agents that
%% interact on a social platform.  In such a case, the system may take
%% responsibility for more of the functions in our model, as well.

%% In Cluster (C), applications are being developed in which there is no
%% user role per se, but rather, humans become stakeholders in the computational process
%% \cite{stakeholder-groups-bookchapter}.  In particular, the
%% ``evaluation step'' is likely to be carried out from several
%% perspectives: serendipity, in terms of our model, may take place in
%% the system or its stakeholders, or both.

%\subsection{Overall summary}

In this paper our primary contributions have been:
\begin{enumerate}[label=(\arabic*),itemsep=0pt]
  \item An analysis of the literature on serendipity and creativity
    that draws out common themes.
  \item A model of serendipity potential inspired by predictive
    processing accounts of cognition, applied here to higher cognitive
    functions.
  \item An analysis of historical examples in light of our model,
    showing the feasibility of its computational realisation.
\end{enumerate}

In addition, we have discussed the relationship of our model to other
theoretical literature on serendipity, to AI ethics, and to
computational design.

To emphasise what \emph{serendipity in the system} brings to the table, consider the example of {\sf Max}
\cite{Figueiredo2001,campos2001searching}.  {\sf Max} is a classic example
of \emph{serendipity as a service}.  It modelled users'
interests as word vectors, extracted from emails; these were converted
to conceptual structures using WordNet; {\sf Max} then suggested new
web pages for the user to read.  {\sf Max} was capable of delivering,
albeit with low probability, recommendations deemed to be of
considerable value.  Examples of pseudoserendipitous and serendipitous
recommendations were produced \cite[p.~59]{Figueiredo2001}.
%%
However, {\sf Max} was not ``open [to] new lines of
investigation in the growth of knowledge''
\cite{swanson1997interactive}, and as such could not carry out
new use-inspired research to improve its performance.  For example,
{\sf Max} applied \emph{term frequency-inverse document frequency}
(tf-idf) to rank the concepts in each user-supplied document
\cite[p.~160]{campos2001searching}---but there is no chance that the
system, as architected, would decide to try reducing the dimensionality
of the associated vector space, and then use declustering (like {\sf
  Auralist} \cite{Zhang2011}) to see if this improved recommendation
quality.  In short, the conditions that led to the historically-significant
extension of tf-idf into \emph{latent
  semantic analysis} (LSA) are simply not modeled in {\sf Max}, even
though the program was built with a somewhat-similar problem in mind:
\begin{itemize}
\item Landauer, who pioneered LSA: ``the words that people wanted to
  use, to give orders to computers, or to look things up, rarely
  matched the words the computer understood'' \cite{landauer2003pasteur}.
\item Campos and Figueiredo, creators of {\sf Max}: ``Information retrieval usually
  assumes that the users know what they are searching for [but information can also be acquired] in an accidental, incidental, or
  serendipitous manner'' \cite{campos2001searching}.
\end{itemize}

% complements emerging environmental/ecological approaches to AI.
Rich functional models of operating domains will be necessary for
systems to recognise their own best and most interesting efforts, to
identify new problems, and to exploit serendipitous outcomes when they
occur.  While {\sf DAYDREAMER} met the basic requirements of our
model, it does not have a robust way to discriminate between
more and less interesting daydreams; nor can it adjust its view on
the world to take in new perceptions based on its creative process.
Similarly, while {\sf HRL} also met the basic requirements of the
model, to make discoveries of significant value it would need to be
revised to draw on a wider range of scientific and mathematical
knowledge.  Regarding computational serendipity,
\citet{pease2013discussion} suggested to ``proceed with caution in
this intriguing area.''  The current paper offers a considered view of
the issues at stake.

%% \begin{table}
%% \begin{tabular}{l} \\
%% \textbf{System-environment relationships differ widely, and develop differently.} \\
%% \textbf{Chance can play various roles in shaping perception.} \\
%% \textbf{The system has limited control.} \\
%% To create the possibility for varied patterns of inference to arise, support rich interfaces.  \\
%% To reduce constraints, allow features to be defined inductively.  \\
%% Organise and process perceptions differently depending on the tasks undertaken.  \\
%% \textbf{Adaptive attention is related to surprise.} \\
%% \textbf{Learning, context, and meaning begin to arise together with attention.} \\
%% \textbf{To some approximation, features of the environment will be attended to.} \\
%% Attention can be understood as competition for scarce processing resources.  \\
%% Attention can be time-delineated.  \\
%% Competition may be less natural when we can take advantage of parallelism.  \\
%% \textbf{Assess the data's potential for strategic usefulness.} \\
%% \textbf{Interest is related to curiosity.} \\
%% \textbf{Context change is a possible basis for belief revision.} \\
%% Interest can be linked to novelty in order to inspire learning.  \\
%% Interest can be linked to aesthetics in order to capture varied notions of fitness.  \\
%% Beauty is in the eye of the beholder.  \\
%% \textbf{A new model yields an improved ability to make a prediction.} \\
%% \textbf{There are different kinds of viable explanations.} \\
%% \textbf{The system creates an explanation of the event for itself.} \\
%% Experiments can have limited scope and still be useful.  \\
%% Given a sufficiently rich background, only a small amount of new data is needed.  \\
%% Learning is less efficient, but more widely applicable, than knowing. \\
%% Communication between agents can transfer causal information.  \\
%% \textbf{It is sometimes necessary or desirable to go beyond explanation.} \\
%% \textbf{Two cases: pseudoserendipity versus true serendipity: ``identifying'' or ``positing'' a problem.} \\
%% \textbf{The bridge is transformational.} \\
%% \textbf{A good problem can be identified by working at a meta-level.} \\
%% Similarity, analogy, and metaphor can be used to retrieve known problems.  \\
%% Concept blending may, but does not necessarily, help identify new problems.  \\
%% Working across domains can give rise to intriguing ideas.  \\
%% Experiments can give surprising insights.  \\
%% \textbf{Affection is based on reflection.} \\
%% Model a sense of taste. \\
%% Allow the system to use the world. \\
%% Allow the system to shape its own goals. 
%% \end{tabular} 
%% \caption{Summary of theoretical foundations (in bold) and heuristics for implementation from Section \ref{sec:our-model}\label{tab:collected-heuristics}}
%% \end{table}
