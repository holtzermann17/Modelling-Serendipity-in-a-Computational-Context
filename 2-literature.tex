\section{The structure of serendipitous occurrences}
\label{sec:literature-review}

To capture the intricate concept of serendipity in a model that is amenable to computational implementation, we first need a thorough understanding of the concept.  Our objective in this section is therefore to identify the factors common to existing theories of serendipity in one unified interpretation.  We will draw on related conceptualisations of \emph{creativity}, a theme that has drawn considerable attention in artificial intelligence research (cf.~\cite{boden1998creativity,colton2009computational,mccormack2012computers}).  


% This motivates our return to the theoretical literature and consider both the origins and development of the serendipity concept as foundation for our framework. We consequently start with a survey of definitions and the etymology of serendipity. We then describe how serendipity is intertwined with creativity, and eventually juxtapose our findings to derive the components of our framework. %Not even unambiguously defined within domain of recommender systems (cf. Lu et al., 2012). Most popular is definition by Herlocker et al.: „A serendipitous recommendation helps the user find a surprisingly interesting item he might not have otherwise discovered.“ (Herlocker et al., 2004, S. 43). McNeww et al. define serendipity as „the experience of receiving an unexpected and fortuitous item recommendation“ (McNee et al., 2006, S. 1099). Zhang et al. distinguish it from novelty by saying “unlike novelty, serendipity encompasses the semantic content of items, and can be imagined as the distance between recommended items and their expected contents.“ (Zhang et al., 2011, S. 16). 

% I must tell you a critical discovery of mine àpropos: in an old book of Venetian arms, there are two coats of Capello, who from their name bear a hat; on one of them is added a fleur-de-lis on a blue ball, which I am persuaded was given to the family by the Great Duke, in consideration of this alliance; the Medicis, you know, bore such a badge at the top of their own arms. This discovery I made by a talisman, which Mr. Chute calls the Sortes Walpolianæ, by which I find every thing I want, à pointe nommee, whenever I dip for it. This discovery, indeed, is almost of that kind which I call Serendipity, a very expressive word, which, as I have nothing better to tell you, I shall endeavour to explain to you: you will understand it better by the derivation than by the definition. I once read a silly fairy tale, called "The Three Princes of Serendip;" as their Highnesses travelled, they were always making discoveries, by accidents and sagacity, of things which they were not in quest of: for instance, one of them discovered that a mule blind of the right eye had travelled the same road lately, because the grass was eaten only on the left side, where it was worse than on the right — now do you understand Serendipity? One of the most remarkable instances of this accidental sagacity, (for you must observe that no discovery of a thing you are looking for comes under this description,) was of my Lord Shaftsbury, who, happening to dine at Lord Chancellor Clarendon's, found out the marriage of the Duke of York and Mrs. Hyde, by the respect with which her mother treated her at table. I will send you the inscription in my next letter; you see I endeavour to grace your present as it deserves. 

\subsection{Etymology and selected definitions}
The English term \emph{serendipity} derives from Horace Walpole's
interpretation of the first chapter of the 1302 poem \emph{Eight
  Paradises}---in a French translation of an intermediate Italian version of the Persian original---written by the Sufi
poet Am\={\i}r Khusrow \cite{van1994anatomy,remer1965serendipity}.  
%% JC: 13 Nov. Double check spelling vs the version in "Fluke",
%% add a reference
The term ``serendipity'' first appears in a 1757 letter from Horace
Walpole to his friend Horace Mann, wherein he describes the heroes of
this story ``making discoveries, by accidents \& sagacity, of things
which they were not in quest of''
\cite[pp.~407--408]{walpole1937yale}.
%% A related folk tale, of which
%% several exist \cite[p.~225]{mazur-fluke}, had informed Voltaire
%% \cite{huxley1903science} in \emph{Zadig}, published ten years earlier.
%% Hm... I wonder what discovery he was referring to?
Following Walpole's coinage, ``serendipity'' was mentioned in print
only 135 times over the next 200 years,
according to a survey carried out by Robert Merton and Elinor Barber, collected in \emph{The Travels and
  Adventures of Serendipity} \citep{merton}.  Merton described his own
understanding of a generalised ``serendipity pattern'' and its
constituent parts as follows:

\begin{quote}
``\emph{The serendipity pattern refers to the fairly common experience of observing an \emph{\textbf{unanticipated}}, \emph{\textbf{anomalous}} \emph{\textbf{and strategic}} datum which becomes the occasion for developing a new theory or for extending an existing theory.}''~\cite[p. 506]{merton1948bearing}~{[}emphasis in original{]}
%%% AJ I have reincorporated some of the below quote that had been commented out, because otherwise the terms used in the previous quote are quite opaque! I had to go look them up before realising we had the quote details all along
    %% The datum [that exerts a pressure for initiating theory] is, first of all, unanticipated. A research directed toward the test of one hypothesis yields a fortuitous by-product, an unexpected observation which bears upon theories not in question when the research was begun.
	%% Secondly, the observation is anomalous, surprising, either because it seems inconsistent with prevailing theory or with other established facts. In either case, the seeming inconsistency provokes curiosity; it stimulates the investigator to "make sense of the datum," to fit it into a broader frame of knowledge....
    %%And thirdly, in noting that the unexpected fact must be "strategic," i. e., that it must permit of implications which bear upon generalized theory, we are, of course, referring rather to what the observer brings to the datum than to the datum itself. %%For it obviously requires a theoretically sensitized observer to detect the universal in the particular....
    %% The serendipity pattern, then, involves the unanticipated, anomalous and strategic datum which exerts pressure upon the investigator for a new direction of inquiry which extends theory. 
\end{quote}
In Merton's account, the \emph{unanticipated} datum is observed while investigating some unrelated hypothesis; it is a ``fortuitous by-product'' (\emph{ibid}.). It is \emph{anomalous} because it is inconsistent with existing theory or established facts, prompting the investigator to try to unravel the inconsistency. The datum becomes \emph{strategic} when the implications of such investigations are seen to suggest new theories, or extensions of existing theories.

%% In 1986, Philippe Qu\'eau described serendipity as ``the art of
%% finding what we are not looking for by looking for what we are not
%% finding'' (\cite{eloge-de-la-simulation}, in the translation of
%% \citet[p. 121]{Campos2002}).  Campbell
%% \cite{campbell2005serendipity} defines it as ``the rational
%% exploitation of chance observation, especially in the discovery of
%% something useful or beneficial.''  Van Andel describes it simply 

Roberts \cite[pp.~246--249]{roberts} records 30 entries for the term ``serendipity'' from English language dictionaries dating from 1909 to 1989.
%
While classic definitions required an accidental discovery, as per Walpole, this criterion was modified
or omitted later on.  Roberts gives the name \emph{pseudoserendipity} to
``sought findings'' in which a desired discovery nevertheless
follows from an accident.
%%   This concept was later elaborated by
%% \citet{chumaceiro1995serendipity}.
\citet{Makri2012a,Makri2012b} point to a continuum between sought and
unsought findings, and highlight the role of subjectivity both in
bringing about a serendipitous outcome, and in perceiving a particular
sequence of events to be ``serendipitous.''
%% \Citet[p.~12]{Makri2012b} also highlight the role of subjective
%% interpretation in ascribing serendipity to a given situation.
%% , pointing
%% out that ``in extreme cases, as asserted by one of our interviewees,
%% `the same event might be serendipitous for me and a disaster for
%% someone else\rq\,{''} (\emph{ibid.}, p.~9).  While Walpole initially
%% described serendipity as an event (i.e., a kind of discovery),
Many of Roberts' collected definitions treat serendipity
as a psychological attribute: a ``gift'' or ``faculty.''
%% Along these lines,
%% %%  following Makri and Blandford, and \citet{de2014structure},
%% Jonathan Zilberg asserts:
%% \begin{quote}
%% ``\emph{Chance is an event while serendipity is a capability dependent
%%     on bringing separate events, causal and non-causal together
%%     through an interpretive experience put to strategic
%%     use.}''~\cite[p.~79]{zilberg2015embedded}
%% \end{quote}
%% % spacing of quotes: http://www.public.asu.edu/~arrows/tidbits/quotes.html
%% %Only one of Roberts' collected definitions defined it solely as an
%% %event, while five define it as both event and attribute.
%% Reflecting on the branching tree
%% of opportunities that formed the foundation of
%% his medical research career \citet[p.~54]{austin1978chase} remarks that there
%% ``have been more meanderings than searches.''
Merton and Barber argue for integrating a social psychology
perspective:
\begin{quote}
``\emph{For if chance favours prepared minds, it particularly favours
    those at work in microenvironments that make for unanticipated
    sociocognitive interactions between those prepared minds. These
    may be described as serendipitous sociocognitive
    microenvironments.}'' \cite[p.~259--260]{merton}
\end{quote}
Thus, consider that between Spencer Silver's creation of high-tack,
low-adhesion glue in 1968, Arthur Fry's invention of a sticky bookmark
in 1973, and the eventual launch of the distinctive canary yellow
re-stickable notes in 1980, there were many opportunities for
Post-its\textsuperscript{\textregistered} to \emph{not} have come to
be \cite{tce-postits}.  Umberto Eco \cite{eco2013serendipities} gives
several examples illustrating the serendipitous impact of mistakes,
falsehoods, and rumours on the production of knowledge.

\subsection{Theories of serendipity and related notions} \label{sec:serendipityInvention}
Serendipity is often discussed in the context of \emph{discovery}.  In
everyday parlance, this concept may be linked with \emph{invention} or
\emph{creativity} \cite{jordanous16plos}.  Henri Bergson made the
following distinction:
\begin{quote}
%% \emph{``La d\'ecouverte porte sur ce qui existe d\'ej\`a, actuellement
%%   ou virtuellement ; elle \'etait donc s\^ure de venir t\^ot ou
%%   tard. L'invention donne l'\^etre \`a ce qui n'\'etait pas, elle
%%   aurait pu ne venir jamais.''}
``\emph{Discovery, or uncovering, has to do with what already exists,
    actually or virtually; it was therefore certain to happen sooner
    or later.  Invention gives being to what did not exist; it might
    never have happened.}''    \cite[p. 58]{bergson1946creative}
\end{quote}
\citet{mckay-serendipity} draws on this Bergsonian distinction to
frame her argument about the role of serendipity in artistic practice,
where discovery and invention can be seen as ongoing and diverse.
The discovery of something unexpected in the world, followed by the
invention of an application for the same closely mirror the two-part
description of serendipity given by \citet{andre2009discovery}, namely
the ``chance encountering of information'' followed by ``the sagacity
to derive insight from the encounter.''

While definitions of creativity vary, two standard criteria are
variously given as ``novelty and utility,'' or ``originality and
effectiveness'' \cite{newell:63,boden,runco2012standard}.  With a
somewhat different emphasis, \citet{cropley2006praise} draws on
\citet{austin1978chase} to infuse his concept of creativity with
features of chance, and understands a creative individual to be
someone who ``stumbles upon something novel and effective when not
looking for it.''  However, Cropley questions ``whether it is a matter
of luck,'' because of the work and knowledge involved in the process
of forming an assessment of one's findings.  \citet{campbell1960blind}
contends that ``all processes leading to expansions of knowledge involve
a blind variation-and-selective-retention process.''
However, \citet[p.~49]{austin1978chase} remarks that: ``Nothing [suggests that]
you can blunder along to a fruitful conclusion, pushed there solely by
external events.''

Cs\'ikszentmih\'alyi describes creativity much along the lines of
Merton's unanticipated, anomalous, and strategic datum, as it arises
and develops in a social context.

\begin{quote}
``{[}C{]}\emph{reativity results from the interaction of a system
    composed of three elements: a culture that contains
   \emph{\textbf{symbolic rules}}, a person who brings
    \emph{\textbf{novelty}} into the symbolic domain, and a
    field of experts who recognize and
    \emph{\textbf{validate}} the innovation.}''
  \cite[p.~6]{csikszentmihalyi1997flow}~{[}emphasis added{]}
\end{quote}
%% Although there are common features to existing definitions of
%% creativity, and again, much in common with serendipity,
%% there is also much disagreement and discussion -- for example,
%% about the relevance of the social context.

In this case, novelty is attributed to ``a person'': even so, it is
reasonable to assume that this person's novel insights rely at least
in part on the observation of data.
%%
Cs\'ikszentmih\'alyi's three-part model of the creative process can be
compared with his five-part phased model, comprising
\emph{preparation}, \emph{incubation}, \emph{insight},
\emph{evaluation}, and \emph{elaboration}
\cite[pp.~79--80]{csikszentmihalyi1997flow} (adapting
\citet{wallas1926art}).  \citet{Campos2002} used this more elaborate
model to describe instances of serendipitous creativity.

This model is also a near match to the process-based model of
serendipity from \citet{lawley2008maximising}, centred on a sequence
of component-steps: \emph{prepared mind}, \emph{unexpected event},
\emph{recognise potential}, \emph{seize the moment}, \emph{amplify
  effects}, and \emph{evaluate effects}.  Lawley and Tompkins's model
introduces a feedback loop between ``recognising potential'' and
``evaluating effects'' that has no parallel in the
Wallas/Cs\'ikszentmih\'alyi model of creativity.  Using the feedback
loop, they make a connection to \emph{learning}: sometimes the process
``further prepares the mind.''

\Citet{Makri2012a} adapt Lawley and Tompkins's model, notably by
combining the ``prepared mind'' and ``unexpected event'' into one
first step, a \emph{new connection}, which involves a ``mix of
unexpected circumstances and insight.''  Expanding on the role of
feedback, they suggest that a process of reflection on the
``unexpectedness of circumstances that led to the connection and/or
the role of insight in making the connection'' is important for the
subjective identification of serendipity.  Projections of value can
also be updated when the new connection is exploited---for example,
when it is discussed with others.

`Insight' could play a role at several stages in the theories
discussed so far.  Klein and Jarosz use this term to denote
``discontinuous discoveries, that is, nonobvious inferences from the
existing evidence'' \cite[p.~335]{Klein2011}.  Out of the pool of
examples they studied, only 18\% involved an insight that was reached
by accident, although 92\% were accompanied by surprise (\emph{ibid.},
pp.~343--345).  These authors criticise the assumptions made by the
Wallas model as being unrealistic for many of the situations they
examined, and present an alternative framing whereby insight arises in
a response to inconsistency, from desperation, or from the discovery
of a new connection.  \citet[p.~6]{simon1958heuristic} had the rosy
view that ``Intuition, insight, and learning are no longer exclusive
possessions of humans: any large high-speed computer can be programmed
to exhibit them also.''  One approach to modelling insight with
computers focused on establishing ``an improved representation of an
important previously unsolved problem''
\cite[p.~118]{demystification}.

\citet{Allen:2013:LOD:2655780.2655790} studied how the term
`serendipity' and its various synonyms and related terms have been
used to describe opportunistic discovery in the biomedical literature.
Three categories of usage were particularly salient:
\emph{inspiration}, \emph{mentioned findings}, and \emph{research
  focus}.  A fourth category, \emph{systematic review}, highlighted
scholarly interest in the topic of serendipity itself.
%%
%\citet{mccay2015investigating} and subsequently
On this note, \citet{bjorneborn2017three} surveys several theoretical
treatments beyond those mentioned above, and extracts diverse personal
and environmental factors that can promote serendipity.  We will
engage with his work later on, but for now, we have enough material to
assemble themes in line with our objective.

 %% The simplest case is \emph{mentioned findings}, which briefly report phenomena that have arisen as a side-effect of research carried out with some other intended purpose. Allen et al distinguish this category from \emph{inspiration}, in which a serendipitous occurrence inspires a research study.  Lastly, \emph{research focus} refers to reports of research that are specifically ``devoted to reporting a serendipitous finding encountered during the conduct of research.''
%%inspiration: ``articles where the serendipity described forms the basis of the research design or seeks to further describe previously recorded ODI phenomena.''
%% mentioned findings: ``Mentioned findings were determined to be instances where the ODI findings described resulted from the study, but were not the major research outcomes. While the context and tenor of the term use is very similar to the statements that indicated a research focus on the ODI phenomenon, the location of the statements within the paper played a key role in determining whether the ODI incident was the focus of the paper or a mentioned finding. The statements related to a research focus were primarily located early in the paper, in abstracts, introductions and statements of purpose. Mentioned findings, however, were not seen until the results, discussion, or conclusion sections.''
%% research focus: ``When the entire article is devoted to reporting a serendipitous finding encountered during the conduct of research, diagnostics or medical therapeutics for another purpose, we considered the term usage to be research focus''
%% NB ODI = Opportunistic Discovery of Information

%%%%%%%%%%%%%%%%%%%%%%%%%%%%%%%%%%%%%%%%%%%%%%%%%%%%%%%%%%%%%%%%%%%%%%%%%%%%%%%%%%%%%%%%%%%%%%%%%%%%
%\setul{0pt}{.4pt}% 1pt below contents
\def\tabularxcolumn#1{m{#1}}
\newcolumntype{Y}{>{\centering\arraybackslash}X}
\begin{table}
{\centering\small\def\arraystretch{1.2}
%% scale the overall result
%% \hspace{-11cm}%
\begin{tabularx}{.98\textwidth}{Yc}  
\textbf{Serendipity is \raisebox{-.5ex}{$\cdots$}} & \phantom{(0)}\\[.5cm]
\end{tabularx}\offinterlineskip

\vspace{.2cm}
\noindent\begin{tabularx}{.98\textwidth}{Y@{\textbullet}Yc}  %\cline{1-2}
{chance encountering of information} & {sagacity to derive insight} & (1) \\
\end{tabularx}\offinterlineskip

\noindent\begin{tabularx}{.98\textwidth}{Y@{\textbullet}Yc}  %\cline{1-2}
 discovery &  invention & (2) \\
\end{tabularx}\offinterlineskip

%% \noindent\begin{tabularx}{.98\textwidth}{Y@{\textbullet}Y@{\textbullet}Yc}
%% %\cline{1-3}
%% symbolic rules & novelty & {validation}& (3)
%% \end{tabularx}\offinterlineskip

\noindent\begin{tabularx}{.98\textwidth}{Y@{\textbullet}Y@{\textbullet}Yc} 
%\cline{1-3}
findings & inspiration & research focus & (4) \\
\end{tabularx}\offinterlineskip

\noindent\begin{tabularx}{.98\textwidth}{Y@{\textbullet}Y@{\textbullet}Y@{\textbullet}Yc} 
%\cline{1-4}
unanticipated datum & anomalous datum & {strategic datum} & {new\slash modified theory} & (5) \\
\end{tabularx}\offinterlineskip

%% \noindent\begin{tabularx}{.98\textwidth}{Y@{\textbullet}Y@{\textbullet}Y@{\textbullet}Y@{\textbullet}Yc}  %\cline{1-5}
%% preparation & incubation & insight & evaluation & {elaboration} & (6) \\
%% \end{tabularx}\offinterlineskip

\noindent\begin{tabularx}{.98\textwidth}{Y@{\textbullet}Y@{\textbullet}Y@{\textbullet}Y@{\textbullet}Y@{\textbullet}Yc}  %\cline{1-6}
prepared mind & unexpected event & recognise potential & seize the moment & \textcolor{white}{xx}amplify\newline effects & \textcolor{white}{x}evaluate\newline effects & (7) \\
\end{tabularx}\offinterlineskip

\noindent\begin{tabularx}{.98\textwidth}{Y@{\textbullet}Y@{\textbullet}Y@{\textbullet}Y@{\textbullet}Y@{\textbullet}Yc}  %\cline{1-6}
\multicolumn{2}{p{.2787\textwidth}@{\textbullet}}{\begin{minipage}{.2787\textwidth}
{\centering new connection

\par}
  \end{minipage}} & project value & \textcolor{white}{xx}exploit\newline connection & valuable outcome & \textcolor{white}{x}reflect\newline on value & (8)
\end{tabularx}\offinterlineskip

\vspace{0cm}
\raisebox{1cm}{\noindent\begin{tabularx}{.98\textwidth}{YYYYYYc}
%% \Cline{1-6}{2pt}
$\downarrow$&$\downarrow$&$\downarrow$&$\downarrow$&$\downarrow$&$\downarrow$&\\[-.3cm]
%\shifttext{1.5em}{$\uparrow$}\newline
%\textbf{\emph{perception} \mbox{of a}} \shifttext{-1.6em}{\mbox{\textbf{chance event}}} &
%\shifttext{1.2em}{$\uparrow$}\newline \textbf{\emph{attention} \mbox{to salient} \mbox{detail}} &
%\shifttext{1.4em}{$\uparrow$}\newline \textbf{\emph{interest} \mbox{achieves a} \mbox{focus shift}} &
%\shifttext{1.4em}{$\uparrow$}\newline \textbf{\shifttext{-.7em}{\emph{explanation}} \mbox{of the} \mbox{event}} &
%\shifttext{1.2em}{$\uparrow$}\newline \textbf{\emph{bridge} \mbox{to a} problem} &
%\shifttext{1.2em}{$\uparrow$}\newline \textbf{\emph{valuation} \mbox{of the} \mbox{result}} & (9) \\ 
\shifttext{1.5em}{}\newline
\textbf{\emph{perception} \mbox{of a}} \shifttext{-.6em}{\mbox{\textbf{chance event}}} &
\shifttext{1.2em}{}\newline \textbf{\emph{attention} \mbox{to salient} \mbox{detail}} &
\shifttext{1.4em}{}\newline \textbf{\shifttext{-.2em}{\mbox{\emph{focus shift}}} \mbox{achieved} \shifttext{-.5em}{\mbox{by interest}}} &
\shifttext{1.4em}{}\newline \textbf{\shifttext{-.5em}{\emph{explanation}} \mbox{of the} \mbox{event}} &
\shifttext{1.2em}{}\newline \textbf{\emph{bridge} \mbox{to a} problem} &
\shifttext{1.2em}{}\newline \textbf{\emph{valuation} \mbox{of the} \mbox{result}} & \raisebox{-.1cm}{(9)} \\ 
\end{tabularx}}

\vspace{-.4cm}
\shifttext{-2.5em}{$\underbrace{\text{\phantom{XXXXXXXXXXXXXXXXXXXXXXXXXXXXXXXXXXXXXXXXXXXXXXXXXXXXXXXXXX}}}$}
\vspace{.2cm}

\begin{tabularx}{.98\textwidth}{Yc} %%\cline{1-1}
\textbf{All of which are operations of a \emph{prepared mind} subject to \emph{chance}.} & \phantom{(0)}\\
 %%\cline{1-1}
\end{tabularx}\offinterlineskip

\par}
\caption{Aligning ideas from several theories of serendipity and
  creativity.  Rows 1-7 show increasing detail, moving from two to six
  phases; row 8 bundles two of the steps together; row 9 summarises our
  analysis and
  provides the framework for Section \ref{sec:our-model}. Sources: (1) 
 \citet{andre2009discovery}; (2) \citet{mckay-serendipity}, citing \citet{bergson1946creative}; (4) \citet{Allen:2013:LOD:2655780.2655790}; (5) \citet{merton1948bearing}; (7) Lawley and Tompkins \cite{lawley2008maximising}; (8) Makri and
  Blandford \cite{Makri2012a}. \label{tab:theory-summary}}
\end{table}
%%%%%%%%%%%%%%%%%%%%%%%%%%%%%%%%%%%%%%%%%%%%%%%%%%%%%%%%%%%%%%%%%%%%%%%%%%%%%%%%%%%%%%%%%%%%%%%%%%%%


%% JC: Added a sentence describing the role of the different elements
%% as per Alison's advice, Nov 13 2016
\subsection{Summary} \label{sec:literature-summary}

Our review of significant literature on serendipity has exposed the
key phases of `serendipitous' system operation.  Highlights are
summarised in Table \ref{tab:theory-summary}, where we have aligned
the concepts discussed by previous authors.  We distil and collect six
cognitively plausible processing steps: perception of a chance event,
attention to salient detail, focus shift achieved by interest,
explanation of the event, bridge to a problem, valuation of the
result.  We will define these steps in the following section.  For
now, we summarise our current understanding that serendipity is a form
of creativity that happens in context, on the fly, with the active
participation of a creative agent, but not entirely within that
agent's control.

% Feedback...
