\section{Attributes of serendipity} \label{attributes}

Building on our summary of the literature in Table
\ref{tab:theory-summary}, this section introduces the concepts that
inform our model and evaluation procedure in Section
\ref{sec:our-model}. We illustrate each concept by means of historical
examples and consider what they might mean in a computational context.
%%
The elements of Table \ref{tab:theory-summary} can be loosely separated into categories insofar as they concern a discoverer, a discovery, or the environment in which discovery takes place.


We follow the divisions and define relevant attributes for each below: discovery ($\S$\ref{discovery}), discoverer ($\S$\ref{discoverer}) and
environment ($\S$\ref{environment}).

%\section{Defining serendipity in a computational context}

\subsection{Attributes of a serendipitous event}\label{discovery}

\begin{itemize}
\item \textbf{Trigger}: a piece of data without which a result would
  not have been found. The trigger does not directly cause the
  outcome, but rather, inspires a potentially new insight.  It was
  long known by Quechua medics that cinchona bark stops shivering.  In
  particular, it worked well to stop shivering in malaria patients, as
  was observed when malarial Europeans first arrived in Peru
  \cite[pp.~75--77]{desowitz1997gave}.  The joint appearance of
  shivering Europeans and a South American remedy was the trigger.
  That an extract from cinchona bark can cure and can even prevent
  malaria was learned subsequently.
\end{itemize}

\begin{itemize}
\item \textbf{Bridge}: the path and set of mechanisms used to get from
  trigger to result.  Mechanisms often include reasoning techniques,
  such as abductive inference (what might cause a clear patch in a
  petri dish?), analogical reasoning (de Mestral constructed a target
  domain from the source domain of burrs hooked onto fabric), or
  conceptual blending (Kekul\'e, discoverer of the benzene ring
  structure, is said to have blended his knowledge of molecule
  structure with his dream image of a snake biting its tail, although
  this story is probably apocryphal
  \cite[p.~xv]{kennedy2016inventology}).  The bridge may rely on new
  social arrangements or physical prototypes.  It may have many steps,
  and may feature chance elements.  Several serendipitous episodes may
  be chained together in sequence, so that the result of one episode
  serves as the trigger for the next, and this sequence as a whole
  forms the bridge to a truly unprecedented result.  For example,
  C\'edric Villani \cite[pp.~15--16]{birth-of-a-theorem} describes
  two surprising conversations that happened in one day early on in
  his work on the Boltzmann equation.  The first conversation was with
  Freddy Bouchet, about the way galaxies stabilise,
%-- ``I was thrilled to see Landau damping suddenly make another
%-- appearance, scarcely more than a week after my discussion with
%-- Cl\'ement [Mouhot]''--
   and the second with his colleague \'Etienne Ghys, who provided an
   unexpected link from the content of the first conversation to ideas
   in Komolgorov-Arnold-Moser theory: ``I didn't really want to say
   anything, C\'edric, but those figures there on the board -- I've
   seen them before.''
\end{itemize}


\begin{itemize}
\item \textbf{Result}: the new product, artefact, process, theory, use
  for a material substance, or other outcome.  The outcome may
  contribute evidence in support of a known hypothesis, or a solution
  to a known problem.  Alternatively, the result may itself {\em be} a
  new hypothesis or problem.  The result may be
  ``pseudoserendipitous'' in the sense that it was {\em sought}, while
  nevertheless arising from an unexpected source.  More classically,
  it is an \emph{unsought} finding, such as the discovery of the
  Rosetta stone.
\end{itemize}

\begin{mdframed}
{\bf Definition}

A {\em potentially serendipitous event} is one which can be decomposed
into Trigger, Bridge and Result.
%[do we want FS in here?]
\end{mdframed}


\subsection{Attributes of a serendipitous person or
  system}\label{discoverer}
We measure the serendipitous potential of a person or system by its
knowledge and abilities. 

\begin{itemize}
\item \textbf{Prepared mind}: this concerns the internal environment
  of a person or system, its knowledge and input data. Alexander
  Fleming's ``prepared mind'' included his focus on carrying out
  experiments to investigate influenza as well as his previous
  experience that showed that foreign substances in petri dishes can
  kill bacteria.  He was concerned above all with the question ``Is
  there a substance which is harmful to harmful bacteria but harmless
  to human tissue?''  \cite[p. 161]{roberts}. A ``prepared mind'' is
  not something that systems either have or not: we can describe the
  contents of any system in terms of prepared mind. However, we can
  identify aspects that are helpful in increasing the likelihood of
  serendipitous discoveries. For instance, maintaining a set of open
  problems, half-formed solutions and unexpected as well as expected
  results are features of the prepared minds in the human context. The
  prepared mind will normally develop over a session of a system run,
  so a trigger might be missed at the start of a run where the same
  system might have done a focus shift on the same trigger later on in
  its run.
\end{itemize}

%\item Bridging techniques. These are a subset of the prepared mind and
 % constitute those techniques actually used to  

\begin{itemize}
\item \textbf{Focus Shift}: this is an ability to make an assessment
  or reassessment -- a focus shift in which something that was
  previously unnoticed or uninteresting, of neutral, or even negative
  value, becomes interesting.  In one famous example, George de
  Mestral, an electrical engineer by training, and an experienced
  inventor who worked in a machine shop, returned from a hunting trip
  in the Alps. He removed numerous burdock burrs from his clothes and
  his dog's fur and became curious about how they worked. After
  examining them under a microscope, he realised the possibility of
  creating a new kind of fastener that worked similarly, laying the
  foundations for the hook-and-loop mechanism in Velcro\texttrademark\
  \cite{roberts}. 

  In the human situation this ability is exhibited when people take
  consequential note of anomalies to arrive at unanticipated
  discoveries. For instance, consider a situtation in which a given
  object in a given context is evaluated as poor, according to given
  evaluation criteria. A focus shift occurs when a second context is
  retrieved or constructed, in which the same object is evaluated more
  highly, or when the evaluation criteria are changed so that the
  object in its original context is now evaluated more highly.
\end{itemize}
%\item Curiosity [include or omit?]

\begin{itemize}
\item \textbf{Sagacity}: connects both the knowledge and abilities on
  the part of the discoverer. This old-fashioned word is related to
  ``wisdom,'' ``insight,'' and especially to ``taste''t.  Merton
  \cite[p.~507]{merton1948bearing} writes: ``{[}M{]}en had for
  centuries noticed such `trivial' occurrences as slips of the tongue,
  slips of the pen, typographical errors, and lapses of memory, but it
  required the theoretic sensitivity of a Freud to see these as
  strategic data through which he could extend his theory of
  repression and symptomatic acts.''  The degree to which such data
  are \emph{prima facie} unanticipated and anomalous is clear.  Merton
  would be prepared to take in stride that Freud's claims surrounding
  this data are part of ``an idealized story'' \cite{freudtheory}.
  For Merton ``what the observer brings to the datum'' is an essential
  aspect of strategy; his key criterion is that the result ``must
  permit of implications which bear upon generalised theory'' -- not
  that it be correct. As well as including the knowledge in the
  prepared mind, and the ability to focus shift and to form the bridge
  between the trigger and the result, this includes enhanced
  evaluation capacities: initial evaluation and re-evaluation, ability
  to generate and evaluate between possible directions that developing
  the trigger might lead, and ability to evaluate the result and
  recognise that it is good.
\end{itemize}

\begin{mdframed}
{\bf Definition}

A {\em potentially serendipitous person or system} is one which can be described as having a Prepared Mind, ability to Focus Shift, and Sagacity.
\end{mdframed}


\subsection{Attributes of a serendipitous
  environment}\label{environment}
Serendipity seems to be more likely for agents who experience and
participate in a \emph{dynamic world}, who are active in
\emph{multiple contexts}, occupied with \emph{multiple tasks}, and who
avail themselves of \emph{multiple influences}.

\begin{itemize}
\item \textbf{Dynamic world}: Information about the world develops
  over time, and is not presented as a complete, consistent whole.  In
  particular, \emph{value} may come later.
  \Citet[p. 643]{van1994anatomy} estimates that in twenty percent of
  innovations ``something was discovered before there was a demand for
  it.''  To illustrate the role of this factor, it may be most
  revealing to examine a remarkable ``near miss,'' in which the state
  of the world changed, but the dynamics and concomitant implications
  were not attended to carefully.
%%
  \citet[pp.~75-76]{cropley2013creativity} describe the case of Eugen
  Semmer, a veterinary pathologist who intended to carry out a post
  mortem analysis on two unwell horses.  As it turned out, there was
  just one problem with his plan: ``when he arrived in the morning he
  discovered that the animals had unexpectedly and inexplicably
  recovered.''  %% With some consternation, he then determined that
%%   ``their recovery was linked to the unintended presence of spores of
%%   the fungus \emph{penicillium notatum} in his laboratory.''  He
%%   proceeded to test this theory with further \emph{in vivo}
%%   experiments on other animals.
%% \begin{quote}  ``\emph{However, apparently blinded by the
%%   narrow nature of his special knowledge {\upshape\ldots}\ he did not
%%   recognise that he had stumbled on an important life-saver (what we
%%   now call `antibiotics'), and instead went to considerable lengths to
%%   eradicate the spores from his laboratory.}''
%% \end{quote}
%% In this example it appears that a focus shift was effected, since
%% Semmer did give the perplexing fungus further attention: however a
%% bridge to a potentially valuable result was not formed, despite of a
%% growing body of evidence.  This example reveals that the mind needs to
%% be prepared to cope effectively with change.
Semmer effected a focus shift and figured out the cause of this change
of affairs -- \emph{penicillium notatum} spores -- but he failed to
establish a bridge to a valuable result because he was missing the
important ability \cite{bereiter1997situated} to revise his basic
approach in response to changes in the underlying situation.
\end{itemize}

\begin{itemize}
\item \textbf{Multiple contexts}: One of the dynamical aspects at play
  may be the discoverer/inventor going back and forth between
  different contexts with different stimuli and affordances.  3M
  employee Arthur Fry sang in a church choir and needed a good way to
  mark pages in his hymn book -- and happened to have been recently
  attending internal seminars offered by his colleague Spencer Silver
  about restickable glue \cite{tce-postits}.
\end{itemize}

%% Einstein's work at the patent office seems to have been fortuitous
%% not because it gave him ideas, but because it gave him time to work
%% on his ideas.  Famously, this resulted in four fundamental papers
%% in the year 1905.

\begin{itemize}
\item \textbf{Multiple tasks}: Two decades after his \emph{annus
  mirabilis}, Einstein's fame drew letters and a copy of a rejected
  paper from Indian physicist Satyendra Nath Bose.  As a recent Nobel
  Prize winner and an institute director, Einstein was in no way
  obliged to take on the extra task of translating Bose's paper from
  English to German.  Nevertheless, he did so out of interest, and in
  the process learned a calculation method that produced accurate
  physical results, despite making nonstandard physical assumptions
  \cite{delbruck1980bose}.  Einstein's subsequent examination of this
  work led to further papers and the idea of Bose-Einstein statistics,
  which describes fundamental particles, now known as bosons, which do
  not obey the Pauli exclusion principle.
\end{itemize}

\begin{itemize}
\item \textbf{Multiple influences}: The bridge from trigger to result
  is often found by availing oneself of others' points of view.  For
  example, Arno Penzias and Robert Wilson, working at Bell Labs, used
  a large antenna to detect radio waves that were relayed by bouncing
  off satellites.  After they had carefully removed interference
  effects due to radar, radio, and heat, they found residual ambient
  noise that couldn't be eliminated.  They then ruled out directional
  effects that would suggest either a terrestrial or galactic origin
  \cite[p.~3]{lachieze1999cosmological}.  They were mystified, and
  only understood the significance of their work after a friend at MIT
  told them about a preprint on the subject written by astrophysicists
  from nearby Princeton University, who had hypothesised the
  possibility of measuring radiation released by the big bang
  \cite[p.~385]{shu1982physical}.
\end{itemize}


\begin{mdframed}
{\bf Definition}

A {\em potentially serendipitous environment} is one which can be
described in terms of a dynamic world, and multiple contexts, in which
agents are concerned with multiple tasks, and have multiple
influences.
\end{mdframed}


\subsection{Serendipity space: uniting our three dimensions}

The [four] components described above have attributes that may be
present to a greater or lesser degree.  These are: \emph{Chance} --
how likely was the trigger to appear?; \emph{Curiosity} -- how likely
was this trigger to be identified as interesting?; \emph{Sagacity} --
how likely was it that the interesting trigger would be turned into a
result?; -- and \emph{Value} (how valuable is the result that is
ultimately produced?). We unite these with the notion of {\bf
  serendipity space}, in which events may lie.

{\em serendipity space} is the space defined by the three dimensions
chance, sagacity and value. We measure {\em position in serendipity
  space} by where in serendipity space a potentially serendipitous
event lies.


\begin{itemize}
\item \textbf{Chance}: relates to the environment: how likely was the
  trigger to occur in that context? Fleming \cite{fleming} noted:
  ``There are thousands of different moulds'' -- and ``that chance put
  the mould in the right spot at the right time was like winning the
  Irish sweep.''  It is important to notice that \emph{he} was in the
  right spot at the right time as well -- this was not a complete
  coincidence.
\end{itemize}

\begin{itemize}
\item \textbf{Sagacity} relates to the discoverer: how sophisticated
  was the insight used to see the relevance of the trigger (the focus
  shift), and the skill and knowledge it took to get from the trigger
  to the result)?
\end{itemize}


\begin{itemize}
\item \textbf{Value}: relates to the result, or discovery. Serendipity
  concerns happy surprises, but different parties may have different
  judgements as to whether a given situation is ``happy'' or
  ``surprising''.  A third party judgement of value can help to
  discriminate between luck, sleight of hand, and \emph{bona fide}
  value creation.  An unconcerned third party is more likely to see
  serendipity when ``One man's trash is another man's treasure'' than
  when ``One man's loss is another man's gain.''  In order to clearly
  distinguish between these two cases, wherever possible we prefer to
  make use of independent judgements of value.  A literal example of
  the trash-to-treasure scenario is provided by the Swiss company
  Freitag, which was started by design students who built a business
  around ``upcycling'' used truck tarpaulins into bags and backpacks.
  Thanks in part to clever marketing \cite[pp. 54--55,
  68--69]{russo2010companies}, their product is now a global brand.
\end{itemize}

%is a dimension (- how good is it?)


\begin{mdframed}
{\bf Definition}

A point in {\em serendipity space} is a {\em potentially serendipitous
  event} along which dimensions of Chance, Sagacity and Value have
been defined. 
\end{mdframed}


%how is context different to prepared mind etc? 


\subsection{A step-by-step example illustrating how these concepts can
  be applied to computational systems} \label{sec:by-example-summary}

In Sections \ref{sec:our-model} and
\ref{sec:computational-serendipity}, we will show how the key
condition, components, dimensions and environmental factors of
serendipity discussed above can be used to model and assess the
potential for serendipity in computational systems.
%%
Here, we develop a preliminary illustration showing how these criteria can be
applied to separate computational examples into instances, non-instances, and
weak instances.
We will consider three systems in the tradition of ``computational
discovery in mathematics'' \cite{colton2007computational}.
%%   These system descriptions illustrate two central points: namely that
%% all of the components are necessary in order for a system to have
%% potential for serendipity, and that they can be active to a greater or
%% lesser degree in different systems.  In addition, systems do not
%% require all (or even any) of the supportive environmental factors to
%% achieve serendipitous results, but that the presence of these factors
%% can go along with several advantages.

\paragraph{System A.~Zero potential for serendipity -- Automatic
  theorem proving}
A user of an automatic theorem proving system typically has in mind
the theorem for which he or she wishes to establish a formal proof.
That is, an informal proof already exists, and when translating this
into the formal language, only minor logical and syntax errors stand
in the way.  These can be straightforwardly debugged.  Once the proof
has been fully specified, the theorem prover will return a
certification.  There seems to be no chance for serendipity here, on
either the user or the system side.  Even if we were to construe an
erroneous formal proof as a \textbf{trigger} and a corresponding error
message as a \textbf{result}, the other components all fail to
materialise.  Nothing can be generated from this trigger except for
the (to-be-expected) error message.
%% Furthermore, if the user
%% happens to be surprised by the error message, they are unlikely to be
%% particularly happy about it.

\paragraph{System B.~Moderate potential for serendipity -- Generating
  conjectures and proofs}
Suppose that rather than checking a known theorem, the user programs
the computer to come up with conjectures, and generate proof-attempts
on its own.  Furthermore, suppose that three out of 100 generated
conjectures turn out to be provable.  In this case, the user may be
interested and pleasantly surprised -- especially if one or more of
the theorems is one that he or she wouldn't have thought of.  In this
case, each of the generated conjectures is a potential
\textbf{trigger} for discovery.  Note that although the system itself
generates these conjectures, in general it cannot determine in advance
which ones, if any, will turn out to be true.  Its ability to assess
the generated conjectures and to construct proof-attempts constitutes
an elementary \textbf{prepared mind}.  The system is (fallibly) able
to apply pre-programmed methods to form a \textbf{bridge} to an
interesting \textbf{result}, namely, a new theorem.  Indeed,
fallibility applies twice over: not only may the system fail to find
the interesting conjectures, it may also fail to find proofs for all
of the (true) conjectures that it does discover.  Furthermore, note
that due to its limited domain knowledge, the system has only a weak
model of the way \textbf{value} will be assigned to any theorem it
finds.
%% Conjectures arise through some
%% combinatorial or other similar process, which, although deterministic,
%% selects a sample from the population of available conjectures.
% Although the conjectures are presumably deterministically constructed,
% they may appear random; more specifically, i
In lieu of further information about the conjecture generation process
(and about those conjectures which are not generated)
it seems that every potential trigger for discovery is
encountered by definition, so that \textbf{chance} does
not play a significant role at that stage.
% \footnote{For example, in lieu of further information
%  about the conjecture-generating algorithm, we might regard a
%  conjecture's ``order of arrival'' as a proxy for its probability of
%  being encountered, after a suitable normalisation.  So that if $n$
%  conjectures are selected, the $j$th conjecture could be assigned the
%  ``subjective probability'' $2(n+1-j)/n(n+1)$.}
However, the system may make use of simple heuristics -- based, for
example, on a computed \emph{plausibility measure}
\cite[p.~193]{colton2007computational} -- to keep it from focusing on
a conjecture that it isn't likely to prove, so it is capable of a
somewhat discriminating form of \textbf{curiosity}.  It will effect a
\textbf{focus shift} to each plausible trigger independently, in turn.
This conservative behaviour also contributes to the system's
\textbf{sagacity}, which is otherwise grounded in proof-generation
techniques.  This system matches all of our criteria for serendipity,
although we should stress that its ability to generate new mathematics
will depend, in part, on the initial selection of problem domain --
and, to a considerable extent, on the programmer's ingenuity.  Among
the environmental factors from Section
\ref{sec:environmental-factors}, this system matches the description
of \textbf{multiple tasks}, but not the others.

\paragraph{System $B^{\prime}$.~Low potential for serendipity -- Zooming in on part of the process}

Here we focus in on the part of \emph{System B} that generates
conjectures, without considering proof attempts.  This was the
historical course initially taken by the {\sf HR} project with the
{\sf NumbersWithNames} program \cite{colton2002numberswithnames}.
%%  --
%% although subsequent versions of the system worked with third-party theorem
%% provers to generate proofs \cite{colton2002hr}.
%%
Intuitively, {\sf NumbersWithNames} can help with ``the discovery
part'' of mathematics \cite[p.~7]{colton2002numberswithnames}.
%%
The \textbf{trigger} for this system was a given integer sequence,
which may have been chosen at random or hand-selected by a user.  An
interesting conjecture (\emph{sans} proof) about the sequence is
considered to be a valuable, albeit preliminary, \textbf{result}.  A
case can be made for the system possessing an entirely nonexceptional
form of \textbf{curiosity}: each trigger is submitted for further
processing, in this case via a range of transformation rules that
explore outwards from the triggering sequence to discover potential
statements that can be made about it.  However, examining the
algorithms used by {\sf NumbersWithNames}, the case for
\textbf{sagacity} initially seems rather weak.  In the first place:
``Even after pruning, the program often produces a plethora of
conjectures'' \cite[p.~4]{colton2002numberswithnames}.  Identifying
the \emph{plausible} conjectures among these requires some further
common-sense ideas and straightforward numerical processing.
Naturally, filtering the results list cannot guarantee that any of the
generated plausible results will actually be interesting.  Here it is
worth emphasising that, in practice, many of the interesting results
from {\sf NumbersWithNames} were found based on intelligent problem
selection on the part of the system's users, who were able to supply a
preselected sequence of interest.  Ultimately, the fact that {\sf
  NumbersWithNames} could surface interesting conjectures about these
sequences suggests that it is sufficiently, if minimally, sagacious,
after all.
%% Moreover, work with this system led to publishable mathematical
%% results from \citet{colton1999refactorable} and others.
Automated problem selection is one of the key advantages \emph{System
  B} obtains by using \emph{System $B^{\prime}$} as a submodule.  As
we saw above, this allows \emph{System B} to be somewhat
discriminating in its curiosity, whereas \emph{System $B^{\prime}$}
seems to have relied alternately on outside help or luck to steer its
initial processing.
%% :\emph{System B}'s ``hit rate'' of 3\% would be much worse if the
%% conjectures did were not filtered.

\paragraph{System C.~High potential for serendipity -- Mining an online domain model} 
In a more futuristic and entirely hypothetical example, we can imagine
a system has at its disposal a large database of formalised proofs,
assorted mathematical concepts, and informal heuristics.
Additionally, suppose that new data in a machine-accessible format is
coming online all the time -- perhaps the system deploys a
next-generation parser on new papers as they are added to the
mathematics Arxiv (cf.~\citet{ginev2009architecture}).  Such a system
could have a large collection of open problems that it is working on
at any given moment, which, together with the aforementioned facts and
heuristic patterns, constitute a considerably more robust
\textbf{prepared mind} than in the previous system.  This system could
take a discriminating approach to generating conjectures, and apply a
range of mathematical techniques to find a \textbf{bridge} from a
conjecture to a proof.
%%
Each new paper or fragment of user interaction it encounters would constitute
a potential \textbf{trigger} for discovery.  Some of these contributions will have
more generative potential than others.  Importantly, the system would
be able to judge for itself whether a given \textbf{result} is globally
new.  The new data expresses highly unpredictable mathematical
content, so \textbf{chance} plays a prominent role in this 
system.  However, like the previous system, this one is fallible: for
all its background knowledge, there is no guarantee that it will find
any worthwhile results on any given day.  Its ``hit rate'' will depend partly on the quality of
the search strategies it uses.  It would be straightforward to
characterise the system's search priorities using the dimension of
\textbf{curiosity}.  Again, the system could afford to be
discriminating, with its allocation of attention driven by an interest in
specific problems.
%%
The system's heuristics for solving these problems would
straightforwardly connect with the dimension of \textbf{sagacity}.
Adding a further layer on top of this, higher-order programming could
be applied that would operationalise the search for new strategies
and heuristics.  The system is clearly situated in a \textbf{dynamic
  world}.  It can avail itself of \textbf{multiple influences} by
reading papers from different mathematical domains.  Switching
attention between proving new theorems and developing new search
strategies and problem solving heuristics would give the system
\textbf{multiple tasks} and \textbf{multiple contexts} for creativity.

\bigskip

The foregoing system descriptions illustrate two central points:
namely that all of the components are necessary in order for a system
to have potential for serendipity, and that they can be active to a
greater or lesser degree.  We will make these assertions more precise
in the following section.




