\subsection{In pursuit of serendipity: An brief account of progress with computational discovery systems} \label{sec:pursuit}

In this section we will consider landmarks in a historical sequence of
development of discovery systems created by two of the authors
\cite{colton2002automated,pease07}.  This is treated via four
snapshots of the development process, each of which is evaluated
using the model, to discern the serendipity potential (if any) for
the system described at that point in time.

\begin{ex}[HR constructs the concept of the central elements in a group]\label{ex:central}
The {\sf HR} system\footnote{Named after mathematicians Hardy (1877 - 1947) and Ramanujan (1887 - 1920).} \citet{colton2002automated} is a general purpose machine learning tool which performs automated discovery in a variety of domains. The user provides background domain knowledge describing axioms, concepts and examples, and the system uses production rules to discover new concepts.  HR then conjectures empirical relationships between concepts, and evaluates its concepts and conjectures according to a set of interestingness measures.  When working in mathematics domains, it can link to McCune's theorem prover {\sf OTTER} (Organized Techniques for Theorem-proving and Effective Research)\cite{mccune:90} and model generator {\sf MACE} (Models And Counter-Examples) \cite{mccune:macemanual,zhang}, in order to prove interesting conjectures or find counterexamples.  In early applications its basic mode of operation was entirely deterministic.
%%

One early success was in the domain of abstract algebra, in which {\sf HR} re-discovered the concept of \emph{the central elements of a group} (the set of elements in a group that commute with every element in the group) \cite{colton2002automated}.  This is a core concept in Group Theory that appears in most if not all basic textbooks on the subject.  HR discovered this concept by applying its {\em compose}, {\em exists} and {\em forall} production rules in the following way:  

\begin{align*}
\left.
\begin{array}{c}
{[}a,b,c{]} : a*b=c\\
{[}a,b,c{]} : a*b=c
\end{array}
\right\}\:\:
&\mathbf{compose}\rightarrow [a,b,c] : a*b=c \:\text{\emph{\&\&}}\: b*a=c\\
&\mathbf{exists}\rightarrow [a,b] : \text{exists}\ c\ (a*b=c \:\text{\emph{\&\&}}\: b*a=c)\\
&\mathbf{forall}\rightarrow [a] : \text{all}\ b\ (\text{exists}\ c\ (a*b=c \:\text{\emph{\&\&}}\: b*a=c))
\end{align*}
\end{ex}

Relative to our model, the operations involved do not go much beyond
those of the pocket calculator discussed in Section
\ref{sec:calculator}.  In addition to perceiving new conjectures
(which it generates), we allow that these are given further attention
(by testing them with examples and checking them with external systems),
but in the context of our model, that is where things stop.

\begin{ex}[HR refutes a boring conjecture in monoid theory]\label{ex:monoid}
With a view to increasing the serendipity potential of {\sf HR} , Colton 
subsequently enhanced the system so that whenever it finds a
counterexample to a new conjecture, it tests to see whether the
counterexample also breaks any previously unsolved open conjecture.
In this case, the system's ``prepared mind'' takes the form of previous
experiences, background knowledge, a store of unsolved problems, as
well as skills and a current focus.  The new counterexample arises
due partly to factors beyond the system's control, in particular, the
built-in structure of the
domain.
%% dynamic complexity, or some other process independent of the
%% current focus.
This version of the system was tested in three test domains:  group theory (associativity, identity and inverse axioms), monoid theory (associativity, identity) and semigroup theory (associativity). When run in breadth first mode, i.e., with no heuristic search, during sessions with tens of thousands of production rule steps, there were no instances of open problems which were solved in this way. Amending the search strategy to random led to one instance of a newly generated counterexample solving a pre-existing conjecture in monoid theory, none at all in group theory and a handful of times in semi-group theory (there were three times when a new counterexample dispatched an open conjecture, and on one occasion, ten open conjectures were dispatched by one counterexample).  Not only was it rare, the conjectures which were disproved in this way could not be considered interesting: for instance, the monoidal conjecture which was an open problem disproved by a later counterexample was the following:
\begin{align*}
\forall b, c, d &\hspace{.2cm}(((b * c = d \wedge c * b = d \wedge c * d = b \wedge (\exists(e * c = d \wedge e * d = c)))\\
&\hspace{.2cm}\leftrightarrow(b * c = d \wedge (\exists f(b * c = f)) \wedge (\exists g(g * c = b)) \wedge d * b = c \wedge c * d = b)))
\end{align*}

\noindent This conjecture does {\em not} appear in textbooks on Monoid Theory.
\end{ex}

Alongside the attributes of perception and attention that were present
in Example \ref{ex:central}, we now have evidence of a focus shift,
since every newly-disproved conjecture is considered in the context of
each open conjecture, a potentially long series of (rather limited)
recontextualisations.  However, this evidence must be seen as rather
weak, since the same basic routine happens with each disproved
conjecture, whereas one might expect a true recontextualisation would
be data-specific.
%%
The explanation phase is somewhat better represented (namely via the
demonstration that the counterexample refutes the earlier conjecture)
but the bridge is wholly missing, so the system's findings can only be
regarded as path-dependent rather than serendipitous.

Note, as well, that mathematical theorems are generally more important
than disproved conjectures, and that the conjectures that were being
disproved were very unlike the ones that might appear in a textbook on
abstract algebra.  So, evidence for the valuation phase would also be
quite weak, if everyday standards of mathematical value were applied.

\begin{ex}[HRL undiscovers the platypus]\label{ex:platypus}
{\sf HRL} was an adaptation of {\sf HR}, developed by \citet{pease07} and 
based on a theory of argumentation that acknowledges the role of conflict and ambiguity in mathematical discovery.  The theory, based on the work of
\citet{lakatos}, can also be used to describe (some) real-world
discoveries in mathematics.  {\sf HRL} is a distributed system,
comprised of ``student'' and ``teacher'' agents, each running a copy
of Colton's {\sf HR}.  The systems have different input knowledge,
measures of interestingness, and different ways of producing concepts.
The overall system is organised into independent work phases, and
discussion phases, in which conjectures, concepts, and counterexamples
are communicated.  Students react to counterexamples using Lakatos's
methods.  One such discussion, developed around a simple theory of
animals, progressed as follows:
\begin{itemize}
\item[\emph{A}:] ``There does not exist an animal which produces milk and lays eggs.''
\item[\emph{B}:] ``The platypus does''
\item[\emph{A}:] {[}Checks new object against current theory. Finds it breaks 11\% of its conjectures{]}\newline ``The platypus is not an animal''
\item[\emph{B}:] {[}Finds that the platypus breaks 31\% of its own conjectures.{]}\newline ``Okay - I'll accept that.''
\end{itemize}
\end{ex}

We will discuss this example together with the following:

\begin{ex}[HRL formulates Goldbach's Conjecture]\label{ex:goldbach}
The same system could also do theory formation in basic number theory.
Here is another dialogue:
\begin{itemize}
\item[\emph{A}:] {[}Knows: numbers 10-20, integer, div, mult{]}\newline ``All even numbers are the sum of two primes.''
\item[\emph{B}:] {[}Knows: numbers 0-10, integer, div, mult{]}\newline ``2 is not the sum of two primes.''
\item[\emph{A}:] {[}Checks new object against current theory. It fits well and doesn't break any further conjectures{]}\newline ``Okay - I'll accept that 2 is a number. Then my conjecture is `All even numbers except 2 are the sum of two primes'.''
\end{itemize}
\end{ex}

Let's consider whether either of examples 3 and 4 meet our criteria.
%%
The system's basic perceptual abilities again rely on its generative
methods, drawing where relevant on external systems.  Agents develop
concepts, conjectures, theorems, and examples that are given
preliminary assessments: the most interesting findings are shared
during the ``discussion phase''.  This is reasonable evidence of
attention.  We saw evidence of a limited focus shift even for {\sf HR}
in Example \ref{ex:monoid}; somewhat more convincing context- and
data-specific behaviour is illustrated with {\sf HRL}'s agent model,
since each agent is working with its own theory, and can independently
decide what to do with the evidence shared by the other agents.
Causal models are established using external systems, which produce
proofs or check for counterexamples.  This allows the system to
generate simple explanations, for example, Agent \emph{A}'s assertion
that ``The platypus is not an animal.''
%%
In Example \ref{ex:platypus}, this explanation is simply accepted, but
in Example \ref{ex:goldbach}, Agent \emph{A} uses \emph{B}'s
counterexample to produce a new conjecture.  The reasoning can be seen
as a bridge to a new problem:
\begin{quote}
``All even numbers are the sum of
two primes.''\:$\rightarrow$\:``2 is not the sum of two
primes.''\:$\rightarrow$\:``All even numbers except 2 are the sum of two
primes.'' 
\end{quote}
This conjecture is considered interesting by the system for roughly
the same reason it is historically interesting: it is succinctly
stated but continues to evade proof.  However, since the conjecture
is already well known (and remains unproved), the simple
fact of its reformulation by {\sf HRL} has little chance of 
receiving the kind of recognition given to original mathematical
discoveries---of the sort that have in fact been made with {\sf HR}
\cite{colton2007computational}.

\subsection{Summary}
%%% Maybe stress here briefly how we differ from Pease et al's earlier paper in which the GH system was evaluated more positively?
%% As Table \ref{tab:systems-analysis} shows, our model
%% can effectively discriminate between systems that have little or no
%% potential to be serendipitous, and computational or interactive
%% systems that possess serendipity potential.
%%
{\sf DAYDREAMER} meets our criteria for \emph{serendipity in the
  system}, though two are only met weakly.  {\sf Colloquy of Mobiles}
meets the criteria only when viewed as a system for \emph{serendipity
  as a service.}  {\sf HRL} seems to have serendipity potential,
though our analysis here shows that in itself this does not
immediately confer tangible benefits over its predecessor system.

Negative examples can also be readily supplied.

The serendipity potential of {\sf DAYDREAMER} and {\sf Colloquy of
  Mobiles} might be increased in further rounds of prototyping.
%%
The source code for {\sf DAYDREAMER} is
online,\footnote{\url{https://github.com/eriktmueller/daydreamer}} and
\citet{pangaro2018serendipity} are building {\sf Colloquy of Mobiles
  2018} using contemporary technologies, intending to ``open-source
everything found and everything generated, including CAD numerical
models and engineering drawings''---so such progress may indeed be
possible.
 
The model also highlights areas where there is further room for improvement
that could confer specific advantages.
In particular, in future development, the system might be redesigned to make more
interesting focus shifts, e.g., rather than undiscovering the
platypus, the system might invent the category of monotremes and look
for any evidence of further examples.  Such a requirement would push
back on the perception phase, since the system might need to select
appropriate sources of data to search for examples.  Taken together,
this description of {\sf HR} and {\sf HRL} shows how a sequence of
development can unfold, bringing the features that our model
describes online one by one, and strengthening them over time.

%% \paragraph{Environmental factors}
%% \begin{itemize}
%% \item Dynamic world: 
%% \item Multiple contexts: Reasoning could operate across different domains, eg one agent in number theory, one in group theory.
%% \item Multiple tasks: 
%% \item Multiple influences: This is a multi-agent system, where
%% multiple software agents with different knowledge and goals
%% interact.
%% \end{itemize}

%% \paragraph{Skills of the discoverer in the computational case}

%% \begin{enumerate}
%% \item Prepared mind: the agent's background knowledge,
%% developed theory, store of unsolved problems, set of
%% production rules, interestingness measure, and revision
%% techniques, and current focus 
%% \item Perception of the new event is when agent receives the
%% conjecture ``all even numbers are the sum of two primes''.
%% Agent pays attention to the conjecture by re-constructing
%% it in order to put it into context with its existing theory
%% \item Bridge: the agent performs the theory formation step to
%% get the new (modified) conjecture. 
%% \item Result: 
%% \end{enumerate}

%% \paragraph{The value of the discovery in the computational case}
