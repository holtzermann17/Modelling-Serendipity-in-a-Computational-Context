\section{Discussion} \label{sec:discussion}

  %% This is related to the notion of \emph{open discovery} which
  %% ``leads the creative knowledge discovery process from a given starting
  %%   domain towards a yet unknown second domain which at the end of
  %%   this process turns out to be connected with the first one.''

  %% However, Connections between these fields seems
  %% unexplored in practical work, though Ada Lovelace had already
  %% acknowledged the role of ``various collateral influences, besides
  %% the main and primary object attained'' \cite[p.~696]{lovelace} in
  %% all additions to knowledge.

% To merge:
%% Austin \cite[p.~78]{austin1978chase} advances two further
%% variations on Pasteur's famous principle, namely: \emph{Chance favours
%%   those in motion} (and generalisations such as ``curiosity about many
%% things, persistence, willingness to experiment and to explore''), and
%% \emph{chance favours the individualised action} (i.e., ``distinctive
%% hobbies, personal life styles, and activities peculiar to [the]
%% individual'').  These variants suggest that certain properties of the
%% investigator both introduce and constrain possibilities for chance
%% encounters and unexpected connections.

%% Our work in this paper does not report on system development, but we
%% nevertheless we have argued that serendipity is not strictly beyond
%% the reach of computational implementations.  Indeed, 

The examples in the previous section show that serendipitous behaviour
can be exhibited in a meaningful sense by computer systems.
%%% Do they? No system has a full YES or SOME. 
%%% Also, this is not the goal: the examples should demonstrate that our evaluation framework is working, not that engineering serendipity in the system is possible. 
The demonstration of this claim has made use of a novel theoretical
synthesis that is compatible with other established perspectives on
serendipity.  We are not the first to argue that the potential for
serendipity can be increased---or, indeed, decreased---because of
technological design choices (e.g., \cite{danzico2010design},
\cite{newman2002designing}, \cite{melo2018}).
%%% Is this still the case, given the paper that you've found last weekend?
However, current thinking about AI policy has focused on establishing
control over the behaviour in autonomous systems
\cite{research-priorities}.  Much less attention has been given to
features that would allow autonomous systems to make beneficial use of
the surprises that are bound to be encountered in real-world
applications.

In Section \ref{sec:literature-summary}, we suggested that serendipity
is \emph{a form of creativity that happens in context, on the fly,
  with the active participation of a creative agent, but not entirely
  within that agent's control.}  We have focused on what Edmonds
referred to as ``open system models'' \cite{Edmonds1994}.  The
perspective we developed is compatible with what
\citet{tonnessen2015uexkullian} calls ``Uexk\"ullian phenomenology.''
This conception of a world rich in interdependence across various
layers of mental processing is also compatible with Copeland's
\cite{copeland2017serendipity} assertion that serendipity is found in
networks and communities, and in mundane social encounters.
%% The basic concern about AI is that given a long enough leash (or,
%% lead) and access to a suitable energy source, more sensors and
%% actuators, more processing power, and so on, computational systems
%% will run amok.  We might do well to compare Bostrom's reflections on
%% superintelligence \cite{bostrom2014superintelligence} with Ostrom's
%% reflections on the commons \cite{ostrom1999revisiting}.  Consider
%% that, from an ecological perspective, we are already grasping with
%% existential risks due to anthropogenic climate change.

Assessing the risks involved in programming for serendipity clearly
depends on how `serendipity' is understood.  Thus for example
\citet[p.~158]{simonton2010creative} has suggested that `exceptional
creativity' is more likely to engage blind selection mechanisms, on
the view that ``blind variation generates the originality of an
idea.''  He cites {\sf BACON} \cite{langley1987scientific} as an
example of a blind but nonetheless ``systematic'' search program,
based on ``heuristic methods in which a solution is no longer
guaranteed'' \cite[p.~169]{simonton2010creative}.  However, this
system, and the related example of a ``blind'' radar search
\cite[p.~383]{campbell1960blind} should be contrasted with Austin's
\emph{`barking up the right tree' phenomenon}: ``if you happen to be
the kind of person who hunts afield, it may be, in fact, your dog who
leads you up to the correct tree, and to a desirable conclusion''
\cite[p.~50]{austin1978chase}.
\citet[p.~720]{kockelman2011biosemiosis} contends that just as ``one
cannot offer an account of significance without an account of
selection'' also ``one cannot offer an account of selection without an
account of significance.''  In order to make good use of serendipity,
systems will need to model both the anticipation and appreciation of
valuable outcomes in an uncertain world.

Notice that if `blindness' came in degrees, we might expect to see the
propagation of prediction error through a broader system that works to
reduce surprise over the long term, in line with predictive processing
accounts \cite{Kiverstein2017,Friston2012}.  Microsoft's
much-publicised misadventure with the Twitter bot {\sf Tay}
\cite{wolf2017we} provides us with an example of a system that
hindsight tells us was destined to fail.  Hopefully we can learn more
from this episode than simply to keep our AI systems locked in a `dark
room'.  We can at least imagine a future research programme that takes
inspiration from the `functionally integrated networks' in the brain
\cite{Pessoa2017} to design systems that interact in a socially and
ecologically responsible manner.
%% Our overall assessment is that ``blindness'' is neither a
%% sufficient nor even a necessary condition for serendipity.
%% %%
%% At least, it is clearly possible for a sequence
%% of accurate local optimizations to result in an unexpected global structure,
%% as demonstrated convincingly in various evolutionary simulations \cite{alife2018cases}.
%% %%% cf. ``Kant and the Platypus: Essays on Language and Cognition'', p. 90.
 
%% While Copeland emphasises that ``serendipity is a category that can
%% only be applied retrospectively to a discovery process'' 
%% \cite[p.~7]{copeland2017serendipity}, she also
%% mentions several skills and cultural traits that can be cultivated to
%% encourage serendipity, such as the early sharing of research results.
%% Although we have presented the steps of our model building on one
%% another in sequence, feedback loops are allowed, and 
%% experimentation with different architectures will be important.  We
%% outline several potential applications later in this section.

Our model of serendipity examines necessary factors of serendipity
potential, but aside from requiring each facet of the model, we have
not had a lot to say about specific techniques or strategies for
encouraging serendipity.  \citet{bjorneborn2017three} expands upon
that theme in considerable detail.  He puts forward three major
``personal factors in serendipitous encounters'': \emph{curiosity},
\emph{mobility}, and \emph{sensitivity}.  These correspond to three
parallel environmental factors or affordances, which he calls
\emph{diversifiability}, \emph{traversability}, and
\emph{sensorability}.  Both sides of this balance are then described
in terms of sub-factors, with ten on each side.  While Bj\"orneborn
here traces an interesting parallel between agent and environment, he
does not comment explicitly on a parallel with the classic theory of
mind in three parts, namely the ``\emph{conative},''
``\emph{cognitive},'' and ``\emph{affective}''
\cite{hilgard1980trilogy}.
%%
Thus for example \citet[p.~347]{boden1998creativity} comments that
creativity ``involves not only a cognitive dimension (the generation
of new ideas) but also motivation and emotion.''
%%
These dimensions may be active throughout the processes we've
described, which is why they did not fit cleanly into the alignment of
theories in Table \ref{tab:theory-summary}.

In a recent paper exploring serendipity in computer-generated fiction,
\citet{mccallum2018} reflect on how structured thinking about
serendipity can help designers of AI systems take advantage of the
``productive and perilous moment [\ldots] in which an unexpected event
or pattern occurs'' (p.~7).  They cast doubt on whether computational
art should be constrained to resemble human art.
\citet{gucklesberger2017addressing} tackle related issues with a
series of ``why questions'' which, they suggest, computational systems
would need to be capable of addressing in order to be seen as
`creative'.  It may also be the case that full and accurate explanations
would have to be highly complex.  \citet{loughran2018serendipity} have
drawn on the concept of ``cybernetic serendipity'' in their design of
a music system driven by a {``}`circular-causal' loop,{''} which
employs a population of evolving critics to build an emergent fitness
function, which in turn guides the evolution of melodies.  The critics
give the musical creators an ever-new problem to solve.

Peffers et al.'s model of design science research, mentioned in
Section \ref{sec:background}, has been employed to scaffold various
design tasks.  One recent example is a methodology for component-based
program synthesis that uses a type theoretic descriptions to align
components \cite{10.1007/978-3-030-03427-6_35}.
\citet{pease2013discussion} had outlined a related experiment that
would introduce serendipity into the construction of program
flowcharts.  Recently, a graphical model of dataflow and execution
dependence was used to establish state of the art results in several
prediction tasks for multiple input source languages
\cite{NIPS2018_7617}.  In contrast to the straw man (straw-bot?)
example from evolutionary computation presented in Section
\ref{sec:why-this-matters}, technologies like type theory and
distributional semantics \cite{DBLP:journals/corr/abs-1803-09473}
could be used by future systems to interpret their results.

% \cite{KWISTHOUT201784} - six different ways to lower prediction error

%% Insofar as human mental processing has evolved to elegantly handle inputs, kinematics, actions, and intentions \cite[p.~85]{KWISTHOUT201784} we could similarly be said to be designed for serendipity.  There is clearly quite a way before we have computational systems with comparable abilities.
